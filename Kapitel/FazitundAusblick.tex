\chapter{Fazit und Ausblicke}
\section{Rückschlüsse}
Über die ganze Arbeit lässt sich einen Sachverhalt klarstellen - der Aufbau und die Funktion eines intelligenten Schweißroboters ist vielfältig und komplex. Es gibt mehrere Hardware- und Softwarekomponenten, die für eine richtige Funktionsweise des Roboters integral sind. Der Roboter ist der Hauptteil der Hardwarekomponenten. Er verleiht dem System eine Handhabungseinrichtung, die für die Manipulation der Werkzeuge, die Schweißpistole, wichtig ist. Um mit Menschen zu kollaborieren muss der Roboter allerdings ein paar Sicherheitsanforderung erfüllen. Neben einer ungefährlichen Geometrie wird die Geschwindigkeit sowie Kraft und das Drehmoment des Roboters beschränkt. Er wird mit verschiedenen Funktionen zur Kollisionsdetektion ausgestattet, um lebensgefährliche Verletzungen zu vermeiden.

Die Sicherheitsanforderungen sind nicht nur durch den Roboter zu erfüllen, sondern durch die ganze Zelle zu gewährleisten. Es gibt Notaustaster, die den Roboter zu einem sofortigen Stopp bringen würden und die Schweißpistole sofort ausschalten würden. Der Arbeitsbereich des Cobots wird während des Betriebs durch eine physische Zelle getrennt, die transparente folierte Wände hat. Dies ermöglicht eine Überwachung des Systems während des Betrieb, ohne die Gefahr einer Verblendung. Der giftige Schweißrauch, welcher bei Schweißen entsteht, enthält aerosolierte krebserzeugende Partikeln. Dieser Rauch wird letztlich durch ein Absauggerät entfernt und gefiltert bevor die Luft wieder in der Umgebung eingeführt wird.

Die Schweißquelle liefert nicht nur den erforderten Strom für die Schweißaufgabe, sondern bietet auch zusätzliche Funktionalitäten zur Stabilisierung und Verbesserung des Schweißprozesses an. Die Stromquelle des Robotersystems erfüllt genau diese Anforderungen und steuert darüber hinaus nicht nur den Drahtvorschub, sondern auch den Fluss des Schutzgases für das Metall-Schutzgas-Schweißen. Letztendlich ist die Schweißquelle auch Kommunikationsfähig. Die Schweißquelle des Robotersystems ist in der Lage, über mehrere gängige Industrieprotokolle mit anderen Geräten zu kommunizieren, die die Schweißquelle auch steuern können. 

Innerhalb des Netzwerks des Robotersystems kommunizieren die Steuereinheit des Roboters sowie der Schweißquelle, der Laserliniensensor und ein Industrierechner. Neben der Kommunikationsschnittstelle, die der Lasersensor anbietet, wurde er aufgrund seiner hohen Ausgabeauflösung, anpassbare Pulsfrequenz und Genauigkeit gewählt. Der IPC, worauf das processit-Programm läuft, ist für die rechenintensive Aufgabe der Detektion sehr leistungsfähig.

Das processit-Programm bietet drei Hauptfunktionen an. Das erste Teil ist dient zur Interaktion mit dem Laserliniensensor. Es übernimmt die Aufgabe der Parametersetzung, vorverarbeitet die Messergebnisse aus dem Sensor und stellt sie zur Verfügung bereit. Das zweite Teil des Programmpakets enthält Methoden zur Erkennung unterschiedlicher Schweißnähte oder einer Offline Werkzeugbahn für die Bearbeitung einer Freiformoberfläche. Das Detektionsverfahren kann durch externe Prozesse gesteuert werden. Der dritte Teil von processit widmet sich der Prozessteuerung und Bahngenerierung. Es steuert die Teilvorgänge des Detektionsverfahrens indem es die Methoden zur Detektion aufruft und kontrolliert. Letztendlich werden die Erkenntnisse aus dem Detektionsverfahren für die Erstellung einer Bahn für den Roboterarm verwendet, die danach an dem Programm für die Trajektorienplanung übermittelt wird. Die Basis von processit ist ROS, welches die Kommunikation zwischen den unterschiedlichen Teilprogramme und Hardwarekomponenten ermöglicht. Hierfür werden verschiedene standard sowie benutzerdefinierte ROS-Messages verwendet, die den Kernpartikel der ROS-basierten Kommunikation bilden. 

Diese Software- sowie Hardwarekomponente erfüllen die Anforderungen eines intelligenten kollaborierenden Robotersystems für die automatische Erkennung der Schweißnaht.

\section{Implikationen für die Industrie und Wissenschaft}

Die Robotik findet bislang überwiegend bei großen Unternehmen in der Produktion eine Anwendung. Roboter bieten bei der Abwicklung Produktionsaufgaben eine hohe Wiederholgenauigkeit an und ermüden sich nicht. Sie steigern durchaus die Effizienz und Produktivität in vielen Prozessen. Konventionelle Ansätze waren allerdings nur für die Fertigung in großen Losgrößen rentabel. Mit einem intelligentem kollaborierendem Schweißroboter-System kann die Einsetzbarkeit der Robotik für kleine und mittelständische Unternehmen verbessert werden. Das intelligente System ist in der Lage, das Schweißverfahren mit der Genauigkeit eines Roboters und die Intuition eines Menschen zu beherrschen. Mit dem Einbinden eines Laserliniensensor wird das System mit einem technischen Auge ausgestattet, welches die Programmierung des Roboters um ein Vielfaches erleichtert. Bauteile können schnell in kleinen Losgrößen ohne hohen Zeit-, Kosten- und Menschenaufwand gefertigt werden. Das demographische Problem des Fachkräftemangels kann auch durch dieses Robotersystem gelöst werden. Es ist fähig, mit anderen Mitarbeitern zusammenzuarbeiten, ohne sie in Gefahr zu bringen.

Da das processit-Paket mit starkem Einfluss des Entwurfsprinzips der losen Kopplung entwickelt wird, kann es sehr einfach durch zusätzliche Funktionalitäten erweitert werden. Mittels ROS können sehr schnell diese Funktionalitäten in das Hauptprogramm integriert oder ausgeschlossen werden. Die Hardwarekomponenten sind genauso modular wie die Softwareteile. Das System lässt sich für die Abwicklung anderer Aufgaben schnell umrüsten. Somit bietet sich das System als eine gute Basis zur Entwicklung neuer intelligenter Verfahren zur Abwicklung anderer Herstellungsprozesse an. Entwicklungen im Bereich der Computer-Vision, Trajektorienberechnung, Robotergenauigkeit und maschinelles Lernen sind wichtige Beitragsleistungen zu der Wissenschaft. Diese bieten Plattformen für weitere Forschungen in diesen Bereichen an.

\section{Limitationen und Zukunftspotenzial}
Der Zweck dieser Arbeit war es, die Zusammensetzung und Funktion des Robotersystems klarzulegen. Dies ist ein großes Projekt, welches durch mehreren Beteiligten betreut wird. Aus diesem Grund reicht die Umfang dieser Arbeit nicht aus, sehr detailliert die einzelnen Komponenten des Systems anzuschauen. Daher konzentriert sich diese Arbeit nur auf die wichtigsten Komponenten des Systems

Das Robotersystem hat auch technische Limitationen. Aufgrund der physischen Einschränkungen des Roboters ist die Größe des zu bearbeitenden Werkstückes beschränkt. Das Schweißen sehr großer Bauteile ist somit nicht mit diesem System möglich. Die Erkennung einer Stumpfnaht sowie einer offline Werkzeugbahn befinden sich in primären Phasen. Diese Methoden sind noch nicht ausgereift und müssen verbessert werden. Das Programm ist aktuell nicht in der Lage, die Schweißparameter während eines laufenden Prozesses auf Basis der Werkstückgeometrie anzupassen. Dies kann Einflüsse auf die Schweißqualität haben.

Es besteht das Potenzial, diese Limitationen grundsätzlich zu beseitigen. Die Methoden zur Erkennung der Stumpfnaht und Offline Werkzeugbahn sollen weiterentwickelt werden, um die Einsatzmöglichkeiten des Robotersystems zu verbessern. Weiterhin soll die Hardware und Software des Systems erweitert werden, um neue Werkzeuge und Herstellungsprozesse einzugliedern. 

Aufgrund des Ansatzes zur Detektion der Schweißnaht ist das Verfahren nur auf die Erkennung von Ebenen aus der Punktwolke beschränkt. Aus diesem Grund können geometrische Merkmale des Werkstückes wie Löcher, Schlitze, Stoßgeometrien, etc. nicht durch das Programm erkannt werden. Dies führt dazu, dass die Schweißparameter auf Basis der Werkstückgeometrie nicht angepasst werden. Es besteht hier das Potenzial, durch eine Überarbeitung des Erkennungsverfahrens, diese geometrische Merkmale ins Betracht zu ziehen. Somit können Schweißparameter auf Basis dieser Merkmale angepasst werden.

Die Automatisierung ist ein brandaktuelles Zukunftsthema. Flexible und intelligente Produktionssysteme, die sich einfach einrichten und anwenden lassen, werden zukünftig ausschlaggebend für die Konkurrenzfähigkeit der kleinen und mittelständischen Unternehmen sein. Die Entwicklung solcher Systeme ist im Gegensatz dazu die Herausforderung von Heute.
