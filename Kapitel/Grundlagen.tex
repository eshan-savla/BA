\chapter{Grundlagen}
Die Robotik deckt ein sehr breites Spektrum ab. Unter dem Begriff sind verschiedene Roboterarten mit unterschiedlichen Merkmalen und Eigenschaften inbegriffen. Aufgrund dieser Diversität ist es wichtig, dass Roboter auf Basis gewisse Merkmale untereinander differenziert werden. Eine Unterscheidung kann nach der Konstruktionsweise, Funktionsweise oder Verwendungszweck erfolgen. Unter Annahme einer Differenzierung nach der Konstruktionsweise können Roboter in vier Arten eingeteilt werden - stationäre Roboter, mobile Roboter, kognitive Roboter und humanoide Roboter. \autocite[24]{maier2022grundlagen}. Da humanoide Roboter keine breite Anwendung in der Industrie finden und durch Firmen wie Tesla, Inc. \autocite{morris2022} entwickelt werden, werden sie zuerst vernachlässigt. Hauptsächlich sind Industrieroboter in der Form von mehrachsigen stationären Robotern in der Produktion zu finden \autocite[25]{maier2022grundlagen}.

\section{Stationärer Industrieroboter}
Die Hauptaufgabe des Industrieroboters befasst das Bewegen und Führen eines Endeffektors zu einer Zielposition in einem vordefinierten Raum. Dieser Endeffektor kann ein Mittel zum Zweck des Greifens, Messens oder der Bearbeitung eines Werkstückes sein. Das Tool Center Point (TCP) oder der Werkzeugmittelpunkt bezeichnet einen charakteristischen Punkt des Endeffektors, welcher als einen Bezugspunkt für die Bewegungen des Roboters benutzt wird. Um den Endeffektor bewegen zu können, wird er als das letzte Armteil einer Aneinanderreihung von Armteilen angebracht, die miteinander durch Gelenke verbunden sind. Diese Gelenke zusammen mit den Größenverhältnisse der Arme spezifizieren die Bewegungsmöglichkeiten des Roboters. \autocite[64]{maier2022grundlagen}

Die Aneinanderreihung der Armteile des Industrieroboters definiert auch gleichzeitig seine kinematische Kette. Die kinematische Kette eines Roboters gibt alle Bewegungsmöglichkeiten und Positionen, die der Roboter annehmen kann, vor. Auf Basis der kinematischen Kette und der mechanische Aufbau kann eine Einteilung nach Bauart und sich daraus darbietendem Arbeitsraum erfolgen. Die Bauart eines Roboters ergibt sich aus der Art der Gelenke, die in der mechanischen Konstruktion benutzt wurden. Schubgelenke ermöglichen dem Roboter eine translatorische Bewegung mittels translatorischen Achsen. Drehgelenke ermöglichen dem Roboter im Gegensatz eine rotatorische Bewegung, indem sie eine rotatorische Achse bilden. \autocite[114-115]{maier2022grundlagen}

Knickarm-Roboter oder Gelenkarmroboter werden aus drei serielle Drehgelenke zusammengesetzt und besitzen somit eine RRR-Kinematik. Die serielle Kinematik bezeichnet eine reihenweise Anordnung einzelner Bewegungsachsen - eine Serie von Bewegungsachsen \autocite[84]{maier2022grundlagen}. Diese Bauart des Roboters findet am häufigsten eine Anwendung in der Industrie. Das Fundament eines vertikalen Knickarm-Roboters ist eine horizontal drehbare Basis, worauf weitere vier bis fünf Gelenke aufgebaut werden. Diese Gelenke sind nicht nur drehbar, sondern auch vertikal beweglich. Eine parallele Bewegung dieser Achsen ermöglicht dem Roboter das Ausführen von komplexen Bewegungsabläufen. Diese Flexibilität führt zu einem fast kugelsegmentförmigen Arbeitsraum, indem der Roboter sich bewegen kann. Aufgrund dieser hohen Beweglichkeit lässt sich der Knickarm-Roboter sehr gut für Produktionsaufgaben wie das Beschichten, Fräsen oder Schweißen einsetzen. \autocite[116-117]{maier2022grundlagen}

\begin{figure}[h]
	\includegraphics[width = \textwidth]{Abbildungen/Knickarm-Roboter.png}
	\centering
	\caption{Kinematische Kette eines Knickarm-Roboters \autocite[116]{maier2022grundlagen}}
\end{figure}

\subsection{Cobots} \label{ssec:cobots}
Einige Aufgaben der automatisierten Produktion sind manchmal sehr komplex. Um diese Komplexität zu überwinden, wird ein kollaboratives Arbeitsverhalten zwischen dem Roboter und Mensch gefördert. Kollaborierende Roboter, oder Cobots, sind Roboter die speziell für die Zusammenarbeit mit Menschen ausgelegt sind. Cobots sind Industrieroboter, die hinsichtlich der Sicherheitsgewährleistung nach dem Norm DIN EN ISO 10218 angepasst und geändert werden. Es bestehen hohe Sicherheitsanforderungen für Cobots, wie die Anwendung einer Leichtbauweise, zur Vermeidung jegliche Menschenverletzungen, da diese Roboter im Gegensatz zu herkömmlichen Industrieroboter in direktem Kontakt mit Menschen kommen. \autocite[181-183]{maier2022grundlagen}

In der Mensch-Roboter-Kollaboration überschneiden sich die Arbeitsräume des Menschen und des Roboters. Deswegen werden sie auch von einander durch eine physische Separation wie einen Schutzzaun oder eine Trennwand nicht abgetrennt. Aufgrund der fehlenden physischen Barriere steigt das Risiko einer Kollision zwischen Mensch und Roboter enorm. Deswegen müssen einige Sicherheitsanforderungen für die Zusammenarbeit erfüllt werden. Bei einer fehlenden physischen Schutztrennung soll innerhalb des Arbeitsraums des Roboters einen Gefahrenbereich identifiziert werden. Diese Zone soll mittels Schutzeinrichtungen wie Lichtschranken oder Laserscanner überwacht werden, sodass beim Betreten der Zone, ein Stoppsignal an dem Roboter sofort versendet werden kann. Cobots werden öfter zur Zielposition oder zwecks der Teach-In-Programmierung durch den Mensch per Hand geführt. Hierbei sollen Kraftsensoren im Roboter die Schubrichtung erkennen und die Bewegung in der Richtung mittels Motoren erleichtern. Gleichzeitig soll dabei die Geschwindigkeit durch den Roboter geregelt werden. Wichtig ist der Einbau einer Kollisionserkennung. Hierbei soll der Roboter eine Kollision mit einem Gegenstand über Kraft- und Impulssensoren erkennen und sofort aufhalten. Somit können ernsthafte lebensgefährliche Verletzungen vermieden werden. \autocite[12-16]{Schleicher2020}

Die Menschen-Roboter-Kollaboration ist neben der herkömmlichen Anwendung in der Produktion auch häufig in der Intralogistik zu sehen. Hinsichtlich der pünktlichen und kostengünstigen Warenlieferung werden Roboter zunehmend in diesem Bereich eingesetzt. Zur Kanalisierung des Warendurchflusses innerhalb des Warenhauses werden mobile Roboter für den Transport verwendet. Allerdings teilen sich diese Roboter eine Fläche mit anderen Menschen und müssen auf diese achten. Somit entsteht ein Netzwerk mobiler Roboter, die dem kollaborierenden Menschen keinen Gefahr stellen. Cobots werden auch anderweitig als Trag- und Hebehilfen für die menschlichen Mitarbeiter verwendet. \autocite[52]{Glück2022}

Eine Kollaboration mit Robotern bringt mit sich einige Verbesserungen der Prozessflexibilität, Prozesseffizienz und Ergonomie der Arbeit. Durch die Verwendung von Cobots können Fähigkeiten der Mitarbeiter und Roboter miteinander harmonisiert werden, sodass es sowohl von den Stärken der Menschen als auch der Robotern profitiert werden kann. Cobots dienen nicht nur als Instrumente der Prozessverbesserung sondern auch zur Erleichterung anstrengender Aufgaben. Somit können für die Mitarbeiter mehr Arbeitssicherheit gewährleistet sowie berufsbedingte körperliche Verletzungen und Krankheiten vermeidet werden. Die Menschen-Roboter-Kollaboration kann auch als Mittel gegen demographischen Probleme wirken. Ländern wo der Durchschnittsalter höher wird stehen vor einem künftigen Fachkräftemangel. Dieser kann durch die Verwendung von Cobots kompensiert werden, die die fehlende Arbeitskraft ersetzen können. \autocite[50]{Glück2022}

Die angesprochenen Problematiken sind in der Schweißindustrie besonders stark zu finden, weswegen Schweiß-Cobots zuletzt einen regelrechten Boom erlebt haben, und insbesondere als erleichterten Einstieg in die Robotik zu sehen sind. Nach dem Verstehen der grundsätzlichen Funktionsweise eines Industrieroboters lohnt sich ein Einblick in einem seiner Anwendungsfälle. Industrieroboter werden neben anderen Bereiche auch in der Schweißtechnik eingesetzt. In diesem Gebiet werden verschiedene übliche Schweißverfahren durch mehrere Modifizierungen angepasst, sodass sie durch Roboter ausgeführt werden können. Das Verstehen in dieser Prozesse und ihrer Parameter sowie deren Effekte ist für einen effektiven Einsatz der Schweißrobotik notwendig.

\section{Die Schweißrobotik}

\subsection{Warum die Schweißrobotik?}
Die wirtschaftlichen Vorteile der Industrieroboter in Einsatz für die Produktion wurden schon in ~\ref{sec:Industrieroboter} diskutiert. Heutige Marktanforderungen für das Schweißen setzen es voraus, dass auch nicht standardisierte, kundenspezifische Werkstücke schnell, kostengünstig und in kleineren Losgrößen hergestellt und verarbeitet werden können. Roboter sind dank ihrer Programmierbarkeit und diversen Effektoren für unterschiedlichen Aufgaben sehr flexibel und eignen sich für den Einsatz in der Produktion unter heutigen Anforderungen. Sie können für das Schweißen von unterschiedlichen Werkstücken mit abweichenden Geometrien schnell angepasst werden. Roboter besitzen auch speicherprogrammierbare Steuerungen (SPS) und sind somit in der Lage, mit Sensoren sowie der Schweißquelle zu kommunizieren und diese zu steuern. Das automatisierte Schweißen benötigt eine Bahnsteuerung und sensorische Information über den Prozess. Eine Steuereinheit und Sensorintegration lassen dem Roboter diese Fähigkeiten zuteilwerden und ermöglichen die Koordination zwischen der Schweißquelle, Schweißpistole und dem Schweißverfahren. \autocite[17-22]{Pires_WeldingRobots_2006}

Es gibt unterschiedlichen Schweißverfahren, die in der Industrie für die Metallverarbeitung verwendet werden. Aufgrund der unterschiedlichen Schweißmethodiken gibt es verschiedenartige Parameter, die den Prozess beeinflussen und somit durch den Schweißroboter kontrolliert werden müssen.

\subsection{Wolfram-Inertgas-Schweißen}
\subsubsection{Verfahren}
Dieses Schweißverfahren (\emph{WIG}) ist auf dem Lichtbogenschweißverfahren basiert und verwendet dazu einen inerten Gas zum Schutz des Schweißbades. Bei dieser Methode entsteht ein Lichtbogen zwischen ein´er Wolfram-Elektrode und den Werkstücken, wodurch Wärme generiert wird. Diese Wärme führt zu dem Schmelzen der Werkstückoberflächen und einer Zusatzelektrode sowie zu der Formation eines Schweißbades, das aus flüssigem Metall besteht. Dieses Schweißbad wird durch das Inertgas umhüllt und wird somit gegen Fremdstoffe und Verunreinigungen in der Luft beschützt. Durch das Abkühlen des Schweißbades entsteht eine solide und starke Verbindung der Werkstücke. Eine wesentliche Merkmale dieses Verfahrens ist die Wolfram-Elektrode. Diese ist eine Dauerelektrode und wird während des Prozesses nicht verbraucht. Die, durch dieses Verfahren hergestellten Teile haben eine höhere Schweißqualität, weniger Bauteilverzug und kaum Metallspritzflecken. \autocite[27-28]{Pires_WeldingRobots_2006}

\subsubsection{Geräte}
Das WIG-Schweißverfahren besteht hauptsächlich aus vier Komponenten. Die Schweißquelle oder Stromquelle ist für die Bereitstellung der nötigen Spannung und Strom für das Schweißen zuständig. Die Schweißquelle besteht aus Stromrichter, Wechselrichter und einen Transformator. Der Wechselstrom aus dem Hauptnetz wird zuerst in Gleichstrom umgewandelt. Danach folgend wird er wieder rückgängig in Wechselstrom mit einer höheren Frequenz als den Netzstrom gewandelt. Dank einer Frequenzerhöhung kann der Transformator die gleiche Leistung bei einem kleineren Form und niedrigeren Gewicht leisten. Somit bleibt die Schweißquelle leicht transportable. Nach einer Invertierung wird der Hochspannungsstrom zu einer niedrigeren, zum Schweißen geeigneten, Spannung gebracht. Dieser Strom wird wahlweise danach zum Gleichstrom verwandelt. Mittels eines Regelkreises wird die Spannung des Ausgangsstroms überwacht. Meisten Stromquellen bieten die Möglichkeit an, zwischen Gleich- oder Wechselstrom für den Ausgangsstrom zu wählen. \autocite[28]{Pires_WeldingRobots_2006}

Die Schweißpistole dient nicht nur als Halterung der Dauerelektrode, sondern auch zur Übertragung des Stroms zu der Dauerelektrode sowie zur richtigen Führung des Inertgases zum Schweißbad. Je nach Leistung und Einsatzdauer der Schweißpistole wird besteht der Bedarf einer Gas- oder Wasserkühlung. \autocite[29]{Pires_WeldingRobots_2006}

Dauerelektroden bestehen aus pures Wolfram oder eine Wolframlegierung. Während pures Wolfram ein paar Vorteile in der Anwendung hat, wird eine Wolframlegierung mit zwei prozentiges Thorium-Oxid (Thorium-Oxid Wolfram) weit in der Industrie verwendet. Diese Legierung hat eine hohe Beständigkeit gegen Kontaminierungen, einen kleiner Wartungsbedarf und erzeugt stabileren Lichtbogen. Aufgrund der Radioaktivität von Thorium-Oxid werden auch andere Alternativlegierungen verwendet. Beispielsweise findet eine Wolframlegierung mit Zirkonium auch eine häufige Anwendung, da sie sehr ähnlichen Eigenschaften zu Thorium-Oxid Wolfram hat. \autocite[29-30]{Pires_WeldingRobots_2006}

Der Inertgas-Regulator wird zur Reglung des Gasdruckes verwendet. Schwankungen in dem Druck der Gasquelle können somit ausgeglichen werden. Die Druckreduzierung und Regulierung kann in einem oder zwei Phasen erfolgen, wobei die zweite Variante eine stabilere Ausströmung anbietet. \autocite[31]{Pires_WeldingRobots_2006}

\subsubsection{Prozessparameter}
Nachfolgend wird diskutiert, wie Prozessparameter des Schweißverfahrens die Qualität der Schweißnaht beeinflussen. Schweißgeräte verleihen die Kontrolle über diese Parameter, wodurch die Qualität des Prozesses geregelt werden kann. 

Der Schweißstrom wirkt auf die Form der Schweißnaht ein. Diese ist als den Querschnitt der Naht vorzustellen, die orthogonal zu der Schweißrichtung ist. Durch die Verwendung von Wechselstrom oder Gleichstrom unterschiedlicher Polaritäten entsteht. Einen Gleichstrom an der negativen Elektrode liefert die besten Ergebnisse mit einer besseren Einbrandtiefe und Schweißgeschwindigkeit. Schweißen mit Gleichstrom an der positiven Elektrode liefert im Gegensatz eine sehr geringe Einbrandtiefe. Schweißen mit Wechselstrom stellt das mittlere Ergebnis dar. Wechselstrom-Schweißen liefert beim Schweißen von Metallen wie Aluminium und Mangan gewisse Vorteile, indem es die oxidierte Oberflächenschicht dieser Stoffe beim Schweißen entfernt. Gepulster Gleichstrom wird häufig zur Reduzierung des Bauteilverzuges verwendet. Die Periode und Stärke des Stromes wird anhand der Materialeigenschaften und Werkstückprofilgeometrie festgelegt. Die Frequenz hat einen Einfluss auf diversen Eigenschaften, wie die Einbrandtiefe, maximale Schweißgeschwindigkeit und reduzierte Porosität der Schweißnaht. Die Einstellung der Schweißstrom-Eigenschaften erfolgt über die Stromquelle. \autocite[31-32]{Pires_WeldingRobots_2006}

Die Schweißgeschwindigkeit hat einen direkten Einfluss auf die Erwärmung der Werkstücke. Dies geschieht ohne eine Änderung der Lichtbogen-Eigenschaften. Mit einer Erhöhung der Schweißgeschwindigkeit wird der Querschnitt der Schweißnaht reduziert sowie, zu einem niedrigeren Grad, die Einbrandtiefe und Breite der Naht. Anhand des Materials und Geometrie des Werkstückes sowie die Eigenschaften des Schweißstroms werden standardmäßig Schweißgeschwindigkeiten von 100 bis zu 500 mm/min gewählt. Da die Schweißpistole als Endeffektor eines Roboters eingesetzt wird, wird die Schweißgeschwindigkeit durch die Roboterbewegung bestimmt. \autocite[33]{Pires_WeldingRobots_2006}

Die Lichtbogenlänge wird aus der Distanz zwischen der Werkstückoberfläche und Dauerelektrode ermittelt. Die Lichtbogenlänge und Spannung aus der Stromquelle stehen in einem direkten Zusammenhang mit einander. Mit einer Vergrößerung der Lichtbogenlänge muss zur Gewährleistung der Lichtbogenstabilität die Phasendifferenz zwischen der Elektrode und Werkstück erhöht werden. Die Lichtbogenlänge darf vergrößert werden, wenn die Erwärmung der Werkstücke reduziert werden soll, da mehr Wärme durch Strahlung verloren wird. Demzufolge wird die Einbrandtiefe und der Querschnitt der Schweißnaht reduziert. Die Lichtbogenlänge hängt von einer physikalischen Größe ab - die Distanz. Deswegen kann sie durch den Schweißroboter beeinflusst werden, indem die Position der Schweißpistole manipuliert wird. \autocite[33]{Pires_WeldingRobots_2006}

Der Auswahl des Schutzgases spielt eine wesentliche Rolle bei der Prävention von Verschmutzung der Schweißnaht durch Umwelteinflüsse. Eine Verunreinigung kann die Qualität der Schweißnaht und die Stärke der Schweißverbindung deutlich beeinträchtigen. Zusätzlich hat der Schutzgas auf die Lichtbogenzündung sowie -Stabilität eine Ausprägung. Am häufigsten wird der Inertgas Argon aufgrund seiner niedrigen Ionisation und schwerem Gewicht als Schutzgas verwendet. Diese Eigenschaften fördern die Lichtbogenentzündung und Stabilität der Luftblase um den Schweißbad herum. Ein weiterer Vorteil dieses Gases ist sein niedriger Preis im Vergleich zu anderen Inertgasen wie Helium. Wenn höherer Temperaturen zum Schweißen dickerer Werkstücke oder leitfähigerer Materialien erfordert werden, wird Helium als Schutzgas verwendet. Aufgrund seiner höheren Ionisation muss die Spannung des Schweißstroms für eine erfolgreiche Lichtbogenentzündung und stabileren Lichtbogen erhöht werden. Folglich entstehen, die zum Schweißen nötigen höheren Temperaturen. \autocite[33-34]{Pires_WeldingRobots_2006}

Nach \textcite[499-503]{Abbasi2018} steht die Flussrate des Schutzgases in einer inversen Korrelation mit der Mikrostruktur und Beschaffenheit der Schweißnaht, zwar die Bildung von Kleinrissen und Pori. Während eine niedrige Flussrate zur Formung von Rissen und Poren führt, können sie durch höhere Flussrate vermeidet und die Zugfestigkeit sowie Duktilität des Materials verbessert werden. Laut \textcite{Tesfaw2022} sind bei anderen Schutzgas-Schweißmethoden auch Korrelationen zwischen der Gasflussrate und Schweißnaht-Qualität zu sehen, beispielsweise die Materialhärte. \textcite{Mvola2017} diskutieren über Methoden zur intelligenten und adaptiven Steuerung des Gasflusses. Mittels verschiedenen Sensoren wird Information über den Gasfluss gesammelt und diese in einer Reglung eingekoppelt. Somit ist nicht nur eine Kontrolle über die Gasflussrate möglich, sondern auch eine Anpassung der Gasmischung.

Der Winkel und die Stellung der Elektrode hat eine Einwirkung auf die Einbrandtiefe und Lichtbogendruck. Je kleiner der Winkel, desto besser ist die Einbrandtiefe und Lichtbogendruck. Allerdings führt es auch zur Verschlechterung der Schweißpistolenspitze. \autocite[34]{Pires_WeldingRobots_2006}

\subsection{Metall-Schutzgas-Schweißen}
\subsubsection{Verfahren}
Dieses Schweißverfahren (\emph{MSG}) ist in der Funktionsweise sehr ähnlich zum WIG-Schweißen. Auch hier wird das Werkstück und die Elektrode mittels eines Lichtbogens erhitzt. Der Unterschied zwischen den beiden Verfahren liegt an der Art der Elektrode und des Schutzgases. Die Elektrode dieses Verfahrens unterscheidet sich von der Dauerelektrode des WIG-Verfahrens dadurch, dass sie abschmelzt und während des Schweißens verbraucht wird. Somit ist kein externes Schweißzusatz für diese Methode benötigt. Aus diesem Zusatz entsteht die Schweißnaht. Es können außer Inertgase auch Aktivgase als Schutzgas verwendet werden. \autocite[36]{Pires_WeldingRobots_2006}

Das MSG-Verfahren findet weit und breit in der Industrie eine Anwendung. Es ist für das Schweißen dünne sowie dicke Materialien in allen möglichen Stellungen der Schweißpistole gleicht effektiv. Aufgrund der höheren Schweißgeschwindigkeit und die niedrigere Anforderung an Erfahrung und Fähigkeit des Bedieners ist diese Methode auch sehr wirtschaftlich. Da die Zufuhr des Schweißzusatzes automatisch erfolgt, kann sie maschinell gesteuert werden. All diese Vorteile machen das MSG-Verfahren für den Einsatz in der Schweißrobotik sehr geeignet. \autocite[37]{Pires_WeldingRobots_2006}

\subsubsection{Geräte}
Da die MSG-Verfahren eine sehr ähnliche Funktionsprinzip zu dem WIG-Verfahren haben, gibt es bei den beiden Verfahrensarten ähnliche Geräte, die dem gleichen Zweck erfüllen.

Die Stromquellen der MSG-Verfahren haben eine ähnliche elektrische Architektur wie die des WIG-Verfahrens. Sie liefern eine konstante Spannung, die zusammen mit einem konstanten Drahtvorschub für die Stabilität des Lichtbogens sorgt und einen Ausgleich bei abweichender Entfernung zwischen der SNeben Geräte teilen sich die beiden Schutzgasschweißverfahren auch sehr ähnliche Prozessparameter. chweißpistole und Werkstückoberfläche ermöglicht. Diese Quellen bieten auch die Möglichkeit an, den Strom in Intervallen zu erzeugen und leiten. Typischerweise werden Strompulse in einer Frequenz von 100 Hz bis 200 Hz generiert. \autocite[38]{Pires_WeldingRobots_2006}

Das Drahtvorschubgerät ist für diese Art des Schutzgasschweißens ein Alleinstellungsmerkmal, da die Zuführung des Schweißzusatzes in dem WIG-Verfahren durch den Anwender kontrolliert wird. Dieses Gerät ist meistens innerhalb des Schweißgerätes integriert. Es ist für die Abwickelung der abschmelzenden Elektrode aus einer Spule und deren Begradigung sowie Führung zu der Schweißpistole zuständig. \autocite[39-40]{Pires_WeldingRobots_2006}

Die Schweißpistole der MSG-Verfahren ist im Aufbau und in der Funktion sehr ähnlich zu der des WIG-Verfahrens. Ihre Hauptfunktion ist die Leitung von Strom zu der abschmelzenden Elektrode und die Schutzgasführung. Abhängig von dem Einsatz wird die Schweißpistole Wasser- oder Gasgekühlt. Diese Pistole ist einzigartig, indem deren Taster Steuersignale nicht nur an der Stromquelle und dem Gasventil sendet, sondern auch an dem Drahtvorschubgerät. \autocite[40]{Pires_WeldingRobots_2006}

\subsubsection{Prozessparameter}
Neben Geräte teilen sich die beiden Schutzgasschweißverfahren auch sehr ähnliche Prozessparameter. Ähnlich zu dem WIG-Verfahren wird abhängig von der Material und dem Einsatz entweder die abschmelzende Elektrode mit Gleichstrom sowie Wechselstrom oder das Werkstück mit Gleichstrom beliefert. Eine höhere Spannung führt zu einer flacheren und breiteren Schweißnaht, wobei eine zu hohe Spannung die Lichtbogenstabilität negativ beeinflusst und die Schweißnaht-Qualität verschlechtert. Eine niedrige Spannung verursachen eine höhere Schweißraupe, die zur besseren Bewehrung der Schweißnaht führt. Bei der Wahl einer richtigen Spannung und eines Stroms wird das Übertragungsmodus ins Betracht genommen. Die Übertragungsmodi bestimmen wie die Wärme sowie das Schweißzusatz an der Werkstückoberfläche angebracht werden und somit zum Werkstück übertragen werden. Der Wahl eines Übertragungsmodus wird auf Basis der Werkstückmaterial, gezielten Schweißqualität und anderen Parameter bestimmt. \autocite[41-42]{Pires_WeldingRobots_2006}

Im Gegensatz zu dem WIG-Verfahren hat die Schweißgeschwindigkeit der MSG-Verfahren keinen linearen Zusammenhang mit der Wärmeeinbringung zu dem Werkstück und Ablagerungsrate des Schweißzusatzes. Eine initiale Erhöhung der Schweißgeschwindigkeit zeigt einen höheren Einbrand in dem Werkstück, da die Werkstückoberfläche eine längere Exposition zu dem Lichtbogen hat. Eine Steigerung dieses Parameters ab einer Grenze führt zu einer Verschlechterung der Einbrandtiefe und Schweißnaht-Qualität, da das Schweißbad nicht mit genügendem Schweißzusatz beliefert wird. \autocite[42]{Pires_WeldingRobots_2006}

Elektrodenverlängerung steht für die Länge der Elektrode, die von der Schweißpistole herausragt. Sie wird durch die Distanz zwischen der Pistolenende und Werkstückoberfläche bestimmt. Dieses Parameter hat eine direkte Korrelation mit der Schmelzrate der abschmelzenden Elektrode und wird anhand des gewählten Übertragungsmodus angepasst. Das Elektrodendurchmesser wird auf Basis des Schweißstromes, der Materialeigenschaften und Werkstückgeometrie gewählt. \autocite[42-43]{Pires_WeldingRobots_2006}

Der Drahtvorschub regelt die Geschwindigkeit, mit der die abschmelzende Elektrode zu der Schweißpistole geführt wird. Somit ist auch die Masse des Schweißzusatzes kontrolliert, die zu dem Schweißbad und der Schweißnaht hinzugefügt wird. Integral für eine gute Schweißqualität ist eine konstante und ruckfreie Zuführung der Elektrode. Nach \textcite{Senthilkumar2017} stehen die Abschmelzrate des Schweißzusatzes und der benötigte Schweißstrom mit dem Drahtvorschub im Zusammenhang. Auch in der gleichen Studie wurde festgestellt, dass ein höherer Drahtvorschub in einer höheren Schweißraupe resultiert.

Neben der in dem WIG-Verfahren verwendetem Inertgas werden üblicherweise auch aktive Gase in den MSG-Verfahren verwendet. Während Inertgase zum Schweißen reaktive Materialien wie Kupfer und Nickel verwendet wird, werden Aktivgase für diverse Stähle benutzt. Aktive Gase werden häufig auch mit Inertgasen gemischt. Das Sauerstoff in Argon-Sauerstoff-Mischungen für Edelstähle und das Kohlendioxid in Argon-Kohlendioxid-Mischungen für Kohlenstoffstähle trägt zur Stabilisierung des Lichtbogens bei und beeinflusst die Schweißnaht-Geometrie. Auch ternäre Gasmischungen werden zur Verbesserung der Materialbeständigkeit gegenüber Verschmutzung häufig für bestimmte Materialien verwendet. \autocite[42-43]{Pires_WeldingRobots_2006}

Es soll bemerkt werden, dass diese Schweißparameter nicht vollständig unabhängig voneinander angepasst und justiert werden können. Im Gegensatz dazu stehen einzelne Schweißparameter in konkreten Zusammenhängen mit einander. Die Änderung gewisser Parameter führt zu einer automatischen Anpassung eines oder mehrerer anderer Parameter. \autocite[40]{Pires_WeldingRobots_2006}

\subsection{Laserstrahlschweißen}
\subsubsection{Verfahren}
Bezugnehmend auf Abschnitt ~\ref{ssec:lasersensoren} besteht ein Laser aus Lichtstrahlen, die durch Stimulierung von Atomen entstehen, verstärkt und ausgestrahlt werden. Die Energie aus dem Laserstrahl wird entweder durch Reflexion abgeführt oder durch das Werkstück absorbiert. Durch diese Energie wird das Bauteil erwärmt, welches zum Abschmelzen oder sogar zur Verdampfung des Materials führt. Auf Basis der Energiedichte kann das Laserstrahlschweißen in zwei Verfahren aufgeteilt werden. Laserschweißen mit einer niedrigeren Energiedichte führt zu einem Energieverlust von bis zu neunzig Prozent. Die absorbierte Energie schmelzt die Werkstückoberfläche aber reicht für die Verdampfung nicht aus. Das Schweißbad dieser Variante kann durch sein breites aber flaches Profil charakterisiert werden. Die zweite Prozessvariante betrifft Laserschweißen mit Energiedichten über $10^{10}\ Wm^{-2}$. Aufgrund dieser hohen Energie schmelzt sowie verdampft sich das Oberflächenmaterial teilweise. Dadurch verteilt sich die Hitze tiefer im Material und schafft eine schmale und tiefe Schweißnaht, als das flüssige Material hinter dem Laser abkühlt. Die zweite Prozessvariante wird häufig zum Schweißen dickere Werkstücke mit hohen Schweißgeschwindigkeiten eingesetzt und kann mit oder ohne Schweißzusatz erfolgen. \autocite[45-47]{Pires_WeldingRobots_2006}

Das Laserstrahlschweißen lässt sich sehr gut einsetzen, wo eine sehr hohe Präzision mit geringem Bauteilverzug erforderlich ist. Allerdings ist die Qualität der Ergebnisse dieses Verfahrens von der Materialeigenschaften wie Reflektivität und Lichtabsorption. Daneben ist die Investition für dieses Verfahren ein Vielfaches der Kapitalanlage für die Lichtbogenschweißverfahren. \autocite[47]{Pires_WeldingRobots_2006}

\subsubsection{Geräte}
Grundsätzlich werden die gewöhnlichen Geräte aller Schweißverfahren auch in diesem Verfahren verwendet. Eine Schweißpistole dient dem Zweck des Führens und Richtens des Laserstrahles sowie des Schutzgases. Ein Schutzgas wird zur Vermeidung von Porenbildung in der Schweißnaht und Verschlechterung der Schweiß-Qualität benutzt. Auch der optionale Einsatz eines Schweißzusatzes kann über eine Drahtvorschubeinrichtung erfolgen. Die Stromquelle bei diesem Verfahren dient allerdings nicht zur Generierung eines Lichtbogens, sonder zur Erzeugung des Laserstrahls. In diesem Schweißverfahren werden zwei Arten von Lasern benutzt - Festkörperlaser und Gaslaser \autocite[47]{Pires_WeldingRobots_2006}.

In Festkörperlasern befinden sich spezielle Kristalle, die erregt werden und einen starken Laserstrahl ausstrahlen. Die Erregung erfolgt mittels Lichtblitze aus Laserdioden. Grundsätzlich kann der Laserstrahl bei geringer Leistung im Dauermodus emittiert werden, allerdings werden sie aufgrund beschränkter Kühlleistung mit sehr hohen Leistungen und konstanter Frequenz gepulst. Der Laserstrahl wird mittels einer Glasfaserleitung zu der Schweißpistole geleitet. Durch die Nutzung solch einer Leitung kommt der Lasersystem eine höhere Flexibilität zugute. \autocite[47-48]{Pires_WeldingRobots_2006}

In Gaslasern wird der Laserstrahl durch die Erregung eines zirkulierenden Gases erzeugt. Dieser Strahl wird mittels Spiegeln und teil-reflektierenden Spiegeln fokussiert und gerichtet. Die Leistung eines Lasers dieser Art wird auch durch die Kühlleistung des Systems begrenzt. Gaslaser werden selten kontinuierlich ausgestrahlt, sondern häufig gepulst. Ähnlich wir bei Festkörperlasern wird der Laserstrahl mittels einer Glasfaserleitung übertragen. \autocite[48-49]{Pires_WeldingRobots_2006}

\subsubsection{Parameter}
Die verfahren-übergreifende Geräte wie die Schweißpistole und Drahtvorschubeinrichtungen bieten Anpassungsmöglichkeiten an, die auch verfahren-übergreifend sind. Effekte dieser Parameter wie der Drahtvorschub und die Elektrodenverlängerung auf die Qualität sind gleich wie bei anderen Schweißverfahren. Einige Parameter zeichnen sich allerdings aus, da sie einzigartig zu den verwendeten Lasern sind. 

Die Leistung des Laserstrahls hat bei einem konstanten Durchmesser einen direkten Einfluss auf die Einbrandtiefe. Mit einer Erhöhung der Laserleistung steigt auch die Energiedichte an, sodass mehr Material verschmolzen und verdampft wird. Auf die Energiedichte hat auch die Brennweite der fokussierenden Linse einen Einfluss. Mit einer Erhöhung dieses Werts wird das Durchmesser des Laserstrahls größer, welches zu einer Reduzierung der Energiedichte führt. Durch die Einstellung der Tiefe des Brennpunktes kann die Geometrie und Tiefe der Schweißnaht kontrolliert werden. Die Setzung des Brennpunktes über die Werkstückoberfläche führt zu einer kleineren Einbrandtiefe und dünneren Schweißnaht. Allerdings steigt die Genauigkeitsanforderung der Werkstückplatzierung bei der tieferen Setzung des Brennpunktes in dem Werkstück. \autocite[50]{Pires_WeldingRobots_2006}

Die Effizienz dieses Schweißverfahrens wird auch durch Materialeigenschaften beeinflusst. Eine hohe Absorptionsfähigkeit des Materials resultiert in einer hohen Energieaufnahme durch das Material, die zu einer besseren Schmelzleistung führt. Konträr ist eine hohe Reflektivität zum Schmelzen des Werkstückes ungeeignet. Die geringe Absorption von Metallen erklärt die Erforderlichkeit der hohen Laserleistung. \autocite[52-53]{Pires_WeldingRobots_2006}

Die Schweißgeschwindigkeit spielt analog zu anderen Schweißverfahren bei der Einbrandtiefe eine Rolle. Eine höhere Schweißgeschwindigkeit reduziert die Einbrandtiefe. Merkwürdig ist die Entstehung eines flacheren Einbrands bei sehr geringen Schweißgeschwindigkeiten. Dies passiert aufgrund Plasmawolken, die den Laserstrahl stören und somit die Wärmeleistung verringern. \autocite[51]{Pires_WeldingRobots_2006}

Der Schutzgas dient bei diesem Verfahren nicht nur zum Schutz des Schweißbades vor atmosphärische Verunreinigungen, sondern auch zur Entfernung von Plasmawolken. Dies geschieht, indem ein Inertgas wie Helium lateral über das Schweißbad gepustet wird. Je nach Reaktivität des Werkstückmetalls wird ein passendes Schutzgas gewählt. \autocite[52]{Pires_WeldingRobots_2006}

\subsection{Durchbrüche in der Schweißrobotik}

\textcite{Wilmsmeyer_2018} haben ein Design für einen Schweißroboter vorgeschlagen, der fähig ist, Bauteilen mit großen Toleranzen präzise ohne Nachverarbeitung zu fertigen. Es werden Geometriedaten des Bauteils über CAD-Daten eingelesen. Für die hoch-genaue Bahnplanung wird allerdings die Bauteilgeometrie mittels 3D-Sensoren optisch getastet. Auch zur Erkennung der Schweißnaht werden diverse Sensoren wie Gasdüsensensoren, Lichtbogensensoren und Lasersensoren benutzt. Obwohl die Autoren zugeben, dass das optische Messverfahren zur Erfassung der Bauteilgeometrie nicht die Genauigkeit des taktilen Verfahrens nachahmen kann, ist dieses bei vielen Fällen deutlich schneller. Diese Arbeit schaffte es, sinnvolle Geometriedaten optisch mittels Sensoren zu erfassen.

Eine andere Arbeit durch \textcite{Yang_2019} schlägt eine weitere Methode für die intelligente Naht-Erkennung vor. Es referenziert auch diverse andere Arbeiten, die mittels unterschiedlichen Sensoren wie Ultraschallsensoren, Infrarotsensoren und RGB-D Tiefenkameras diverse Erkennungen während des Schweißprozesses machen könnten. Die Autoren dieser Arbeit entwickelten eine Methode zur Extraktion der Schweißnaht-Geometrie aus einer Punktewolke des Bauteils. Ihre Methode für die Erkennung könnte im Vergleich zu der traditionellen Teach-In-Methode eine deutliche Effizienzverbesserung und Zeitersparnis bringen. 

Eine ähnliche Arbeit wurde durch \textcite{Geng2021} geleistet, wo sie eine Methode zur Bahnplanung des Schweißroboters ohne den Bedarf an Teach-In-Programmieren vorschlagen. Dies geschieht unter der Verwendung eines industriellen 3D-Kameras, welches zum Einscannen der Werkstückoberfläche verwendet wird. Diese Kamera ermittelt Entfernungen auf Basis der Stereo-Vision-Technologie. Aus der 3D-Daten des Werkstückes wird eine Punktwolke erstellt. Nach einer Vorverarbeitung, wird mathematisch die Naht aus zwei deckungsgleiche Ebenen ermittelt. Somit kann eine präzise Bahnplanung der Schweißnaht entlang ohne eine Teach-In-Programmierung erfolgen. Mit einer geringen Zeitanforderung ist diese Methode auch für die Industrie interessant. 

Mittlerweile ist die Sensorik ein integraler Bestandteil der Schweißrobotik geworden, sondern auch anderer Roboteranwendungen. Die Sensorik ermöglicht sowohl die kontinuierliche Überwachung der Position und Stellung der Roboterarme, als auch die Wahrnehmung der Umwelt um den Roboter. Um die Sensordaten aus der Umgebung sinnvoll für die Roboterprozesse zu verwenden wird eine zentrale Steuerung notwendig. Dies erfolgt mittels einer zentralen Steuereinheit, die die Aufgabe der Überwachung aller Bewegungen übernimmt und diese anhand der Sensordaten steuert. Letztendlich ist auch eine Programmiereinrichtung, meistens ein multifunktionales Programmierpanel nötig, das zur Weisungsbefugnis dient. \autocite[114]{maier2022grundlagen}

\section{Sensoren eines Industrieroboters}
Analog zu den Sinnesorgane der Menschen ermöglichen Sensoren den Maschinen oder Geräten es, auf die Umwelt zu reagieren. Konkret bereichern Sensoren Maschinen, Geräte oder Anlagen mit der Fähigkeit, bestimmte Zustände oder Zustandsänderungen wahrzunehmen und auf diese zu reagieren. \autocite[1-2]{HesseStefan2018SFDP}. Robotersteuerungen können mit der aus Sensoren gewonnen Informationen den Ist-Zustand des Roboters mit dem Soll-Zustand vergleichen und darauf basierend Anweisungen generieren. Sensoren für Roboter können auf Basis ihrer erfassten Datenarten in zwei Klassen unterteilt werden - interne und externe Sensoren. \autocite{maier2022grundlagen}

Interne Sensoren fassen Information über den Zustand der intrinsischen physikalischen Eigenschaften des Roboters auf, beispielsweise dessen Innentemperatur, Geschwindigkeit oder Position. Sie liefern Aktualisierungen über die Position und Achsenstellung des Roboters sowie den Status des Endeffektors. Beschleunigungssensoren und Gyroskopen werden häufig zur Erfassung der Beschleunigung, Kräfteverhältnis oder Orientierung eingesetzt. \autocite[220]{maier2022grundlagen}

Externe Sensoren erfassen die qualitativen und quantitativen physikalischen Eigenschaften der Umgebung des Roboters. Somit ermöglichen sie die Wahrnehmung des Arbeitsraumes sowie des zu verarbeitenden Objektes. Die Beschaffenheit von Oberflächen, Ausrichtung von Objekten oder die Entfernung zu einem Gegenstand zählen unter gängigen physikalischen Größen, die durch externen Sensoren wie Kameras, optische Sensoren oder Tastensensoren erfasst werden. Diese Sensoren nehmen die Werten der physikalischen Größen auf und wandeln diese in eindeutig verwertbare, herkömmlicherweise in elektrische Signale um\autocite[221]{maier2022grundlagen}

\subsection{Lasersensoren zur Distanzermittlung} \label{ssec:lasersensoren}
Das Laser - ein Kürzel für \emph{light amplification by stimulated emission of radiation} - ist ein kohärenter und monochromatischer Lichtstrahl, der im Vergleich zu Lichtstrahlen aus anderen Quellen intensiver und feiner ist. Ein Laserstrahl kann über großen Entfernungen mit wenigem Verlust reflektiert werden, welches dem Lasersensor zu einem geeigneten Distanzermittler macht. Mittels einer Laserdiode kann ein Laserstrahl induziert werden. \autocite[126]{HesseStefan2018SFDP}

\subsubsection{Die Laserdiode}
Die Laserdiode setzt sich aus drei integralen Bestandteile zusammen - die p- und n-Elektroden, einen Kristall, und einen Spiegel. Die p-Elektrode wird durch Strom erregt, welches dazu führt, dass erregte Elektronen ein Photon freilassen, um wieder zu stabilisieren. Die freigesetzten Photonen werden mehrmalig in dem Kristall reflektiert, welches dazu führt, dass dieser erregt wird. Der Kristall setzt wiederum Photonen der gleichen Frequenz und Phase als die aus der p-Elektrode freigesetzten Photonen frei. Aus der Resonanz dieser gleichphasigen Photonen entsteht ein intensiver Lichtstrahl. \autocite[4-5]{PeterWEpperlein20013} \autocite[16-21]{Hooshang2004} \autocite[221-232]{DomingoGeorge2007}

Die Distanzermittlung funktioniert grundsätzlich auf Basis der Reflektion. Der Laserstrahl, meistens ein rotes oder unsichtbares infrarotes Licht, wird auf dem Objekt projiziert. Dieser Strahl wird durch das Objekt zurück zu dem Lasersensor reflektiert, welcher aus dieser Information ein elektrisches Signal berechnet. Dieses Signal dient zur Ermittelung der Objektdistanz. Das Messverfahren zu Grunde liegend kann der Lasersensor in drei Unterklassen eingeteilt werden.\autocite[167]{Hering2018}

\subsubsection{Pulszeitverfahren}
Bei dem Pulszeitverfahren wird ein Lichtpuls in Intervallen ausgestrahlt und es wird die Zeit bis zum Rückkehr des Lichts gemessen. Aufgrund zwei bekannte und konstant bleibende Größen lässt sich diese Zeit \emph{$\Delta$t} als einen Maß für den Objektabstand \emph{d} verwenden. Die Lichtgeschwindigkeit \emph{c} ist bekannt und konstant sowie das Brechungsindex \emph{n} des Mediums, in dem der Laserstrahl durchläuft. Somit lässt sich der Objektabstand einfach errechnen (Gl. 2.1). \autocite[171]{Hering2018}

\begin{equation}
	d = \frac{c \cdot \Delta t}{2 \cdot n}
\end{equation}

Um Fehlmessungen zu verringern wird intern in dem Lasersensor mittels einer Stoppuhr und Signale werden Lichtpulse von weiter entfernten Objekten elektronisch ausgeblendet. Es wird bei dem Aussenden des Lichtes einen Startsignal ausgelöst und die Uhr gestartet. Sobald der Empfänger einen Lichtimpuls mit einer Mindestlichtintensität erhält, wird ein Stoppsignal gesendet und die Stoppuhr aufgehalten und ausgelesen. \autocite[171]{Hering2018}.

\subsubsection{Frequenzlaufzeitverfahren}

Ein ähnliches Verfahren zu dem Pulszeitverfahren ist das Phasen- oder Frequenzlaufzeitverfahren. Der Objektabstand \emph{d} wird durch Vermessen der Phasendifferenz \emph{$\Delta$ $\varphi$} zwischen dem ausgesendeten und eingegangenen Licht errechnet (Gl. 2.3). Integral für die Berechnung ist die Modulationswellenlänge \emph{$\lambda_m$} des Lichtpulses, welches mit der Periodendauer \emph{$T_m$} moduliert wird (Gl 2.2). \autocite[171]{Hering2018}

\begin{equation}
	\lambda_m = \frac{c \cdot T_m}{n}
\end{equation}

\begin{equation}
			d = \frac{\Delta \varphi}{2 \cdot \pi} \cdot \frac{\lambda_m}{2} + i \cdot \frac{\lambda_m}{2}, i=0,1,2,...
\end{equation}

Um sowohl möglichst großen Abstände eindeutig zu messen und als auch eine hohe Messgenauigkeit zu erzielen müssen mehrfache Modulationswellenlängen verwendet werden. Dies wird möglich, indem der Puls eine konstante und zyklische Frequenzänderung zwischen einem Minimum- und Maximumwert durchlebt. \autocite[172-173]{Hering2018}

\subsubsection{Interferometrisches Verfahren}

Bei dem interferometrischen Messverfahren wird das Laserlicht mittels eines teildurchlässigen Spiegels in zwei Strahlen aufgeteilt. Diese Strahlen kehren voneinander ab und legen unterschiedliche Wege zurück. Ein Teilbündel wird innerhalb des Gerätes mittels eines anderen Spiegels zurückreflektiert, während das andere Teilbündel mit der Objektoberfläche begegnet. Nach dem Rückkehr beider Teilbündel werden sie zusammengeführt und die Lichtintensität nach der Überlagerung ermessen. Bei einer Änderung des Objektabstandes ändert sich die Lichtintensität proportional. 

\subsubsection{Triangulationsverfahren}

Das Triangulationsverfahren zur Abstandsermittlung ist ein geometrisches Verhalten. Hierbei spielen zeitabhängige Größen wie die Lichtgeschwindigkeit keine Rolle. Ein dichtes Lichtbündel wird von dem Lasersensor gesendet und trifft die Oberfläche des zu detektierenden Objektes. Nach der Reflektion, die von der Oberflächenkontur und Eigenschaften abhängt, wird der Strahl durch den Sensor empfangen und in die Detektionsebene abgebildet. Auf diese Ebene entsteht eine Lichtverteilung mit einem Schwerpunkt \emph{x}, dessen Position mit einer Änderung der Objektdistanz \emph{d} auch verschiebt. Da der Basisabstand \emph{B} zwischen dem Sender und Empfänger des Lasersensors sowie der Abstand \emph{F} zwischen der Empfänger-Optik und Detektionsebene bekannt ist, kann mit dem Vermessen des Schwerpunktes \emph{x} auch der Objektabstand \emph{d} errechnet werden (Gl. 2.4). \autocite[170]{Hering2018}

\begin{equation}
	x = \frac{B \cdot F}{d}
\end{equation}

Optoelektronischen Abstandssensoren auf Basis des Triangulations- sowie Laufzeitverfahrens finden aufgrund ihrer einfacherer Technik und gute Reichweite Anwendung in der Abstandmessung, Konturbestimmung, Dickenbestimmung sowie die Füllstandskontrolle. Interferometrische Abstandssensoren können aufgrund ihrer intrinsischen Eigenschaften sehr hohe Auflösungen realisieren, welches sie zur Abstandsmessung in dem Mikrometerbereich geeignet macht. Diese Art der Lasersensoren finden bei der Feineinstellung von Maschinen und Anlagen sowie in der Oberflächenqualitätskontrolle eine Anwendung. \autocite[174-177]{Hering2018}

Mit dem Einsatz Lasersensoren in Industrierobotern können diese Geräte die Fähigkeit verleiht werden, mit einer hohen Genauigkeit Werkstücke zu erkennen und innerhalb ihren eigenen Koordinatensysteme zu lokalisieren und abzubilden. Allerdings ist für das Zusammenspiel zwischen dem Roboter und dem Lasersensor die richtige Auslegung von Kommunikationswegen zwischen den jeweiligen einzelnen Komponenten sehr wichtig.

\section{Das Robot Operating System (ROS)} \label{sec:ROS}
Das Robot Operating System (ROS) ist ein Sammelwerk von Werkzeugen, Software-Bibliotheken und Normen, die zusammen einen Rahmenwerk für die Erstellung von Robotersoftware bilden. Der Hintergrund hinter ROS ist es, komplexe aber robuste Lösungen für Roboteraufgaben plattformübergreifend zu ermöglichen. Auch simpel erscheinende Probleme umfassen die Lösung sehr komplexer Teilaufgaben wie Planungs-, Koordinierungs- und Erkennungsaufgaben, die auf einander Einflüsse haben, mit einander zusammenarbeiten sollen und jeweils komplexe Teilprobleme mit sich bringen. Zwecks der Simplifizierung und Förderung von Kollaboration ermöglicht ROS die separate Entwicklung von Lösungen der Teilaufgaben und übernimmt die Aufgabe der Kollaboration der jeweiligen Lösungen. \autocite[3]{QuigleyROS2015}

\subsection{Philosophie hinter der Entwicklung}

ROS wurde auf Basis einiger Grundprinzipien geschaffen, die zur Effizienz, Anpassbarkeit, Flexibilität und Wiederverwendbarkeit eines Systems beitragen. 

\subsubsection{Peer-to-Peer}
	ROS-Systeme bilden sich aus verschiedenen kleinen Teilprogramme zusammen, die auf die Lösung einzige Probleme spezialisiert sind. Diese Teilprogramme kommunizieren mittels eines direkten Nachrichtenaustausches miteinander, ohne einer zentralen Routing. Dank dieses \emph{peer-to-peer} Prinzips können ROS-Systeme auch mit enormen Datenmengen umgehen. \autocite[3]{QuigleyROS2015} 
\subsubsection{Tools-basiert}
	ROS besitzt auch keine kanonische, übergeordnete Entwicklungs- und Laufzeitumgebung. Inspiriert durch das \emph{Tools-basierte} Prinzip, werden die jeweiligen Aufgaben einer solchen Umgebung durch kleinen, eigenständigen Programme oder Werkzeuge erledigt. Somit ist nicht nur eine Weiterentwicklung oder Verbesserung des Programms einfach, sondern auch das Ersetzen oder die Überarbeitung. \autocite[3]{QuigleyROS2015} 
\subsubsection{Multilingual}
	Die Entwickler von ROS haben die Stärken und Schwächen unterschiedlicher Programmiersprachen erkannt, wodurch ihren situationsabhängigen Einsatz gewisse Vorteile bringen können. Deswegen wurde ein \emph{multilingualer} Ansatz übernommen, sodass ROS-Module in verschiedenen unterstützten Hochsprachen geschrieben werden können. Auch neue Programmiersprachen können jederzeit hinzugefügt werden, solange Software-Pakete für die Interpretation und Implementierung der ROS-Kommunikationsnormen auch bereitgestellt werden. \autocite[3]{QuigleyROS2015} 
\subsubsection{Schlank}
	Zur Realisierung einer besseren Wiederverwendbarkeit der ROS-Module werden Entwickler empfohlen einen \emph{schlanken} Ansatz zu verwenden, indem eigenständige Module ohne den Bedarf einer ROS-Umgebung entwickelt werden. Zur Integration mit ROS sollen die bestehenden Module mit Methoden zur Kommunikation mit der ROS-Umgebung und anderen ROS-Modulen erweitert werden. \autocite[3]{QuigleyROS2015} 
\subsubsection{Quelloffen}
	Letztlich wird ROS unter einem \emph{quelloffenen} BSD Lizenz veröffentlicht, sodass kommerzielle Systeme mit proprietären Modulen sowie quelloffene akademische und Hobbyprojekte ohne Hindernisse entwickelt werden können. \autocite[4]{QuigleyROS2015}
	
\subsection{ROS-Bausteine}
Zur besseren Verständnis seiner Funktionsweise lohnt sich ein tieferer Blick in den Aufbau von ROS.

\subsubsection{ROS-Nodes}
Ein ROS-System setzt sich aus verschiedenen unabhängigen Programmen oder Modulen zusammen. Diese Module erfüllen roboter-spezifische Aufgaben der Navigation, Computervision, Greifen usw. Graphisch können diese Module als Knoten dargestellt werden. Deswegen werden sie auch als \emph{ROS-Nodes} genannt. ROS-Nodes können mit einander kommunizieren und Information in Form von Nachrichten austauschen. \autocite[9-10]{QuigleyROS2015} 

\subsubsection{ROS-Master}
Das ROS-Master, auch \emph{roscore} genannt, ist ein ROS-Node, deren Aufgabe die eines DNS-Servers ähnelt. Der ROS-Master koordiniert die Kommunikation zwischen einzelnen Nodes, indem es einen Sammelwerk von Information über aktiven ROS-Nodes erstellt. Dies enthält unter anderem den Namen des aktiven ROS-Nodes sowie dessen Adresse. Die Adresse des ROS-Masters wird durch den Benutzer vorgegeben und somit allen Nodes verfügbar. Sobald ein ROS-Node gestartet wird, nutzt sie diese Adresse, um sich bei dem ROS-Master zu registrieren. Dieser Node gibt dem ROS-Master bekannt, mit welchen anderen ROS-Nodes sie kommunizieren will und die Art der Nachrichten, die sie sendet. Der ROS-Master meldet sich mit der Adresse des relevanten ROS-Nodes zurück. Somit können ROS-Nodes mit anderen Nodes Kontakt aufnehmen. Sobald der Kontakt zwischen den Nodes entsteht, tretet der ROS-Master zurück. Änderungen in aktiven Nodes werden auch dem ROS-Master bekannt gegeben. In einem ROS-System darf es nur eine Instanz des ROS-Masters laufen.  \autocite[11-12]{QuigleyROS2015} \autocite[6-7]{NewmanWyattS2018ASAt} \autocite[14]{LentinMasteringROS2021}

\subsubsection{ROS-Messages}
ROS-Messages sind Nachrichten, die während der Kommunikation zwischen Nodes ausgetauscht werden. Ein ROS-Message wird durch den Sender in Form einer serialisierten Datenpaket gesendet, welches mit einem entsprechenden Schlüssel interpretiert und rekonstruiert werden kann. ROS-Messages können standardmäßig Information als Zahlen in Form von (vorzeichenlosen) Integern und Gleitkommazahlen, als Text in Form von Strings und als boolesche Werte enthalten. Allerdings ist es auch möglich, benutzerdefinierte ROS-Messages zu erstellen. Das ermöglicht beispielsweise die Übermittlung von translatorischen und rotatorischen Geschwindigkeiten eines Roboters in Form drei dimensionale Matrizen oder Arrays. \autocite[6]{NewmanWyattS2018ASAt}

\subsubsection{ROS-Topics}
ROS-Topics sind speziell ausgelegte Wege, die analog zu Datenbussen funktionieren. Innerhalb eines ROS-Topics befinden sich ROS-Messages zu einem spezifischen Thema, beispielsweise die Temperaturwerte aus einem Temperatursensor. ROS-Topics unterscheiden sich von Datenbussen, indem sie eine entkoppelte Kommunikation auf Basis des TCP/IP zwischen ROS-Nodes ermöglichen. ROS-Messages werden zu einem Topic durch einen ROS-Node veröffentlicht und können durch anderen ROS-Nodes empfangen werden, indem sie dem ROS-Topic abonnieren. Somit bleiben diese ROS-Nodes anonym. Aufgrund der entkoppelten Kommunikation können mehrere ROS-Nodes gleichzeitig zu dem ROS-Topic veröffentlichen sowie dem abonnieren. \autocite[14]{LentinMasteringROS2021} \autocite[23]{LentinMasteringROS2018} \autocite[6]{NewmanWyattS2018ASAt}

\subsubsection{Publisher und Subscriber}
Publishers und Subscribers sind ROS-Nodes, die eine unidirektionale Kommunikation über ROS-Topics ausführen. Publishers sind meistens ROS-Nodes, die unter allem Daten erfassen und sammeln. Subscribers sind ROS-Nodes, die diese Daten für die Ausführung verschiedener Aufgaben benötigen. Die Daten des Publishers werden als ROS-Messages strukturiert und in einem kontinuierlichen Datenstrom zu einem ROS-Topic veröffentlicht und stehen dem Subscriber zur Verfügung. Um diese Daten zu erhalten, muss der Subscriber dem Topic abonnieren.

\subsubsection{ROS-Services}
Die Kommunikation über ROS-Topics erfolgt unidirektional, die für den Einsatz bestimmter Zwecke ungeeignet ist. Manche Roboteraufgaben erfordern eine Kommunikation basiert auf dem Anfrage/Antwort Modell. Dies wird durch den Einsatz von ROS-Services ermöglicht. ROS-Services setzen sich aus zwei ROS-Nodes zusammen - einen Server und einen Client. Der Client kann dem Server die Ausführung eines Dienstes, zwar eine Aufgabe, anfordern. Nachdem der Server diesen Dienst ausgeführt hat, werden dessen Ergebnisse als eine Antwort an dem Client zurückgesendet. Die Anfrage und Antwort werden beide als ROS-Messages gesendet, die Information eines bestimmten Datentypes oder Formates enthalten. \autocite[24]{LentinMasteringROS2018}. Ein Beispiel für die Anwendung eines ROS-Services in der Robotik sind Bestückungsaufgaben. Der Greifer und ein optischer Sensor eines Roboters werden in einem ROS-System als ROS-Nodes abgebildet. Der Sensor-Node, in Rolle eines Clients, erkennt ein Werkstück. Der Client fordert das Aufheben des Werkstückes an, indem er eine ROS-Message mit den Werkstückkoordinaten dem Greifer-Node sendet, der die Rolle des Servers spielt. Nachdem der Greifer das Werkstück aufgehoben hat, schickt er dem Client eine Antwort in Form einer ROS-Message zurück. Da ROS-Services eine synchrone Kommunikationsform anwenden, führt der Client bis zur Erhaltung einer Antwort keine weitere Schritte aus. Eine asynchrone Form der Kommunikation auf Basis der ROS-Services kann mittels ROS-Actions realisiert werden.

Dank der quelloffenen Philosophie und die Möglichkeit der Kollaboration in ROS sind viele quelloffene Module - sogenannte ROS-Packages - entstanden, die sehr geschickte und umfangreiche Lösungen zu verschiedenen Roboter spezifische Probleme wie die Simulation, Koordinatentransformation und Visualisierung präsentieren. Einige dieser ROS-Packages sind in vielen Entwicklungen Roboterlösungen ein integraler Bestandteil geworden.

\subsection{Simulation mit Gazebo}
Die Simulation in der Robotik dient hauptsächlich zwei Zwecke - zur Verständnis des Systemverhaltens und zur Analyse der Systemleistung \autocite[110]{DatteriSchiaffonati2019}. Beispielsweise kann zur Verständnis der Roboterbewegungsmöglichkeiten, das Robotermodell in einer Simulationsumgebung modelliert werden. Wichtig ist die Festlegung der Knoten oder Gelenke sowie der Festkörper des Modells. In der Simulation können die unterschiedlichen Bewegungsmöglichkeiten geprüft sowie Kollisionsgefahren untersucht werden. \autocite[11]{LentinMasteringROS2018}. Zur Überprüfung der Sicherheitskriterien von kollaborativen Industrierobotern, sogenannten Cobots, wird Simulation als eine kostengünstigere Alternative zu physischen Testverläufe benutzt. Da Cobots Aufgaben in Zusammenarbeit mit Menschen erledigen und ihrer Arbeitsraum nicht abgesichert ist, müssen sie einige Sicherheitskriterien erfüllen, damit sie nicht den Menschen verletzen. Eine Simulation ermöglicht die Überprüfung der Sicherheitsgewährleistung, ohne dass teure Cobots physikalisch beschädigt werden oder die Menschen in Gefahr kommen. Ein geometrisches Modell des Cobots kann in der Simulationsumgebung geladen werden und sein Verhalten bei dem Auftreten von simulierten Störungen und risikobehaftete Situationen beobachtet und bewertet werden. \autocite{ore_vemula_hanson_wiktorsson_fagerström_2019}

Simulatoren können auf Basis der Anzahl an abgebildeten Dimensionen unterteilt werden - 2D- und 3D-Simulatoren. 2D-Simulatoren sind sehr rechen-effizient und eignen sich insbesondere für die Simulation planarer Navigation. Hierbei wird die Fläche eines Raumes zweidimensional abgebildet mit Verzicht auf die Höhendimension. Allerdings ist für vielen Aufgaben, beispielsweise die Manipulationsaufgaben eines Industrieroboters, eine dreidimensionale  Simulation unabdingbar. \emph{Gazebo} ist eine quelloffene Software für Physiksimulation, die häufig mit ROS benutzt wird. Innerhalb der Simulationsumgebung werden physischen Zusammenhänge der Festkörperphysik imitiert. Eine besondere Eigenschaft dieser Software ist, dass sie eine Vielzahl an verschiedenen Sensoren realistisch simulieren kann. Wichtig für eine Simulation mit Gazebo ist auch eine Modellierung der Umgebung des Roboters - das sogenannte Weltmodell. Dieses Modell enthält statische Gegenstände wie Gebäude, Gelände und Hindernisse sowie dynamische Objekte wie greifbare und bewegbare Werkstücke, weitere Roboter und andere bewegliche Gegenstände. Das \emph{gazebo \textunderscore ros} Modul ermöglicht eine bidirektionale Kommunikation zwischen der Gazebo-Software und ROS. Somit können die künstlichen Sensordaten sowie Physikdaten aus der Simulationsumgebung in ROS einfließen sowie Aktuator-Befehle von ROS zu Gazebo zurückgeleitet werden. Im Falle der korrekten Konfiguration könnte ein ROS-System in der Simulationsumgebung identisch zu dem realen Roboter laufen, und das ohne großen Anpassungen. \autocite[95-96]{QuigleyROS2015} \autocite[114]{NewmanWyattS2018ASAt}

\subsection{Koordinatentransformation mit Transform-Library} \label{ssec:tf2}
Wie in Kapitel 2.1 Festgestellt, besteht ein Roboter aus einer Aneinanderreihung von Armteile, die miteinander mit Gelenken verbunden sind. Aufgrund der Freiheitsgrade der einzelnen Gelenke vervielfachen sich Bewegungsmöglichkeiten des Roboters. Um die Gesamtbewegung des Roboters für die Robotersteuerung auszurechnen, wird jedes Arm sein eigenes Koordinatensystem verleiht. Mit Hilfe mehrerer Koordinatentransformationen wird aus der Stellung der einzelnen Arme, beginnend mit dem ersten, die Stellung des Endeffektors berechnet. Dieses nennt sich die Vorwärtstransformation. Bei der Rückwertskinematik handelt es sich um die Berechnung der Gelenkstellungen der jeweiligen Armteile aus der Stellung des Endeffektors. Diese Berechnung ist komplexer, schwieriger und rechenintensiver als die Vorwärtstransformation. Entscheidend für die Trajektorienplanung und Bewegung in der Robotik ist die Berechnung der Rückwertskinematik in Echtzeit. \autocite[65-66]{maier2022grundlagen}

Da Koordinatentransformationen in der Robotik sehr gebräuchlich sind, bietet ROS ein mächtiges Modul namens Transform-Library (tf2). Dieses Modul ist die zweite Generation und erweitert die erste Version dieses Moduls, \emph{tf}. \autocite{foote_tf2_wiki_2019}. Die grundsätzliche Funktion bleibt allerdings unverändert. Das tf2 Modul ermöglicht es, auf die Transformation zwischen zwei Koordinatensysteme des Roboters jederzeit zuzugreifen. Diese berechneten Werte werden im Form einer ROS-Message zu einem ROS-Topic regelmäßig veröffentlicht. Diese Message wird ein \emph{TFMessage} genannt. Die Transformation wird in zwei Komponenten aufgeteilt - die Translationswerte als einen dreidimensionalen Vektor und die Orientierungswerte als eine Quaternion. Außerdem enthält das TFMessage einen Zeitstempel sowie Angaben zu dem Referenz- und transformierten Koordinatensystem. Der Zeitstempel ist für statische sowie dynamische Transformationen wichtig. Eine statische Transformation zwischen zwei Koordinatensysteme bleibt immer gleich, bis deren Ausrichtung zu einander explizit geändert wird. Der Zeitstempel einer statischen Transformation gibt an, ab wann eine Transformation als gültig gilt. Sie bleibt valid, bis sie aktualisiert wird. Eine dynamische Transformation bezeichnet eine ständig wechselnde Ausrichtung zweier Koordinatensysteme zu einander. Der Zeitstempel dieser Art der Transformation gibt an, an welchem Zeitpunkt sie gültig ist. Die Simulationssoftware Gazebo bietet hierzu ein paar hilfreiche Funktionalitäten. Mit der richtigen Aufstellung eines ROS-Systems ahmt Gazebo die Bewegung des Roboters zeitgleich nach. Diese Gelegenheit wahrnehmend werden die aktuellen Orientierungen und Winkeln der Roboter-Gelenke und deren Koordinatensysteme zu einem ROS-Topic \emph{joint\textunderscore states} veröffentlicht. Mittels der Information aus diesem ROS-Topic lassen sich diverse Koordinatentransformationen berechnen. Da diese Berechnungen sehr häufig gebraucht werden, wird durch ROS ein Modul namens \emph{robot \textunderscore state\textunderscore publisher} angeboten, welches die Winkeln der Gelenke annimmt und aus diesen die drei-dimensionale Pose der Roboterarme berechnet. Daraus werden dann die Koordinaten"-transformationen automatisch errechnet und publiziert. \autocite[156-160]{NewmanWyattS2018ASAt}

Das tf2 Modul baut auf die bestehende Publisher/Subscriber Kommunikation in ROS auf. Diese Kommunikationsarchitektur wurde für die Design- und Funktionsanforderungen des tf2 Moduls erweitert. Es besteht aus \emph{Broadcasters} und \emph{Listeners}. Ein Broadcaster dient dem Zweck des Übertragens. Er übertragt mit einer regelmäßigen Frequenz Aktualisierungen über die Transformationen. Eine Änderung der Transformation sei dabei nicht vorausgesetzt, sondern gilt das Vergehen der Zeit auch als eine Aktualisierung. Ein Listener sammelt die, durch den Broadcaster publizierten Werte in einer, nach der Zeit sortierten Liste und speichert sie. Mittels einer Abfrage an dem Listener können Auskünfte über spezifischen Transformationen zu einem Zeitpunkt gewonnen werden. Da Transformationen diskret publiziert werden, besteht die Möglichkeit, dass der Listener keine Transformationsdaten zu einem Zeitpunkt besitzt. In dem Fall wird mittels einer sphärischen linearen Interpolation (SLERP) die Transformation interpoliert. Dies trägt zusätzlich zu der Robustheit dieses Moduls bei. \autocite{Foote_tfPaper_2013} \autocite[161-167]{NewmanWyattS2018ASAt}

Das tf2 Modul vereinfacht das Berechnen, Speichern und Organisieren mehrere Koordinatentransformationen, allerdings ist es für einige Aufgaben der Robotik ungeeignet. Tf2 lässt sich für die Roboterbahnplanung schlecht einsetzen, da es zukünftige hypothetische kinematische Transformationen nicht vorausberechnen kann. Ein weiteres Nachteil dieses Moduls ist es, dass die Transformationen zeitverzögert gespeichert und bereitgestellt werden. Dies macht es für die Anwendung in zeitkritischen Reglungen wie Kraftreglungen ungeeignet. In diesen Fällen wird es empfohlen, die kinematischen Transformationen programm-intern auszurechnen. \autocite[174-175]{NewmanWyattS2018ASAt}

\subsection{Visualisierungen mit Rviz}
\emph{Rviz}, eine Abkürzung von \emph{ROS Visualization}, ist eine Umgebung für die dreidimensionale Visualisierung von reellen oder virtuellen Sensoren, Robotern und ihren Bewegungen sowie Aktionen. Dieses Modul ermöglicht das Überwachen aller Prozesse aus der Sichtweise des Roboters, indem Dateneingänge aus Sensoren visualisiert werden. Beispielsweise werden Bilddaten aus einer Kamera als Bilder oder Videos dargestellt sowie Tiefenwerte aus einem Lasersensor als eine dreidimensionale Punktwolke angezeigt werden. Auch zurückgelegte Bahnen und Bewegungen des Roboters können mittels visueller Anhaltspunkte verfolgt werden. Die Rviz Erweiterung der später vorgestellten MoveIt-Bibliothek bietet auch einige Funktionalitäten für die Fernsteuerung des Roboters an. Das Modul lässt sich auch sehr gut in einem ROS-System und mit anderen ROS-Modulen integrieren. Sensordaten sowie die Pose des Roboters erhält Rviz, indem es den entsprechenden ROS-Topics abonniert. \autocite[177-180]{NewmanWyattS2018ASAt} \autocite[126-127]{QuigleyROS2015} \autocite[243]{LentinMasteringROS2018} \autocite[36-40]{FairchildROS2017}
\begin{figure}[h]
	\includegraphics[width = \textwidth]{Abbildungen/Rviz_vis.png}
	\centering
	\caption{Visualisierung eines Roboters in Rviz}
	\label{fig:Rviz_vis}
\end{figure}

In Abbildung~\ref{fig:Rviz_vis} zu sehen ist ein in Rviz visualisierter Schweißroboter. In dieser Umgebung wird die aktuelle Pose des Roboters während einer Schweißaufgabe illustriert. Hierbei werden nicht nur die Transformationen und Gelenkwinkeln des Roboters eingelesen, sondern auch die Sensordaten aus einem Triangulation-Lasersensor. Diese Daten können sowohl aus dem physischen Roboter und Sensor entstehen, als auch in Gazebo emuliert werden. In Abbildung~\ref{fig:Gazebo_vis} wird der Roboter zusammen mit dem Werkstück (in Orange dargestellt) modelliert. Der Laser aus dem Lasersensor ist ein Linienlaser und wird als die blaue Projizierung abgebildet. In Rviz werden die Tiefenwerte aus dem Sensor ausgewertet und farblich kodiert. Die aktuelle Projizierung des Lasers auf dem Werkstück wird als einen roten Streifen nachgebildet. Durch die Bewegung des Roboters in der dritten Dimension des Lasersensors werden dreidimensionale Punktewolken erstellt. Die Kontur des Werkstückes wird aus dieser Punktewolken interpretiert und als das gelbe Profil nachgezeichnet. Als letztes ist die, durch den Endeffektor beziehungsweise die Schweißpistole zurückgelegte Bahn als die grüne Linie reproduziert worden. Mit Hilfe dieser Animationen kann die Roboterbewegung aus einer sicheren Entfernung überwacht werden sowie die Sensordaten auf Fehlern und Störungen überprüft werden.

\begin{figure}[t]
		\includegraphics[width = \textwidth]{Abbildungen/Gazebo_vis.png}
		\centering
		\caption{Visualisierung des gleichen Roboters in Gazebo}
		\label{fig:Gazebo_vis}
\end{figure}

\subsection{Bahnplanung mit MoveIt}
In Abschnitt~\ref{ssec:tf2} wurde die Signifikanz der Vorwärts- und Rückwertstransformation angedeutet. Diese sind für die Berechnung der direkten und inversen Kinematik des Roboters sehr wichtig \autocite[41-118]{MareczekRobKin2020}. Das tf2 Modul wurde für die Berechnung Koordinatentransformationen spezialisiert, allerdings eignet sich das Modul nicht für die Planung und Berechnung der Kinematik. Für die Erfüllung dieser Aufgaben wurde \emph{MoveIt} konzipiert. Das ist ein ROS-Modul für die Berechnung inverser Kinematik, Bewegungs- und Bahnplanung, Kollisionsprüfung und weiteres mittels unterschiedlicher Ansätze wie stochastische Algorithmen, deterministische Bahnplaner und echtzeitfähige, lokale Planungsmodule. \autocite[160]{GandhinathanROSProjects2019}

Das wichtigste Element des MoveIt-Moduls ist der \emph{move \textunderscore group} ROS-Node. Dieser Node erfüllt die Aufgabe des Integrators, indem es alle anderen Komponenten und Funktionalitäten des MoveIt Moduls zusammen bündelt. Dem Benutzer wird die Interaktion mit diesem Modul mittels ROS-Services und ROS-Actions ermöglicht. ROS-Actions sind eine Erweiterung der ROS-Services und ermöglichen den asynchronen Aufruf von ROS-Services sowie, unter anderem, den Abbruch eines Aufrufs \autocite[61]{QuigleyROS2015}. Den Aufruf dieser Services und Actions könnte mittels eines Python-Skriptes, C++ Programms oder der graphischen Benutzeroberfläche von Rviz gesteuert werden. Details über die Gestaltung und Form des Roboter erhält der \textit{move \textunderscore group} Node über ein 3D-Modell des Roboters. \autocite[161]{GandhinathanROSProjects2019}

Allerdings reicht nur ein Modell des Roboters für eine online Bahnplanung nicht aus. Andere Details besorgt sich MoveIt von dem Roboter mittles der Publisher/Subscriber Kommunikation und ROS-Topics sowie ROS-Actions. Information über die Gelenkstellungen wird per Abonnement zu dem \textit{/joint\textunderscore states} ROS-Topic erhalten. MoveIt überwacht die einzelnen Transformationen des Roboters mit Abfragen zu dem tf2 Modul um aktuelle Information über die Roboterpose zu erhalten. Diese Werte werden intern für die Berechnung der Kinematik verwendet. Das MoveIt Modul kann auch mit der Steuereinheit des Roboters über ROS-Actions interagieren. Der Steuerung des Roboters wird über diese Schnittstelle Befehle für die Bewegung der Gelenke gegeben. Um einen Überblick über die Stellung des Roboters und die Welt zu behalten erstellt MoveIt eine virtuelle Szene, überwacht diese und benutzt sie für die Planung. \autocite{moveit_concepts_2021}

\subsection{ROS für Industrieroboter} \label{ssec:ros_industrial}
ROS-Module erweitern und bereichern die Fähigkeiten von ROS und realisieren eine deutliche Erleichterung des Aufstellens eines ROS-basierten Robotersystems. Wie es aus dem Roboter-System der Abbildung~\ref{fig:Rviz_vis} und Abbildung~\ref{fig:Gazebo_vis} zu entnehmen ist, könnten ROS-basierte Roboter unter Anwendung dieser Module auch industrielle Fertigungsaufgaben erledigen. Um diesen Konzept zu realisieren und ROS für Industrieroboter anwendbar zu machen, wurde \emph{ROS-Industrial} vorgestellt. Dieses Konzept bringt mit sich einige Ziele. Es soll die Entstehung einer Gemeinschaft bestehend aus Akademiker, Forscher und Experten für Industrieroboter unterstützen. Die fortgeschrittenen Fähigkeiten von ROS sollen möglichst gut mit existierenden Industrietechnologien harmonisiert werden und somit ein robustes und zuverlässiges Software-Paket für Industrieforschung sowie Industrieanwendungen anbieten. Grundsätzlich soll es ROS dazu unterstützen, einen Industriestandard für die Robotik zu werden. \autocite[476]{LentinMasteringROS2018}. Die meisten gängigen Roboterherstellern bieten mittlerweile ROS-Schnittstellen an. Diese befinden sich allerdings in unterschiedlich weiten Entwicklungs- und Wartungszuständen.

Die Theorie hinter der Funktionsweise eines Roboters, der automatisierten Schweißverfahren, optischen Sensoren und ROS dient dem Aufbau von Grundlagen, die zur Verständnis der Funktionsweise des kollaborierenden Schweißroboters des Fraunhofer-IPA notwendig sind. 
