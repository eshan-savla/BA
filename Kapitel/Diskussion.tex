%Struktur:
% Erkenntnisse aus der Reproduktion
%PCL-Version und UBuntu
%Bounding Box mit Lasersensor
%Diskussion Test 1
%Diskussion Test 2
%Diskussion Test 3

\chapter{Diskussion}
%Auf einem Ryzen 5 3600 Prozessor \autocite{noauthor_amd_2022} mit sechs Kernen und 12 Threads und einer Basistaktrate von 3,6 GHz kann eine siebenfache Leistungsverbesserung beobachtet werden. Eine Punktwolke mit ca. 450.000 Punkte könnte innerhalb 162 Sekunden verarbeitet werden. Nach der Parallelisierung erfolgte die Kantenerkennung innerhalb 22,5 Sekunden. Es ließ sich postulieren, dass Kanten durch die Verwendung eines Prozessors mit mehr Kernen noch schneller erkannt werden könnten. Die Verwendung eines Grafikprozessors, die deutlich mehr Kernen besitzen, hätte die Leistung der Kantenerkennung für sehr großen Punktwolken enorm steigern können. Allerdings, durfte diese aufgrund der Software-Voraussetzungen in Abschnitt \ref{soft_voraus} nicht an einem Grafikprozessor delegiert werden.  SOLL IN DISKUSSION KOMMEN