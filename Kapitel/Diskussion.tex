%Struktur:
% Erkenntnisse aus der Reproduktion
%PCL-Version und UBuntu
%Bounding Box mit Lasersensor
%Diskussion Test 1
%Diskussion Test 2
%Diskussion Test 3

\chapter{Diskussion}
\section{Zusammenfassung der Methodik und Ergebnisse}
Das in dieser Arbeit vorgelegte Verfahren wurde mit dem Hintergrund entwickelt, eine Methode zur Erkennung geometrischer Merkmale in wachsenden Punktwolken zu präsentieren. Dieses Verfahren sollte es ermöglichen, während der Erzeugung der Punktwolke relevante geometrische Informationen nebenbei bereitzustellen. Grundsätzlich sollte die sequenzielle Identifizierung der Geometrien eines Objektes während es Abtastung ermöglicht werden. Hierfür wurde zuerst ein bereits etabliertes Verfahren nach \autocite{ni_edge_2016} aus der Literatur gewählt und implementiert. Das Verfahren bestand aus zwei Teile - die Kantenerkennung sowie die Kantensegmentierung. Bei der Kantenerkennung wurden zuerst die \textit{K\textsubscript{1}} nächsten Nachbarpunkte eines Punktes \textit{o} gesucht um eine Nachbarschaft \textit{N\textsubscript{o}} zu erstellen. Aus diese Nachbarschaft wurden mittels eines RANSAC-Verfahrens die Punkte ausgesucht, die auf der gleichen Ebene \textit{E\textsubscript{N\textsubscript{o}}} lagen. Falls \textit{o} zu dieser Ebene gehörte, wurden alle Inliers der Ebene \textit{E\textsubscript{N\textsubscript{o}}} weiterhin auf ihren Winkelabstand überprüft. Mittels des Verfahrens aus Abschnitt \ref{edge_detection_reprod} wurden die Winkelabstände \textit{G\textsubscript{$\theta$}} von zwei konsekutiven Nachbarpunkten berechnet. Danach wurde es überprüft, ob der maximale Wert von \textit{G\textsubscript{$\theta$}} einen Schwellwert \textit{$\alpha$} überstieg.

%Auf einem Ryzen 5 3600 Prozessor \autocite{noauthor_amd_2022} mit sechs Kernen und 12 Threads und einer Basistaktrate von 3,6 GHz kann eine siebenfache Leistungsverbesserung beobachtet werden. Eine Punktwolke mit ca. 450.000 Punkte könnte innerhalb 162 Sekunden verarbeitet werden. Nach der Parallelisierung erfolgte die Kantenerkennung innerhalb 22,5 Sekunden. Es ließ sich postulieren, dass Kanten durch die Verwendung eines Prozessors mit mehr Kernen noch schneller erkannt werden könnten. Die Verwendung eines Grafikprozessors, die deutlich mehr Kernen besitzen, hätte die Leistung der Kantenerkennung für sehr großen Punktwolken enorm steigern können. Allerdings, durfte diese aufgrund der Software-Voraussetzungen in Abschnitt \ref{soft_voraus} nicht an einem Grafikprozessor delegiert werden.  SOLL IN DISKUSSION KOMMEN