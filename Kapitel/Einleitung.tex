\chapter{Einleitung}
Moderne Technologien wie das maschinelle Sehen finden in der heutigen Ära weit und breit in diversen industriellen und technischen Sektoren eine Anwendung. Optische Sensoren werden in diversen Prozessen eingesetzt, um Informationen aufzunehmen und zu verarbeiten, sodass darauf basierend (autonom) Entscheidungen über die Prozessreglung getroffen werden können. Einer der häufigsten Anwendungsfälle für optischen Sensoren und das maschinelle Sehen ist die Qualitätskontrolle. Hierzu werden Bilder aufgenommen und durch spezielle Verfahren verarbeitet, um beispielsweise Defekte zu erkennen. \autocite[3-11]{beyerer_machine_2015}

Maschinelles Sehen wird nicht nur auf die Auswertung und Verarbeitung zweidimensionaler Bilder beschränkt, sondern lässt es sich auch auf dreidimensionale Abbildungen anwenden \autocite{biegelbauer_model-based_2010}. Um solche Abbildungen zu erstellen werden Objekte häufig mit bestimmten Sensoren wie Lasersensoren abgetastet, um ihre Oberflächen in einem dreidimensionalen virtuellen Raum abzubilden \autocite[20-22]{savla_intelligente_2022}. Diese Aufnahmen werden in Punktwolken gespeichert, die alle Oberflächen des Objektes mit einer Vielzahl an dreidimensionalen Punkten modellieren. \Textcite{lougheed_3-d_1988} stellen Abbildungssysteme und Verarbeitungsverfahren für dreidimensionalen Abbildungen vor, die in der Robotik angewendet werden können. Es handelt sich dabei um die Entwicklung eines Verfahrens zur Erkennung von Flächen, wo Gegenstände durch den Roboter gegriffen werden sollten. 

Neben Flächen setzen sich Objekte aus mehr dreidimensionalen Merkmalen zusammen und die Erkennung solcher Merkmale eines Objektes hat viele relevante Anwendungen in der Industrie. Eine Teildisziplin des dreidimensionalen maschinellen Sehens beschäftigt sich mit der Erkennung von Kanten. Zum heutigen Stand sind in der Literatur diverse Beiträge zu finden, die unterschiedlichen Verfahren zur Erkennung dieser Kanten in Punktwolken vorstellen.

\section{Stand der Technik} \label{Stand_der_Technik}
Grundsätzlich bestehen zwei Gattungen von Verfahren zur Kantenerkennung, die unterschiedliche Stärken und Schwächen aufweisen. Die meisten Vorschläge für die Kantenerkennung machen sich entweder numerischen Methoden oder neuronalen Netzen zunutze.

\Textcite{hu_jsenet_2020} schlagen ein neuartiges Verfahren zur Kantenerkennung sowie Oberflächensegmentierung vor. Es wird als Eingabe eine Punktwolke vorausgesetzt und gleichzeitig durch zwei unterschiedlichen Operationen verarbeitet. Bei der ersten Operation werden mittels eines faltenden neuronalen Netzwerk (englisch: Convolutional Neural Network oder CNN) die Oberflächen der Punktwolke segmentiert. Bei der Segmentierung handelt es sich um die Gruppierung aller gleichartigen Punkte. Im zweiten Datenstrom werden die Kanten erkannt, die eine Punktwolke repräsentieren können. Schließlich werden beide Datenströme zusammengeführt, um die Informationen beider Ströme zu kombinieren. \Textcite{bazazian_edc-net_2021} schlagen auch ein Verfahren vor, das mittels eines neuronalen Netzes alle Randpunkte der Punktwolke klassifiziert. \Textcite{himeur_pcednet_2021} schlagen ähnlich ein Verfahren vor, welches auch ein CNN anwendet, um Kanten zu erkennen. 

\Textcite{choi_rgb-d_2013} schlagen in ihrem Werk ein neuartiges Verfahren zur Kantenerkennung in dreidimensionalen Farbbilder oder RGB-D Bilder vor. Diese Bilder haben neben den drei gewöhnlichen Farbkanäle auch einen vierten Kanal, welcher die Entfernung jedes Pixels von dem Sensorursprung angibt. Zur Erkennung der Kanten werden große Sprünge oder Unterbrechungen in der Tiefe des Bildes gesucht und die entsprechenden Pixel als Kanten erkannt. Darüber hinaus werden zweidimensionale Kantenerkennungsverfahren auf die Bildkomponente angewendet, um Kanten zwischen Flächen mit einer starken Krümmung zu erkennen. 

\Textcite{mineo_novel_2019} schlagen ein neuartiges Verfahren zur Kantenerkennung in unorganisierten Punktwolken, deren Punkte im Vergleich zu organisierten Punktwolken wie RGB-D Bilder nicht in einer vordefinierten Anordnung zu einander gespeichert werden. Das vorgeschlagene Verfahren bestimmt Randpunkte mittels der Analyse einer Nachbarschaft von Punkten innerhalb einer Sphäre. Das Verfahren ist in der Lage, konvexe und konkave Kanten zu erkennen, sowie den Umfang des Objekts nahtlos nachzuzeichnen.

\Textcite{lu_fast_2019} präsentieren ein Verfahren zur Kantenerkennung, welches etablierte Methoden wie Segmentierung, kleinste Quadrate (englisch: Least Mean Squares) und Projektion verwendet. In diesem Verfahren werden zuerst die Oberflächen der Punktwolke segmentiert. Danach werden alle Punkte der jeweiligen Oberflächen auf eine Ebene projiziert, um mittels der kleinsten Quadrate Methode die Kanten zu erkennen. Diese Kanten werden wieder zurück in die dritte Dimension projiziert.

\Textcite{ahmed_edge_2018} verwenden ein statistisches Verfahren zur Ermittlung eines mittleren Punktes oder eines Centroids einer Nachbarschaft von Punkten. Liegt der Abstand eines Punktes der Nachbarschaft über einen statistisch bestimmten Schwellwert zu dem Centroid, wird der Punkt als einen Randpunkt klassifiziert. So ermittelt das Verfahren nach den Autoren statistisch die Kanten. Es wird ein zusätzliches Verfahren zur Erkennung von Eckpunkten aus Randpunkten auf Basis der Krümmung der Randpunkte vorgeschlagen.

\Textcite{ni_edge_2016} legen ein Verfahren zur Erkennung und Segmentierung von Kanten dar. Die Kantenerkennung erfolgt durch die Bestimmung lokaler Nachbarschaften von Punkten und die Analyse der Winkelabstände zwischen den zugehörigen Punkte. Die Kantensegmentierung erweitert das Verfahren durch das sinnvolle Clustern von Randpunkte, die eine hohe Kollinearität aufweisen und lückenlos an einander anschließen. 

\section{Motivation und Zielsetzung} \label{Motivation}
Bemerkenswert ist die Tatsache, dass sehr viele Beiträge in der Literatur verschiedene neuartige Methoden zur Kantenerkennung und Segmentierung für offline Fällen präsentieren, während der Bereich der Kantenerkennung in online Fällen nach aktuellem Kenntnisstand weitgehend unerforscht blieb. Die Gewinnung von Informationen über Kanten in Punktwolken, während sie nebenbei wachsen, hat für viele Anwendungsgebiete eine große Bedeutung. Die Arbeit von \textcite{savla_intelligente_2022} untersucht eine Implementierung eines \textit{kognitiven} Schweißroboters, der die Schweißnaht des Bauteils während des Schweißvorgangs mittels eines Lasersensors bestimmt und verfolgt. Eine aktuelle Limitation dieses Roboters ist die fehlende Wahrnehmung der Bauteilgeometrie während des Schweißvorgangs. Mittels einer online Kantenerkennung wäre der Schweißroboter nicht nur in der Lage, Erkenntnisse über die Bauteilgeometrie zu sammeln, sondern auch die Laufbahnplanung und Prozessparameter dementsprechend anzupassen.

Für den Einsatz in einem automatischen Schweißvorgang des obigen Falls muss die Kantenerkennung einige Voraussetzungen erfüllen. Das wichtigste Kriterium für die Kantenerkennung ist die hohe Genauigkeit und Robustheit des Verfahrens. Das Verfahren soll unter reellen Bedingungen zuverlässige Ergebnisse liefern, um möglichst genau die Schweißnaht zu bestimmen sowie Hindernisse zu erkennen. Zur korrekten Erkennung von Kehlnähte soll das Verfahren auch Innen- sowie Außenkanten erkennen können. Die Anwendung des Verfahrens darf nicht nur auf organisierten Punktwolken beschränkt sein, sondern auch unorganisierten Punktwolken. Die online Funktionalitäten des Roboters setzen voraus, dass das Verfahren performant und in einer kurzen Zeitspanne die Kanten erkennen soll. Letztlich wird die Hardwarebeschleunigung mit einem Grafikprozessor ausgeschlossen, um Konflikte mit dem Echtzeitkernel der Robotersoftware zu verhindern. 

Im Rahmen dieser Arbeit wird angestrebt, ein Verfahren zur Erkennung der Kanten geometrischer Merkmale in wachsenden Punktwolken unter Einhaltung der obigen Anforderungen vorzulegen. Aus der Vielzahl der Verfahren aus Abschnitt~\ref{Stand_der_Technik} wird das AGPN-Verfahren (englisch: Analysis of Geometric Properties of Neighborhoods) nach \textcite{ni_edge_2016} als Grundlage für die online Kantenerkennung gewählt und mit weiteren Funktionalitäten für die Online-Erkennung erweitert. Für ihre Wahl über anderen Verfahren der Literatur hat die Methode ihre hohen Genauigkeit zu verdanken, insbesondere bei der Trennung zwei naheliegender Kanten. Darüber hinaus bietet das Verfahren der Autoren zur Kantensegmentierung eine elegante Möglichkeit an, zwischen den Kanten eines Objektes zu unterscheiden. Da das Verfahren auf numerischen Methoden zurückgreift und keine neuronale Netze verwendet, muss kein rechenaufwändiges Training mit einem großen Datensatz auf einem Grafikprozessor durchgeführt werden.  

Es wird im Laufe dieser Arbeit der Frage nachgegangen, wie ein nummerisches Verfahren aus der Literatur für die Kantenerkennung und Segmentierung für eine Anwendung auf wachsenden Punktwolken in der Online-Erkennung erweitert werden kann. Dabei ist das Ziel dieser Arbeit die Überprüfung der Effektivität eines solchen Verfahrens. Um die vagen Rahmen dieser Forschungsfrage zu konkretisieren wird sie in drei weiteren Forschungsfragen unterteilt. Im Umfang dieser Frage wird hinterfragt, zu welcher Genauigkeit das Verfahren Kanten erkennen kann, welcher Einfluss die Punktdichte auf die Genauigkeit hat und wie Robust das Verfahren gegen Unregelmäßigkeiten ist. Kurzgefasst sind dies die Forschungsfragen:

\begin{itemize}
	\item Wie Effektiv ist ein numerisches Verfahren bei der Kantenerkennung und Segmentierung von wachsenden Punktwolken?
	\begin{itemize}
		\item Zu welcher Genauigkeit kann ein solches Verfahren Kanten erkennen und segmentieren?
		\item Welcher Einfluss hat die Punktdichte auf die Genauigkeit?
		\item Wie Robust ist das Verfahren gegen Unregelmäßigkeiten?
	\end{itemize}
\end{itemize}

Anhand quantitativen Metriken wird das Verfahren nach den Richtlinien der drei Forschungsfragen ausgewertet. Dabei werden sowohl synthetische Ground-Truth Dateien verwendet, als auch reelle Aufnahmen von echten Bauteilen. Durch die Verwendung eines Ground-Truths können die Ergebnisse der jeweiligen Untersuchungen mit höchster Zuversicht ausgewertet werden, während die reellen Aufnahmen eine Beurteilung der \textit{tatsächlichen} Effektivität bei der Anwendung unter reellen Bedingungen ermöglichen werden. Die quantitativen Ergebnisse der Untersuchungen ermöglichen auch eine vernünftige Gegenüberstellung des Verfahrens, um seine Genauigkeit mit anderen Verfahren aus der Literatur zu vergleichen. 

\section{Aufbau und Struktur dieser Arbeit}
Im folgenden Kapitel werden die nötigen theoretischen Grundlagen zur Methodik besprochen. Hierbei werden wichtige Kenntnisse über allgemeine Datenstrukturen und Algorithmen vermittelt. Darüber hinaus werden besondere Datenstrukturen und Algorithmen detaillierter behandelt, die in der Methodik eine Anwendung finden. 

In der Methodik wird darauf eingegangen, wie auf dem neuartigen AGPN-Verfahren aus der Literatur aufgebaut wird, um es mit neuen Funktionalitäten zur Kantenerkennung in wachsenden Punktwolken zu bereichern. Dabei wird zuerst behandelt, wie das Verfahren nachgestellt wird. Danach werden die konkreten Schritte zur Erweiterung des Verfahrens genannt. Dabei werden konkrete Maßnahmen vorgestellt, wodurch die Leistung oder Genauigkeit des Verfahrens verbessert wird. 

Im nächsten Kapitel wird die Genauigkeit und Einsatzfähigkeit des Verfahrens überprüft. Hierzu werden anhand der Forschungsfrage sowie der drei Teilforschungsfragen Tests konzipiert, die gezielt die Genauigkeit und Robustheit des Verfahrens ausmessen und darlegen. Neben der Beschreibung der drei Tests werden auch Testdateien und Auswertungsmetriken vorgestellt, die für die Tests verwendet werden. Unter den Testdaten wird eine synthetische Ground-Truth Datei sowie vier Aufnahmen reeller Bauteile vorgestellt. 

Als Vorletztes werden die Methodik und Ergebnisse dieser Arbeit diskutiert. Zuerst werden diese zusammengefasst und die wichtigsten Erkenntnisse aus den jeweiligen Kapiteln nochmals genannt. Darüber hinaus werden besondere Erkenntnisse der Methodik genannt, welcher zu einer Leistungsverbesserung geführt haben. Es werden die Ergebnisse der drei Tests interpretiert. Letztlich werden Limitationen dieser Arbeit diskutiert, die die Entwicklung des Verfahrens sowie seine Auswertung beeinflusst haben. 

Zuletzt wird der Beitrag dieser Arbeit zur Forschung zusammengefasst und über die Zukunftspotentiale hinsichtlich weiterer Untersuchungen sowie Erweiterungsmöglichkeiten diskutiert.