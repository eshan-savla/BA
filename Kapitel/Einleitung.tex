\chapter{Einführung in der intelligenten Robotik}
\section{Robotik in der Industrie} \label{sec:Industrieroboter}
Industrieroboter werden primär in der Industrie für die Aufgabe der Handhabung, Montage oder Verarbeitung von Werkstücken und Teile eingesetzt. Häufig finden Industrieroboter als Schweißroboter, Lackierroboter oder Montageroboter Anwendung. Diese Art der Roboter ermöglichen in diesem Umfeld eine günstigere, genauere und zuverlässigere Verarbeitung von Produkten und Werkstücken. \autocite{maier2016grundlagen}

Industrierobotern fanden ihre erste Anwendung in der 1960er Jahren in der Automobilindustrie in den USA. Durch die Einführung von japanischen Industrierobotern in den 1980er Jahren erlebten japanischen Unternehmen zur Folge von Lean-Product¸ion und Kaizen-Konzepten eine erhebliche Steigerung ihrer Produktivität und Qualität. Unter Konkurrenzdruck  forderten europäischen und nordamerikanischen Automobilehesteller leistungsstärkerer und bessere Industrieroboter an und setzten diese ein. Seitdem entsteht die Nachfrage für Industrieroboter hauptsächlich von großen Unternehmen. Im Jahr 2008 entstand 70\% dieser Nachfrage aus der Automobilindustrie.\autocite{Bierfreund+2008+135+140}

Von den Vorteilen der Automatisierung und Robotik in einem industriellen Umfeld können nicht nur großen Unternehmen profitieren, sondern auch kleinen und mittelständischen Unternehmen. Die Robotik bringt insbesondere den Firmen einen Vorteil, die in einem Arbeitsmarkt agieren, wo qualifizierte und ausgebildete Arbeitnehmer eine Mangelware sind und Arbeitskosten allgemein hoch oder sehr hoch sind. Durch den Einsatz von Industrierobotern in der Produktion könnte dieser Mangel an ausgebildeten Arbeitskräften ausgeglichen werden. 80\% der, durch Personalkosten entstehenden Prozesskosten des Schweißens in der Produktion, können mittels Industrieroboter abgeschafft werden. \autocite{glaser2008industrial}

Für den Einsatz in kleinen und mittelständischen Unternehmen müssen Industrieroboter gewisse Anforderungen dieser Unternehmen erfüllen. Um die Ansprüche auf kleineren Losgrößen und der Einzelfertigung kleiner und mittelständiger Unternehmen erfüllen zu können, müssen Robotersysteme nicht nur mit unterschiedlichen Werkstückgeometrien zurechtkommen, sondern diverse Aufgaben ausführen können. Ein häufiges Programm- und Werkzeugwechsel des Roboters ohne die Voraussetzung von Expertenwissen soll bei dieser neuen Generation von Robotersysteme ein Standard werden. \autocite{Bierfreund+2008+135+140}

Obwohl die Technik der Industrieroboter seit den letzten Jahren rasant weiterentwickelt wurde, sind diese Art der Roboter  meistens nicht autonom. Sie müssen bei Bedarf für ihre Aufgabenausführung an neuen Produkte oder Werkstücke angepasst und umprogrammiert werden. \autocite{maier2016grundlagen}. Eine schnelle Anpassung oder Umprogrammierung des Roboters ist ein wesentliches Kriterium für die Erreichung eines hohen Flexibilitätsgrades. Diese Erhöhung der Flexibilität ist durch die Errichtung von \emph{intelligenten} Prozessen möglich, die mit der Verwendung von Sensoren und Reglungen den herkömmlichen Prozessen gegenüber eine Verbesserung darstellen. Roboter wandeln somit von normalen Handhabungsgeräten zu cyberphysische Systeme, die ihre Umgebung wahrnehmen können und sich somit auf Störungen oder Änderungen anpassen können. \autocite{MeyerChristian2007RaSd}
\autocite{HägeleMartinFraunhoferIPA2016SfdP}

Auch für die sichere Mensch-Roboter-Kollaboration sind sensor-ausgerüstete Roboter nicht nur vorteilhaft, sondern auch nötig, insbesondere für kleine und mittelständische Unternehmen. Mit kollaborierenden Robotern können auf teure Sicherheitseinrichtungen verzichtet werden, welches für diese Unternehmen eine große Ersparnis anbietet. Eine Entwicklung der Fraunhofer Institut für Produktions- und Automatisierungstechnik (IPA) - ein kognitiver Schweißroboter - ist ein Beispiel eines, durch Sensorik und Robotik verbesserten Produktionsprozesses. Die Lokalisierung des Produktes mit einer Genauigkeit von unter einem Fünftel Millimeter erfolgt durch die Verwendung von 3D-Sensordaten. Vorteile davon sind eine verkürzte Einrichtungszeit und das wirtschaftliche Schweißen in kleinen Losgrößen, da es keine komplizierten Fixierungen benötigt werden. \autocite{HägeleMartinFraunhoferIPA2016SfdP}

\section{Motivation} \label{sec:motivation}
Es gibt in der Industrierobotik einige Durchbrüche, die zu der Weiterentwickelung von intelligenten Robotern geführt haben. Industrieroboter wurden hinsichtlich ihrer Flexibilität, Genauigkeit, Einsatzmöglichkeit und Kollaboration mit Menschen verbessert.

Die Arbeit von \textcite{Liu2022} umfasst die Mitbetrachtung der Umwelt und dynamischer externer Störungen bei der Bahnplanung von Robotern. Es wird über die Überlagerung der statischen oder offline Planung mit einer dynamischen oder online Planung diskutiert. Die offline Planung umfasst die Navigation eines Roboters unter Betrachtung aller globalen Elemente. Demgegenüber handelt es sich bei der online Planung um die Anpassung der vorgeplanten Bahn des Roboters auf Basis der Umgebungsänderungen in Echtzeit. Durch die Verwendung maschinellen Sehens haben die Autoren Methoden zur Erkennung dynamischer Änderungen der Umwelt entwickelt, sodass Roboter beispielsweise spontan erschienene Hindernisse auf der Bahn erkennen und vermeiden können. 

\textcite{Luo2016} haben Roboter in cyber-physischen Systemen durch die Implementierung von IoT Methoden umgewandelt. Es wurden Aktoren und Sensoren mit einander vernetzt, um einen Informationsaustausch für die Erledigung Produktionsaufgaben zu ermöglichen. Dabei wurden Methoden basiert auf maschinellem Sehen für Aufgaben wie Objekterkennung, Greifen von Objekten, Hindernisvermeidung und Sicherheitsgewährleistung vorgeschlagen und entwickelt. 

\textcite{Wang2012} schlagen einen Plan für multiple kollaborierende Roboter die mit einander vernetzt sind und kommunizieren können. Information aus dem Sensor eines Roboters steht anderen beteiligten Robotern des Netzwerks zur Verfügung. Beispielsweise sind diese vernetzten Roboter über die aktuellen Positionen anderer Roboter bewusst. Mit der Anwendung von Methoden aus dem maschinellem Lernen sind diese Schwarmroboter in der Lage, komplexe logistische Probleme zu lösen.

Über die Zeit sind verschiedene Projekte entstanden, die ein autonomes Handeln für Roboter entwickelt haben, das einen flexibleren Einsatz dieser Roboter in dynamischen Produktionssysteme ermöglichen sowie mehr Menschen-Roboter-Kollaboration fördern. Diese Projekte werden auch überall in der Welt gefördert, da steigender Lohnkosten und ein steigendes Durchschnittsalter in vielen Ländern ein drohendes Problem darstellt. \autocite{Wang2018} \autocite{Marguglio2022}

Die Fraunhofer Institut für Produktions- und Automatisierungstechnik entwickelt einen intelligenten kollaborierenden Roboter, einen Cobot, um das automatisierte Schweißen für kleinere Losgrößen zu realisieren. Es soll die Programmierung eines solchen Roboters für die Industrieanwendung vereinfacht werden, sodass trotz mangelnden Expertenwissens und eines Fachkräftemangels kleinere Betriebe von erhöhten Produktionsfähigkeiten gewinnen können. Dabei sind die Investitionskosten eines Cobots deutlich geringer sowie verfallen die Installationskosten, da keine Sicherheitsvorkehrungen bei Cobots notwendig sind. \autocite{automationspraxis_2021} 

Die Programmierung bestehender Cobots mit dem Teach-In-Verfahren ist zwar einfach, allerdings ist sie auch sehr zeitintensiv. Eine Programmierung auf Basis CAD-Daten eines Bauteils ist nicht immer möglich, da diese Daten manchmal bei kleinen und mittelständischen Unternehmen nicht vorhanden sind. Mittels eines Laserliniensensors wird die Programmierung und Trajektorienplanung des Roboters einfacher und effizienter gemacht. Der Sensor erkennt optisch die Kehlnaht und ihrer Position in Echtzeit und benutzt diese Information zur Steuerung der Roboterbewegung. Der Anwender muss dem Roboter und den Schweißbrenner grob über den Werkstück positionieren, danach den Start- und Endpunkt festlegen. Nach dem Programmstart erkennt der Roboter die Kehlnaht und positioniert den Schweißbrenner genau auf sie hin. Danach wird der Kehlnaht entlang geschweißt sowie Änderungen in der Naht-Geometrie erkennt und die Trajektorie angepasst. \autocite{automationspraxis_2021}

Das Ziel dieser Arbeit ist es, den kollaborierenden Schweißroboter der Fraunhofer IPA und seine Funktionsweise zu untersuchen. Dabei sollen die einzelnen Hardware- und Softwarekomponenten genauer betrachtet werden. Es soll die Frage beantwortet werden: Welche Hardware- und Softwareanforderungen gibt es für die automatische Schweißnaht-Erkennung durch ein intelligentes kollaborierendes Robotersystem.

\section{Struktur dieser Arbeit}
Im folgenden wird die Aufbau dieser Arbeit und die einzelnen Inhaltsteile kurzgefasst. Nach der obigen Zielsetzung dieser Arbeit werden theoretische Grundlagen zur Verständnis der Funktionsweise diskutiert. Danach folgt eine detaillierte Beschreibung der aktuellen Technik und Funktionsweise des Schweißroboters. Schließlich werden die wesentlichen Erkenntnisse dieser Arbeit zusammengefasst und präsentiert.

Es wird in den Grundlagen der Robotermechanik und -Kinematik eingetaucht. Hier wird der Grundaufbau eines Industrieroboters zusammen mit der Funktion seiner Einzelteile vorgestellt. Beleuchtet wird auch wie die Gelenke den Arbeitsraum eines Roboters definieren und die Bewegung des Roboters beeinflussen. 

Aufbauend auf dem Thema der Industrieroboter wird das Thema der Schweißrobotik eingeführt. Neben eine Argumentation für den Einsatz von Schweißroboter in der Industrie wird über unterschiedlichen Schweißverfahren diskutiert, die häufig in der Schweißrobotik eine Anwendung finden. Es werden die Funktionsweise, benötigten Hardwareeinrichtungen und anpassbaren Parameter sowie ihren Einfluss untersucht. 

Die Sensorik findet zunehmend in der Robotik jetzt eine Anwendung. Neben anderen Sensoren zur Erkennung der Temperatur, Kraft, Haptik und Geräusch werden auch optische Sensoren als das technische Auge eines Roboters verwendet. Es werden die Grundlagen der Sensorik für die Anwendung in der Automatisierung besprochen und die Methoden zur Distanz Ermittlung diverser Lasersensoren untersucht. 

Als letztes wird das Verbindungselement der einzelnen Hardwarekomponenten vorgestellt - die Software. Hier das Robot Operating System vorgestellt, welches die Kommunikation zwischen unterschiedlichen Komponenten eines Robotersystems ermöglicht und weitere fortgeschrittene Funktionalitäten wie Simulationen, Visualisierungen, und Trajektorienplanung für Roboter anbietet.

Nach der Behandlung der Theorie wird der aktuelle technischer Stand des Roboters erläutert. Hierbei wird grundsätzlich zwischen der Hardware und Software des Robotersystems unterschieden. Es werden die Daten und Einzelfunktionen der physischen Geräte kurzgefasst. Danach wird die Software, die das autonome Roboterschweißen ermöglicht, detailliert behandelt. Hierbei wird nicht nur die Funktionsweise des Softwaremoduls abstrahiert, sondern werden die einzelnen Teilmodule betrachtet, die auf die Ausführung spezieller Teilaufgaben ausgelegt sind. Danach werden wichtige Nachrichtenprotokolle für den Austausch diverser Informationstypen wie Koordinatentransformationen oder Sensorwerte analysiert. Schließlich werden andere Softwareabhängigkeiten des Softwarepakets angeschaut, die die Softwareentwickelung für das Robotersystem erleichtern.

Schließlich werden die Erkenntnisse dieser Arbeit zusammengefasst sowie die Applikationen, Limitationen und Zukunftspotenzial des Robotersystems behandelt.