\chapter{Einleitung}
Moderne Technologien wie das maschinelle Sehen finden in der heutigen Ära weit und breit in diversen industriellen und technischen Sektoren eine Anwendung. Optische Sensoren werden in diversen Prozessen eingesetzt, um visuelle Informationen aufzunehmen und zu verarbeiten, sodass darauf basierend (autonome) Entscheidungen über die Prozessreglung getroffen werden können. Einer der häufigsten Anwendungsfällen für optischen Sensoren und das maschinelle Sehen ist die Qualitätskontrolle. Hierzu werden Bilder aufgenommen und durch spezielle Verfahren verarbeitet, um beispielsweise Defekte zu erkennen. \autocite[3-11]{beyerer_machine_2015}

Maschinelles Sehen wird nicht nur auf die Auswertung und Verarbeitung zweidimensionaler Bilder beschränkt, sondern lässt sich das Konzept des maschinellen Sehens auch auf dreidimensionale Abbildungen anwenden \autocite{biegelbauer_model-based_2010}. Um solche Abbildungen zu erstellen werden Objekte häufig mit bestimmten Sensoren wie Lasersensoren abgetastet, um ihre Oberflächen in einem dreidimensionalen virtuellen Raum abzubilden \autocite[20-22]{savla_intelligente_2022}. Diese Aufnahmen werden in Punktwolken gespeichert, die alle Oberflächen des Objektes mit einer Vielzahl an dreidimensionalen Punkten modellieren. \Textcite{lougheed_3-d_1988} stellen Abbildungssysteme und Verarbeitungsverfahren für dreidimensionalen Abbildungen vor, die in der Robotik angewendet werden können. Es handelt sich dabei um die Entwicklung eines Verfahrens zur Erkennung von Flächen, wo Teile durch den Roboter gegriffen werden sollten. 

Objekte besitzen mehr als ein dreidimensionales Merkmal und die Erkennung unterschiedlichen dreidimensionalen Merkmale eines Objektes hat viele relevante Anwendungen in der Industrie. Eine Teildisziplin des dreidimensionalen maschinellen Sehens beschäftigt sich mit der Erkennung von Kanten. Zum heutigen Stand sind in der Literatur diverse Beiträge zu finden, die unterschiedlichen Verfahren zur Erkennung dieser Kanten in Punktwolken vorstellen.

\section{Stand der Technik}
Grundsätzlich bestehen zwei Gattungen von Verfahren zur Kantenerkennung, die unterschiedlichen Stärken und Schwächen aufweisen. Die meisten Vorschläge für die Kantenerkennung machen entweder numerischen Methoden oder neuronalen Netzen zunutze.

\Textcite{hu_jsenet_2020} schlagen ein neuartiges Verfahren zur Kantenerkennung sowie Oberflächensegmentierung vor. Es wird als Eingabe eine Punktwolke eingenommen und gleichzeitig durch zwei Datenströme durchgeführt. Bei dem ersten Strom werden mittels eines faltendes neuronales Netzwerk die Oberflächen der Punktwolke segmentiert. Bei der Segmentierung handelt es sich um die Gruppierung aller gleichartigen Punkte. Im zweiten Datenstrom werden alle Kanten der Punktwolke erkannt. Schließlich werden beide Datenströme zusammengeführt um die Informationen beider Ströme zu kombinieren. \Textcite{bazazian_edc-net_2021} schlagen auch ein Verfahren vor, das mittels eines neuronalen Netzes alle Randpunkte der Punktwolke klassifiziert. \Textcite{himeur_pcednet_2021} schlagen ähnlich ein Verfahren vor, welches auch ein faltendes neuronales Netzwerk anwendet, um Kanten zu erkennen. 

