\section*{\centering Zusammenfassung}
Industrieroboter finden bislang eine breite Anwendung in Großunternehmen, insbesondere in der Produktion. Herkömmliche Industrieroboter lohnen sich meistens nicht für kleine und mittelständische Unternehmen, da sie nur in kleinen Losgrößen fertigen. Um die Anforderungen dieser Unternehmen zu erfüllen, müssen Robotersysteme flexibler und intelligenter werden, um Werkstücke schnell und kostengünstig fertigen zu können. Diese Arbeit hat das Ziel, die Anforderungen eines solchen Systems herauszufinden. Hierfür wird ein kollaborierendes, intelligentes Robotersystem für das sensorgestützte Schweißen in kleinen Losgrößen vorgestellt. Das System wird in seiner Hardware- und Softwarekomponenten zerlegt, sodass die Einzelteile genauer untersucht werden können. Neben einem kollaborierenden Schweißroboter werden die anderen Elemente - ein Laserliniensensor, Schweißgeräte und Sicherheitseinrichtungen vorgestellt und ihrer Funktionen klargelegt. Danach wird die Software zur automatischen Erkennung der Schweißnaht analysiert und ihr Funktionsprinzip erklärt. Letztlich wird über die Software- und Hardwarelimitationen, sowie künftige Einsatzmöglichkeiten sowie Verbesserungspotenziale dieses Systems diskutiert.

\section*{\centering Abstract}
Industrial robots are widely used by large corporations in the production sector. Conventional applications for robotics are usually not justifiable for small and mid-sized companies, as they only produce in smaller batch sizes. Robotic applications start becoming applicable for these companies, once they become more flexible at intelligently handling manufacturing tasks. This paper aims to uncover the requirements of such a system to be viable for small and midcap manufacturers. For this, a collaborative robot for sensor based welding will be introduced. The hardware and software components will be isolated and will be closely examined. Besides the robot itself, other elements such as the laser sensor, welding equipment and safety features of the system will be presented, and their functioning explained. Secondly, the software needed to automatically recognise the welding seam will also be illustrated. Lastly the software and hardware limitations, possible applications and scope for improvement will be discussed.
