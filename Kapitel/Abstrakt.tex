\section*{\centering Zusammenfassung}
Die Erkennung von geometrischen Merkmalen eines Objektes, während es durch einen optischen Sensor abgetastet wird, hat für verschiedenen Anwendungen eine hohe Relevanz. In dieser Arbeit wird ein numerisches Verfahren für diesen Zweck vorgestellt. Das Ziel dieser Arbeit ist die Bewertung der Effektivität des Verfahrens. Dazu ergänzend wird diese Forschungsfrage gestellt: Wie Effektiv ist ein numerisches Verfahren bei der Kantenerkennung Segmentierung von wachsenden Punktwolken? Zuerst wurde ein anderes Verfahren aus der aktuellen Forschung reproduziert bevor es für den Einsatzzweck angepasst und erweitert wurde. In dieser Arbeit werden die Schritte zur Erweiterung des Verfahrens detailliert beschrieben sowie auf Methoden zur Entfernung falsch-detektierten Kanten eingegangen. Zur Beantwortung der Forschungsfrage wurden drei Untersuchungen konzipiert. Konkret wurde die Genauigkeit des Verfahrens unter verschiedenen Bedingungen überprüft und auf Basis der korrekt erkannten und segmentierten Kanten bewertet. Hierfür wurden reeller sowie synthetische Objekte verwendet. Diese Arbeit lieferte das Ergebnis, dass das Verfahren zu einer hohen Genauigkeit Kanten von geometrischen Merkmalen erkennen und segmentieren kann. Daneben wurden auch wichtige Erkenntnisse über die Schwachstellen und Grenzen des Verfahrens gewonnen. Schließlich werden aufbauende Forschungsmöglichkeiten im Bereich der Robotik und Automatisierung besprochen.

\section*{\centering Abstract}
The recognition of geometric features of an during the process of being scanned by an optical sensor is of high relevance for various applications. This paper presents a numerical method for this purpose and aims to evaluate its effectiveness. The research question posed is: How effective is a numerical method for edge detection and segmentation in growing point clouds? Initially, another method from current research was reproduced before being adapted and extended for the specific application. The steps taken to extend the method are detailed in this paper, including methods for removing falsely detected edges. Three studies were designed to answer the research question, where the accuracy of the method was tested under different conditions and evaluated based on correctly detected and segmented edges. Real and synthetic objects were used for this purpose. The result showed that the method can detect and segment edges of geometric features with high accuracy, while also highlighting weaknesses and limitations. Finally, future research opportunities on the basis of this method in the field of robotics and automation are discussed.
