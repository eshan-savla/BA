\chapter{Analyse des IEFD}
Das IEFD-Verfahren dieser Arbeit war noch in der Prototypenphase. Wie in Abschnitt \ref{label} erwähnt, sollte dieses Verfahren unter anderem, eine Anwendung bei der Online-Erkennung von Schweißnähten finden. Hierfür muss das Verfahren unter bestimmten Kriterien evaluiert werden. Zur Evaluierung des Verfahrens wurden Anhand der Forschungsfrage sowie den drei Teilforschungsfragen Tests entworfen. Der erste Test sollte die Genauigkeit des IEFD-Verfahrens überprüfen. Mittels des zweiten Tests wurde der Einfluss der Punktdichte auf dem Verfahren überprüft. Der letzte Test überprüfte die Robustheit des IEFD-Verfahrens gegen Objekten mit unterschiedlichen geometrischen Eigenschaften.

\section{Testdaten}
Für alle drei Tests mussten Testdaten erstellt werden. Nach Bedarf der Tests wurden hierzu Punktwolken künstlich entworfen oder mit einem Lasersensor aufgenommen. Bei der Aufnahme wurden unterschiedliche Werkstücke mit eindeutigen Kanten sowie Geometrien und/oder sichtbaren Knicken ausgesucht, um eine möglichst vielfältige Datenbasis anzubieten. Für das Scannen wurde der Laserliniensensor - \textit{scanControl 3000} verwendet, der auf dem Schweißroboter nach \textcite[39]{savla_intelligente_2022} montiert wurde. Obwohl die Entfernung des Lasersensors von der Werkstückoberfläche über die gesamte Abtastung möglichst konstant gehalten wurde, waren die Punktabstände der Punktwolke manchmal leicht unregelmäßig. Laut \textcite[9]{ni_edge_2016} hatte der Punktabstand den größten Einfluss auf die Leistung von AGPN. Es wurde behauptet, dass das \textit{UniformSampling} diese Verzerrungen der Punktabstände ausgleichen würde. 

Die künstliche Punktwolke wurde auf Basis vorgegebener Parameter mittels eines Python-Skriptes erstellt. Der Zweck dieser Punktwolke war es, ein Objekt mit ein paar einfachen geometrischen Merkmalen zu simulieren. Dieses Objekt durfte dann als ein Ground-Truth verwendet werden. Unter Berücksichtigung des Einsatzzwecks wurde eine treppenförmige Oberfläche abgebildet, sodass es eine Innenkante sowie eine Außenkante vorhanden ist. Diese Kanten sollten Schweißnähte nachahmen, entlang der durch den Schweißroboter geschweißt werden sollte. Das Weltkoordinatensystem wurde als Referenz für die Erstellung verwendet. Die \textit{x}, \textit{y} und \textit{z} Dimensionen entsprachen den Werten \textit{1000}, \textit{2000} beziehungsweise \textit{500} Punkte. Als den Start wurde der Punkt $\left(\begin{smallmatrix}
	0,01 & 0,01 & 0,01
\end{smallmatrix}\right)$ gewählt. Der erste Teil der Punktwolke bestand aus eine Oberfläche auf der \textit{xy}-Ebene mit der Dimension $\left(\begin{smallmatrix}
500 & 2000
\end{smallmatrix}\right)$ und mit $1.000.000$ Punkten. Der zweite Teil bestand aus einer Oberfläche auf der \textit{yz}-Ebene mit der Dimension $\left(\begin{smallmatrix}
2000 & 500
\end{smallmatrix}\right)$ mit $1.000.000$ Punkten, die am Ende des ersten Teils begann. Die Schnittlinie der ersten beiden Teile bildete die Innenkante ab. Der letzte Teil bestand aus einer Oberfläche auf der \textit{xy}-Ebene mit der gleichen Dimension und Punktezahl wie der erste Teil. Diese Oberfläche schloss sich am Ende des zweiten Teils an und die daraus entstandene Schnittlinie bildete die Außenkante ab. Somit ergaben sich genau $3.000.000$ Punkte in der Punktwolke. Es wurde auch ein konstanter Punkteabstand zwischen allen benachbarten Punkten gewährleistet. Für den zweiten Test wurde allerdings der Punkteabstand variiert und wird näher im Abschnitt \ref{label} behandelt. Abbildung \ref{fig: ground_truth} zeigt die künstlich erstellte Punktwolke, die weiterhin als \testcloud referiert wird. 

\begin{figure}[t]
	\includegraphics[scale = 0.5]{Abbildungen/ground_truth.png}
	\centering
	\caption[ground truth]{Die Testdatei beziehungsweise das \textit{Ground Truth}}
	\label{fig: ground_truth}
\end{figure}

\subsection{Evaluierungsmetriken} \label{evaluations_metrics}
Zur Evaluierung der Leistung des IEFD-Verfahrens wurden ein paar quantitative Metriken nach \textcite[10]{ni_edge_2016} entwickelt. Diese sollten die Genauigkeit der Kantenerkennung sowie die Genauigkeit der Segmentierung überprüfen. Wie in Abschnitt \ref{edge_detection_reprod} betont, sollte die Differenz der Winkelabstände \textit{G\textsubscript{$\theta$}} Randpunkte zu einem anderen Randpunkt mindestens $\frac{\pi}{2}$ betragen. Die Metrik \textit{p\textsubscript{dc}} gibt die Genauigkeit des IEFD-Verfahrens zur Kantenerkennung als einen Prozentsatz an. Es werden die Anzahl der erkannten Kanten \textit{N\textsubscript{dc}} mit der Anzahl der tatsächlich vorhanden Kanten \textit{N\textsubscript{gc}} verglichen. Gleichzeitig wurde auch die Ungenauigkeit \textit{p\textsubscript{mj}} des Verfahrens geprüft, indem die Anzahl der fälschlicherweise markierten Kanten mit der Anzahl der tatsächlich vorhanden Kanten vergleicht wurden. Wie in Abschnitt \ref{edge_segmentation} ertönt, dürften die Kanten zusammen gruppiert werden, deren Randpunkte in unmittelbarer Nähe zu einander stehen und deren Richtungsvektoren nicht ruckartig voneinander abweichen. Die Metrik \textit{p\textsubscript{dct}} zur Überprüfung der Genauigkeit der Kantensegmentierung entworfen. Es wurden die Anzahl der korrekt segmentierten Kanten \textit{N\textsubscript{tc}} mit der Anzahl der korrekt erkannten Kanten \textit{N\textsubscript{dc}} sowie der Anzahl der fälschlicherweise erkannten Kanten \textit{N\textsubscript{mj}} verglichen. Gleichzeitig wurde auch die Ungenauigkeit \textit{p\textsubscript{mjt}} überprüft, indem die Anzahl der falsch segmentierten Kanten \textit{N\textsubscript{mjt}} mit der Anzahl der korrekt erkannten Kanten \textit{N\textsubscript{dc}} sowie der Anzahl der fälschlicherweise erkannten Kanten \textit{N\textsubscript{mj}} verglichen wurde. Falls \textit{N\textsubscript{mjt}} größer als \textit{N\textsubscript{dc}} und \textit{N\textsubscript{mj}} summiert war, wurde \textit{N\textsubscript{mjt}} mit der gesamten Anzahl der erkannten Segmente $\textit{N\textsubscript{mjt}} + \textit{N\textsubscript{tc}}$ verglichen. Die fälschlicherweise erkannten Kanten wurden bei der Metrik \textit{p\textsubscript{mj}} auch mitbetrachtet, da diese vor der Kantensegmentierung nicht verworfen wurden. Die Gleichungen \ref{pdc} bis \ref{pmjt} zeigen die vier Metriken.

\begin{equation}
	\label{pdc}
	p_{dc} = \frac{N_{dc}}{N_{gc}}
\end{equation}
\begin{equation}
	\label{pmj}
	p_{mj} = \frac{N_{mj}}{N_{gc}}
\end{equation}
\begin{equation}
	\label{pdct}
	p_{dct} = \frac{N_{tc}}{N_{dc} + N_{mj}}
\end{equation}
\begin{equation}
	\label{pmjt}
	p_{mjt} =
	\begin{cases}
		\frac{N_{mjt}}{N_{dc} + N_{mj}} & if\ N_{mjt} \leq N_{dc} + N_{mj}\\
		\frac{N_{mjt}}{N_{tc} + N_{mjt}}
	\end{cases}
\end{equation}

\subsection{Überprüfung der Genauigkeit des IEFDs}
Die erste Forschungsfrage stellt die Genauigkeit des Verfahrens in Frage. Hierbei sollte geprüft werden wie viele Kanten durch das Verfahren richtig sowie falsch erkannt werden, und wie viele davon korrekt oder inkorrekt segmentiert werden. Für diesen Test wurden die beiden Verfahren auf Basis des \testcloud geprüft. Hierzu wurde ein Punkteabstand von genau \textit{0,0001} festgelegt. Somit hatte die Testdatei eine Dimension von $\left(\begin{smallmatrix}
	0,1m & 0,2m & 0,05m
\end{smallmatrix}\right)$. Die Überprüfung der Genauigkeit erfolgte durch zwei Methoden. Es wurden die Anzahl der, durch das Verfahren erkannten Randpunkte und Segmente gezählt und mit der Anzahl der tatsächlich vorhandenen Randpunkte beziehungsweise Segmente verglichen. Für die zweite Methode wurden die Metriken aus Abschnitt \ref{evaluations_metrics} verwendet, um die Verhältnisse der richtig sowie falsch erkannten Kanten sowie Segmente zu bestimmen. Bei der Festlegung der Parameter wurden die Erkenntnisse aus der Literaturquelle betrachtet. Es wurde erkannt, dass die Parameter \textit{K\textsubscript{1}} und \textit{K\textsubscript{2}} nahezu keinen großen Einfluss auf die Genauigkeit des AGPNs hatten. Der Schwellwert oder Glättungsfaktor $\phi$ bestimmte die maximale Abweichung zwischen zwei fugenlosen Segmente. Die Grenzwerte \textit{d\textsubscript{t1}} und \textit{d\textsubscript{t2}} für den Punkteabstand der RANSAC-Verfahren aus der Kantenerkennung und Kantensegmentierung, die die Bestimmung von Inliers der Verfahren regeln, hatten die größten Einflüsse auf die Genauigkeit. Bei dem Auswahl eines Wertes für \textit{d\textsubscript{t1}} und \textit{d\textsubscript{t2}}, der dem durchschnittlichen Punkteabstand der Punktwolke entspricht, wurden die besten Ergebnisse erzielt. \autocite[10-11]{ni_edge_2016}

Unterberücksichtigung der Erkenntnisse aus dem Referenzwerk wurden die Parametert für \testcloud festgelegt. Die Größe einer Iteration wurde auf 0,01m festgelegt. Die letzten 0,001m einer Iteration wurden wiederholt sowie einen Bereich von 0,0002m zur Entfernung falscher Kanten festgelegt. Die Tabelle \ref{table: parameters_test1} listet die Parameter für das IEFD auf. Aufgrund der relativ hohen Punktdichte wurde festgelegt, dass Kanten und Randpunkte, die weniger als ein halber Millimeter von einander entfernt sind, zusammen segmentiert werden durften. Deswegen wurde ein \textit{d\textsubscript{t2}} von 0,0005m gewählt. Im Gegenzug wurde zwecks einer hohen Genauigkeitsanforderung ein \textit{d\textsubscript{t1}} von 0,0001m für die Kantenerkennung gewählt.

\begin{table}
	\centering
	\begin{tabular}[width=\textwidth]{l *{7}{c}}
		\hline
		\multirow{2}{*}{\textbf{Datei}}&\multirow{2}{*}{\textbf{Anzahl der Punkte}}&\multirow{2}{*}{\textbf{Punkteabstand}}&\multicolumn{5}{c}{\textbf{Parameter}}\\
		& & & \textbf{K\textsubscript{1}} & \textbf{d\textsubscript{t1}} & \textbf{K\textsubscript{2}} & \textbf{d\textsubscript{t2}} & \textbf{$\phi$} \\
		\hline
		\testcloud & 3000000 & 0,0001 & 200 & 0,0001 & 30 & 0,001 & 0,2 \\
		\hline
	\end{tabular}
	\caption{Parameter für den Test auf die Testdatei}
	\label{table: parameters_test1}
\end{table}

Die Überprüfung der Genauigkeit der Kantenerkennung und Segmentierung nach beiden obigen Methoden wurde auch für das AGPN ausgeführt, um Vergleichswerte zu erzeugen. Dank der Einheitlichkeit der Testdatei war es schon fundiert, dass es insgesamt 7.000 Punkte auf den äußeren Rändern sowie 4.000 auf die Innen- und Außenkanten gaben. Daneben war es auch bekannt, dass es insgesamt acht Außenränder sowie zwei Innen- beziehungsweise Außenkanten gaben. Somit ergab sich der Wert von \textit{N\textsubscript{gc}} als 10. Auf Basis dieser Größen wurde die Genauigkeit der IEFD- und AGPN-Verfahren überprüft.

Um die Einflüsse von Ausreißer aus den Ergebnissen zu reduzieren wurden die AGPN und IEFD Verfahren jeweils sechs Mal auf das \testcloud ausgeführt. Die Ergebnisse daraus wurden gemittelt und präsentiert. Das IEFD-Verfahren konnte durchschnittlich 6550 Randpunkte erkennen. Dies entsprach eine Erfolgsrate von $59,54\%$. Die erkannten Randpunkte befanden sich über alle Außenränder sowie Innen- und Außenkanten verteilt. Jede Kante der Testdatei wurde durch das Verfahren erkannt und eindeutig bestimmt. Die Kantensegmentierung des IEFD-Verfahrens lieferte sehr gute Ergebnisse. Es wurden durchschnittlich $97,29\%$ der erkannten Randpunkte richtig segmentiert. Bis auf einen Fall wurden in den restlichen fünf Ausführungen des Verfahrens alle 10 Kanten der Testdatei erkannt. Im Falle des Ausreißers wurde die Innenkante nicht vollständig segmentiert, sondern wurden irrigerweise zwei zusätzliche Segmente auf der Kante erzeugt. Das AGPN-Verfahren konnte im Gegensatz durchschnittlich 3450 Randpunkte erkennen und entsprach eine Erfolgsrate von $31,36\%$. Die erkannten Randpunkte befanden sich lediglich auf die Außenränder verteilt. Die Innen- sowie Außenkante wurden überhaupt nicht durch das Verfahren erkannt. Die Kantensegmentierung des AGPN-Verfahrens lieferte im Gegensatz sehr gute Ergebnisse. Hierbei ist zu betrachten, dass nur die Kantensegmentierung nur auf Basis der erkannten Kanten bewertet wurde, und die fehlende Innen- sowie Außenkante nicht in Betracht gezogen wurden. Es wurden durchschnittlich $99,27\%$ der Randpunkte korrekt segmentiert. Anders als im Falle des IEFD-Verfahrens wurden in keiner der Fälle unvollständige Segmente erzeugt. Obwohl die Genauigkeitsraten der Kantenerkennung beider Verfahren nicht vielversprechend sind, bildeten die erkannten Randpunkte der Verfahren alle Kanten der Testdatei zu einer hohen positionellen Genauigkeit ab. Die fehlenden Randpunkte verursachten keine Lücken in den Kanten und beeinträchtigen die Glätte der erkannten Kanten nicht. Nur für einen Anwendungsfall mit sehr hohen Genauigkeitsanforderung, wo jeder einzelner Randpunkt richtig erkannt werden soll, würden die AGPN- und IEFD-Verfahren unzureichend sein. Aus diesem Grund wurde entschlossen, dass die Metriken nach Abschnitt \ref{evaluations_metrics} zur Bewertung der Verfahren besser geeignet waren. 

Die Ergebnisse aus der sechs Ausführungen wurden für die Auswertung nach Abschnitt \ref{evaluations_metrics} wiederverwendet. Das IEFD-Verfahren hatte einen durchschnittlichen \textit{p\textsubscript{dc}} Wert von 1,00, da in allen sechs Ausführungen alle Kanten der Testdatei erkannt wurden. Spiegelbildlich zu dieser Metrik hatte die Metrik \textit{p\textsubscript{mj}} einen Wert von 0, da das Verfahren keine Kanten fälschlicherweise erkannt hat. In fünf der Ausführungen wurden alle erkannten Kanten richtig und vollständig segmentiert. Im Falle des einen Ausreißers wurde die Innenkante teilweise richtig segmentiert, sowie zwei weitere Segmente auf der Kante erzeugt. Somit ergibt sich einen durchschnittlichen Wert von 1,00 für die Metrik \textit{p\textsubscript{dct}} allerdings auch einen Wert von 0,03 für die Metrik \textit{p\textsubscript{mjt}}. Insgesamt funktionierte das IEFD-Verfahren mit einer Genauigkeit von circa 98,61\%. Das AGPN-Verfahren hatte einen durchschnittlichen \textit{p\textsubscript{dc}} Wert von 0,83, da die Innen- und Außenkante in allen sechs Ausführungen nicht erkannt wurden. Fälschlicherweise hat das AGPN-Verfahren auch keine Punkte als Kanten markiert und erzielte somit eine Bewertung von 0 für die Metrik \textit{p\textsubscript{mj}}. Die Kantensegmentierung erfolgte in allen sechs Ausführungen fehlerfrei, abgesehen davon, dass die Innen- und Außenkanten nicht mitbetrachtet wurden. Das AGPN-Verfahren erzielte eine Bewertung von 1,00 für die Metrik \textit{p\textsubscript{dct}} und 0 für die Metrik \textit{p\textsubscript{mjt}}. Insgesamt konnte das AGPN-Verfahren Kanten mit einer Genauigkeit von 83,33\% erkennen und segmentieren. Die Tabelle \ref{table: metric_values} fasst diese Ergebnisse zusammen.

\begin{table}[h]
	\centering
	\begin{tabular}{l *{9}{c}}
		\hline
		\textbf{Verfahren} & \textbf{N\textsubscript{dc}} & \textbf{N\textsubscript{mj}} & \textbf{N\textsubscript{gc}} & \textbf{p\textsubscript{dc}} & \textbf{p\textsubscript{mj}} & \textbf{N\textsubscript{tc}} & \textbf{N\textsubscript{mjt}} & \textbf{p\textsubscript{dct}} & \textbf{p\textsubscript{mjt}} \\
		\hline
		AGPN & 10 & 0 & 12 & 0,83 & 0 & 10 & 0 & 1,00 & 0 \\
		IEFD & 12 & 0 & 12 & 1,00 & 0 & 12 & 0,33 & 1,00 & 0,03 \\
		\hline
	\end{tabular}
	\caption{Diese Tabelle stellt die durchschnittlichen Werte der jeweiligen Metriken und zusammengehörigen Werte dar}
	\label{table: metric_values}
\end{table}

Die Abbildung \ref{fig: segments_comparision_grnd_trth} visualisiert die Qualität der erkannten und segmentierten Kanten. Wie es in dieser Abbildung zu sehen ist, verlaufen die erkannten Kanten nahezu lückenlos. Trotz der fehlenden Randpunkte sind alle Kanten und somit die Struktur und das Rahmen der Testdatei eindeutig zu erkennen. Die Ecken dieser Kanten stellen Bereiche dar, wo ein paar Randpunkte zu keiner der angrenzenden Kanten zugewiesen werden konnten. Allerdings sind diese Randpunkte so niedrig in der Anzahl, dass es keinen großen Ausmaß machte.

\begin{figure}[h]
	\centering
	\begin{subfigure}[h]{0.49\textwidth}
		\includegraphics[width=\textwidth]{Abbildungen/ground_truth_segments_agpn.png}
		\centering
		\caption{Durch AGPN erkannten Segmente}
		\label{fig: agpn_segments_grnd_trth}
	\end{subfigure}
	\hfil
	\begin{subfigure}[h]{0.49\textwidth}
		\includegraphics[width=\textwidth]{Abbildungen/ground_truth_segments_iefd.png}
		\centering
		\caption{Durch IEFD erkannten Segmente}
		\label{fig: iefd_segments_grnd_trth}
	\end{subfigure}
	\caption{Die Segmente der Testdatei, die durch beide Verfahren erkannt wurden}
	\label{fig: segments_comparision_grnd_trth}
\end{figure}

Somit ließ sich die erste Forschungsfrage beantworten - das IEFD-Verfahren bietet eine vergleichbare hohe Genauigkeit wie das AGPN-Verfahren und kann sogar Kanten besser erkennen. Um dieses Verfahren weiterhin auf seine Einsatzfähigkeit zu überprüfen erfolgten die nächsten Tests.

\subsection{Überprüfung des Einflusses der Punktedichte}