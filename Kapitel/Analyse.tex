\chapter{Analyse des Prototyps}
Zur Evaluierung des Prototyp-Verfahrens wurden Anhand der Forschungsfrage sowie den drei Teilforschungsfragen Tests entworfen. Der erste Test sollte die Genauigkeit des Prototyps überprüfen. Mittels des zweiten Tests wurde der Einfluss der Punktdichte auf dem Verfahren überprüft. Der letzte Test überprüfte die Robustheit des Prototyp-Verfahrens gegen Objekten mit unterschiedlichen geometrischen Eigenschaften.

\section{Testdaten}
Für alle drei Tests mussten Testdaten erstellt werden. Nach Bedarf der Tests wurden hierzu Punktwolken künstlich entworfen oder mit einem Lasersensor aufgenommen. Bei der Aufnahme wurden unterschiedliche Werkstücke mit eindeutigen Kanten sowie Geometrien und/oder sichtbaren Knicken ausgesucht, um eine möglichst vielfältige Datenbasis anzubieten. Für das Scannen wurde der Laserliniensensor - \textit{scanControl 3000} verwendet, der auf dem Schweißroboter nach \textcite[39]{savla_intelligente_2022} montiert wurde. Obwohl die Entfernung des Lasersensors von der Werkstückoberfläche über die gesamte Abtastung möglichst konstant gehalten wurde, waren die Punktabstände der Punktwolke manchmal leicht unregelmäßig. Laut \textcite[9]{ni_edge_2016} hatte der Punktabstand den größten Einfluss auf die Leistung von AGPN. Es wurde behauptet, dass das \textit{UniformSampling} diese Verzerrungen der Punktabstände ausgleichen würde. 

Die künstliche Punktwolke wurde auf Basis vorgegebener Parameter mittels eines Python-Skriptes erstellt. Der Zweck dieser Punktwolke war es, ein Objekt mit ein paar einfachen geometrischen Merkmale abzubilden. Dieses Objekt durfte dann als ein Ground-Truth verwendet werden.