\chapter{Evaluierung und Ergebnisse}
Das IEFD-Verfahren dieser Arbeit ist noch in der Prototypenphase. Wie in Abschnitt~\ref{Motivation} erwähnt, sollte dieses Verfahren unter anderem, eine Anwendung bei der Online-Erkennung von Schweißnähten finden. Um dieses zu erreichen, wurde das Verfahren unter bestimmten Kriterien evaluiert. Zur Evaluierung des Verfahrens wurden Anhand der Forschungsfrage sowie den drei Teilforschungsfragen Tests entworfen. Der erste Test sollte die Genauigkeit des IEFD-Verfahrens überprüfen. Mittels des zweiten Tests wurde der Einfluss der Punktdichte auf das Verfahren überprüft. Der letzte Test überprüfte die Robustheit des IEFD-Verfahrens gegen Objekte mit unterschiedlichen geometrischen Eigenschaften.

\section{Vorbereitungsmaßnahmen für die Untersuchungen}
\subsection{Testdaten}\label{test_data}
Für alle drei Tests wurden Testdaten erstellt. Nach Bedarf der Tests wurden hierzu Punktwolken künstlich entworfen oder mit einem Lasersensor aufgenommen. Bei der Aufnahme wurden unterschiedliche Werkstücke mit eindeutigen Kanten sowie Geometrien und/oder sichtbaren Knicken ausgesucht, um eine möglichst vielfältige Datenbasis zu erzeugen. Für das Scannen wurde ein Laserliniensensor von \textit{scanControl} verwendet. Die Testdaten entstanden aus Aufnahmen vier reeller Bauteile mit verschiedenen geometrischen Merkmalen. Außerdem ersten wurden die restlichen drei Bauteile auf einem Schweißtisch (PROFIPlusLINE D16) mit Löchern von einem Durchmesser von ca. 15\,mm gespannt. Das erste Bauteil bestand aus einem Blech mit einer kreisförmigen Aussparung sowie zwei runden schmalen Nuten mit einer durchschnittlichen Breite von ca. 3,5\,mm sowie einer durchschnittlichen Tiefe von ca. 3\,mm. Über dem Blech befand sich auch ein halbkreisförmiges Objekt, welches auch aufgenommen wurde. Die nächsten zwei Bauteile bestanden aus zwei senkrecht zueinanderstehenden Bleche mit rechteckigen Aussparungen. Das letzte Bauteil war geometrisch sehr ähnlich zu den letzten zwei Bauteile, allerdings hat es keine rechteckige Aussparungen. Die Innenkante des zweiten Bauteils war rund. Auf dem dritten sowie des vierten Bauteil befanden sich rechteckige Aussparungen unterschiedlicher Dimensionen. Darüber hinaus hatte die Innenkante dieser beiden Bauteile ein stufenartiges Form. Teile des Schweißtisches sowie dessen Löcher wurden bei dem Scan der letzten drei Bauteile mitaufgenommen. Obwohl die Entfernung des Lasersensors von der Werkstückoberfläche über die gesamte Abtastung möglichst konstant gehalten wurde, waren die Punktabstände der Punktwolken manchmal leicht unregelmäßig. Laut \textcite[9]{ni_edge_2016} hatte der Punktabstand den größten Einfluss auf die Leistung von AGPN. Es wurde behauptet, dass das \textit{UniformSampling} diese Verzerrungen der Punktabstände ausgleichen würde. 

Die künstliche Punktwolke wurde auf Basis vorgegebener Parameter mittels eines Python-Skriptes erstellt. Der Zweck dieser Punktwolke war es, ein Objekt mit ein paar einfachen geometrischen Merkmalen zu simulieren. Dieses Objekt wurde dann als ein Ground-Truth verwendet. Unter Berücksichtigung des Einsatzzwecks wurde eine treppenförmige Oberfläche erzeugt, welche eine Innenkante sowie eine Außenkante vorhanden war. Diese Kanten sollten Schweißnähte nachahmen, entlang der durch den Schweißroboter geschweißt werden sollte. Das Weltkoordinatensystem wurde als Referenz für die Erstellung verwendet. Die \textit{x}, \textit{y} und \textit{z} Dimensionen entsprachen den Werten \textit{0,1\,m}, \textit{0,2\,m} beziehungsweise \textit{0,05\,m}. Als den Start wurde der Punkt $\left(\begin{smallmatrix}
	0,01\,m & 0,01\,m & 0,01\,m
\end{smallmatrix}\right)$ gewählt. Der erste Teil der Punktwolke bestand aus eine Oberfläche auf der \textit{xy}-Ebene mit der Dimension $\left(\begin{smallmatrix}
0,05m & 0,2m
\end{smallmatrix}\right)$. Der zweite Teil bestand aus einer Oberfläche auf der \textit{yz}-Ebene mit der Dimension $\left(\begin{smallmatrix}
0,2m & 0,05m
\end{smallmatrix}\right)$, die am Ende des ersten Teils begann. Die Schnittlinie der ersten beiden Teile bildete die Innenkante ab. Der letzte Teil bestand aus einer Oberfläche auf der \textit{xy}-Ebene mit der gleichen Dimension wie der erste Teil. Diese Oberfläche schloss sich am Ende des zweiten Teils an und die daraus entstandene Schnittlinie bildete die Außenkante. Somit ergab sich eine Punktwolke von einer konstanten Größe. Der Punkteabstand dieser Punktwolke wurde für alle Tests unterschiedlich gewählt. Für den ersten und dritte Test wurde ein konstanter und einheitlicher Punkteabstand zwischen allen benachbarten Punkten festgelegt. Für den zweiten Test wurde allerdings der Punkteabstand variiert und wird näher im Abschnitt~\ref{test_2} behandelt. Abbildung~\ref{fig: ground_truth} zeigt die künstlich erstellte Punktwolke, die weiterhin als Ground-Truth benannt wird. 

\begin{figure}[t]
	\includegraphics[scale = 0.5]{Abbildungen/ground_truth.png}
	\centering
	\caption[ground truth]{Die Testdatei beziehungsweise das Ground-Truth}
	\label{fig: ground_truth}
\end{figure}

\subsection{Evaluierungsmetriken} \label{evaluations_metrics}
Zur Evaluierung der Leistung des IEFD-Verfahrens wurden ein paar quantitative Metriken nach \textcite[10]{ni_edge_2016} entwickelt. Diese sollten die Genauigkeit der Kantenerkennung sowie die Genauigkeit der Segmentierung überprüfen. Wie in Abschnitt~\ref{edge_detection_reprod} erwähnt, sollte die Differenz der Winkelabstände \textit{G\textsubscript{$\theta$}} Randpunkte zu einem anderen Randpunkt größer oder gleich dem Schwellwert $\alpha$ sein, welcher für diesen Test $\frac{\pi}{2}$ betrug. Die Metrik \textit{p\textsubscript{dc}} wurde zur Angabe der Genauigkeit des IEFD-Verfahrens bei der Kantenerkennung verwendet. Es wurden die Anzahl der korrekt erkannten Kanten \textit{N\textsubscript{dc}} mit der Anzahl der tatsächlich vorhanden Kanten \textit{N\textsubscript{gc}} verglichen. Gleichzeitig wurde auch die Ungenauigkeit \textit{p\textsubscript{mj}} des Verfahrens geprüft, indem die Anzahl der fälschlicherweise markierten Kanten \textit{N\textsubscript{mj}} mit der Anzahl der tatsächlich vorhanden Kanten \textit{N\textsubscript{gc}} verglichen wurden. Wie in Abschnitt~\ref{edge_segmentation} erwähnt, dürfen die Kanten zusammengefasst werden, deren Randpunkte in unmittelbarer Nähe zu einander stehen und deren Richtungsvektoren nicht zu sehr voneinander abweichen. Die Metrik \textit{p\textsubscript{dct}} wurde zur Überprüfung der Genauigkeit der Kantensegmentierung entworfen. Hierbei wurde die Anzahl der korrekt erzeugten Segmente \textit{N\textsubscript{tc}} mit der Anzahl der korrekt erkannten Kanten \textit{N\textsubscript{dc}} sowie der Anzahl der fälschlicherweise erkannten Kanten \textit{N\textsubscript{mj}} verglichen. Gleichzeitig wurde auch die Ungenauigkeit \textit{p\textsubscript{mjt}} überprüft, indem die Anzahl der fälschlicherweise erzeugten Segmente \textit{N\textsubscript{mjt}} mit der Anzahl der korrekt erkannten Kanten \textit{N\textsubscript{dc}} sowie der Anzahl der fälschlicherweise erkannten Kanten \textit{N\textsubscript{mj}} verglichen wurde. Falls \textit{N\textsubscript{mjt}} größer als \textit{N\textsubscript{dc}} + \textit{N\textsubscript{mj}} betrug, wurde \textit{N\textsubscript{mjt}} mit der gesamten Anzahl der erkannten Segmente $\textit{N\textsubscript{mjt}} + \textit{N\textsubscript{tc}}$ verglichen. Die fälschlicherweise erkannten Kanten durch die \textit{FindEdgePoints}-Methode wurden bei der Metrik \textit{p\textsubscript{dct}} sowie \textit{p\textsubscript{mj}} auch mitbetrachtet, da diese vor der Kantensegmentierung nicht verworfen werden. Die Gleichungen \ref{pdc} bis \ref{pmjt} zeigen die vier Metriken.

\begin{equation}
	\label{pdc}
	p_{dc} = \frac{N_{dc}}{N_{gc}}
\end{equation}
\begin{equation}
	\label{pmj}
	p_{mj} = \frac{N_{mj}}{N_{gc}}
\end{equation}
\begin{equation}
	\label{pdct}
	p_{dct} = \frac{N_{tc}}{N_{dc} + N_{mj}}
\end{equation}
\begin{equation}
	\label{pmjt}
	p_{mjt} =
	\begin{cases}
		\frac{N_{mjt}}{N_{dc} + N_{mj}} & wenn\ N_{mjt} \leq N_{dc} + N_{mj}\\
		\frac{N_{mjt}}{N_{tc} + N_{mjt}} & sonst\ N_{mjt} > N_{dc} + N_{mj}
	\end{cases}
\end{equation}

\section{Erprobung des Verfahrens}
\subsection{Überprüfung der allgemeinen Genauigkeit} \label{test_1}
Die erste Forschungsfrage stellt die Genauigkeit des Verfahrens in Frage. Hierbei sollte geprüft werden wie viele Kanten durch das Verfahren richtig sowie falsch erkannt wurden, und wie viele davon korrekt oder inkorrekt segmentiert wurden. Es wurde postuliert, dass das IEFD-Verfahren zu einer hohen Genauigkeit dank der Einheitlichkeit der Ground-Truth alle Randpunkte erkennen würde. Für diesen Test wurden die beide Verfahren - das AGPN und IEFD - auf Basis des Ground-Truth geprüft. Hierzu wurde ein Punkteabstand von genau 0,00025\,\si{\metre} festgelegt. Die Überprüfung der Genauigkeit erfolgte durch zwei Methoden. Es wurde die Anzahl der, durch das Verfahren erkannten, Randpunkte und Segmente gezählt und mit der Anzahl der tatsächlich vorhandenen Randpunkte beziehungsweise Segmente verglichen. Für die zweite Methode wurden die Metriken aus Abschnitt \ref{evaluations_metrics} verwendet, um die Verhältnisse der richtig sowie falsch erkannten Kanten sowie Segmente zu bestimmen. Bei der Festlegung der Parameter wurden die Erkenntnisse aus der Literaturquelle betrachtet. Es wurde erkannt, dass die Parameter \textit{K\textsubscript{1}} und \textit{K\textsubscript{2}} nahezu keinen großen Einfluss auf die Genauigkeit des AGPNs hatten. Der Schwellwert beziehungsweise Glättungsfaktor $\phi$ bestimmte die maximale Abweichung zwischen zwei fugenlosen Segmente. Die Grenzwerte \textit{d\textsubscript{t1}} und \textit{d\textsubscript{t2}} für den Punkteabstand der RANSAC-Verfahren aus der Kantenerkennung und Kantensegmentierung, die die Bestimmung von \textit{Inliers} der Verfahren regeln, hatten die größten Einflüsse auf die Genauigkeit. Es wurde wie in dem Referenzwerk einen Wert für \textit{d\textsubscript{t1}} gleich dem durchschnittlichen Punkteabstand gewählt. Der Wert für \textit{d\textsubscript{t2}} wurde größer gewählt und betrug das dreifache des Punkteabstandes. \autocite[10-11]{ni_edge_2016}

Unter Berücksichtigung der Erkenntnisse aus dem Referenzwerk wurden die restlichen Parameter für das Ground-Truth festgelegt. Die Größe einer Iteration wurde auf 0,01m festgelegt. Die letzten 0,001\,\si{\m} einer Iteration werden wiederholt sowie einen Bereich von 0,0002\,\si{\m} zur Entfernung falscher Kanten festgelegt. Die Tabelle \ref{table: parameters_test1} listet die Parameter für das IEFD auf. Aufgrund der relativ hohen Punktdichte wurde festgelegt, dass Kanten und Randpunkte, die weniger als ein drittel Millimeter voneinander entfernt sind, zusammen segmentiert werden durften. Deswegen wurde ein \textit{d\textsubscript{t2}} von 0,0003\,\si{\m} gewählt. Im Gegenzug wurde zwecks einer hohen Genauigkeitsanforderung ein \textit{d\textsubscript{t1}} von 0,0001\,\si{\m} für die Kantenerkennung gewählt.

\begin{table}
	\centering
	\begin{tabular}[width=\textwidth]{l *{8}{c}}
		\hline
		\multirow{2}{*}{\textbf{Datei}}&\multirow{2}{*}{\textbf{Punktezahl}}&\multirow{2}{*}{\textbf{Punkteabstand}}&\multicolumn{6}{c}{\textbf{Parameter}}\\
		& & & \textbf{K\textsubscript{1}} & \textbf{d\textsubscript{t1}} & \textbf{$\alpha$} & \textbf{K\textsubscript{2}} & \textbf{d\textsubscript{t2}} & \textbf{$\phi$} \\
		\hline
		Ground-Truth & 12000000 & 0,0001 & 200 & 0,0001 & $\frac{\pi}{2}$ & 30 & 0,0003 & 0,2 \\
		\hline
	\end{tabular}
	\caption{Parameter für den Test auf die Ground-Truth}
	\label{table: parameters_test1}
\end{table}

Die Überprüfung der Genauigkeit der Kantenerkennung und Segmentierung nach beiden obigen Methoden wurde auch für das AGPN ausgeführt, um Vergleichswerte zu erzeugen. Dank der Einheitlichkeit des Ground-Truths war bekannt, dass insgesamt 14.000 Punkte auf den äußeren Rändern sowie 8.000 auf die Innen- und Außenkanten gaben. Daneben war auch bekannt, dass es insgesamt acht Außenränder sowie zwei Innen- beziehungsweise Außenkanten gibt. Somit ergab sich der Wert von \textit{N\textsubscript{gc}} als 10. Auf Basis dieser Größen wurde die Genauigkeit der IEFD- und AGPN-Verfahren überprüft.

Um die Einflüsse von Ausreißer aus den Ergebnissen zu reduzieren wurden die AGPN und IEFD Verfahren jeweils sechs Mal auf das Ground-Truth ausgeführt. Die Ergebnisse daraus wurden gemittelt und präsentiert. Das IEFD-Verfahren konnte durchschnittlich 13.100 Randpunkte erkennen. Dies entsprach eine Erfolgsrate von $59,54\%$. Die erkannten Randpunkte befanden sich über alle Außenränder sowie Innen- und Außenkanten verteilt. Jede Kante der Ground-Truth wurde durch das Verfahren erkannt und eindeutig bestimmt. Die Kantensegmentierung des IEFD-Verfahrens lieferte sehr gute Ergebnisse. Es wurden durchschnittlich $97,29\%$ der erkannten Randpunkte richtig segmentiert. Bis auf einen Fall wurden in den restlichen fünf Ausführungen des Verfahrens alle 10 Kanten der Ground-Truth erkannt. Im Falle des Ausreißers wurde die Innenkante nicht vollständig segmentiert, sondern wurden zwei zusätzliche Segmente auf der Kante erzeugt. Das AGPN-Verfahren konnte im Gegensatz durchschnittlich 6.900 Randpunkte erkennen und entsprach eine Erfolgsrate von $31,36\%$. Die erkannten Randpunkte befanden sich lediglich auf den Außenränder verteilt. Die Innen- sowie Außenkante wurden durch das Verfahren überhaupt nicht erkannt. Die Kantensegmentierung des AGPN-Verfahrens lieferte im Gegensatz sehr gute Ergebnisse. Hierbei ist zu bemerken, dass die Kantensegmentierung nur auf Basis der erkannten Kanten bewertet wurde, und die fehlende Innen- sowie Außenkante nicht in Betracht gezogen wurden. Es wurden durchschnittlich $99,27\%$ der Randpunkte korrekt segmentiert. Anders als im Falle des IEFD-Verfahrens wurden in keiner der Ausführungen unvollständige Segmente erzeugt. Da die Anzahl der korrekt erkannten und segmentierten Randpunkte nicht im Umfang dieser Arbeit wichtig waren, wurden die Metriken nach Abschnitt \ref{evaluations_metrics} zur Bewertung der Verfahren weiterhin verwendet. 

%Obwohl die Genauigkeitsraten der Kantenerkennung beider Verfahren nicht vielversprechend sind, bildeten die erkannten Randpunkte der Verfahren alle Kanten der Testdatei zu einer hohen positionellen Genauigkeit ab. Die fehlenden Randpunkte verursachten keine Lücken in den Kanten und beeinträchtigen die Glätte der erkannten Kanten nicht. Nur für einen Anwendungsfall mit sehr hohen Genauigkeitsanforderung, wo jeder einzelner Randpunkt richtig erkannt werden soll, würden die AGPN- und IEFD-Verfahren unzureichend sein. SOLL IN DISKUSSION

Die Ergebnisse aus den sechs Ausführungen wurden für die Auswertung nach Abschnitt \ref{evaluations_metrics} wiederverwendet. Das IEFD-Verfahren hatte einen durchschnittlichen \textit{p\textsubscript{dc}} Wert von 1,00, da in allen sechs Ausführungen alle Kanten der Ground-Truth erkannt wurden. Spiegelbildlich zu dieser Metrik hatte die Metrik \textit{p\textsubscript{mj}} einen Wert von 0,00, da das Verfahren keine Kanten fälschlicherweise erkannt hat. In fünf der Ausführungen wuren alle erkannten Kanten richtig und vollständig segmentiert. Im Falle des einen Ausreißers wurde die Innenkante teilweise richtig segmentiert, sowie zwei weitere Segmente auf der Kante erzeugt. Somit ergibt sich einen durchschnittlichen Wert von 0,98 für die Metrik \textit{p\textsubscript{dct}} allerdings auch einen Wert von 0,02 für die Metrik \textit{p\textsubscript{mjt}}. Insgesamt funktionierte das IEFD-Verfahren mit einer Genauigkeit von circa 98,61\%. Das AGPN-Verfahren hatte einen durchschnittlichen \textit{p\textsubscript{dc}} Wert von 0,80, da die Innen- und Außenkante in keiner der sechs Ausführungen erkannt wurden. Das AGPN-Verfahren hat zwei Kanten nicht erkannt und erzielte somit eine Bewertung von 0,2 für die Metrik \textit{p\textsubscript{mj}}. Die Kantensegmentierung erfolgte in allen sechs Ausführungen fehlerfrei, abgesehen davon, dass die Innen- und Außenkanten nicht erkannt wurden. Das AGPN-Verfahren erzielte eine Bewertung von 1,00 für die Metrik \textit{p\textsubscript{dct}} und 0,00 für die Metrik \textit{p\textsubscript{mjt}}. Insgesamt konnte das AGPN-Verfahren Kanten mit einer Genauigkeit von 80,00\% erkennen und segmentieren. Die Tabelle \ref{table: metric_values} fasst diese Ergebnisse zusammen.

\begin{table}[h]
	\centering
	\begin{tabular}{l *{9}{c}}
		\hline
		\textbf{Verfahren} & \textbf{N\textsubscript{dc}} & \textbf{N\textsubscript{mj}} & \textbf{N\textsubscript{gc}} & \textbf{p\textsubscript{dc}} & \textbf{p\textsubscript{mj}} & \textbf{N\textsubscript{tc}} & \textbf{N\textsubscript{mjt}} & \textbf{p\textsubscript{dct}} & \textbf{p\textsubscript{mjt}} \\
		\hline
		AGPN & 8 & 0 & 10 & 0,80 & 0 & 8 & 0 & 1,00 & 0 \\
		IEFD & 10 & 0 & 10 & 1,00 & 0 & 9 & 2 & 0,9 & 0,2 \\
		\hline
	\end{tabular}
	\caption{Die schlechtesten Werte der jeweiligen Metriken aus sechs Wiederholungen}
	\label{table: metric_values}
\end{table}

Die Abbildung \ref{fig: segments_comparision_grnd_trth} visualisiert die Qualität der erkannten und segmentierten Kanten und vergleicht die Ergebnisse aus beiden Verfahren. Wie es zu sehen ist, wurden durch das AGPN-Verfahren nach Abbildung \ref{fig: agpn_segments_grnd_trth} die Innen- und Außenkanten nicht erkannt. Im Gegensatz dazu ist lauf Abbildung \ref{fig: iefd_segments_grnd_trth} zu sehen, dass das IEFD-Verfahren alle Ränder sowie Innen- und Außenkanten erkennen konnte. 
%Wie es in dieser Abbildung zu sehen ist, verlaufen die erkannten Kanten nahezu lückenlos. Trotz der fehlenden Randpunkte sind alle Kanten und somit die Struktur und das Rahmen der Testdatei eindeutig zu erkennen. Die Ecken dieser Kanten stellen Bereiche dar, wo ein paar Randpunkte zu keiner der angrenzenden Kanten zugewiesen werden konnten. Allerdings sind diese Randpunkte so niedrig in der Anzahl, dass es keinen großen Ausmaß machte. SOLL IN DISKUSSION

\begin{figure}[h]
	\centering
	\begin{subfigure}[h]{0.49\textwidth}
		\includegraphics[width=\textwidth]{Abbildungen/ground_truth_segments_agpn.png}
		\centering
		\caption{Durch AGPN erkannten Segmente}
		\label{fig: agpn_segments_grnd_trth}
	\end{subfigure}
	\hfil
	\begin{subfigure}[h]{0.49\textwidth}
		\includegraphics[width=\textwidth]{Abbildungen/ground_truth_segments_iefd.png}
		\centering
		\caption{Durch IEFD erkannten Segmente}
		\label{fig: iefd_segments_grnd_trth}
	\end{subfigure}
	\caption{Die Segmente der Ground-Truth, die durch beide Verfahren erkannt werden}
	\label{fig: segments_comparision_grnd_trth}
\end{figure}

Somit ließ sich die erste Forschungsfrage beantworten - das IEFD-Verfahren bietet eine vergleichbare hohe Genauigkeit wie das AGPN-Verfahren und kann sogar Kanten besser erkennen. Um dieses Verfahren weiterhin auf seine Einsatzfähigkeit zu überprüfen erfolgten die nächsten Tests.

\subsection{Überprüfung des Einflusses der Punktedichte} \label{test_2}
Das Referenzwerk \autocite{ni_edge_2016} deutete auf den Einfluss der Punktedichte auf die Genauigkeit des Algorithmus hin. Die Punktedichte einer Punktwolke korreliert mit schärfer umrissenen Abbildungen, da es eine höhere Anzahl von Punkten auf die Kanten und andere geometrischen Merkmale vorhanden sind. Somit sollten diese Merkmale und deren Kanten einfacher erkannt werden. Deswegen wurde die Hypothese aufgestellt, dass sich die Genauigkeit des IEFD-Verfahrens proportional zu der Punktedichte verhalten würde - eine Abnahme der Punktdichte sollte eine Verringerung der Genauigkeit entsprechen. Um diese Hypothese auszutesten wurde auch das Ground-Truth verwendet. Um unterschiedlichen Punktedichten zu simulieren wurde der Punkteabstand der Ground-Truth zwischen einen Bereich von 0,00005\,\si{\m} zu 0,005\,\si{\m}m diskret variiert. Die Schrittgröße lässt sich aus der Tabelle~\ref{table: test_2_results} ermitteln. Somit ergaben sich 9 unterschiedlichen Varianten der Ground-Truth mit verschiedenen Punktedichten. Die Dimensionen aller Varianten der Ground-Truth blieben dabei konstant. Abbildung \ref{fig: testdata_pointdensity_comparision} zeigt den Unterschied der Punktedichte zwischen der ersten Variante der Ground-Truth mit einem Punkteabstand von 0,00005\,\si{\m}m und der neunten Variante der Ground-Truth mit einem Punkteabstand von 0,005\,\si{\m}m.

\begin{figure}[h]
	\centering
	\begin{subfigure}{0.49\textwidth}
		\includegraphics[width=\textwidth]{Abbildungen/ground_truth_0,00005.png}
		\centering
		\caption{Die erste Variante der Ground-Truth mit Punkteabstand 0,00005m}
		\label{fig: testdata_0,00005m}
	\end{subfigure}
	\hfill
	\begin{subfigure}{0.49\textwidth}
		\includegraphics[width=\textwidth]{Abbildungen/ground_truth_0,005.png}
		\centering
		\caption{Die neunte Variante der Ground-Truth mit Punkteabstand 0,005m}
		\label{fig: testdata_0,005m}
	\end{subfigure}
	\caption{Vergleich zwischen der Varianten der Ground-Truth mit der höchsten und niedrigsten Punktedichte}
	\label{fig: testdata_pointdensity_comparision}
\end{figure}

Da der Punkteabstand in diesem Test variiert wurde, durfte die Anzahl der Scan-Linien \textit{n} in einer Iteration nicht auf Basis der Fensterbreite von 0,4\,\si{\mm} bestimmt werden. Stattdessen wurde \textit{n} auf die fest Zahl von 80 festgelegt. Bei der Variante mit einem Punkteabstand von 0,001\,\si{\m} wurde \textit{n} auf 50, bei 0,0025\,\si{\m} auf 20 und bei 0,005\,\si{\m} auf 10 . Dabei wurde es versucht, dass mindestens fünf Iterationen ausgeführt wurden. Auch die Anzahl \textit{k} der wiederholten Scan-Linien aus Abschnitt~\ref{Testumgebung} wurde so festgelegt, dass immer 10 Scan-Linien wiederholt wurden. Für die Entfernung falscher Kanten wurde ein Bereich definiert, der maximal zwei Scan-Linien am Anfang oder am Ende der Iteration entfernt hat. Außer den Parametern \textit{d\textsubscript{t1}} und \textit{d\textsubscript{t2}} wurden alle Parameter konstant gehalten und mit den gleichen Werten aus Abschnitt \ref{test_1} verwendet. Die Parameter \textit{d\textsubscript{t1}} und \textit{d\textsubscript{t2}} wurden wieder abhängig von dem Punkteabstand gewählt. \textit{d\textsubscript{t1}} wurde dem Punkteabstand gleich groß gewählt und \textit{d\textsubscript{t2}} betrug das dreifache von \textit{d\textsubscript{t1}}. Tabelle \ref{table: test_2_results} stellt die Ergebnisse dieses Tests dar.

\begin{table}[b]
	\centering
	\begin{tabular}[width = \textwidth]{l *{8}{c}}
		\hline
		\multirow{2}{2em}{\textbf{Id.}} & \multirow{2}{3em}{\textbf{Punkte-zahl}} & \multirow{2}{3em}{\textbf{Punkte-abstand}} & \multicolumn{2}{c}{\textbf{Parameter}} & \multirow{2}{*}{\textbf{p\textsubscript{dc}}} & \multirow{2}{*}{\textbf{p\textsubscript{mj}}} & \multirow{2}{*}{\textbf{p\textsubscript{dct}}} & \multirow{2}{*}{\textbf{p\textsubscript{mjt}}} \\
		& & & \textbf{d\textsubscript{t1}} &\textbf{d\textsubscript{t2}} & & & & \\
		\hline
		1 & 48.000.000 & 0,00005 & 0,00005 & 0,00015 & 0,8 & 0 & 1 & 0 \\
		2 & 21.326.667 & 0,000075 & 0,000075 & 0,000225 & 1 & 0 & 1 & 0 \\
		3 & 12.000.000 & 0,0001 & 0,0001 & 0,0003 & 1 & 0 & 1 & 0 \\
		4 & 1.920.000 & 0,00025 & 0,00025 & 0,00075 & 1 & 0 & 1 & 0 \\
		5 & 480.000 & 0,0005 & 0,0005 & 0,0015 & 1 & 0 & 1 & 0 \\
		6 & 212.667 & 0,00075 & 0,00075 & 0,00225 & 1 & 0 & 1 & 0 \\
		7 & 120.000 & 0,001	& 0,001 & 0,003 & 1 & 0 & 1 & 0 \\
		8 & 19.200 & 0,0025 & 0,0025 & 0,0075 & 1 & 0 & 0,9 & 0,1 \\
		9 & 4.800 & 0,005 & 0,005 & 0,015 & 0,7 & 0 & 0,86 & 0,14 \\
		\hline 
	\end{tabular}
	\caption{Ergebnisse des zweiten Tests}
	\label{table: test_2_results}
\end{table}

Die Varianten \textit{2-7} liefern perfekte Ergebnisse. Das IEFD-Verfahren leistet trotz einer absteigende Punktedichte für die Varianten eine sehr hohe Genauigkeit. Bei der achten Variante wurden trotz der höheren Punktedichte alle Kanten der Ground-Truth richtig erkannt. Allerdings war die Kantensegmentierung nicht genau so erfolgreich. Die Außenkante, die in der y-Richtung und somit der Scan-Richtung zeigte, wurde nicht vollständig segmentiert. Stattdessen wurden durch das Verfahren zwei Segmente auf der Kante erzeugt. Somit erzielte der Testdurchlauf der achten Variante für \textit{p\textsubscript{dct}} eine Bewertung von \textit{0,9} sowie für die Metrik \textit{p\textsubscript{mjt}} einen Wert von \textit{0,1}. Bei der neunten Variante der Ground-Truth konnten weder die Kantenerkennung noch die Kantensegmentierung vergleichsweise gut abschneiden. Es wurden bei dieser Variante der Ground-Truth drei der sechs Seitenkanten nicht vollständig erkannt. Diese Kanten lagen auf der ersten sowie der dritten \textit{xy}-Ebene und zeigten in der x-Richtung. Die Randpunkte auf die, durch das Verfahren erkannten, Kanten waren dürftig verteilt, wodurch die Kanten nicht lückenlos erschienen. Allerdings wurden keine zusätzliche falsche Kanten durch as Verfahren erkannt. Aus diesem Grund erlangte der Testdurchlauf der neunten Variante einen \textit{p\textsubscript{dc}}-Wert von \textit{0,7} und sowie einen \textit{p\textsubscript{mj}}-Wert von \textit{0}. Die Kantensegmentierung konnte aus den sieben erkannten Kanten nur fünf richtig segmentieren. Hierbei wurde eine Seitenkante der dritten \textit{xy}-Ebene nicht von der Längskante unterschieden und wurde somit falsch segmentiert. Auch die Außenkante konnte nicht vollständig segmentiert werden, sondern wurden auf die Kante zwei Segmente erzeugt. Der Wert für \textit{p\textsubscript{dct}} betrug somit \textit{0,86} und \textit{0,14} für die Metrik \textit{p\textsubscript{mjt}}. Die erste Variante der Ground-Truth mit einem Punkteabstand von 0,00005m lieferte auch unerwartete Ergebnisse. Trotz der sehr hohen Punktedichte wurden die Innen- und Außenkanten von Ground-Truth durch das Verfahren nicht erkannt, weswegen dieser Testdurchlauf eine \textit{p\textsubscript{dc}} Bewertung von \textit{0,8} sowie eine \textit{p\textsubscript{mj}} Bewertung von \textit{0,2} erhielt. Diese acht Randelemente wurden korrekt und vollständig durch das Verfahren segmentiert und erhielten eine perfekte Bewertung \textit{1,0} für die Metrik \textit{p\textsubscript{dct}}. Abbildung \ref{fig: point_density_bar_chart} visualisiert die Leistung des IEFD-Verfahrens.

\begin{figure}[t]
	\includegraphics[width=\textwidth]{Abbildungen/point_density_influence_bar.png}
	\centering
	\caption{Vergleich der Genauigkeit des IEFD-Verfahrens bei einer Änderung der Punktedichte}
	\label{fig: point_density_bar_chart}
\end{figure}

Nachdem der Einfluss der Punktedichte auf die Genauigkeit des Verfahrens überprüft wurde, wurde der nächste Test ausgeführt, um gezielt die Einsatzfähigkeit des Verfahrens unter reellen Bedingungen zu überprüfen.

\subsection{Überprüfung der Robustheit} \label{test_3}
Wie in Abschnitt \ref{Motivation} erwähnt, soll das Verfahren unter reellen Bedingungen verwendbar sein und unterschiedlichen Geometrien erkannt werden. Dieses verlangt eine hohe Robustheit des Verfahrens. Anhand der Erkenntnisse aus den vorherigen Tests wurde die Hypothese aufgestellt, dass das Verfahren unter rellen Bedingungen zuverlässig funktionieren würde. Um diese Hypothese zu überprüfen wurden zwei Tests konzipiert. Da reelle Aufnahmen durch einen Lasersensor häufig sehr unregelmäßig sind, wurde für den ersten Test der Einfluss einer steigenden Unregelmäßigkeit des Punkteabstandes auf die Genauigkeit des Verfahrens überprüft. Bei dem zweiten Test wurde das IEFD-Verfahren auf verschiedene Aufnahmen von reellen Bauteilen angewendet und anhand der Metriken aus Abschnitt \ref{evaluations_metrics} bewertet.

Für den ersten Test wurde die Ground-Truth verwendet. Hierbei wurde einen Punkteabstand von 0,00025\,\si{\m} festgelegt. Die reelle Punkte aus dem Lasersensor, wiesen keinen regelmäßigen Punkteabstand auf. Um diese Unregelmäßigkeit nachzuahmen, wurde eine künstliche Verzerrung des Punktmusters dem Ground-Truth implementiert. Hierfür wurde ein zufälliger Versatz \textit{d} auf dem ursprünglichen Positionen der Punkte aufaddiert. Zur Errechnung des Versatzes wurde eine zufällige reelle Zahl zwischen -1 und 1 gewählt und mit einem Verzerrungsfaktor \textit{\^{r}} multipliziert. Das Faktor ließ sich errechnen, indem der konstante Punkteabstand \textit{r} mit einer Zahl \textit{k} zwischen 0 und 1,5 multipliziert wurde. Für jeden Punkt wurde ein unterschiedlicher Versatz \textit{d} in allen drei Richtungen des Koordinatensystems errechnet, um den Abstand zwischen Punkten möglichst unregelmäßig zu gestalten. Abbildung \ref{fig: point_pattern_comparision} zeigt die unterschiedlichen Mustern der Punkte in dem Ground-Truth mit und ohne einem Rauschen. 

\begin{figure}[b]
	\centering
	\begin{subfigure}{0.49\textwidth}
		\includegraphics[width=\textwidth]{Abbildungen/point_pattern_without_dist.png}
		\centering
		\caption{Punktmuster von Ground-Truth ohne Verzerrung}
		\label{fig: point_pattern_without_dist}
	\end{subfigure}
	\hfill
	\begin{subfigure}{0.49\textwidth}
		\includegraphics[width=\textwidth]{Abbildungen/point_pattern_with_dist.png}
		\centering
		\caption{Punktmuster von Ground-Truth mit Verzerrung und \textit{\^{r}=1,5}}
		\label{fig: point_patter_with_dist}
	\end{subfigure}
	\caption{Vergleich der Punktmuster des Ground-Truths mit und ohne Verzerrung}
	\label{fig: point_pattern_comparision}
\end{figure}

Für diesen Test wurden die Parameter außer \distthresha und \distthreshb des IEFD-Verfahrens konstant gehalten und aus Abschnitt \ref{test_1} übernommen. Zwecks der kürzeren gesamten Verarbeitungszeit wurde ein höherer Punktabstand von 0,00025\,\si{\m}, da es eine optimale Bilanz zwischen Punktedichte und Punktezahl angeboten hat sowie den durchschnittlichen Punkteabstand der Aufnahmen entsprochen hat, die mittels des Lasersensors aufgenommen wurden. Dementsprechend wurde \distthresha auf den Wert 0,00025 und \distthreshb auf den Wert 0,00075 festgelegt. Um das Verzerrungsfaktor zu manipulieren, wurde die Zahl \textit{k} diskret von der Untergrenze \textit{0} bis zur Obergrenze \textit{0,15} mit einer Schrittgröße \textit{0,1} inkrementiert. Die Größe der Iteration~\textit{n} wurde ähnlich wie zuvor dynamisch mit einer Mindestzahl von 80 errechnet. Die Anzahl der wiederholten Scan-Linien \textit{k} betrugen wiederum mindestens 15. Zur Entfernung falscher Kanten wurde eine Region festgelegt, die maximal zwei Scan-Linien am Anfang oder am Ende einer Iteration umfasste. Nach Festlegung dieser Parameter wurde der Test ausgeführt und die Ergebnisse wurden in Tabelle \ref{table: point_distortion_results} aufgelistet.

\begin{table}[t]
	\centering
	\begin{tabular}[width=\textwidth]{l *{5}{c}}
		\hline
		\textbf{Ausführung} & \textbf{Verzerrungsfaktor \^{r}} & \textbf{p\textsubscript{dc}} & \textbf{p\textsubscript{mj}} & \textbf{p\textsubscript{dct}} & \textbf{p\textsubscript{mjt}} \\
		\hline
		1 & 0,1 & 1 & 0 & 1 & 0 \\
		2 & 0,2 & 1 & 0 & 1 & 0 \\
		3 & 0,3 & 1 & 0 & 1 & 0 \\
		4 & 0,4 & 1 & 0 & 1 & 0 \\
		5 & 0,5 & 1 & 0 & 1 & 0 \\
		6 & 0,6 & 1 & 0 & 1 & 0 \\
		7 & 0,7 & 1 & 0 & 1 & 0 \\
		8 & 0,8 & 1 & 0 & 1 & 0 \\
		9 & 0,9 & 1 & 0,2 & 0,92 & 0,25 \\
		10 & 1,0 & 1 & 0,8 & 0,67 & 0,17 \\
		11 & 1,1 & 1 & 1 & 0,6 & 0,35 \\
		12 & 1,2 & 1 & 1,3 & 0,53 & 0,43 \\
		13 & 1,3 & 1 & 1,6 & 0,39 & 0,34 \\
		14 & 1,4 & 1 & 1,5 & 0,4 & 0,36 \\
		15 & 1,5 & 0,8 & 1,4 & 0,42 & 0,33 \\
		\hline
	\end{tabular}
	\caption{Die Genauigkeit des Verfahrens gegen steigender Verzerrung der Punkte}
	\label{table: point_distortion_results}
\end{table}

Die Ausführungen \textit{1-8} lieferten perfekte Ergebnisse, wo das IEFD-Verfahren trotz steigender Verzerrung der Punkte alle zehn Kanten des Ground-Truths erkennen sowie vollständig segmentieren konnte. Als die Verzerrung der Punkte erhöht wurde, wurden neben allen Kanten auch weitere Einzelpunkte als Randpunkte erkannt. Diese Einzelpunkte \(Störpunkte\) lagen zufällig gestreut und es wurde weder eine linienartige Kante noch ein anderes geometrisches Muster visuell erkannt. Ab der neunten Ausführung konnten Spuren der falschen Kanten aus Abschnitt \ref{false_edges} erkannt werden. Neben den zehn Kanten des Ground-Truths waren zwei falsche Kanten leicht erkennbar, weswegen die neunte Ausführung eine \textit{p\textsubscript{dc}} und \textit{p\textsubscript{mj}} Bewertung von \textit{1,0} beziehungsweise \textit{0,2} erhalten hat. Aufgrund der Dürftigkeit der falschen Kanten konnte eine dieser Kanten nicht vollständig segmentiert wurden. Stattdessen wurden an der Stelle zwei Segmente erzeugt. Darüber hinaus wurde die Innenkante des Ground-Truths auch nicht vollständig segmentiert. An den Stellen, wo die falschen Kanten die Innenkante geschnitten haben, wurde ein neues Segment erzeugt. Somit erhielt diese Variante eine Bewertung von \textit{0,92} und \textit{0,25} für die Metriken \textit{p\textsubscript{dct}} beziehungsweise \textit{p\textsubscript{mjt}}. Ab einem Verzerrungsfaktor \textit{\^{r}} von \textit{1,0} der Verzerrung konnten mehr falschen Kanten zwischen jeder Iteration erkannt werden. Diese waren allerdings ähnlich dürftig und undeutlich definiert.  Dieser Trend der steigenden Verzerrung wurde bis zu der 13. Ausführung bemerkt, wo der Höchstpunkt erreicht wurde. Relativ zur früheren Ausführungen gab es die meisten Störpunkte und sowie die meisten falschen Kanten, die nicht entfernt wurden. Ab dieser Stelle wurde ein Plateau erreicht. Als die Verzerrung der Punkte erhöht wurde, wurden ungefähr die gleichen Anzahl an falschen Kanten erkannt. Gleichzeitig stiegen die Anzahl der Störpunkte, die durch die Kantenerkennung erkannt wurden. Trotzdem wurden alle Randelemente sowie die Innen- und Außenkante des Ground-Truths durch das Verfahren richtig erkannt. Aufgrund der hohen Anzahl an Störpunkte waren allerdings die Innen- und Außenkanten leicht dürftig. Insgesamt erzielten die Ausführungen 13 bis 15 einen \textit{p\textsubscript{dc}} Wert von 1.0, allerdings waren die Werte für \textit{p\textsubscript{mj}} deutlich höher. Die höhere Anzahl der falschen Kanten hat dazu geführt, dass die Randelemente in der y-Richtung teilweise nicht vollständig segmentiert wurden, sondern wurden bis zu zwei zusätzlichen Segmente erzeugt. Darüber hinaus hat die hohe Anzahl der Störpunkte dazu geführt, dass weder die Innenkante noch die Außenkante vollständig segmentiert wurde. Stattdessen wurde in jeder Iteration ein neues Segment gestartet. Dadurch erhielten beide Metriken \textit{p\textsubscript{dct}} und \textit{p\textsubscript{mjt}} nahezu die gleiche Werte von ca. 0,4 beziehungsweise 0,34. Abbildung \ref{fig: point_distortion_comparision} visualisiert den Trend der Metriken. 

\begin{figure}[h]
	\includegraphics[scale=0.85]{Abbildungen/point_distortion_influence_line.png}
	\centering
	\caption{Das Verhältnis zwischen der Genauigkeit des Verfahrens und dem Verzerrungsfaktor der Punktverzerrung.}
	\label{fig: point_distortion_comparision}
\end{figure}

Nach der Überprüfung der Genauigkeit bei einer Änderung der Punktverzerrung wurde die Robustheit des IEFD-Verfahrens gegen unterschiedlichen Geometrien überprüft. Dabei wurden reelle 3D-Aufnahmen von vier verschiedenen Bauteilen verwendet. Ähnlich wie zuvor wurden alle Parameter außer \distthresha und \distthreshb aus Abschnitt \ref{test_1} übernommen. Zur Festlegung dieser Parameter war noch einen Schritt nötig. Die Punktedichte der Aufnahmen war an manchen Stellen der Punktwolke deutlich höher als im Durchschnitt. Diese waren die Stellen, wo der Lasersensor im Vergleich zu anderen Stellen länger aufhielt, wodurch mehrere Scan-Linien an den Stellen generiert wurden. Abbildung \ref{fig: point_density_before} visualisiert diese Anomalie. 

\begin{figure}[t]
\includegraphics[width=1\textwidth]{Abbildungen/point_density_distribution_blech_before.png}
\centering
\caption{Die Anzahl der Nachbarpunkte jedes Punktes vor dem Downsampling im Umfang von einem Millimeter}
\label{fig: point_density_before}
\end{figure}

Diese lokalisierten Fällen der hohen Punktedichte hatten eine schlechte Auswirkung auf der Kantensegmentierung und führte dazu, dass diese Kanten nicht vollständig segmentiert wurden. Durch die Anwendung der \textit{UniformSampling} Methode konnte dieses Problem gelöst werden und eine einheitlichere Punktedichte sichergestellt werden. Es wurde für jede Aufnahme eine Blattgröße festgelegt, die den durchschnittlichen Punkteabstand nach dem Downsampling regelte. Somit hatte jeder Punkt ungefähr die gleichen Anzahl an Nachbar innerhalb eines bestimmten Radius. Dieses wird in Abbildung \ref{fig: point_density_after} dargestellt. Dieser Wert wurde für den Parameter \distthresha festgelegt. Der Parameter \distthreshb wurde in manchen Fällen höher gesetzt, da die Kantensegmentierung dabei bessere Ergebnisse lieferte.

\begin{figure}[t]
	\includegraphics[width=\textwidth]{Abbildungen/point_density_distribution_blech_after.png}
	\centering
	\caption{Die Anzahl der Nachbarpunkte jedes Punktes nach dem Downsampling}
	\label{fig: point_density_after}
\end{figure}

Nachdem die Parameter des Verfahrens festgelegt wurden, wurde jeder Scan drei Mal durch das Verfahren verarbeitet und die Ergebnisse davon wurden gemittelt, um den Einfluss von Ausreißer zu verringern. Diese Scans enthielten Bauteile mit kreisförmigen Löchern, Innenkanten, Nuten, Schlitze und weiteren geometrischen Merkmalen. Die Erkennung dieser Merkmale galt als erfolgreich, wenn die Kanten zu diesen Merkmalen erkannt wurden. Für jede Punktwolke wurden auch die die Anzahl der Punkte sowie die Anzahl der erkennbaren Kanten \textit{N\textsubscript{gc}} manuell bestimmt´. Drei der vier Punktwolken der Bauteile wiesen einen höheren durchschnittlichen Punkteabstand von ca. 0,00028\,\si{\m} in einem bestimmten Bereich am Ende des Scans auf. Die restlichen Punkte standen im Gegensatz ca. 0,00008\,\si{\m}m voneinander entfernt. Dieses war vor dem Downsampling der Fall. Aus diesem Grund wurde für diese drei Punktwolken eine \textit{Leaf-Size} von 0,0003m gewählt. Die Tabelle \ref{table: test_3-2_results} fasst diese Erkenntnisse und Ergebnisse zusammen. 

\begin{table}[t]
	\centering
	\begin{tabularx}{\textwidth}{l c >{\centering}X *{7}{c}}
		\hline
		\textbf{Id.} & \textbf{Punktezahl} & \textbf{Punkteabstand} & \textbf{d\textsubscript{t1}} & \textbf{d\textsubscript{t2}} & \textbf{N\textsubscript{gc}} & \textbf{p\textsubscript{dc}} & \textbf{p\textsubscript{mj}} & \textbf{p\textsubscript{dct}} & \textbf{p\textsubscript{mjt}}\\
		\hline
		1 & 2324097 & 0,0001 & 0,0001 & 0,00015 & 22 & 0,82 & 0 & 0,83 & 0,39 \\
		2 & 813997 & 0,0003 & 0,0003 & 0,0003 & 49 & 0,94 & 0 & 0,89 & 0,19 \\
		3 & 795356 & 0,0003 & 0,0003 & 0,00045 & 92 & 0,95 & 0 & 0,92 & 0,09 \\
		4 & 707337 & 0,0003 & 0,0003 & 0,0003 & 80 & 0,96 & 0,05 & 0,91 & 0,18 \\
		\hline
	\end{tabularx}
	\caption{Ergebnisse des Verfahrens mit vier reellen Punktwolken von Bauteilen}
	\label{table: test_3-2_results}
\end{table}

Im Vergleich zu den anderen drei wurde bei dem ersten Bauteil die wenigsten Kanten erkannt. Auch die Kantensegmentierung war für dieses Bauteil am wenigsten erfolgreich. Aus den 22 gezählten Kanten wurden nur 18 vollständig richtig erkennt. Darüber hinaus wurden nur 15 dieser 18 Kanten auch richtig segmentiert. Es wurden alle Ränder des Bauteils richtig erkannt sowie die kreisförmige Aussparung. Die Erkennung der Nuten lieferte im Gegensatz dazu mangelnde Ergebnisse. Die Qualität der Punktewolke verschlechterte sich in den Nuten und die Abbildung der Kanten war nicht sehr deutlich definiert. Die Auflösung hatte trotzdem gereicht, visuell die Kanten der Nuten zu erkennen. Aus den zwölf Kanten der Nuten konnten lediglich acht richtig erkannt werden. Die Segmentierung der Außenränder erfolgte auch sehr gut. Bis auf eine Kante, wurden alle vollständig segmentiert. Die Kantensegmentierung konnte zwischen zwei parallelen Kanten unterscheiden, die ca. einen Millimeter voneinander entfernt waren. Die Segmentierung der erkannten Kanten der Nuten erfolgte auch nur teilweise richtig. Während manche Kanten mit einer minderen Anzahl an lokalen Störungen vollständig segmentiert wurden, wurden Kanten mit einer höheren Anzahl an Störpunkte unvollständig segmentiert. Auf diesen Kanten wurden bis zu zwei weiteren Segmente erzeugt. Die kreisförmige Aussparung wurde auch unvollständig segmentiert. An der Stelle wurden drei zusätzliche Segmente erzeugt. 

Am Anfang der Punktwolken der restlichen drei Bauteile wurde eine Besonderheit erkannt. Das Anfangsstück des Bauteils wurde bereits einmal gescannt, bevor das gesamte Bauteil von Anfang an mit einer höheren Auflösung gescannt wurde. Dies resultierte in einem Überlappungsbereich, wo es Punktwolken mit zwei unterschiedlichen Auflösungen existierten. Dieses wird in Abbildung \ref{fig: scan_overlap} visualisiert.

\begin{figure}[h]
	\includegraphics[width=0.8\textwidth]{Abbildungen/scan_overlap.png}
	\centering
	\caption{Unterschiedliche Punktedichten der Bauteile zwei bis vier}
	\label{fig: scan_overlap}
\end{figure}

In der Punktwolke des zweiten Bauteils wurden 49 Kanten visuell erkannt. Die Kantenerkennung konnte 46 dieser Kanten vollständig erkennen. Besondere Schwierigkeiten hatte die Kantenerkennung bei der vollständigen Erkennung der drei kreisförmigen Löcher des Schweißtisches. Diese Kreise wurden nur teilweise korrekt als Halbkreise oder Teilkreise erkannt. Es wurden alle Ränder sowie die Innenkante des Bauteils richtig erkannt. Die Kantensegmentierung für das zweite Bauteil schnitt im Vergleich zu dem ersten Bauteil besser ab. Es wurden bis auf einem alle erkannten Ränder des Bauteils vollständig segmentiert. Die Innenkante des Bauteils wurde auch vollständig richtig segmentiert. Schwierigkeiten hatte die Kantensegmentierung wieder bei den Halb- und Teilkreisen. Auf diesen wurden insgesamt zwei bis vier Segmente erzeugt. Insgesamt konnte die Kantenerkennung 41 der 46 erkannten Kanten vollständig segmentieren, während es 9 zusätzliche falsche Segmente durch das Verfahren erzeugt wurden. Bei dem dritten Bauteil wurden ähnlich gute Ergebnisse bei der Kantenerkennung sowie Kantensegmentierung erzielt. Aus den 92 gezählten visuellen Kanten wurden 87 durch die Kantenerkennung erkannt. Schwierigkeiten hatte das Verfahren wiederum bei der vollständigen Erkennung der Löcher des Schweißtisches. Diese wurden ähnlich wie zuvor nur als Halb- oder Teilkreise erkannt. Die Ränder sowie die Innenkante des Bauteils wurden auch vollständig erkannt. Die Segmentierung dieses Bauteils hat auch einen ähnlichen Erfolgsquote wie zuvor. Aus den 87 erkannten Kanten wurden 80 vollständig segmentiert. Unter diesen Kanten zählten die Ränder des Bauteils. Die Stufen der stufenartigen Innenkante wurden auch überwiegend richtig voneinander unterschieden und richtig segmentiert. Diese Stufen konnten allerdings in dem Überlappungsbereich nicht mehr vollständig segmentiert werden. Die rechteckigen Aussparungen wurden auch durch das Verfahren überwiegend richtig und vollständig segmentiert. Bei dem letzten Bauteil mit ähnlichen geometrischen Merkmalen wie dem dritten wurde auch eine ähnliche Genauigkeit des Verfahrens beobachtet. Aus den 80 gezählten Kanten wurden 77 richtig erkannt. Dazu gehörten wiederum die Außenränder sowie die Innenkante des Bauteils. Gleichermaßen wurden die Löcher des Schweißtisches nicht vollständig erkannt. Die Segmentierung lieferte auch ein ähnlich gutes Ergebnis. Aus den 77 Kanten wurden 70 vollständig segmentiert. Hierunter zählten die Außenränder und eine überwiegenden Anzahl an Stufen der Innenkante. Die meisten rechteckigen Aussparungen wurden auch richtig segmentiert. Es wurde eine Längskante einer Aussparung detektiert, die nicht vollständig segmentiert wurde. Stattdessen wurden insgesamt zwei Segmente an der Stelle erzeugt. Die Kantenerkennung sowie Kantensegmentierung konnten die zwei Bereiche unterschiedlicher Punktedichten am Anfang der Punktwolken voneinander unterscheiden. Die Grenzen dieser Bereiche wurden richtig erkannt sowie vollständig segmentiert.

Die in diesem Kapitel entworfenen Untersuchungen der Genauigkeit des Verfahrens konnten die Leistung des Verfahrens unter diversen Bedingungen austesten. Aus diesen Untersuchungen konnten diverse Erkenntnisse gesammelt werden, die in dem folgenden Kapitel detaillierter behandelt werden.