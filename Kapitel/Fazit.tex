\chapter{Fazit}

Das Ziel dieser Arbeit war die Vorstellung und Evaluierung des IEFD-Verfahrens zur Erkennung geometrischer Merkmale in unvollständigen und wachsenden Punktwolken. Die Evaluierung des Verfahrens sollte ergeben, ob es für eine Anwendung unter reellen Bedingungen geeignet ist. Um eine möglichst ausgewogene Aussage über die Leistung und Genauigkeit des Verfahrens zu treffen, wurde das Verfahren hinsichtlich seiner Genauigkeit und Robustheit gründlich getestet. Dabei wurden verschiedene Testdateien verwendet und unterschiedliche reelle Störungen simuliert. 

Als Grundlage des IEFD-Verfahrens wurde das neuartige AGPN-Verfahren aus der Literatur implementiert, die unter Verwendung von kd-Bäumen, dem RANASC-Algorithmus sowie die Region-Growing-Segmentierung zu einer hohen Genauigkeit Kanten erkennen und segmentieren kann. Das AGPN-Verfahren wurde im Rahmen dieser Arbeit mit zusätzlichen Funktionalitäten bestattet, um Kanten korrekt und vollständig in iterativ wachsenden Punktwolken zu erkennen und zu segmentieren. Auch zusätzliche Korrekturmaßnahmen mussten implementiert werden, um eine Anomalie der Kantenerkennung zu kompensieren. Auch zur Verbesserung der Leistung und Rechenzeit wurden im Laufe dieser Arbeit wichtige Erkenntnisse gewonnen. Ausschlaggebend ist die siebenfache Leistungsverbesserung der Kantenerkennung, die durch eine Parallelisierung des Verfahrens erzielt werden kann.

Die Bestätigung der ersten Hypothese hat grundsätzlich unter Beweis gestellt, dass das Verfahren richtig funktioniert. Während jeder Randpunkt einer Punktwolke nicht korrekterweise markiert oder erkannt wird, kann das Verfahren ganzheitlich alle Kanten ausgeglichen und lückenlos erkennen sowie darstellen. Wichtiger ist die Erkenntnis, dass das IEFD-Verfahren unter Verwendung der gleichen Parameter wie das AGPN-Verfahren mehr Kanten richtig erkennen kann. Mit nahezu einer Genauigkeit von 100\% für die Kantenerkennung sowie die Kantensegmentierung kann die erste Teilforschungsfrage beantwortet werden. 

Es wurde auch nachgewiesen, dass die Genauigkeit des IEFD-Verfahrens nach keinem erkennbaren Trend mit dem Punkteabstand korreliert. Das Verfahren liefert trotz einem steigenden Punkteabstand eine sehr hohe Genauigkeit von 100\% für die Kantenerkennung sowie eine Genauigkeit von 90\% für die Kantensegmentierung. Einen wirkungsvolleren Einfluss hat möglicherweise die Anzahl der Punkte in der Punktwolke auf die Genauigkeit des IEFD-Verfahrens. Somit lässt sich die zweite Teilforschungsfrage beantworten: Der Punktabstand hat auf die Genauigkeit des Verfahrens keine beeinträchtigende Wirkung. 

Das IEFD-Verfahren ist gegen Verzerrungen innerhalb der Punktwolke sowie gegen andere reellen Bedingungen und Störfaktoren sehr Robust. Bei einer zufälligen Verzerrung der Punkte bis zu einer Amplitude von 1.0 liefert das IEFD-Verfahren für die Kantenerkennung sowie Segmentierung sehr gute Ergebnisse von 100\% beziehungsweise 92\%. Diese Ergebnisse können ohne Anpassung der Prozessparameter trotz einer Erhöhung des durchschnittlichen Punktabstands um ca. 1,5 erzielt werden. Die Kantenerkennung funktioniert für unterschiedlichen Geometrien auch sehr stabil und kann gerade sowie runde Kanten zu einer hohen Genauigkeit erkennen. Eine Schwachstelle des IEFD-Verfahrens präsentiert sich im Form von Übergangsbereiche, wo eine Kontur nahtlos von einer Seitenkante zur einer Innen- oder Außenkante übergeht. Diese sind für die Kantenerkennung besonders schwierig zu erkennen. Auch soll betont werden, dass die Anzahl der wiederholten Punkte in jeder Iteration einen unbestimmten Einfluss auf die Güte der Kantensegmentierung hat. Im Grunde lässt sich die Aussage zur Beantwortung der dritten Teilfrage treffen, dass das IEFD-Verfahren sehr Robust mit einer durchschnittlichen Genauigkeit von 92\% sowie 89\% Kanten unter reellen Bedingungen erkennen und segmentieren kann. 

Die Ergebnisse dieser Arbeit stellen unter Beweis, dass das IEFD-Verfahren zu eine gute Robustheit und eine hohe Genauigkeit aufweist, wodurch es für den Einsatz unter reellen Bedingungen eignet. Dieses Verfahren eignet sich insbesondere zur Erkennung der scharfen Konturen und Kanten von Bauteile in der Produktion und Verarbeitung. Das IEFD-Verfahren wird in dieser Arbeit in einer Prototypenphase vorgestellt und eignet sich in seinem aktuellen Stand nicht zur direkten Umsetzung in der Industrie oder für praktischen Probleme. 

Es besteht noch viel Potenzial, dieses Verfahren weiterhin zu überprüfen sowie weiterzuentwickeln. Mögliche Arbeiten könnten die Einflüsse der unterschiedlichen Prozessparameter auf die Genauigkeit des Verfahrens unter Betracht ziehen, sowie die Leistung des Verfahrens hinsichtlich der Performanz und Verarbeitungszeit auswerten. Das weitere Auslasten des Verfahrens mit diversen Aufnahmen reeller Objekte könnte die Grenzen des Verfahrens erproben und wichtige Erkenntnisse über die Leistung und Genauigkeit des Verfahrens liefern. Weitere Arbeiten sollen sich damit beschäftigen, die Performanz des Verfahrens zu optimieren, um seine Eignung für zeitkritischen Aufgaben zu verbessern. Eine weitere Möglichkeit zur Verbesserung der Genauigkeit besteht in der Nachbearbeitung der Segmente, um zwei kollineare Segmente zu vergleichen und zusammenzufügen, falls Sie auf der selben Kante liegen. Letztlich kann auf das IEFD-Verfahren aufgebaut werden, um beispielsweise Methoden zur Laufbahnplanung für Roboter auf Basis der erkannten Kanten zu implementieren. Dieser Arbeit stellt ein funktionsfähiges Konzept vor, welches in den Bereichen des maschinellen Sehens, der Robotik und Automatisierung große Implikationen für die Wissenschaft und Industrie haben kann.