%Datenstrukturen und Algorithmen
%	Was sind Algorithmen und Datenstrukturen
% 	Arten von Basis Datenstrukturen
%		Vector/Liste
%		Queue(Warteschlange)
%		Stack??
%		map und hash tabellen
%	besondere Datenstruktur - PCL Punktwolke
%	
%	Komplexität und Notation von Algorithmen
%	Arten von Basis Algorithmen
%		
% 		
%	besondere Algorithmen:
%		kd-tree
%		octree(vielleicht)
%		RANSAC
%		Segmentierung

\chapter{Theorie}
Im Kern dieser Arbeit steht eine Rechenaufgabe an. Die positionellen Informationen über Objekte und Bauteile müssen sinnvoll verarbeitet werden, um die Lage und Form der geometrischen Merkmale des Objektes zu bestimmen. Bei der Entwicklung eines allgemeinen Verfahrens zur Erkennung der geometrischen Merkmale in Abschnitt~\ref{Methodik} werden Algorithmen und Datenstrukturen verwendet. 

\section{Datenstrukturen} \label{Datenstrukturen}

Datenstrukturen dienen der Organisation und Speicherung von Daten sodass, die Beziehung zwischen einzelnen Elemente auch bewahrt wird. In einer Datenstruktur werden darüber hinaus auch Zugriffsmethoden für den Zugriff auf die gespeicherten Daten definiert sowie Angaben über Möglichkeiten zur Verarbeitung der Daten gemacht. Eine gute Datenstruktur setzt voraus, dass die Beziehung zwischen der Daten bewahrt und gut definiert wird sowie die Verarbeitung der Daten leicht gemacht wird. Eine Datenstruktur soll auch bestimmte Operationen auf die Daten ermöglichen, beispielsweise die Hinzufügung oder Entfernung von Datenpunkte, die Zusammenführung oder Sortierung der Daten sowie das Durchqueren der Datenstruktur nach bestimmten Daten. In der Informatik gibt es bereits etablierte Datenstrukturen, die sich nach unterschiedlichen Einsatzzwecken richten und eine sehr breite Anwendung finden. Diese lassen sich nach Abbildung~\ref{fig: datastructures} nach lineare und nichtlineare Datenstrukturen unterteilen. \autocite[1-2]{mohanty_data_2021}

\begin{figure}[h]
	\includegraphics[width=\textwidth]{Abbildungen/Datenstruktur_arten.png}
	\centering
	\caption[Datenstrukturarten]{Die unterschiedlichen Arten von linearen und nichtlinearen Datenstrukturen nach \textcite[2]{mohanty_data_2021}}
	\label{fig: datastructures}
\end{figure}

\subsection{Lineare Datenstrukturen}
Eine lineare Datenstruktur ist eine Datenstruktur, bei der die Elemente in einer sequentiellen Reihenfolge angeordnet sind. Dies bedeutet, dass jedes Element genau einen Vorgänger und einen Nachfolger hat, außer dem ersten und letzten Element. Lineare Datenstrukturen können als eine tabellarische oder eine verkettete Liste implementiert werden. \autocite[314-315]{hoffmann_einfuhrung_2011}

Die Operationen, die auf einer linearen Datenstruktur ausgeführt werden können, sind in der Regel das Initialisieren der Datenstruktur als leere Menge, das Einfügen eines Elements in die Datenstruktur und das Entfernen eines Elements aus der Datenstruktur. Es ist auch möglich, andere Operationen auszuführen, die nicht unbedingt auf der Ordnungsbeziehung zwischen den Elementen basieren. Unter diesen Operationen zählen beispielsweise das Suchen nach einem Element oder das Ersetzen eines Elements in der Datenstruktur. \autocite[314-315]{hoffmann_einfuhrung_2011}

Die Implementierung einer linearen Datenstruktur kann je nach Anforderungen und verfügbaren Ressourcen variieren. Es ist jedoch wichtig sicherzustellen, dass die Datenstruktur korrekt implementiert ist und dass alle Operationen den Zustand der Datenstruktur ordnungsgemäß ändern. \autocite[314-315]{hoffmann_einfuhrung_2011}

Unter den linearen Datenstrukturen finden hauptsächlich drei Datenstrukturen eine breite Anwendung, nämlich Felder (Arrays), Queues (Warteschlangen) und Stapel (Stacks).

\subsubsection{Felder}

Das Feld ist einer der einfachsten linearen Datenstrukturen. Ein Feld besteht aus mehreren Daten des gleichen Formats oder Datentyps, die je nach Implementierung in aufeinanderfolgenden Speicherorten gespeichert werden. Diese werden sequenziell hintereinander angeordnet und zusammen gespeichert. Elemente eines Feldes dürfen eindimensional oder auch mehrdimensional gespeichert werden, wodurch diese Datenstruktur mit Matrizen oder Vektoren aus der Mathematik verglichen werden kann. Die Größe oder Dimension des Feldes wird immer vorgegeben und bleibt in der Regle statisch. Jedes Element eines Feldes besitzt einen sogenannten Index, welcher auf die Position des Elements in dem Feld deutet. Im Falle eines zwei Dimensionalen Feldes besitzt jedes Element der Datenstruktur zwei Indizes. In der regel deutet der erste Index auf die Zeile und der zweite Index auf die Spalte des Datenfeldes hin, allerdings kann diese Regel abhängig von der Implementierung variieren. Felder dürfen eine beliebig Anzahl \textit{n} Dimensionen besitzen, allerdings steigt somit auch die Anzahl der Indizes aller Elemente. Datenfelder über vier Dimensionen können sogar räumlich nicht vorgestellt werden, jedoch stellt dies für einen Rechner keine Probleme dar. Zugriffsoperationen auf Elemente sowie Operationen zur Einfügung und Entfernung dieser Elemente verwenden ihre Indizes, um den Eintrag an einer bestimmten Position des Feldes aufzurufen oder zu manipulieren. \autocite[35-36]{ollmert_datenstrukturen_2020}

Eine andere Variante der Felder ist die sogenannte lineare Liste. Diese Liste unterscheidet sich von Felder, indem sie dynamisch initialisiert werden darf. Im Gegensatz zu der statischen Größe oder Dimension eines Feldes, darf die Größe einer linearen Liste beliebig geändert werden. Während ein Feld mit maximale Größe \textit{n} und \textit{n} Elemente nicht um ein weiteres Element \textit{k} erweitert werden darf, kann eine lineare Liste der gleichen Größe mit der gleichen Anzahl an Elementen um das \textit{k-te} Element erweitert werden. Das Element darf auch an einer beliebigen Stelle der Liste eingefügt werden. Die Reihenfolge beziehungsweise die Positionen der anderen Elemente werden automatisch angepasst. Dies könnte auch dazu führen, dass ein Element Z, welches zum Zeitpunkt t\textsubscript{1} vor der Einfügung eines neuen Elements einen bestimmten Index \textit{i} besaß, zu einem Zeitpunkt t\textsubscript{2} nach der Einfügung nicht mehr an der gleichen Position zu finden ist. Das gleiche kann auch durch das Entfernen von Elementen an beliebigen Positionen geschehen. Die Positionen und somit die Indizes aller Elemente dürfen auch geändert werden, indem sie Beispielsweise nach einem Kriterium sortiert werden. \autocite[40-42]{ollmert_datenstrukturen_2020}

Lineare Listen bieten viele ähnlichen Operationen wie Feldern an, um mit den gespeicherten Elementen zu interagieren. Dazu gehört das Abrufen eines bestimmten Elements, das Einfügen eines neuen Elements zwischen zwei benachbarten Elementen (sowie Spezialfälle für das Hinzufügen eines neuen ersten oder letzten Elements), das Entfernen eines bestimmten Elements, das Bestimmen der aktuellen Länge der Liste anhand der Elementanzahl, die Suche nach einem Element mit einem bestimmten Wert, das Zusammenführen von zwei linearen Listen und das Aufteilen einer Liste in zwei Teillisten.\autocite[42-43]{ollmert_datenstrukturen_2020}


Verkettete lineare Listen sind eine bestimmte Art der Implementierung von linearen Listen. Eine solche Kette unterscheidet sich grundsätzlich nach ihrem Aufbau und Speicherverfahren von gewöhnlichen Listen und Feldern. Die Elemente einer verketteten linearen Liste werden nicht aufeinanderfolgend gespeichert. Stattdessen wird mit jedem Element dieser Liste auch ein Zeiger beigegeben, der entweder die Speicheradresse des nächsten Elements angibt oder auf das Ende der Liste hinweist. Der Anker ist ein besonderer Zeiger, der auf den Anfang der Liste deutet. Die Struktur und Operationen einer linearen Liste ist in Abbildung~\ref{fig:linear_list} abgebildet. Bei der doppelt-verketteten linearen Listen werden mit jedem Element nicht nur einen Zeiger zu dem nächsten Element beigegeben, sondern auch ein Zeiger zu dem vorigen Element. Bei dieser Art der verketteten Liste werden zwei Anker verwendet - der Vorwärtsanker für den Anfang und der Rückwärtsanker für das Ende der Liste. Die Verarbeitung mancher linearen Listen stellt eine Herausforderung dar, insbesondere wenn sie gleichzeitig durch mehrere Programme verarbeitet werden. In solchen Fällen könnte es beispielsweise dazu führen, dass ein Programm auf ein bestimmtes Element mittels seines Index zugreift, während ein anderes Programm ein Element davor hinzufügt oder entfernt. \autocite[43-44]{ollmert_datenstrukturen_2020}

\begin{figure}[t]
	\includegraphics[width=\textwidth]{Abbildungen/Verkettete_lineare_liste.png}
	\centering
	\caption[Lineare Liste]{Die Struktur und Operationen einer linearen Liste. Die Großbuchstaben deuten auf die Elemente der Liste während die \textit{z}-Buchstaben die Zeiger repräsentieren. Die Kopfzelle steht in dieser Abbildung für den Anker.\autocite[611]{ernst_grundkurs_2020}}
	\label{fig:linear_list}
\end{figure}

Für alle gängigen Implementierungen von linearen Listen ist die Implementierung einer sequentiellen Zugriffsfunktion, die das nächste Element in der Liste ausgehend von einem bestimmten Element bereitstellt, einfach. Die Implementierung einer direkten Zugriffsfunktion, die das \textit{i}-te Element in der Liste bereitstellt, ist zwar ebenfalls einfach, aber die benötigte Zeit hängt von der Position des Elements in der Liste ab und nimmt mit zunehmender Länge der Liste zu.\autocite[45]{ollmert_datenstrukturen_2020}

\subsubsection{Queues}
Queues sind besondere sequentielle Datenstrukturen, die Daten nur in einer bestimmten Reihenfolge speichern. Dadurch wird auch die Entnahme und Einfügung der Daten geregelt. Queues folgen dem Prinzip nach \textit{First-In-First-Out} (FIFO), also werden die Elemente zur Verfügung gestellt, die zuerst zu der Datenstruktur hinzugefügt wurden. Elemente dürfen in der Regel nur am Ende der Queue eingefügt und vom Anfang der Queue entnommen werden. \autocite[371]{gumm_band_2016}

Das Prinzip nach \textit{FIFO} wird üblicherweise mittels verketteten linearen Listen ermöglicht. Der Anfang sowie das Ende der Queue werden mittels Zeiger gekennzeichnet. Jedes Element, welches zu der Queue hinzugefügt wird, erhält je nach Art der verketteten Liste einen Zeiger zu dem Element davor oder danach. Im Falle des ersten oder letzten Elements wird ihm ein besonderer Zeiger - der Nullzeiger - zugewiesen. Dieses deutet darauf hin, dass es keine weiteren Elemente zu finden sind. Der Zeiger \textit{anfang} deutet somit auf das vorderste und älteste Element der Queue während der Zeiger \textit{ende} auf das hinterste und neuste Element deutet. Sobald der erste Eintrag der Queue gelesen oder entnommen wird, wird der Zeiger \textit{anfang} inkrementiert, sodass es auf das nächste Element zeigt. Wenn ein neues Element zu der Queue hinzugefügt wird, wird der Zeiger \textit{ende} inkrementiert, sodass es auf den letzten Eintrag der Queue deutet. \autocite[48-49]{ollmert_datenstrukturen_2020} \autocite[371]{gumm_band_2016}

Queues bieten ähnliche Operationen wie Felder und Listen an. Im Gegensatz zu Listen oder Felder, wo Elemente mit einem bestimmten Funktionsaufruf an beliebige Positionen innerhalb der Datenstruktur platziert werden dürfen, dürfen Elemente nur am Ende einer Queue eingefügt werden. Abhängig von der Implementierung wird eine Methode bereitgestellt, die das Einfügen eines Elements ermöglicht. Allgemein wird das Einfügen einer Datei in einer Queue als \textit{enqueue} benannt. Felder und Liste ermöglichen mittels bestimmter Funktionen einen Zugriff auf beliebige Elemente innerhalb der Datenstruktur in einer beliebigen Reihenfolge. Bei Queues darf nur das erste Element aus der Datenstruktur entnommen werden, welches am längsten in der Datenstruktur enthalten war. Auch wird der Zugriff auf dieses Element abhängig von der Implementierung durch eine bestimmte Methode ermöglicht. Allgemein wird das entfernen oder auslesen eines Elements einer Queue als \textit{dequeue} bezeichnet. Die Anzahl der Elemente \textit{count} in einer Queue lässt sich bestimmen, indem die Differenz zwischen den Zeigern \textit{anfang} und \textit{ende} berechnet wird. In den meisten Implementierungen werden diese Schritte innerhalb einer Methode zur Bestimmung der Länge eingekapselt. Diese Operationen zusammen mit den zur Erstellung und Vernichtung von Queues sind die einzigen Methoden, die für diese Datenstruktur in der Regel bereitgestellt werden. \autocite[71-72]{hubwieser_fundamente_2015} \autocite[371]{gumm_band_2016}

Die Ausführung der Prozeduren \textit{enqueue} und \textit{dequeue} erfordert bei einer Queue einen konstanten Zeitaufwand, der unabhängig von der Anzahl der Elemente in der Datenstruktur ist. Die Suche nach einem Bestimmten Element in einer Queue erfordert jedoch einen Zeitaufwand, der linear von der Anzahl der aktuellen Elemente in der Datenstruktur abhängt. Zur Überprüfung, ob ein bestimmtes Element in einer Queue vorhanden ist, muss im Durchschnitt die Hälfte der Queue durchsucht werden. Die Quantifizierung des Zeitaufwands erfolgt in Abschnitt~\ref{o-notation}. \autocite[318]{hoffmann_einfuhrung_2011}

\begin{figure}[t]
	\includegraphics[width=\linewidth]{Abbildungen/Queue.png}
	\centering
	\caption[Eine Queue]{Eine Queue mit den Operationen zum Einfügen und Entfernen von Elementen dargestellt \autocite[371]{gumm_band_2016}}
	\label{fig: queue}
\end{figure}

\subsubsection{Stapel}
Die Datenstruktur des Stapels, auch als Stack bekannt, ist eine homogene, sequentielle Struktur, die nur das Einfügen und Lesen von Elementen am Anfang der Struktur erlaubt. Beim Lesen eines Elements wird dieses gleichzeitig entfernt, so dass das folgende Element an den Anfang rückt. Stapeln folgen das Prinzip nach \textit{First-In-Last-Out (FILO)} oder \textit{zuerst-rein-zuletzt-raus}. Die Anzahl der Speicherplätze des Stacks ist im theoretischen Sinne einseitig unbegrenzt, so dass der Stack dynamisch wachsen und schrumpfen kann. \autocite[614]{ernst_grundkurs_2020}

Stapel werden üblicherweise als verketteten linearen Listen implementiert, die nur in eine Richtung expandieren oder reduzieren dürfen. Somit erfolgt das Einfügen sowie Entnehmen von Elementen im Gegensatz zu Queues nur an einem Ende des Stapels. Beim Einfügen wird ein neues Element auf die vorhandenen Elemente des Stapels platziert und dem neuen Element wird ein Zeiger zum vorigen Element beigegeben. Im Falle des ersten Elements wird ihm ähnlich wie zuvor ein Nullzeiger zugewiesen, der auf das Ende des Stapels hinweist. Elemente des Stapels dürfen nur in der umgekehrten Reihenfolge ausgelesen werden, in der sie eingefügt wurden. Somit wird das allererste Element nur als allerletztes Element ausgelesen. \autocite[363]{gumm_band_2016}

Ähnlich wie Queues bieten Stapel im Vergleich zu Felder und Listen nur begrenzte Funktionalitäten an. Elemente eines Stapels dürfen nicht an beliebigen Stellen zwischen vorhandenen Elementen des Stapels platziert werden, sondern nur am \textit{Kopf} des Stapels. Dies wird durch die Operation \textit{push} ermöglicht, welches ein neues Element an der obersten Ebene platziert und den Zeiger für den Kopf nach oben nachrückt. Genau wie das Speichern darf ein Elementes nicht aus einer beliebigen Stelle des Stapels ausgelesen werden, sondern nur aus dem Kopf des Stapels. Die Operation \textit{pop} ist dafür zuständig, das erste beziehungsweise oberste Element des Stapels zu entfernen und für das Auslesen bereitzustellen. Gleichzeitig wird der Zeiger für den Kopf nach hinter gerückt, sodass es auf das vorige Element zeigt. Neben dieser Operationen werden auch zusätzliche Operationen bereitgestellt, die Stapel erzeugen, vernichten oder seine Größe bestimmen. Wie diese Operationen definiert werden, hängt allerdings von der Implementierung ab. \autocite[614]{ernst_grundkurs_2020} \autocite[45-46]{ollmert_datenstrukturen_2020}

Im Prinzip funktioniert ein Stapel ähnlich zu einer Queue, indem er das Auslesen oder Einfügen von Elementen nur an bestimmten Stellen erlaubt. \textit{Push} und \textit{pop} sind beide Operationen, die ähnlich wie \textit{enqueue} und \textit{dequeue} funktionieren und unabhängig von der Größe der Datenstruktur sind. Somit sollte der Zeitaufwand dieser Operationen auch konstant bleiben. Die Suche nach einem bestimmten Element in einem Stapel sollte ähnlich wie bei Queues direkt von der Anzahl der Elemente abhängen.

\begin{figure}[t]
	\includegraphics[width=\textwidth, height=0.25\textheight]{Abbildungen/Stack.png}
	\centering
	\caption[Stapel]{Ein Stapel mit den Operationen zur Einfügung und Entfernung von Elementen dargestellt \autocite[371]{gumm_band_2016}}
	\label{fig: stack}
	\hfill
\end{figure}

\subsection{Nichtlineare Datenstrukturen} \label{nicht_lineare_datenstrukturen}
Bei nichtlinearen Datenstrukturen stehen im Gegensatz zu linearen Datenstrukturen die einzelnen Elemente nicht in einer sequentiellen Reihenfolge zueinander \autocite[321]{hoffmann_einfuhrung_2011}. 

Die Funktionsweise sowie die Operationen nichtlinearer Datenstrukturen lassen sich nicht verallgemeinern, sondern können abhängig von der Implementierung abweichen. Unter nichtlinearen Datenstrukturen finden hauptsächlich vier Arten dieser eine breite Anwendung, nämlich Bäume (Tree), Graphen (Graph), Tabellen (Tables) und Mengen (Sets). Im Rahmen dieser Arbeit werden nur Bäume und Graphen behandelt.

\subsubsection{Bäume}
Bäume stellen eine grundlegende Datenstruktur in der Informatik dar, die eine zweidimensionale Verallgemeinerung von Listen darstellen. Im Gegensatz zu Listen erlauben Bäume die Speicherung von Daten sowie die relevanten Beziehungen zwischen diesen Daten, wie beispielsweise Ordnungs- oder hierarchische Beziehungen. Daher sind Bäume besonders gut geeignet, um gesuchte Daten schnell wiederzufinden. Ein Baum besteht aus einer Menge von Knoten, die miteinander durch Kanten verbunden sind. Wenn eine Kante von Knoten A zu Knoten B führt, wird dies als A$\rightarrow$B notiert und A wird als Vater von B oder B als Kind von A bezeichnet. Ein Knoten ohne Kinder wird als Blatt bezeichnet, während alle anderen Knoten als innere Knoten bezeichnet werden. \autocite[389]{gumm_band_2016}

Ein Pfad von A nach B führt als Folge von Knoten und Pfeilen von A nach B, wobei die Knoten durch Pfeile verbunden sind. Dieser Pfad wird als A $\rightarrow$ X1 $\rightarrow$ X2 $\rightarrow$ $\cdots$ $\rightarrow$ B notiert, wobei die Länge des Pfades durch die Anzahl der Knoten bestimmt wird. Diese Anzahl kann 0 oder mehr betragen. Wenn es einen Pfad von A nach B gibt, so gilt B als Nachkomme von A, während A als Vorfahre von B bezeichnet wird. \autocite[389]{gumm_band_2016}

Ein Baum erfüllt verschiedene Axiome, die ihn von anderen Datenstrukturen unterscheiden. So gibt es genau einen Knoten, der als Wurzel bezeichnet wird und keinen Vater hat. Jeder andere Knoten hat genau einen Vater und ist ein Nachkomme der Wurzel. Zudem darf es keine zyklischen Pfade im Baum geben, da dies zu Widersprüchen führen würde. Ein weiteres Axiom besagt, dass es von der Wurzel zu jedem anderen Knoten genau einen Pfad gibt. Das bedeutet, dass jeder Knoten eindeutig bestimmt werden kann und es keine mehrdeutigen Wege durch den Baum gibt. Schließlich bilden die Nachkommen eines beliebigen Knotens K zusammen mit allen ererbten Kanten einen zusätzlichen Baum mit K als Wurzel. Dieser Baum wird als Unterbaum mit Wurzel K bezeichnet und erfüllt ebenfalls alle Axiome eines Baumes. Aufgrund dieser Eigenschaften sind Bäume besonders nützlich, hierarchische Strukturen abzubilden und Daten effizient abzuspeichern und wiederzufinden. \autocite[389]{gumm_band_2016}

Aufgrund des hierarchischen Aufbaus und der Form dieser Datenstruktur besitzt sie eine besondere Eigenschaft - die Tiefe. Die Tiefe wird anhand der Anzahl der Ebenen eines Baums bestimmt. Zur Bestimmung der Tiefe eines Baums wird der kürzeste Pfad zu der tiefsten Ebene des Baums verfolgt und dabei die Anzahl der Knoten aufgezählt. Diese Anzahl repräsentiert die Tiefe eines Baums. \autocite[390]{gumm_band_2016}

Eine exemplarische Abbildung eines Baums ist in Abbildung~\ref{fig: tree} zu sehen.

\begin{figure}[t]
	\includegraphics[width=\linewidth]{Abbildungen/Tree.png}
	\centering
	\caption[Binärbaum]{Ein Baum mit einer Tiefe von Fünf sowie die Darstellung der Beziehungen zwischen einzelnen Knoten \autocite[390]{gumm_band_2016}}
	\label{fig: tree}
	\hfill
\end{figure}

Ein binär Baum ist die Bezeichnung einer spezifischen Implementierung dieser Datenstruktur. Sie dient als einer der wichtigsten dynamischen Datenstrukturen. Binäre Bäume ermöglichen das Einfügen und Entfernen von Elementen in der gleichen Geschwindigkeit wie Felder und einfachen linearen Listen. Darüber hinaus bieten Sie auch eine Möglichkeit an, die Suche nach bestimmten Elementen sowie das Durchsuchen innerhalb der Datenstruktur im Vergleich zu konventionellen sequentiellen Datenstrukturen zu beschleunigen. \autocite[617]{ernst_grundkurs_2020}

Alle Knoten eines binären Baums ausschließlich der Blätter haben genau zwei Nachkommen oder Kinder. Darüber hinaus darf ein Vorfahre nicht zwei leere, inhaltslose Kinder haben, sondern darf maximal nur ein Kind keinen Wert haben. Durch die Festlegung der ersten Bedingung wird erzielt, dass jeder Vorfahre immer zu zwei Unterbäumen führt. Die zweite Bedingung verhindert das Aufblasen eines Baums, indem keine unnötigen leere Unterbäume entstehen. Binäre Bäume sind für die Aufgabe des Suchens besonders geeignet. Sie dienen der effizienten binären (Ja/Nein) Auswertung von Aussagen. Bäume für diesen Zweck werden oft als binäre Suchbäume bezeichnet. Ein binärer Suchbaum ist genau so wie ein allgemeiner binärer Baum aufgebaut, indem jeder Knotenpunkt einen linken und rechten Unterbaum besitzt. Jeder Knoten enthält einen Schlüsselwert, der ihm zugeordnet ist. Eine wichtige Eigenschaft von binären Suchbäumen ist, dass der Schlüsselwert jeden Knotens größer als alle Schlüsselwerte im linken Unterbaum des Knotens ist und kleiner als alle Schlüsselwerte im rechten Unterbaum des Knotens ist. \autocite[94-95]{ollmert_datenstrukturen_2020}

Um nach einem Element in einem binären Suchbaum zu suchen, wird zuerst der Wert der Wurzel überprüft. Falls dieser nicht dem Element entspricht, werden die nächsten Nachkommen der Wurzel überprüft. Hierbei werden allerdings nicht beide Kinder der Wurzel mit dem Element verglichen, sondern der Wert des Elements zuerst mit dem der Wurzel verglichen. Ist das Element kleiner als die Wurzel, wird nur das linke Kind beziehungsweise Unterbaum weiter untersucht. Ist es hingegen größer, wird der rechte Unterbaum untersucht. Dieses wird rekursiv wiederholt, bis ein Konten mit dem Wert des Elements gefunden wird. Bei den meisten Implementierungen der Suchfunktion erfolgt das allerdings iterativ und nicht rekursiv. Dieser Vergleich wird solange wiederholt, bis das Element gefunden wird. \autocite[139-140]{knebl_algorithmen_2021}

Um ein Element in dem Suchbaum einzufügen erfolgen die folgenden Schritte. Mittels der Suchfunktion wird zuerst überprüft, ob das Element bereits in dem Suchbaum vorhanden ist. Im Falle des Vorhandenseins wird die Funktion abgebrochen. Ansonsten wird der Wert des Blattes aufgerufen, wo die Suchfunktion aufgehört hat. Dank der Funktionsweise eines binären Baums wird der Wert des neuen Elements ähnlich groß wie des Blattes sein. Das Blatt wird zu einem inneren Knoten eines Unterbaums gewandelt. Falls der Wert des neuen Elements kleiner als der des Blattes ist, wird es links im Unterbaum gespeichert. Ansonsten wird es rechts im Unterbaum gespeichert. \autocite[140]{knebl_algorithmen_2021}

Beim Löschen eines Elements wird der gleiche erst Schritt wie beim Einfügen verwendet. Mittels der Suchfunktion wird sichergestellt, dass das Element in dem Baum vorhanden ist. Falls das Element in einem Blatt gespeichert ist, erfolgt die Löschung sehr einfach, indem das Blatt vom Baum entfernt wird und der Vorfahre aktualisiert wird, somit dieser nicht auf das leere Blatt zeigt. Die Entfernung des Elements erfolgt auch relativ simpel im Falle, dass der Knoten \textit{v} des Elements nur einen Nachkommen hat. In diesem Fall wird der Vorfahre des Knotens \textit{v} geändert, sodass es auf den Nachkommen des Knotens \textit{v} zeigt. Danach wird der Knoten \textit{v} gelöscht. Problematischer wird die Löschung wenn der Knoten \textit{v} des Elements zwei Nachkommen besitzt. Hierfür wird in dem linken Unterbaum von \textit{v} möglichst tief nach dem größten Wert \textit{$\overline{e}$} gesucht. Dieser Wert wird möglichst tief und rechts liegen. Dieser Wert wird sowohl größer als andere Werte im linken Unterbaum als auch kleiner als \textit{v} sein. Dieser Wert darf maximal einen linken Nachkommen besitzen. Sobald \textit{$\overline{e}$} bestimmt wird, wird dieser Wert anstelle des Elements in \textit{v} eingefügt. Der linke Nachkomme von \textit{$\overline{e}$} wird somit der rechte Nachkomme von dem vorigen Vorfahre von \textit{$\overline{e}$}. Somit wird erzielt, das Element aus dem Baum zu löschen und die binäre Eigenschaft des Suchbaums aufrechtzuerhalten. Die Abbildung~\ref{fig: tree_delete} visualisiert dieses Verfahrens. \autocite[140-141]{knebl_algorithmen_2021}

\begin{figure}[t]
	\includegraphics[width=\linewidth]{Abbildungen/tree_delete.png}
	\centering
	\caption[Löschvorgang binären Baums]{Eine Veranschaulichung des Löschvorgangs in einem binären Suchbaum \autocite[140]{knebl_algorithmen_2021}}
	\label{fig: tree_delete}
\end{figure}

Neben dem Suchbaum gibt es auch andere Varianten des binären Baums, die unterschiedlichen Eigenschaften aufweisen und besondere Funktionalitäten anbieten. Unter diesen sind B-Bäume und AVL-Bäume einer der gängigsten Varianten, die häufig eine Anwendung finden. B-Bäume eignen sich insbesondere für das Speichern und effiziente Verwalten von sehr großen Datenmengen während AVL-Bäume, aufgrund ihrer bilanzierten Struktur, Suchfunktionen besonders schnell ausführen können. Die genauere Untersuchung der Struktur und Funktionsweise dieser Varianten liegt allerdings außerhalb des Umfangs dieser Arbeit \autocite[407-412]{gumm_band_2016}.

\subsubsection{Graphen}
Diese Datenstruktur findet ihre Herkunft in dem Teilgebiet der diskreten Mathematik in der sogenannten Graphentheorie. Graphen werden anschaulich als eine Menge von Knoten dargestellt, die durch Kanten miteinander verbunden sind. Sie umfassen eine sehr allgemeine Klasse von Datenstrukturen, welche andere Strukturen wie Bäume und lineare Listen als Teilmenge enthalten. In der Praxis haben Graphen eine große Bedeutung, da sich sehr viele statische und dynamische Strukturen der realen Welt darauf abbilden lassen. Straßenverbindungen, Kommunikations- und Rechnernetze, Flussdiagramme, Automaten, elektronische Schaltpläne sind Beispiele dafür. \autocite[215]{knebl_algorithmen_2021} \autocite[654]{ernst_grundkurs_2020}

Durch die Verbindung von Knoten mit Kanten in einem Graph werden die Relationen zwischen den Datenelementen dargestellt. Die Kanten verbinden jeweils zwei Knoten miteinander und können ungerichtet oder gerichtet sein. Ungerichtete Kanten stellen eine Verbindung zwischen zwei Knoten dar, ohne eine bestimmte Richtung anzugeben. Gerichtete Kanten hingegen haben eine bestimmte Richtung und stellen eine Verbindung von einem Knoten zu einem anderen dar. \autocite[221-222]{knebl_algorithmen_2021}

Für die Suche nach einem Datenelement innerhalb des Graphen müssen Graphen durchquert werden. Die Breitensuche und Tiefensuche sind zwei etablierte Verfahren, die dazu dienen. Bei der Breitensuche (Breadth-First Search, BFS) wird der Graph schichtenweise durchsucht. Zu Beginn wird ein Startknoten gewählt und seine direkten verbundenen benachbarten Knoten (Nachbarn) werden besucht. Anschließend werden die Nachbarn der Nachbarn sequentiell weiterhin besucht, bis alle Knoten besucht wurden oder ein bestimmter Zielknoten gefunden wurde. Das besondere an diesem Verfahren ist, dass ein Knoten nur dann besucht wird, wenn er noch nicht besucht wurde. Diese Tatsache verleiht diesem Verfahren seine besonders hohe Geschwindigkeit bei einem Suchvorgang. Um zu verhindern, dass der Algorithmus in einer Endlosschleife landet, werden die besuchten Knoten in einem Stapel gespeichert. \autocite[227-228]{knebl_algorithmen_2021} \autocite[666]{ernst_grundkurs_2020}

Bei der Tiefensuche (Depth-First Search, DFS) wird der Graph hingegen rekursiv durchsucht. Auch hier wird ein Startknoten gewählt, dessen erster direkter Nachbar besucht wird. Nachdem der erster Nachbar besucht wurde, wird dessen direkter Nachbarn wieder besucht. So werden sequentiell die Nachbarknoten von Nachbarknoten durchquert, bis ein Endknoten erreicht wird. Danach wird der vorherige Knoten wieder ausgewählt und sein nächster Nachbar besucht. Das Verfahren wird solange fortgesetzt, bis alle Knoten besucht wurden. Auch hier werden die besuchten Knoten markiert, um zu verhindern, dass der Algorithmus bereits besuchte Knoten wieder besucht und in einer Endlosschleife landet. Zur Aufspeicherung der bereits besuchten Knoten wird an der Stelle eine Queue verwendet. \autocite[231-232]{knebl_algorithmen_2021} \autocite[666]{ernst_grundkurs_2020}

\begin{figure}[!b]
	\centering
	\begin{subfigure}[h]{0.49\textwidth}
		\includegraphics[width=\linewidth]{Abbildungen/Breitensuche.png}
		\centering
		\caption[Breitensuche in Graphen]{Eine Visualisierung der Breitensuche nach \textcite[228]{knebl_algorithmen_2021}}
		\label{fig: breitensuche}
	\end{subfigure}
	\hfill
	\begin{subfigure}[h]{0.49\textwidth}
		\includegraphics[width=\linewidth]{Abbildungen/Tiefensuche.png}
		\centering
		\caption[Tiefensuche in Graphen]{Eine Visualisierung der Tiefensuche nach \textcite[232]{knebl_algorithmen_2021}}
		\label{fig: tiefensuche}
	\end{subfigure}
	\caption[Suchverfahren von Graphen]{Eine Visualisierung beider Suchverfahren für Graphen}
	\label{fig: graph_search_functions}
\end{figure}

Wie bereits erwähnt, Graphen eignen sich zur Speicherung von statischen und dynamischen Strukturen der reellen Welt. Mittels Algorithmen wie Dijkstras Algorithmus können gewichtete Graphen kostengünstig durchquert werden. Die Kanten eines gewichteten Graphen erhalten auf Basis der Kosten (Beispielsweise Zeit oder Streckenlänge) zur Durchquerung über der Kante eine Gewichtung. Die kürzesten oder schnellsten Pfade zwischen Knoten können somit gefunden werden. Andere Algorithmen wie Prims Algorithmus dienen der Ermittelung des minimalen Spannbaums eines Graphen. Ein Spannbaum könnte als ein Teilgraph \textit{T} definiert werden, der alle Knoten eines Graphen \textit{G} enthält, allerdings weniger Kanten. Trotzdem sind alle Knoten eines Spannbaums vollständig vernetzt. Ein minimaler Spannbaum ist ein Teilgraph eines gewichteten Graphen, dessen Summe der Gewichte weniger als alle andere möglichen Spannbäume ist. Prims Algorithmus lässt sich besonders gut für Routing-Probleme in lokalen Rechnernetze verwenden. \autocite[277-282]{hubwieser_fundamente_2015}

\subsection{Datenstruktur zur Speicherung mehrdimensionalen Daten}

Ein k-dimensionaler Baum, ein kd-Baum, ist eine geometrische Datenstruktur, die dazu dient, mehrdimensionale Punkte in einem effizientem binären Suchbaum zu organisieren. Dabei werden raumbezogene Suchanfragen bearbeitet, indem über Dimensionen abwechselnd gesucht wird. \autocite[92]{saha_advanced_2019} \autocite{bentley_fast_1978}

Der kd-Baum ist die gängigste räumliche Datenstruktur für die Durchführung von Bereichs- und nächsten Nachbarschafts-Suchanfragen. In jeder Ebene des Baums werden alle Kinder entlang einer bestimmten Dimension partitioniert. Danach wird in der nächsten Ebene des Baums die nächste Dimension der Datenstruktur verwendet. Zu Beginn des Baums werden alle Punkte basierend auf der ersten Dimension des Wurzelknotens partitioniert, sodass alle Punkte des rechten Teilbaums eine größere erste Dimension als die Wurzel haben. Im Umkehrschluss ist der Wert der ersten Dimension aller Punkte im linken Teilbaum kleiner. Jede Ebene des Baums partitioniert den Suchraum basierend auf der nächsten Dimension und kehrt periodisch zur ersten Dimension zurück, wenn die letzte verwendet wurde. Der effizienteste Weg zur Erstellung eines statischen KD-Baums besteht darin, eine Partitionierungsmethode zu verwenden, die den Medianwert der Raumpunkte als den Wurzelknoten verwendet. Dadurch wird es erzielt, die Struktur des kd-Baums möglichst symmetrisch zu gestalten. Auf Basis der Dimension werden Punkte entweder zum linken oder rechten Unterbaum des Wurzelknotens sortiert. Dieser Vorgang wird dann rekursiv auf die nächsten Teilbäume wiederholt, bis nur noch ein Element übrig bleibt. Dieses Element wird als Blatt des Baumes gespeichert. \autocite[92]{saha_advanced_2019}

Die Erstellung eines statischen kd-Baums für \textit{n} Raumpunkte in \textit{d} Dimensionen erfolgt nach den folgenden Schritten. Zuerst wird der Median der Randpunkte ermittelt und in dem Wurzelknoten des Baums platziert. Danach wird die erste Dimension der \textit{d} Dimensionen verwendet, um eine Hyperebene zu erstellen. Falls \textit{d} gleich zwei ist, wird eine Linie erstellt. Falls \textit{d} drei beträgt, wird eine zweidimensionale Ebene erstellt. Diese Hyperebene teilt die Raumpunkte gleichmäßig auf. Alle Punkte mit einer kleineren ersten Dimension befinden sich somit links zur Hyperebene und Punkte mit einer größeren ersten Dimensionen an der rechten Seite. Die linken und rechten Seiten der Hyperebenen werden in Teilbäumen aufgespeichert. Danach werden die Medianwerte der Punkte der jeweiligen Teilbäume ermittelt und auf Basis der nächsten Dimension wieder mittels Hyperebenen aufgeteilt. Diese Schritte werden rekursive für alle weitere Teilbäume wiederholt, wobei die Dimension zur Erstellung der Hyperebene für jede sukzessive Ebene zyklisch gewechselt wird. Bis zum Ende dieses Zyklus werden alle Punkte so partitioniert, dass am Ende des Baums nur Blätter vorhanden sind. Abbildung~\ref{fig: kd-tree_creation} bildet einen kd-Baum ab und visualisiert seine Erstellung. \autocite[93-94]{saha_advanced_2019} 

% Die Erstellung eines kd-Baums erfolgt mit einer Komplexität von $\mathcal{O}(n\log n)$, wobei \textit{n} die Anzahl der Punkte beträgt.

\begin{figure}[!b]
	\includegraphics[width=\textwidth]{Abbildungen/2d_kd-tree.png}
	\centering
	\caption[kd-Baum einer zweidimensionaler Datenstruktur]{Ein kd-Baum mit für eine Datenstruktur mit zwei Dimensionen sowie der Prozesses zur Erstellung dieses Baums \autocite[60]{garcia-garcia_alberto_towards_2015}}
	\label{fig: kd-tree_creation}
\end{figure}

Da das kd-Baum einen binären Baum als die grundlegende Datenstruktur verwendet, können alle Operationen eines allgemeinen binären Baums auf kd-Bäume angewendet werden. Mittels einer Suchoperation wird die Suche nach einem bestimmten Raumpunkt innerhalb der Baumstruktur ermöglicht. Bei der Suche nach einem bestimmten Punkt \textit{p} wird die erste Dimension mit der Wurzel verglichen. Falls der Punkt kleiner als der Median ist, wird im linken Teilbaum weitergesucht. Ansonsten gelangt das Verfahren zum rechten Teilbaum. Danach wird die zweite Dimension von \textit{p} mit der zweiten Dimension der Wurzel des Teilbaums verglichen, um den nächsten Nachkommen zu bestimmen. Diese Schritte werden rekursiv wiederholt, bis der Punkt \textit{p} gefunden wird. Das Einfügen sowie das Entfernen von Elementen aus einem kd-Baum erfolgt analog zu den allgemeinen Verfahren für binären Bäume aus Abschnitt~\ref{nicht_lineare_datenstrukturen}. \autocite[94]{saha_advanced_2019}

Eine besondere Eigenschaft des kd-Baums ist die Möglichkeit schnell und effizient nach einer bestimmten Anzahl von Nachbarpunkte eines Raumpunktes zu suchen. Dies erfolgt nach zwei Verfahren - die Bereichssuche und die Suche der nächsten Nachbarn. Gewöhnlich wird die Bereichssuche von kd-Bäume eingesetzt, um alle Punkte innerhalb eines vorbestimmten sphärischen Radius \textit{R} zu suchen. Hierfür wird ein Stapel verwendet. Zuerst wird der Mittelpunkt \textit{m} des sphärischen Bereiches festgelegt, der häufig ein bereits vorhandener Raumpunkt des kd-Baums ist. Die Suchfunktion des Baums beginnt bei dem Wurzel und quert durch die Unterbäume durch. Dabei werden die Dimensionen des angefragten Punktes abwechselnd mit dem Knoten verglichen, um den Pfad zum nächsten Unterbaum zu bestimmen. Bei dem Suchverlauf werden alle durchquerten Knoten sowie deren Nachkommen zu dem Stapel hinzugefügt. Wenn ein Blatt des Baumes erreicht wird, werden die eingespeicherten Punkte in dem Stapel der Reihe nach ausgelesen und die Abstände zwischen dem Punkt und \textit{m} bestimmt. Liegt der Abstand eindeutig unter dem Wert \textit{R}, wird der Punkt akzeptiert und abgespeichert. Befindet sich ein Punkt auf den Umfang der Sphäre, werden die Abstände seiner Kinder zu \textit{m} bestimmt. Dabei werden alle Kinder ausgeschlossen, die außerhalb des Radius \textit{R} liegen. Abhängig von der Implementierung ergibt sich auch die Möglichkeit, die Suche auf die ersten \textit{n} Punkte zu beschränken. \autocite[95]{saha_advanced_2019}

Bei einer Suche nach dem nächsten Nachbar in einem kd-Baum wird ähnlich wie bei einer Bereichssuche vorgegangen, nur dass der Radius \textit{R} als aktuelles Minimum und Ablehnungskriterium anstatt als akzeptierte Bedingung behandelt wird. Die Suche nach dem nächsten Nachbar wird verwendet, um den nächsten Punkt in unmittelbarer Nähe zu einem bestimmten Punkt \textit{p} zu finden. Auch für diese Suche wird ein Stapel verwendet. Mit der Suchfunktion wird ein binärer Vergleich zwischen dem Wert von \textit{p} und einem Knoten (am Anfang die Wurzel) ausgeführt und abhängig davon durch den linken oder rechten Teilbaum durchquert. Dabei werden alle besuchten Knoten auf dem Suchpfad und deren Kinder zu dem Stack so lange eingefügt, bis ein Blatt des Baums erreicht wird. Die Entfernung dieses Blatts zu dem Abfragepunkt \textit{p} wird ermittelt und als aktuelles Minimum \textit{R} betrachtet. Danach werden die Punkte aus dem Stapel sukzessiv ausgelesen und der Abstand zwischen ihnen und \textit{p} berechnet. Ist dieser Abstand kleiner als \textit{R}, wird der Wert von \textit{R} aktualisiert, um den neuen kürzesten Abstand zu entsprechen. Wenn der Stapel leer ist, wird der Punkt mit dem kürzesten Abstand zu \textit{p} als der nächste Nachbar zurückgegeben. Abhängig von der Implementierung wird es auch ermöglicht, die ersten \textit{k} nächsten Nachbarn von \textit{p} zu bestimmen. \autocite[96]{saha_advanced_2019}

Da kd-Bäume in der Regel sehr hohen Datenmengen mit einer hohen Dimensionalität speichern, ist es von hohem Wert, wenn die Laufzeiten der einzelnen Operationen eines kd-Bäums geschätzt werden könnten. Hierfür wird die O-Notation verwendet, die in Abschnitt~\ref{o-notation} detaillierter behandelt wird. Die Zeitkomplexität des Initialisierungs- und Erstellungsverfahrens beträgt $\mathcal{O}(n\log n)$. Die Suchfunktionen für einen bestimmten Punkt in dem Baum hat im Gegensatz dazu eine niedrigere Zeitkomplexität von $\mathcal{O}(\log n)$. Das Einfügen und Entfernen eines Elements sowie die Bereichssuche und Suche nach nächsten Nachbarn hat normalerweise auch die gleiche Zeitkomplexität. Allerdings sei es mit einer sehr hohen Anzahl an Dimensionen sowie einem unsymmetrischen  kd-Baum möglich, dass die Zeitkomplexität der Bereichssuche sowie der Suche des nächsten Nachbars auf $\mathcal{O}(k \cdot n^{1-\frac{1}{k}})$ steigt. \autocite[104-105]{bentley_fast_1978} \autocite[94-96]{saha_advanced_2019}

\section{Algorithmen}
Ein Algorithmus ist ein Verfahren, welcher zur Bestimmung einer oder mehrerer Lösungen eines bestimmten Rechenproblems verwendet wird \autocite[1]{knebl_algorithmen_2021}. In einem Algorithmus wird ein Lösungsansatz möglichst präzise ausformuliert, indem kleine, isolierte und klar definierte Verarbeitungsschritte definiert werden. Auch ein simples Verfahren zur Summierung zweier Zahlen lässt sich als ein Algorithmus nennen. \autocite[9-10]{hubwieser_fundamente_2015}

Gewisse strukturelle Gemeinsamkeiten sind bei allen Algorithmen vorhanden, auch wenn sie verschiedene Darstellungsarten haben können. Laut \textcite[12-14]{hubwieser_fundamente_2015} können Algorithmen durch elementare Verarbeitungsschritten sowie Sequenzen, bedingte Verarbeitungsschritten und Wiederholungen von elementaren Verarbeitungsschritten beschrieben werden. Diese Bausteine werden auch als Strukturelemente von Algorithmen bezeichnet.

Es gibt simple, nicht-zusammengesetzte  Verarbeitungsschritte, die zwingend ausgeführt werden müssen. Zusätzlich zu diesen elementaren Verarbeitungsschritten sind noch drei Arten von zusammengesetzten Verarbeitungsschritten erforderlich, um Abläufe zu beschreiben: Sequenzen, bedingte Verarbeitungsschritte und Wiederholungen. 

Elementare Verarbeitungsschritte können zu Sequenzen zusammengefasst werden, um hintereinander auszuführende Schritte darzustellen. Jeder Schritt übernimmt dabei das Ergebnis seines Vorgängers. Zur Trennung der einzelnen Komponenten einer Sequenz wird ein festes Trennzeichen festgelegt, wie beispielsweise ein Strichpunkt oder ein Zeilenwechsel. In manchen Programmiersprachen können Sequenzen auch mit Begrenzungssymbolen (beispielsweise geschweiften Klammern) zu einem Block zusammengefasst werden.

Bei einigen Verarbeitungsschritten ist es notwendig, dass sie nur unter bestimmten Bedingungen ausgeführt werden. In solchen Fällen kann es auch erforderlich sein, alternative Verarbeitungswege festzulegen, die ausgeführt werden, wenn die Bedingung nicht erfüllt ist. Diese Art eines Verarbeitungsschrittes wird als ein bedingter Verarbeitungsschritt bezeichnet. Bei bedingten Verarbeitungsschritten dürfen sowohl elementare als auch zusammengesetzte Verarbeitungsschritte ausgeführt werden.

Es ist häufig notwendig, dass Sequenzen von Verarbeitungsschritten mehrfach ausgeführt werden. Dabei werden bestimmte Bedingungen definiert, um die Anzahl der Wiederholungen zu regeln. Es ist möglich, sowohl elementare Verarbeitungsschritte als auch zusammengesetzte Verarbeitungsschritte zu wiederholen. Dabei gibt es einen Unterschied zwischen zwei Arten von Wiederholungen: solchen mit einer vorgegebenen Anzahl von Wiederholungen und solchen mit einer Bedingung, die zu Beginn oder am Ende der Wiederholung geprüft wird. Die letztere Art ist sicherer, während die erstere flexibler ist und eine größere Vielfalt von Funktionen ermöglicht.

\subsection{Bestimmung der Komplexität eines Algorithmus} \label{o-notation}
Die asymptotische Analyse ist eine Disziplin der Informatik, die sich mit der Bestimmung des Aufwands von Algorithmen zur Lösung von Problemen beschäftigt. Dabei werden insbesondere die Zeitkomplexität und die Speicherkomplexität betrachtet. In diesem Zusammenhang soll im Folgenden nur auf die zeitliche Komplexität eingegangen werden, da der verfügbare Speicherplatz in der heutigen Zeit üblicherweise ausreichend ist und zudem immer preisgünstiger wird. Dennoch lassen sich die angewandten Analyse-Techniken auch auf die Untersuchung des Speicherbedarfs übertragen. \autocite[201]{hubwieser_fundamente_2015}

Eine Möglichkeit, die Laufzeit eines Algorithmus zu bestimmen, wäre die Durchführung von Tests mit verschiedenen Eingabegrößen. Dieses Vorgehen ist jedoch häufig unpraktikabel, da manche Algorithmen für größere Eingaben so lange laufen, dass eine Messung nicht durchführbar ist. Zudem würde die zum Testen verwendete Hardware das Ergebnis beeinflussen, was für eine aussagekräftige Effizienzbewertung unerwünscht ist.\autocite[201]{hubwieser_fundamente_2015}

Aus diesen Gründen werden in der asymptotischen Analyse Techniken verwendet, die unabhängig von der Hardware sind und auf mathematischen Überlegungen basieren. Dabei wird der Aufwand eines Algorithmus in Abhängigkeit von der Größe der Eingabe analysiert. Das Ergebnis dieser Analyse sind sogenannte Laufzeitfunktionen, die das Wachstum des Algorithmus in Bezug auf die Größe der Eingabe beschreiben. Durch die Verwendung von Notationsmethoden wie der O-Notation können die Laufzeitfunktionen kompakt und übersichtlich dargestellt werden. Dies ermöglicht es, Algorithmen auf ihre Effizienz hin zu vergleichen und geeignete Algorithmen für spezifische Problemstellungen auszuwählen.\autocite[201-203]{hubwieser_fundamente_2015}

Es hat sich bei der asymptotischen Analyse etabliert, drei verschiedenen Maße für die Zeitkomplexität zu verwenden. Bei dem besten Fall (englisch: best case) T\textsubscript{best} wird der Fall des Algorithmus betrachtet, der am schnellsten gelaufen ist. Bei dem schlimmsten Fall (englisch: worst case) T\textsubscript{worst} wird die langsamste Berechnungsdauer eines Algorithmus in Betracht gezogen. Um die durchschnittliche Dauer der Berechnung anzugeben, wird der Durchschnittsfall (englisch: average case) T\textsubscript{avg} verwendet. Der beste Fall eines Algorithmus ist in den meisten Fällen irrelevant, da Programme meistens gegen den schlimmsten Fall gesichert werden müssen. Hier sind die Maße T\textsubscript{worst} oder T\textsubscript{avg} von höherer Bedeutung. \autocite[202]{hubwieser_fundamente_2015}

\begin{table}[t]
	\centering
	\begin{tabular}{{l}{r}}
		\hline
		\textbf{Notation} & \textbf{Sprechweise} \\
		\hline
		$\mathcal{O}(1)$ & konstant \\
		$\mathcal{O}(\log (n))$ & logarithmisch \\
		$\mathcal{O}(n)$ & linear \\
		$\mathcal{O}(n\log (n))$ & log-linear \\
		$\mathcal{O}(n^2)$ & quadratisch \\
		$\mathcal{O}(n^k, k > 2)$ & polynomial \\
		$\mathcal{O}(k^n), k \geq 2$ & exponentiell \\
		\hline
	\end{tabular}
	\caption[Notationen der Zeitkomplexität]{Die gängigen Varianten der O-Notation, sortiert nach steigender Zeitkomplexität \autocite[205]{hubwieser_fundamente_2015}}.
	\label{table: o-notation}
\end{table}

Die Angabe der Zeitkomplexität mittels Einheiten der Zeit wie Sekunden lässt das Treffen einer allgemeinen Aussage über die Effizienz des Algorithmus nicht zu. Vielmehr hilft die Modellierung der Zeitkomplexität als eine Funktion der Eingabegröße, da sie eine Abstrahierung der Effizienz ohne den Einfluss von Faktoren wie Hardware ermöglicht. Zur Angabe der Zeitkomplexität wird am gängigsten das Landau-Symbol oder die \textit{O-Notation} verwendet. Das Wachstumsverhalten der benötigten Berechnungsdauer eines Algorithmus wird gegen eine unendliche große Eingabegröße aufgezeichnet, indem die Eingabegröße möglichst hohe Werte annimmt. Zeigt die Zeitkomplexität eine polynomialen Trend, wird zur Darstellung der Zeitkomplexität der höchste Grad der Funktion verwendet. Konstante werden bei der O-Notation auch nie berücksichtigt. Der Verlauf der Zeitkomplexität gegen steigender Eingabegröße kann auch einen logarithmischen Trend aufweisen. Die Tabelle~\ref{table: o-notation} fasst alle gängige Komplexitätsmaße zusammen, die häufig verwendet werden. Abbildung~\ref{fig: time_complexity} vergleicht die Zeitkomplexitätsmaße graphisch miteinander. \autocite[203]{hubwieser_fundamente_2015}

\begin{figure}[t]
	\centering
	\includegraphics[scale=0.8]{Abbildungen/Time_complexities_lower.png}
	\caption[Zeitkomplexität von Algorithmen]{Visualisierung der unterschiedlichen Größen der Zeitkomplexität}
	\label{fig: time_complexity}
\end{figure}

\subsection{Grundlegende Algorithmen}
Viele alltäglichen Rechenaufgaben haben einen häufigen Bedarf an Algorithmen, die bestimmte Aufgaben ausführen. Hierunter zählen das Sortieren von Elementen oder das Suchen nach bestimmten Elementen innerhalb einer Datenstruktur. Im folgenden werden Algorithmen diskutiert, die standardisierte und effiziente Methoden zur Ausführung dieser Aufgaben anbieten.

\subsubsection{Sortieren}
Sortieren wird als das Arrangieren der Elemente oder einer Teilmenge der Elemente einer Datenstruktur in einer bestimmten Reihenfolge verstanden. Sortieren findet häufig als eine Vorverarbeitungsmaßnahme für komplexere Algorithmen eine Anwendung, da sie sich sowohl menschlich als auch maschinell deutlich schneller verarbeiten lässt. Die Suche nach einem bestimmten Element sowie bestimmten Elementen innerhalb eines Bereiches wird durch die Sortierung effizienter gestaltet. Auch die Gruppierung gleicher oder ähnlicher Elemente in einer Datenstruktur erfolgt deutlich schneller, da diese durch eine Sortierung näher aneinander gebracht werden. \autocite[153-154]{sanders_sequential_2019}

Die einfachsten Verfahren zur Sortierung linearer Datenstrukturen erfolgen nach dem Prinzip des Auswählens und Einfügens (Insertion). Bei der Sortierung durch Auswählen wird in jeder Iteration das kleinste oder größte Element ausgelesen, entfernt und am Ende des Ausgabefeldes eingefügt. Am Ende stehen die Elemente in einer aufsteigenden beziehungsweise absteigenden Reihenfolge in dem Ausgabefeld. Die Elemente dürfen auch rekursiv innerhalb des Eingabefelds ohne den Bedarf an einem zusätzlichen Ausgabefeld sortiert werden. Konsequenterweise ist diese rekursive Sortierung besonders effizient bei der Speicherverwendung, da die Feldgröße konstant bleibt. Die Zeitkomplexität bei dieser Art der Sortierung lässt sich anhand der Anzahl der Elemente bestimmen. Da diese Variante der Sortierung durch \textit{n} Elemente des Felds iteriert, wird dies genau \textit{n} mal ausgeführt. In jedem Durchlauf der Schleife werden alle verbliebenen Elemente des Eingabefeld nochmal durchquert, um den minimalen oder maximalen Wert zu bestimmen. Dies resultiert in zusätzlichen Kosten von $n\cdot c$ für jeden Durchlauf. Insgesamt hat die Sortierung durch Auswählen eine Zeitkomplexität von $\mathcal{O}(n^2)$, wenn die Konstanten fallen gelassen werden. \autocite[156]{sanders_sequential_2019} \autocite[212]{hubwieser_fundamente_2015}

Bei der Sortierung durch das Einfügen werden die Elemente aus dem Eingabefeld der Reihe nach ausgelesen und, basierend  auf ihren Werten, in ein neues Ausgabefeld eingefügt. Beim Einfügen wird der Wert des ausgelesenen Elements mit den Werten der bereits sortierten Elemente in dem Ausgabefeld verglichen. Abhängig von dem Kriterium der Sortierung wird das ausgelesene Element zwischen zwei Elemente eingefügt, die jeweils weniger oder mehr betragen. Im Falle des ersten Elements wird es am Anfang des Feldes platziert. Eine Sortierung durch Einfügen ist auch ohne die Verwendung eines Ausgabefelds möglich. Dabei werden die unsortierten Elemente um eine Stelle nach hinten gerückt, um Platz für das ausgelesene Element zu schaffen. Danach wird der Wert des ausgelesenen Elements mit den der bereits sortierten Werte am Anfang des Eingabefelds verglichen, um seine neue Position zu bestimmen. Auch bei dieser Variante der Sortierung werden für \textit{n} Elemente genau \textit{n} Iterationen ausgeführt. Beim Einfügen eines Elementes werden zur Bestimmung seiner neuen Position alle bereits sortierten Elemente durchquert. Dieser Wertvergleich zusammen mit der Einrückung der unsortierten Punkten führt dazu, dass höchstens \textit{$n \cdot c$} Elemente in jeder Iteration durchquert werden. Somit ergibt sich eine Zeitkomplexität für die Sortierung durch Einfügen von $\mathcal{O}(n^2)$. \autocite[157]{sanders_sequential_2019} \autocite[210-211]{hubwieser_fundamente_2015}

Zur Sortierung eines Feldes bietet sich eine andere Möglichkeit in Form des Quicksorts (deutsch: schnelle Sortierung) an. Bei diesem Algorithmus wird die Strategie des \enquote{Teilen und Herrschens} (englisch: Divide-and-Conquer) angewendet. Im Grunde wird das zu sortierende Feld rekursiv aufgeteilt, um die Sortierung durch die Lösung kleinerer Teilprobleme auszuführen. Diese Teilprobleme können möglicherweise auch rekursiv durch das Quicksort-Verfahren geteilt werden. Schließlich werden alle Teillösungen zur Lösung des Gesamtproblems zusammengeführt, um ein sortiertes Ausgabefeld zurückzugeben. Zur Aufteilung eines Feldes \textit{F} wird ein zufälliges Pivotelement \textit{x} aus der Datenstruktur gewählt. Alle Elemente mit einem kleineren Wert als \textit{x} werden zum ersten Teilfeld \textit{F\textsubscript{1}} hinzugefügt, während die restlichen Elemente zum zweiten Teilfeld \textit{F\textsubscript{2}} hinzugefügt werden. Danach wird das Quicksort Verfahren rekursiv auf die Teilfelder angewandt, um weitere Teilfelder zu erzeugen. In jedem Teilfeld sind alle Werte entweder größer oder kleiner als das Pivotelement. Durch die rekursive Zerlegung der Teilfelder werden die Teilprobleme immer kleiner, bis jedes Teilproblem nur ein Element enthält. Dank der Teilsortierung in jeder Rekursion stehen auch alle Teilprobleme am Ende sortiert und können zusammengefügt werden. Diese Schritte werden anhand des Beispiels einer Zahlenfolge in Tabelle~\ref{table: quick_sort} verdeutlicht. \autocite[76-77]{knebl_algorithmen_2021}

Zur Bestimmung der Zeitkomplexität des Quicksort-Verfahrens hilft es, die einzelnen Teilprobleme in einer Baumstruktur sich vorzustellen. Jeder Knoten des Baums besitzt maximal zwei Kinder aber mindestens ein Kind. Ausgeschlossen von dieser Verallgemeinerung sind die Blätter des Baums. Unter Berücksichtigung dieser Eigenschaft kann die Höhe des Baums mit $|\log_{2}(n)|$ bestimmt werden, wobei \textit{n} die Anzahl der Elemente beträgt. In jeder Ebene des Baums finden bis zu \textit{n} Vergleiche mit dem Pivotelement statt. Somit ergibt sich eine Zeitkomplexität von $\mathcal{O}(n\log_{2}(n))$, die sich als $\mathcal{O}(n\log(n))$ verallgemeinern lässt. Für den schlimmsten Fall, wo jedes Teilproblem nur ein Kind hat, beträgt die Zeitkomplexität \textit{T\textsubscript{worst}} $\mathcal{O}(n^2)$. \autocite[216-217]{hubwieser_fundamente_2015} \autocite[79]{knebl_algorithmen_2021} \autocite[169-170]{sanders_sequential_2019}

\begin{table}[t]
	\centering
	\begin{tabular}{l *{8}{c}}
		\hline
		Array: & 67 & 56 & 10 & 41 & 95 & 18 & 6 & \textit{\underline{42}} \\
		Pivotelement: & & & & & \textit{\underline{42}} & & & \\
		Aufteilung: & 6 & 18 & 10 & \textit{\underline{41}} & & 56 & 67 & \textit{\underline{95}} \\
		Pivotelemente: & & & & \textit{\underline{41}} & & & & \textit{\underline{95}} \\
		Aufteilung:  & 6 & 18 & \textit{\underline{10}} & & & 56 & \textit{\underline{67}} & \\
		Pivotelemente: & & \textit{\underline{10}}  & & & & & \textit{\underline{67}} & \\
		Aufteilung: & 6 & & 18 & & & 56 & & \\
		Sortiertes Array: & 6 & 10 & 18 & 41 & 42 & 56 & 67 & 95 \\
		\hline
	\end{tabular}
	\caption[Beispiel eines Sortieralgorithmus]{Die schrittweise Sortierung eines Feldes mit dem Quicksort Algorithmus}
	\label{table: quick_sort}
\end{table} 

\Textcite[213-214]{hubwieser_fundamente_2015} und \textcite[582-585]{ernst_grundkurs_2020} beschreiben einen weiteren Algorithmus namens Bubblesort. Allerdings wird im Umfang dieser Arbeit dieses Algorithmus nicht detaillierter behandelt, da es zugunsten einer niedrigeren Zeitkomplexität eine inferiore Korrektheit anbietet. Sortieralgorithmen wie Heapsort oder Shellsort können auch laut den Autoren eine hohe Korrektheit mit einer ähnlichen oder niedrigeren Zeitkomplexität leisten, indem sie besondere Datenstrukturen verwenden. Die ausführliche Behandlung dieser Varianten des Suchalgorithmus liegt nicht innerhalb des Rahmens dieser Arbeit. 

\subsubsection{Suchen}
Beim Suchen handelt es sich um das Finden eines bestimmten Elements innerhalb einer Datenstruktur. Hierzu wurden bereits ein paar Verfahren in Abschnitt~\ref{Datenstrukturen} erwähnt. In diesem Abschnitt werden sie zusammengefasst und ihre Zeitkomplexität ermittelt. 

Die sequentielle Suche ist das einfachste Verfahren, das zum Finden eines Elements in linearen Datenstrukturen verwendet werden kann. Bei der sequentiellen Suche wird durch die lineare Datenstruktur iteriert und der Wert jedes Elements mit dem des angefragten Elements verglichen. Wird das Element gefunden, liefert das Verfahren eine positive Rückmeldung. Im konversen Fall gelangt die Suche zum Ende der Datenstruktur und findet das angefragte Element nicht. In diesem Fall liefert der Algorithmus eine negative Rückmeldung. Da es bei diesem Verfahren durch die Datenstruktur durchquert wird, müssen in der Regel \textit{n} Elemente verglichen werden. Somit beträgt die Zeitkomplexität der sequentiellen Suche $\mathcal{O}(n)$. \autocite[224]{hubwieser_fundamente_2015}

Die binäre Suche erfolgt ähnlich wie das Quicksort-Algorithmus nach der Strategie des \enquote{Teilen und Herrschens}. Eine wichtige Voraussetzung der binären Suche ist die aufsteigende Sortierung der Datenstruktur. In diesem Verfahren wird zuerst ein Element \textit{m} an der mittleren Position der Datenstruktur, in diesem Fall eine Liste, gewählt. Ist das angefragte Element kleiner als \textit{m}, wird in der linken Teilliste weitergesucht. Ist das angefragte Element größer als \textit{m}, wird in der rechten Teilliste weitergesucht. Beträgt das angefragte Element zufälligerweise gleich \textit{m}, wird die Suche erfolgreich beendet. Ansonsten werden die obigen Schritte auf die Teilliste rekursiv angewendet, bis das Element gefunden wird. Zur Bestimmung der Laufzeit dieses Verfahrens soll bemerkt werden, dass die Anzahl der zu durchsuchenden Elemente in jeder Teilliste halbieren. Diese Anzahl steigt mit jeder weiteren Teilliste logarithmisch ab, wodurch es sich eine Zeitkomplexität von $\mathcal{O}(\log(n))$ ergibt. \autocite[224-226]{hubwieser_fundamente_2015}

Suchfunktionen werden mit den meisten nichtlinearen Datenstrukturen wie Bäume gebündelt geliefert. Binärer Suchbäume dienen dem Zweck des Suchens, indem das angefragte Element rekursiv mit der Wurzel sowie dem Knoten jedes Teil- beziehungsweise Unterbaums verglichen wird. Die genaue Funktionsweise des binären Suchbaums wurde in Abschnitt~\ref{nicht_lineare_datenstrukturen} detaillierter behandelt. Bei einem bilanzierten oder symmetrischen Baum wird die Anzahl der zu untersuchenden Elemente auf jeder Ebene halbiert und folgt einen logarithmischen Trend. Somit ergibt sich im besten Fall eine Zeitkomplexität T\textsubscript{best} von $\mathcal{O}(\log(n))$. Im Schlimmsten Fall T\textsubscript{worst} wird durch einen degenerierten Baum durchquert, der wie eine lineare Liste strukturiert ist. So werden \textit{n} Elemente des Baums durchsucht, weswegen die Zeitkomplexität $\mathcal{O}(n)$ beträgt. \autocite[226-228]{hubwieser_fundamente_2015}

\subsubsection{Hashing}

Hashing Algorithmen dienen dem effizienten Einfügen, Löschen und Suchen von Elementen innerhalb einer Datenstruktur. Gewöhnlich werden für diese drei Operationen Vergleiche der Werte oder der Schlüssel ausgeführt, die in der Regel kostspielig sind, da mehrere Elemente der Datenstruktur dabei durchquert werden. Hashing Algorithmen machen von einzigartigen Schlüsseln für jedes Element einer Datenstruktur Gebrauch. Beim Hashing wird der eindeutige Schlüssel eines Elements verwendet, um es zu finden. \autocite[229]{hubwieser_fundamente_2015}

Hashing lässt sich natürlich nur auf Datenstrukturen verwenden, deren Elemente mit einzigartigen Schlüssel verknüpft sind. Bei Hashing werden besondere Datenstrukturen namens Hashtabellen verwendet, die unter zwei assoziierten Feldern einen Wert mit einem Schlüssel speichern. Dies heißt, dass jedes Element seinen eindeutigen Schlüssel hat. Zur Ermittelung der Position eines Elementes aus seinem Schlüssel wird eine sogenannte Hashfunktion verwendet. Bei einer Hashfunktion wird auf mathematischer Basis die Speicheradresse oder Position eines Elements anhand des Schlüssels berechnet. Nachdem die Position des Elements ermittelt wird, kann das dort gespeicherte Element abhängig von der Operation ausgelesen oder entfernt werden sowie ein neues Element eingefügt werden. Somit werden alle Operationen einer Datenstruktur auf das Aufrufen des Elements reduziert. Die Länge der Datenstruktur verliert somit ihren Einfluss auf die Zeitkomplexität der Operationen und ermöglicht theoretisch eine Zeitkomplexität von $\mathcal{O}(1)$. Um den Einfluss der Hashfunktion auf die Zeitkomplexität möglichst gering zu halten, sollte die Berechnung der Speicheradresse aus dem Schlüssel möglichst effizient erfolgen und dabei auch Kollisionen vermeiden. Kollisionen entstehen dann, wenn zwei eindeutige Schlüssel für unterschiedlichen Elemente auf die gleiche Position innerhalb der Datenstruktur zeigen. \autocite[230-231]{hubwieser_fundamente_2015}

Eine der einfachsten Hashfunktionen ist die Divisions-Rest-Methode oder die Modulo-Methode. In dieser Funktion wird die Position eines Elements ermittelt, indem der Restwert nach einer Division des Schlüssels durch eine konstante Größe \textit{T} berechnet wird. Eine gute Auswahl des Teilers \textit{T} ist für die Kollisionsvermeidung von höchster Bedeutung. Primzahlen, die weit von einer Zweierpotenz liegen, eignen sich zur Verwendung als den Teiler besonders gut. Herkömmlicherweise wird die Größe \textit{T} durch die Anzahl der verfügbaren Schlüssel bestimmt, also muss zur Manipulierung des Teilers die Anzahl der Schlüssel angepasst werden. \autocite[230-231]{hubwieser_fundamente_2015}

\Textcite[232]{hubwieser_fundamente_2015} stellen die perfekte Hashfunktion oder "Perfektes Hashing" vor: ein theoretisches Schema zur kollisionsfreien Ermittlung der Position eines Elementes. Dabei wird die Voraussetzung gesetzt, dass die Anzahl der Elemente nicht die Anzahl der verfügbaren Schlüsseln übersteigt. Mathematisch ausgedrückt sei die Anzahl der Elemente $|K| \leq T$. Grundsätzlich wird die Reihenfolge der Schlüssel nach ihrer Anordnung verwendet, um eindeutig die Positionen von Elementen zu bestimmen. Die genauere Funktionsweise sowie eine Untersuchung der Kollisionsvermeidung liegt außerhalb des Umfangs dieser Arbeit und kann in \textcite[92-94]{mehlhorn_algorithms_2008} nachgelesen werden.

Eine universelle Hashfunktion ist eine spezielle Art von Hashfunktion, die so konstruiert ist, dass sie eine geringe Wahrscheinlichkeit von Kollisionen aufweist. Im Gegensatz zu anderen Hashfunktionen ist eine universelle Hashfunktionen nicht speziell auf einen bestimmten Datentyp oder eine bestimmte Anwendung zugeschnitten, sondern kann auf verschiedene Datentypen und Anwendungen angewendet werden. Die Idee hinter einer universellen Hashfunktionen ist, dass sie zufällig ausgewählt wird, wodurch die Wahrscheinlichkeit von Kollisionen reduziert wird. Eine universelle Hashfunktionen ist im Allgemeinen eine Familie von Hashfunktionen, die auf eine bestimmte Größe von Hashtabellen abgestimmt sind. Um eine universelle Hashfunktionen auszuwählen, wird in der Regel eine zufällige Hashfunktionen aus der Familie ausgewählt, um eine gute Streuung der Schlüssel in der Tabelle zu erreichen. \autocite[232-234]{hubwieser_fundamente_2015} \autocite[114-116]{knebl_algorithmen_2021}

Je nach Auswahl einer Hashfunktion besteht die Gefahr, dass für zwei unterschiedliche Schlüssel der gleiche Wert zurückgegeben wird. Dies fordert eine Methode zur Kollisionsbehandlung auf. Dieses Gefahr besteht insbesondere beim Einfügen eines neuen Elements in der Hashtabelle. Hierzu schlagen \textcite[568-572]{ernst_grundkurs_2020} Methoden vor, die zur Behebung dieser Konflikte verwendet werden. 

\subsection{Verfahren zur datenbasierten Anpassung eines Modells}

\subsubsection{RANSAC}
Der RANSAC-Algorithmus ist eine Methode zur Schätzung von Parametern eines mathematischen Modells aus einer Menge von beobachteten Datenpunkten, die mit Rauschen überlagert sind. Der Name steht für "Random Sample Consensus", was auf die Tatsache zurückzuführen ist, dass der Algorithmus zufällige Stichproben aus den Daten verwendet, um ein Modell zu schätzen, das am besten zu den Daten passt. Das Ziel des RANSAC-Algorithmus besteht darin, ein Modell zu finden, das die meisten Datenpunkte erklären sowie die Ausreißer ignorieren kann. Ausreißer sind Datenpunkte, die nicht zum Modell passen und aufgrund von Fehlern in der Messung, unerwarteten Ereignissen oder anderen Faktoren auftreten können. \autocite[381-383]{fischler_random_1981}

Der Algorithmus beginnt damit, eine zufällige Stichprobe von Datenpunkten aus der Gesamtmenge auszuwählen. Diese zufällige Teilmenge wird verwendet, um eine vorläufige Schätzung der Parameter zu erstellen, die die Daten am besten beschreiben. Als nächstes wird jeder Datenpunkt in der Gesamtmenge überprüft, um festzustellen, ob er innerhalb einer vordefinierten Toleranzgrenze der vorläufigen Schätzung liegt. Wenn ein Datenpunkt innerhalb dieser Toleranzgrenze liegt, wird er als \textit{konsistent} oder \textit{Inlier} genannt. Andernfalls wird er als \textit{inkonsistent} oder \textit{Ausreißer} bezeichnet. Die konsistenten Datenpunkte werden verwendet, um eine neue Schätzung der Parameter zu erstellen. Dieser Prozess wird iterativ wiederholt, indem in jeder Iteration eine neue zufällige Stichprobe von Datenpunkten ausgewählt wird. Dadurch gelingt es dem Verfahren, eine neue vorläufige Schätzung der Parameter zu erstellen. Die Anzahl der zufälligen Stichproben und die Toleranzgrenze werden so gewählt, dass eine hinreichende Wahrscheinlichkeit besteht, dass der Algorithmus ein sinnvolles Modell findet. Am Ende des Algorithmus wird die Schätzung mit den meisten konsistenten Datenpunkten als endgültige Schätzung der Parameter verwendet. Diese endgültige Schätzung minimiert die Auswirkungen von Rauschen und Ausreißern und ist daher robust gegenüber unvorhergesehene Störungen in den Daten. \autocite[383-384]{fischler_random_1981} \autocite[3]{martinez-otzeta_ransac_2023}

Die Funktionsweise des Verfahrens lässt sich besser anhand eines Beispiels verstehen. Angenommen, dass es eine Menge von \textit{n} Punkten auf einer zweidimensionalen Fläche verteilt sind. Zu diesen Punkten gehören auch Rauschen oder Ausreißer. Mit dem RANSAC-Algorithmus wird angestrebt, eine Gerade zu schätzen, die möglichst gut durch die Punkte verläuft. Zuerst werden zwei Punkte aus \textit{n} zufällig gewählt, um eine vorläufige Gerade zu definieren. Danach werden alle Punkte identifiziert, die innerhalb einer gewissen Toleranz zu der Gerade liegen und somit als \textit{Inliers} der Gerade zählen. Diese Anzahl der Inliers wird gespeichert und zur späteren Bewertung dieser Iteration verwendet. Anhand der \textit{Inliers} wird eine Geradengleichung für diese Iteration bestimmt. Danach werden zwei neue zufällige Punkte gewählt, und die vorigen Schritte werden für eine vordefinierte Anzahl von Iterationen wiederholt. Schließlich wird die Geradengleichung ausgewählt, mit der die höchsten Anzahl an Inliers zustimmten. 

\begin{figure}[!b]
	\centering
	\begin{subfigure}{0.43\textwidth}
		\includegraphics[width=\textwidth]{Abbildungen/Line_with_outliers.png}
		\centering
		\caption[Linie mit Ausreißer]{Punkte auf einer zwei dimensionalen Linie mit Ausreißern}
		\label{fig: line_with_outliers}
	\end{subfigure}
	\hfill
	\begin{subfigure}{0.43\textwidth}
		\includegraphics[width=\textwidth]{Abbildungen/Fitted_line.png}
		\centering
		\caption[Linienmodell für eine Linie]{Ein angepasstes Linienmodell auf die Punkte}
		\label{fig: fitted_line}
	\end{subfigure}
	\caption[Visualisierung des RANSAC-Verfahrens]{Eine durch RANSAC erkannte Gerade durch Punkte. Die blauen Punkte stellen Inliers und roten Punkte die Outliers dar.}
	\label{fig: ransac_line}
\end{figure}

Insgesamt ist der RANSAC-Algorithmus eine robuste Methode zur Schätzung von Parametern aus beobachteten Datenpunkten, die durch Rauschen beeinträchtigt sind. Aufbauend auf dem nach \textcite{fischler_random_1981} vorgeschlagenen Verfahren wurden zusätzliche Abzweigungen des Verfahrens konzipiert, die eine höhere Robustheit oder Leistung anbieten. Das RANSAC Verfahren findet in diversen Bereichen wie maschinelles Sehen (Computer Vision), Robotik und Geodäsie eine sehr breite Anwendung \autocite[2]{martinez-otzeta_ransac_2023}.

\subsubsection{Region-Growing}
Region-Growing-Segmentierung ist ein Verfahren der Bildsegmentierung, welches auf der Idee basiert, dass Pixel, die ähnliche Eigenschaften aufweisen, zu einer zusammenhängenden Region zusammengefasst werden können. Das Ziel dabei ist es, die relevanten Objekte im Bild zu identifizieren und von der Hintergrundinformation zu trennen. \autocite[641]{adams_seeded_1994}

Das Verfahren beginnt mit einem ausgewählten Startpixel, welches als Samen oder Seed bezeichnet wird. Anschließend werden benachbarte Pixel des Seeds analysiert und geprüft, ob sie ähnliche Eigenschaften aufweisen wie der Seed. Wenn dies der Fall ist, werden diese Pixel zu einer Region hinzugefügt. Der Prozess wird wiederholt, bis keine weiteren Pixel gefunden werden, die zur dieser Region hinzugefügt werden können. Dadurch wird eine Region des Bildes gewachsen und identifiziert, die homogene Bildeigenschaften besitzt. ´\autocite[641-642]{adams_seeded_1994}

Ein wichtiger Faktor bei der Region-Growing-Segmentierung ist die Wahl der Kriterien für die Ähnlichkeit von Pixeln. Es gibt verschiedene Methoden zur Bestimmung von Ähnlichkeitskriterien, darunter Intensität, Farbe, Textur und Form. Die Wahl hängt von der Art des zu segmentierenden Bildes und den gewünschten Ergebnissen ab. Darüber hinaus spielt die Wahl des Seeds auch eine wichtige Rolle. In Daten mit wenigen Ausreißern hat eine falsche Wahl des Seeds keinen wesentlichen Effekt auf die Korrektheit des Algorithmus, allerdings können viele Ausreißer den Algorithmus stören. In diesem Fall ist es wichtig, dass kein Ausreißer als Seed gewählt wird, um die Erzeugung falscher Segmente zu vermeiden. \autocite[641-643]{adams_seeded_1994}

Region-Growing-Segmentierung lässt sich auf eine breite Auswahl an Bildern anwenden. Dies ermöglicht zahlreiche Anwendungen dieses Algorithmus in der medizinischen Bildgebung, z.B. bei der Segmentierung von Tumoren oder Organen. Diese hohe Anwendbarkeit des Algorithmus lässt auch eine Anwendung in der industriellen Bildverarbeitung und der Robotik zu.\autocite[646]{adams_seeded_1994}

