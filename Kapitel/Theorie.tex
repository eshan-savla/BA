%Datenstrukturen und Algorithmen
%	Was sind Algorithmen und Datenstrukturen
% 	Arten von Basis Datenstrukturen
%		Vector/Liste
%		Queue(Warteschlange)
%		Stack??
%		map und hash tabellen
%	besondere Datenstruktur - PCL Punktwolke
%	
%	Komplexität und Notation von Algorithmen
%	Arten von Basis Algorithmen
%		
% 		
%	besondere Algorithmen:
%		kd-tree
%		octree(vielleicht)
%		RANSAC
%		Segmentierung

\chapter{Theorie}
Im Kern dieser Arbeit steht eine Rechenaufgabe vor. Die positionellen Informationen über Objekte und Bauteile müssen sinnvoll verarbeitet werden, um die Lage und Form der geometrischen Merkmale des Objektes zu bestimmen. Bei der Entwicklung eines allgemeinen Verfahrens zur Erkennung der geometrischen Merkmale in Abschnitt~\ref{Methodik} werden Algorithmen und Datenstrukturen verwendet. 

\section{Datenstrukturen}

Datenstrukturen dienen der Organisation und Speicherung von Daten so, dass die Beziehung zwischen einzelnen Elemente auch aufbewahren wird. In einer Datenstruktur werden darüber hinaus auch Zugriffsmethoden für den Zugriff auf die gespeicherten Daten definiert sowie Angaben über Möglichkeiten zur Verarbeitung der Daten gemacht. Eine gute Datenstruktur setzt voraus, dass die Beziehung zwischen der Daten aufbewahren und gut definiert wird sowie die Verarbeitung der Daten leicht gemacht wird. Eine Datenstruktur soll auch bestimmte Operationen auf die Daten ermögliche, beispielsweise die Hinzufügung oder Entfernung von Datenpunkte, die Zusammenführung oder Sortierung der Daten sowie das Durchqueren der Datenstruktur und die Suche nach bestimmten Daten. In der Informatik gibt es bereits etablierte Datenstrukturen, die sich nach unterschiedlichen Einsatzzwecken richten und eine sehr breite Anwendung finden. Diese lassen sich nach Abbildung \ref{fig: datastructures} nach lineare und nichtlineare Datenstrukturen unterteilen. \autocite[1-2]{mohanty_data_2021}

\begin{figure}[h]
	\includegraphics[width=\textwidth]{Abbildungen/Datenstruktur_arten.png}
	\centering
	\caption{Die unterschiedlichen Arten von linearen und nichtlinearen Datenstrukturen nach \textcite[2]{mohanty_data_2021}}
	\label{fig: datastructures}
\end{figure}

\subsection{Lineare Datenstrukturen}
Eine lineare Datenstruktur ist eine Datenstruktur, bei der die Elemente in einer sequentiellen Reihenfolge angeordnet sind. Dies bedeutet, dass jedes Element genau einen Vorgänger und einen Nachfolger hat, außer dem ersten und letzten Element. Lineare Datenstrukturen können als tabellarische Liste oder als verkettete Liste implementiert werden. \autocite[314-315]{hoffmann_einfuhrung_2011}

Die Operationen, die auf einer linearen Datenstruktur ausgeführt werden können, sind in der Regel das Initialisieren der Datenstruktur als leere Menge, das Einfügen eines Elements in die Datenstruktur und das Entfernen eines Elements aus der Datenstruktur. Es ist auch möglich, andere Operationen auszuführen, die nicht unbedingt auf der Ordnungsbeziehung zwischen den Elementen basieren. Unter diesen Operationen zählen beispielsweise das Suchen nach einem Element oder das Ersetzen eines Elements in der Datenstruktur. \autocite[314-315]{hoffmann_einfuhrung_2011}

Die Implementierung einer linearen Datenstruktur kann je nach Anforderungen und verfügbaren Ressourcen variieren. Es ist jedoch wichtig sicherzustellen, dass die Datenstruktur korrekt implementiert ist und dass alle Operationen den Zustand der Datenstruktur ordnungsgemäß ändern. \autocite[314-315]{hoffmann_einfuhrung_2011}

Unter der linearen Datenstrukturen finden hauptsächlich drei Datenstrukturen eine breite Anwendung, nämlich Felder (Arrays), Schlangen (Queues) und Keller (Stacks).

\subsubsection{Felder}

Das Feld ist einer der einfachsten linearen Datenstrukturen. Ein Feld besteht aus mehreren Daten des gleichen Formats oder Datentyps, die je nach Implementierung in aufeinanderfolgenden Speicherorten gespeichert werden. Diese werden sequenziell hintereinander angeordnet und zusammen gespeichert. Elemente eines Feldes dürfen eindimensional oder auch mehrdimensional gespeichert werden, wodurch diese Datenstruktur mit Matrizen oder Vektoren aus der Mathematik verglichen werden kann. Die Größe oder Dimension des Feldes wird immer vorgegeben und bleibt in der Regle statisch. Jedes Element eines Feldes besitzt einen sogenannten Index, welcher auf die Position des Elements in dem Feld deutet. Im Falle eines zwei Dimensionalen Feldes besitzt jedes Element der Datenstruktur zwei Indizes. In der regel deutet das erste Index auf die Zeile und das zweite Index auf die Spalte des Datenfeldes hin, allerdings kann dieser Regel von Implementierung zu Implementierung variieren. Felder dürfen eine beliebig Anzahl \textit{n} Dimensionen besitzen, allerdings steigt somit auch die Anzahl der Indizes aller Elemente. Datenfelder über vier Dimensionen können sogar räumlich nicht vorgestellt werden, jedoch macht es für einen Rechner kein Problem. Zugriffsoperationen auf Elemente sowie Operationen zur Einfügung und Entfernung dieser Elemente verwenden ihre Indizes, um den Eintrag an einer bestimmten Position des Feldes aufzurufen oder zu manipulieren. \autocite[35-36]{ollmert_datenstrukturen_2020}

Eine andere Variante der Felder ist die sogenannte lineare Liste. Diese Liste unterscheidet sich von Felder, indem sie dynamisch initialisiert werden darf. Im Gegensatz zu der statischen Größe oder Dimension eines Feldes, darf die Größe einer linearen Liste beliebig geändert werden. Während ein Feld mit maximale Größe \textit{n} und \textit{n} Elemente nicht um ein weiteres Element \textit{k} erweitert werden darf, kann eine lineare Liste der gleichen Größe mit den gleichen Anzahl an Elementen um das \textit{k-te} Element erweitert werden. Das Element darf auch an einer beliebigen Stelle der Liste eingefügt werden. Die Reihenfolge beziehungsweise die Positionen der anderen Elemente werden automatisch angepasst. Dies könnte auch dazu führen, dass ein Element Z, welches zum Zeitpunkt T1 vor der Einfügung eines neuen Elements einen bestimmten Index \textit{i} besaß, zu einem Zeitpunkt T2 nach der Einfügung nicht mehr an der gleichen Position zu finden ist. Das gleiche kann auch durch die Entfernung von Elementen an beliebigen Positionen geschehen. Die Positionen und somit die Indizes alle Elemente dürfen auch geändert werden, indem sie Beispielsweise nach einem Kriterium sortiert wurden. \autocite[40-42]{ollmert_datenstrukturen_2020}

Lineare Listen bieten viele ähnlichen Operationen wie die von Feldern an, um mit den eingespeicherten Elementen zu interagieren. Dazu gehört das Abrufen eines bestimmten Elements, das Einfügen eines neuen Elements zwischen zwei benachbarten Element (sowie Spezialfälle für das Hinzufügen eines neuen ersten oder letzten Elements), das Entfernen eines bestimmten Elements, das Bestimmen der aktuellen Länge der Liste anhand der Elementanzahl, die Suche nach einem Element mit einem bestimmten Wert, das Zusammenführen von zwei linearen Listen und das Aufteilen einer Liste in zwei Teillisten.\autocite[42-43]{ollmert_datenstrukturen_2020}


Verketteten linearen Listen sind eine bestimmte Art der Implementierung von linearen Listen. Eine solche Kette unterscheidet sich grundsätzlich nach ihrem Aufbau und Speicherverfahren von gewöhnlichen Listen und Feldern. Die Elemente einer verketteten linearen Liste werden nicht aufeinanderfolgend gespeichert. Stattdessen wird mit jedem Element dieser Liste auch ein Zeiger beigegeben, der entweder die Speicheradresse des nächsten Elements angibt oder auf das Ende der Liste hinweist. Der Anker ist ein besonderer Zeiger, der auf den Anfang der Liste deutet. Die Struktur und Operationen einer linearen Liste ist in Abbildung \ref{label} abgebildet. Bei der doppel-verketteten linearen Listen werden mit jedem Element nicht nur einen Zeiger zu dem nächsten Element beigegeben, sondern auch einen Zeiger zu dem vorigen Element. Bei dieser Art der verketteten Liste werden zwei Anker geliefert - der Vorwärtsanker für den Anfang und der Rückwärtsanker für das Ende der Liste. Die Verarbeitung mancher linearen Listen stellt eine Herausforderung dar, insbesondere wenn sie gleichzeitig durch mehrere Programme verarbeitet werden. In solchen Fällen könnte es beispielsweise dazu führen, dass ein Programm auf ein bestimmtes Element mittels seines Index zugreift, während ein anderes Programm ein Element davor hinzufügt oder entfernt. \autocite[43-44]{ollmert_datenstrukturen_2020}

\begin{figure}[t]
	\includegraphics[width=\textwidth]{Abbildungen/Verkettete_lineare_liste.png}
	\centering
	\caption{Die Struktur und Operationen einer linearen Liste. Die Großbuchstaben deuten auf die Elemente der Liste während die \textit{z}-Buchstaben die Zeiger repräsentieren. Der Anker wird durch den Kopf abgebildet \autocite[611]{ernst_grundkurs_2020}}
\end{figure}

Für alle gängigen Implementierungen von linearen Listen ist die Implementierung einer sequentiellen Zugriffsfunktion, die das nächste Element in der Liste ausgehend von einem bestimmten Element bereitstellt, einfach. Die Implementierung einer direkten Zugriffsfunktion, die das \textit{i}-te Element in der Liste bereitstellt, ist zwar ebenfalls einfach, aber die benötigte Zeit hängt von der Position des Elements in der Liste ab und nimmt mit zunehmender Länge der Liste zu.\autocite[45]{ollmert_datenstrukturen_2020}

\subsubsection{Schlangen}
Schlangen sind besondere Datenstrukturen, die Daten nur in einer bestimmten Reihenfolge speichern. Dadurch wird auch die Entnahme und Einfügung der Daten geregelt. Schlangen folgen das Prinzip nach \textit{First-In-First-Out} (FIFO), also werden die Elemente zur Verfügung gestellt, die zuerst zu der Datenstruktur hinzugefügt wurden. Elemente dürfen in der Regel nur am Ende der Schlange eingefügt und vom Anfang der Schlange entnommen werden. \autocite[371]{gumm_band_2016}

\section{Algorithmen}

Ein Algorithmus ist ein Verfahren, welches zur Bestimmung einer oder mehreren Lösungen eines bestimmten Rechenproblems verwendet wird \autocite[1]{knebl_algorithmen_2021}. In einem Algorithmus wird ein Lösungsansatz möglichst präzise ausformuliert, indem kleine, isolierte und klar definierte Verarbeitungsschritte definiert werden. Auch ein simples Verfahren zur Summierung zwei Zahlen lässt sich als ein Algorithmus nennen. \autocite[9-10]{hubwieser_fundamente_2015}