\chapter{Theorie}
\section{Algorithmen und Datenstrukturen}
Im Kern dieser Arbeit steht eine Rechenaufgabe vor. Die positionellen Informationen über Objekte und Bauteile müssen sinnvoll verarbeitet werden, um die Lage und Form der geometrischen Merkmale des Objektes zu bestimmen. Bei der Entwicklung eines allgemeinen Verfahrens zur Erkennung der geometrischen Merkmale in Abschnitt~\ref{Methodik} werden Algorithmen und Datenstrukturen verwendet. 

Ein Algorithmus ist ein Verfahren, welches zur Bestimmung einer oder mehreren Lösungen eines bestimmten Rechenproblems verwendet wird \autocite[1]{knebl_algorithmen_2021}. In einem Algorithmus wird ein Lösungsansatz möglichst präzise ausformuliert, indem kleine, isolierte und klar definierte Verarbeitungsschritte definiert werden. Auch ein simples Verfahren zur Summierung zwei Zahlen lässt sich als ein Algorithmus nennen. 

Datenstrukturen 