%Struktur
% -Festlegung von Software-Kriterien
% -Auswahl des Verfahrens
% -Reproduktion des Verfahrens aus der Literatur
% 	-Kantenerkennung
% 	-Segmentierung
%	-Probleme & besondere Erkenntnisse
% -Erweiterung des Verfahrens
%	-Grundlegende Änderungen des Verfahrens
%	-False edge removal
%	-Point marking and cloud filtering
%	-misc.


\chapter{Entwicklungsprozess der Software}
Die Rücksichtnahme des Einsatzzwecks bei der Design und Entwurf des Verfahrens sowie die Entwicklung der Software war erforderlich, um die gewünschte Funktionalitäten gewährleisten zu können. Das Verfahren soll Kanten und Geometrien nicht nur in vollständig generierten Punktwolken erkennen, sondern auch in unvollständige Punktwolken, die iterativ wachsen. Hierbei wird ein Laserliniensensor eine Kante eines Werkstücks oder Objektes entlang geführt und somit sequentiell abgetastet. Deswegen wird die räumliche Struktur des Objektes nicht in einer einzigen Aufnahme abgebildet, sondern durch mehrere kleine Einzelaufnahmen. Der intelligente Schweißroboter, der durch das Fraunhofer Institut für Produktions- und Automatisierungstechnik entwickelt wird, verwendet ein solches Verfahren zum Scannen eines Werkstückes und zur Erkennung Schweißnähte \autocite[39]{savla_intelligente_2022}. Mittels eines Lasersensors wird die Oberfläche des Werkstückes dreidimensional abgebildet. Aktuell wird eine Schweißkegelnaht durch die Erkennung der Schnittlinie zwei Ebenen markiert, die mittels RANSAC-Algorithmen auf die Punktwolke des Werkstückes gefittet werden. Dieses Verfahren zur Erkennung der Schweißnaht bietet allerdings kaum detaillierte Informationen über die Geometrie des Werkstückes an.\autocite[39-52]{savla_intelligente_2022}. Das, in dieser Arbeit entwickelte Verfahren soll das bestehende Verfahren ersetzen und somit seine Limitationen überwinden.

\begin{figure}[h]
	\includegraphics[width = \textwidth]{Abbildungen/collage.jpg}
	\centering
	\caption{Der Laserliniensensor} 
\end{figure}

\section{Vorbereitungen} \label{agpn_reproduction}
\subsection{Software-Voraussetzungen}\label{soft_voraus}
Bei der Auswahl eines geeigneten Verfahrens zur Detektierung Kanten in einer Punktwolke wurden einige Voraussetzungen festgelegt. Die Methode sollte in der Lage sein, nicht nur Außenkanten zu erkennen, sondern auch Innenkanten beziehungsweise Faltungen. Neben dem originellen Einsatzzweck sollte das Verfahren möglichst breit anwendbar sein und eine hohe Modularität aufweisen. Die Funktionen der Kantenerkennung und -Segmentierung sollten unabhängig von einander aufrufbar gestaltet werden, um dem Benutzer eine möglichst hohe Flexibilität anzubieten. Die Kantenerkennung sollte performant erfolgen und Punktwolken innerhalb eines praktischen Zeitraums verarbeiten. Letztlich soll das Programm in dem bestehenden Programmpaket des Schweißroboters integrierbar sein. Die Hardwarebeschleunigung des Verfahrens mittels eines Grafikprozessors wurde ausgeschlossen, da ihrer Verwendung mit dem Echtzeitkernels des Programmpakets zur Konflikte führt. 

\subsection{Auswahl eines Verfahrens}
Eine Literatursuche nach Verfahren zur adaptiven Erkennung von Kanten in wachsenden 3D Punktwolken für den Einsatzzweck ergab nichts. Die meisten Verfahren eigneten sich für die Kantenerkennung nur in vollständigen Punktwolken. Aus diesem Grund wurde die Entscheidung getroffen, ein vorhandenes Verfahren aus der Literatur zu wählen und es für den Einsatzzweck anzupassen. Drei unterschiedlichen Verfahren nach \textcite{bazazian_edc-net_2021}, \textcite{himeur_pcednet_2021} und \textcite{rachmadi_road_2017} zeigten viel versprechende Ergebnisse. Allerdings wurden neuronale Netze in dieser Verfahren verwendet, welches zu zwei Problemen geführt hätte. Aufgrund der Funktionsweise neuronaler Netze wäre es schwierig gewesen, diese für den Einsatzzweck ohne eine umständliche Anpassung des neuronalen Netzes anzupassen. Das zweite Hindernis entsteht durch die Einschränkung bei der Verwendung von Grafikprozessoren. Diese Prozessoren hätten die Rechenzeit neuronaler Netze sehr stark verringert und die schnelle Performanz des Verfahrens gewährleistet \autocite[625]{luo_artificial_2005}. Das numerische Verfahren nach \textcite{choi_rgb-d_2013} war auch für den Einsatzzweck ungeeignet, da es für ein Eingangsparameter eine RGB-D Datei erfordert. Somit wäre das Verfahren nur für eine Anwendung auf organisierten, gefärbten Punktwolken eingeschränkt. Es wurden zwei weitere Verfahren gefunden, die sich zur Erkennung Kanten in organisierten sowie unorganisierten Punktwolken eignen würden. \textcite{mineo_novel_2019} stellten eine numerische Methoden vor, welche zu einer hohen Genauigkeit Kanten erkennen konnte. Allerdings wurden keine Angaben über die Erkennung Innenkanten in dieser Arbeit gemacht. \textcite{ni_edge_2016} schlagen im Gegensatz eine Methode namens AGPN vor, die nicht nur Kanten und Faltungen erkennt, sondern diese zusammen clustert, um geometrisch  voneinander unterschiedlichen Kanten zu trennen. Diese Studie präsentierte ein Verfahren mit einer hohen Genauigkeit sowie eine Möglichkeit, die Kanten sinnvoll zusammen zu gruppieren. Aus diesem Grund wurde dieses Verfahren als Grundlage für das adaptive Verfahren dieser Arbeit gewählt.

\section{Reproduktion des AGPNs}
Bevor das Verfahren für den Einsatzzweck angepasst wurde, wurde es zuerst zwecks einer Überprüfung unverändert implementiert. Es sollte sichergestellt werden, dass das Verfahren für die Erkennung Innenkanten und potenzielle Schweißnähte geeignet ist. Da die Autoren das Quellcode ihres Verfahrens nicht öffentlich zugängig gemacht haben, musste das Programm händisch reproduziert werden. Die Reproduktion des Programms erfolgte in zwei Schritten - die Reproduktion des Verfahrens zur Kantenerkennung und dessen zur Kantensegmentierung. Obwohl andere Skriptsprachen wie Python und MATLAB hinsichtlich des Prototypings Vorteile anbieten, wurde das Programm in C++ wegen seiner besseren Leistungsfähigkeit implementiert \autocite{svensson_performance_2021}. Viele Funktionalitäten der PCL-Bibliothek \autocite{rusu_3d_2011} wurden auch zum Entwurf des Verfahrens verwendet.

\subsection{Verfahren zur Kantenerkennung} \label{edge_detection_reprod}
Während Randelemente in zweidimensionale Bilder eine klare Definition haben, fehlt eine solche Definition für Randelemente und Kanten in 3D-Punktwolken. Es wurde erkannt, dass Kanten einer dreidimensionalen Punktwolke aus Randpunkten besteht. In diesem Verfahren wurden die geometrischen Eigenschaften einer Kollektion von Punkten zur Erkennung Randpunkte berücksichtigt. Randpunkte weisen diese besondere geometrische Eigenschaft auf - der Winkelabstand zwischen benachbarten Randpunkte ist im Vergleich zu anderen benachbarten Punkten deutlich größer. Faltungen stellen den Grenzbereich zwischen zwei angrenzenden Ebenen dar, deren Normale in unterschiedlichen Richtungen zeigen. Diese geometrischen Eigenschaften wurden zur Erkennung Randpunkte verwendet. \autocite[1-2]{ni_edge_2016}

\begin{figure}[h]
	\includegraphics[scale=0.12]{Abbildungen/ablauf_plan_edge.png}
	\centering
	\caption{Das Programmablaufplan zur Erkennung Randpunkte \autocite{ni_edge_2016}.}
	\label{flow_chart}
\end{figure}

Im folgenden wird das \textit{FindEdgePoints} Verfahren detaillierter erläutert. Zwecks der Übersichtlichkeit wurde dieses Verfahren als \textit{FindEdgePoints} referiert. Für einen Punkt \textit{o} wurde eine Sammlung von \textit{K\textsubscript{1}} benachbarten Punkten mittels eines kd-trees erstellt. Diese Sammlung wird als eine Nachbarschaft \textit{N\textsubscript{o}} referiert. Danach wurde mittels eines RANSAC-Algorithmus eine Ebene \textit{E\textsubscript{N}} auf diese Nachbarschaft gefittet, um Ausreißer herauszufiltern und zwei angrenzenden Flächen innerhalb der Nachbarschaft voneinander zu trennen. Danach wurde Überprüft, ober der Punkt \textit{o} auf der RANSAC-Ebene lag. Falls dieser Punkt ein Ausreißer der Ebene \textit{E\textsubscript{N}} war, wurde er nicht als ein Randpunkt markiert. Ansonsten wurden weiterhin die geometrischen Eigenschaften der Nachbarschaft überprüft. Abbildung \ref{RANSAC-Ebene} visualisiert die Trennung zwischen unterschiedlichen Flächen einer Punktwolke mittels des RANSAC-Verfahrens. 

\begin{figure}[h]
	\includegraphics[width=\textwidth]{Abbildungen/RANSAC-Ebene.png}
	\centering
	\caption{Eine lokale RANSAC-Ebene (rot dargestellt) neben anderen Oberflächen (blau dargestellt). In \textbf{a} sind drei Ebenen zu sehen, wobei in \textbf{b} nur zwei zu sehen sind. \autocite{ni_edge_2016}}
	\label{RANSAC-Ebene}
\end{figure} 

Im Falle, dass der Punkt \textit{o} ein Inlier war und zu der Ebene \textit{E\textsubscript{N\textsubscript{o}}} gehörte, fang die tatsächliche Überprüfung der geometrischen Eigenschaften der Nachbarschaft \textit{N\textsubscript{o}} an. Um die Ebenengleichung von \textit{E\textsubscript{N}} näherungsweise zu schätzen, wurde zuerst die Normale \textit{$\vec{n}$} der Ebene geschätzt. In einer effizienten Weise wurde die Ebenengleichung durch das RANSAC-Verfahren näherungsweise geschätzt. Diese Gleichung wurde weiterhin auf die RANSAC-Inliers optimiert und daraus die Normale \textit{$\vec{n}$} ermittelt. Danach erfolgte die Errechnung des Winkelabstands zwischen den jeweiligen Punkten von \textit{N\textsubscript{o}}. Hierfür wurden für die Ebene \textit{E\textsubscript{N}} die jeweiligen Eigenvektoren $\vec{u}$ und $\vec{v}$ aus der Normale $\vec{n}$ errechnet. Das fertige Verfahren von PCL zur Ausrechnung der Eigenvektoren lieferte ungenaue Ergebnisse. Stattdessen wurden zur Ermittelung \textit{$\vec{u}$} zwei zufällig gewählte Punkten aus der Inliers verwendet. Es wurde dabei sichergestellt, dass keiner der Punkten den Punkt \textit{o} entsprachen. Zur Errechnung des Winkelabstands wurden zuerst die Winkel aller Punkte der lokalen Ebene \textit{E\textsubscript{N}} errechnet. Mit dem Punkt \textit{o} als Ursprung wurde für jeden Punkt \textit{p\textsubscript{i}}, aus \textit{N\textsubscript{r}} Punkten, der Winkel \textit{$\theta_i$} zu einer Nulllinie errechnet. Danach wurde die Differenz zwischen zwei konsekutiver Punktwinkel $\theta_i$ und $\theta_{i+1}$ errechnet, welcher den Winkelabstand \textit{G\textsubscript{$\theta$}} zwischen zwei Punkten \textit{p\textsubscript{i}} und \textit{p\textsubscript{i+1}} betrug. Bei einer Winkelabstand größer als einem bestimmten Schwellenwert wurde der Punkt \textit{o} als ein Randpunkt markiert. Das Referenzwerk verwendete einen Schwellenwert von $\frac{\pi}{2}$. Abbildung \ref{edge_boundary} zeigt, wie der Winkelabstand zwischen Punkten am Rand der Punktwolke aussieht.

\begin{figure}[h]
	\includegraphics[width=0.5\textwidth]{Abbildungen/angular_gap_boundary}
	\centering
	\caption{Der Winkelabstand \textit{G\textsubscript{$\theta$}} zwischen Punkten am Rand der Punktwolke. \textbf{\(a\)} zeigt ein interner Punkt \textit{o}und ein Nachbarpunkt \textit{p\textsubscript{i}}. Im Vergleich dazu zeigt \textbf{\(b\)} \textit{o} am Rand und den großen Winkelabstand \textit{G\textsubscript{$\theta$}} zwischen Punkte \textit{p\textsubscript{i}} und \textit{p\textsubscript{i + 1}}. \autocite{ni_edge_2016}}
	\label{edge_boundary}
\end{figure}

Auch die Erkennung von Punkten in Innen- und Außenkanten war durch diese Berechnungen möglich. Wie bereits erwähnt, wurden zwei angrenzenden Flächen mittels das RANSAC-Verfahren voneinander getrennt. Falls der Punkt \textit{o} auf der Schnittlinie beider Flächen sowie auf der lokalen RANSAC-Ebene \textit{E\textsubscript{N}} liegt, dann gehört es zum lokalen Rand der Ebene. Falls der Punkt \textit{o} auf der Schnittlinie beider Flächen liegt, aber nicht zu der RANSAC-Ebene gehört, wird er automatisch nicht als ein Randpunkt gemerkt. \textit{E\textsubscript{N}}. Die Abbildung \ref{edge_fold} zeigt, wie der Winkelabstand zwischen Punkten auf einer Schnittlinie zwischen zwei Flächen der Punktwolke aussieht. Die Errechnungen des Winkelabstands erfolgte nach den Gleichungen \ref{first_equation} - \ref{last_equation}.

\begin{figure}[h]
	\includegraphics[width=\textwidth]{Abbildungen/angular_gap_fold}
	\centering
	\caption{Der Winkelabstand zwischen Punkten auf einer Schnittlinie zwei angrenzender Flächen. \textbf{\(a\)} zeigt ein interner Punkt \textit{o} der RANSAC-Ebene. \textbf{\(b\)} zeigt \textit{o} am lokalen Rand der RANSAC-Ebene und den Winkelabstand \textit{G\textsubscript{$\theta$}} zwischen Punkte \textit{p\textsubscript{i}} und \textit{p\textsubscript{i + 1}}. \textbf{\(c\)} zeigt \textit{o} als ein Ausreißer der RANSAC-Ebene. \autocite{ni_edge_2016}}
	\label{edge_fold}
\end{figure}

\begin{equation}
\label{first_equation}
d_i^u = \vec{{op}_i} \cdot \vec{u}
\end{equation}
\begin{equation}
d_i^v = \vec{{op}_i} \cdot \vec{v}
\end{equation}
\begin{equation}
\theta_i = \arctan{\frac{d_i^u}{d_i^v}}
\end{equation}
\begin{equation}
G_\theta = \max(\theta_{i + 1} - \theta_i), i \in \{1, \ldots, N_r - 1\},
\label{last_equation}
\end{equation}


Um die Genauigkeit des Verfahrens zu versichern und die Rechenarbeit des Verfahrens zu verringern, wurden gezielt zwei zusätzliche Schritte vor dem Erkennungsverfahren eingeführt. Um die Anzahl der Punkte in der Punktwolke zu verringern wurde ein Voxel-Grid basiertes Downsampling-Verfahren implementiert, um die Punktdichte der Punktwolke künstlich anzupassen und den Abstand zwischen Punkten zu vereinheitlichen. Hierfür wurde die PCL-Funktion \textit{UniformSampling} verwendet \autocite{noauthor_point_2023}. Um Ausreißer aus der Punktwolke zu entfernen, wurde ein statistische Verfahren der PCL-Bibliothek namens \textit{StatisticalOutlierRemoval} zur Entfernung von Ausreißer verwendet \autocite{rusu_towards_2008}. Zur Korrekten Ausrechnung des maximalen Winkelabstands einer Nachbarschaft \textit{G\textsubscript{$\theta$}} war eine aufsteigende Sortierung der Winkel \textit{$\theta_i$} notwendig. Diese Sortierung entsprach eine Sortierung der Punkte \textit{p\textsubscript{i}} nach ihrer aufsteigenden polaren Entfernung von der Nulllinie. Die Abbildung \ref{vector_graph} dient zur Visualisierung der Methode zur Ausrechnung von $\theta_i$. Die Eigenvektoren $\vec{u}$ und $\vec{v}$ bildeten das zweidimensionale Koordinatensystem, wobei $\vec{v}$ analog zu einer x-Achse agierte. Die Skalarprodukte \textit{d\textsubscript{i}\textsuperscript{u}} und \textit{d\textsubscript{i}\textsuperscript{v}} repräsentierten die parallelen Anteile des Vektors $\vec{{op}_i}$ der jeweiligen Eigenvektoren $\vec{u}$ und $\vec{v}$. Somit ließ sich der Winkel $\theta_i$ eines Punktes \textit{p\textsubscript{i}} zu der Nulllinie beziehungsweise dem Vektor $\vec{v}$ errechnen. Abbildung \ref{edge_points_table} zeigt die erkannten Kanten einer dreidimensionalen Abbildung eines Tisches.

\begin{figure}[t]
	\includegraphics[scale=0.7]{Abbildungen/vector_graph.png}
	\centering
	\caption{Diese Abbildung stellt die Berechnung des Winkels $\theta_i$ graphisch dar}
	\label{vector_graph}
\end{figure}

\begin{figure}[h]
	\includegraphics[scale=0.37]{Abbildungen/table_edge_overlay.png}
	\centering
	\caption{Die braune Punkte bilden die Ränder bzw. Kanten ab, die in der blauen Punktwolke durch das Verfahren erkannt wurden}
	\label{edge_points_table}
\end{figure}

Zusammenfassend wurde für jeden Punkt \textit{o} aus der Punktwolke eine RANSAC-Ebene \textit{E\textsubscript{N}} aus einer lokalen Nachbarschaften des Punktes mit \textit{K\textsubscript{1}} Punkten erstellt. Falls \textit{o} in der Ebene lag, und der größte Winkelabstand zwischen Punkten der Ebene mit Ursprung \textit{o} größer als $\frac{\pi}{2}$ betrug, wurde \textit{o} als ein Randpunkt markiert und gespeichert. Durch die Wiederholung dieser Schritte für alle Punkte wurden alle Randpunkte und somit aller Kanten der Punktwolke identifiziert. Abbildung \ref{flow_chart} stellt den Programmablaufplan für dieses Verfahren dar. Die Probleme bei der Reproduktion dieses Verfahrens werden im Abschnitt \ref{label} besprochen.

\subsection{Verfahren zur Kantensegmentierung} \label{edge_segmentation}
Nachdem die Kanten der Punktwolke erkannt wurden, folgte ihrer Segmentierung. Zwecks der Übersichtlichkeit wurde dieses Verfahren als \textit{SegmentEdges} referiert. Hierbei wurden alle Kanten zusammen gruppiert, die zu einem geometrischen Merkmal des gescannten Objektes gehörten. Hierfür wurde ein Region-Growing Verfahren verwendet. Punkte wurden auf Basis zwei Kriterien segmentiert. Das erste Kriterium besagte, dass nur Punkte, die nah aneinander lagen, einen Cluster bilden können. Das zweite Kriterium besagte, dass nur Punkte, die in einer ähnlichen Hauptrichtung zeigten, zusammen geclustert werden dürfen. Die Segmentierung erfolgte hauptsächlich in zwei Schritten - die Erstellung und Exaktifizierung von Nachbarschaften sowie die Region-Growing Segmentierung der Kanten.

Bei dem ersten Schritt handelte es sich um die Berechnung der Richtungsvektoren und die Bestimmung der exakten Nachbarpunkte jedes Randpunktes aller Kanten. Für einen Randpunkt \textit{p} wurden \textit{K\textsubscript{2}} Nachbarpunkte aus den, in Abschnitt \ref{edge_detection_reprod} gefunden Randpunkten, gesucht. Diese Sammlung wurde als die Nachbarschaft \textit{N\textsubscript{p}} referenziert. Danach wurde eine Linie \textit{L\textsubscript{N}} mittels eines RANSAC-Verfahrens auf die Nachbarschaft gefittet, um alle Punkte zu finden, die in der gleichen Hauptrichtung zeigten. Falls der Randpunkt \textit{p} nicht zu den Inliers der RANSAC-Linie gehörte, wurde iterativ auf die Ausreißer das RANSAC-Verfahren wieder implementiert, bis \textit{p} zu den Inliers von \textit{L\textsubscript{N}} gehörte. Danach wurden alle Punkte der Linie \textit{L\textsubscript{N}} als die exakten Nachbarpunkte des Punktes \textit{p} gespeichert. Der, aus \textit{L\textsubscript{N}} ermittelte Richtungsvektor wurde dem Punkt \textit{p} zugeordnet. Dieses Verfahren wurde für jeden Randpunkt wiederholt und in der Methode \textit{ComputeVectors} implementiert. Somit wurde die Vorarbeit zur Erfüllung des ersten Kriteriums erfüllt.

Nachdem es für jeden Punkt einen Hauptrichtungsvektor und eine exakte Nachbarschaft ermittelt wurde, wurde das Region-Growing Verfahren für die Kantensegmentierung implementiert. Hierfür wurde das bestehende Region-Growing Verfahren der PCL-Bibliothek adaptiert \autocite{rusu_3d_2011}. Zuerst wurden alle Punkte mit dem Label \textit{-1} markiert, um diese als \textit{unsegmentiert} zu kennzeichnen. Das Referenzwerk deutete auf die Irreversibilität des Verfahrens hin, weswegen die Auswahl eines guten Startpunktes sehr wichtig war. Randpunkte mit einer hohen Anzahl von exakten Nachbarn konnten mit einer höheren Wahrscheinlichkeit eine Kante oder ein geometrisches Merkmal abbilden. Deswegen wurden alle Randpunkte nach einer absteigenden Anzahl von Nachbarpunkte sortiert, sodass die Auswahl eines guten Startpunktes gewährleistet wurde. Für den Zuwachs eines Segments \textit{C} wurde ein initialer Seedpunkt \textit{s\textsubscript{i}} gewählt, welcher im Falle des ersten Segments der Startpunkt war. Für jeden unmarkierten (durch \textit{-1} gekennzeichnet) exakten Nachbarpunkt \textit{n\textsubscript{s}} von \textit{s} wurde geprüft, ob dessen Hauptrichtungsvektor mit dem des Seedpunktes näherungsweise übereinstimmten. Dies erfolgte durch die Berechnung des Winkelabstands zwischen beiden Hauptrichtungsvektoren, der nicht einen Schwellwert, den sogenannten Glättungsfaktor´ $\phi$ nicht übersteigen durfte. Falls der Richtungsvektor des Nachbarpunktes mit dem des Seedpunktes übereinstimmte, wurde es dem Segment \textit{C} hinzugefügt und mit dem Index des Segments markiert. Danach wurde \textit{n\textsubscript{s}} zu einer Sammlung neuer Seedpunkte \textit{s\textsubscript{c}} hinzugefügt, die für den weiteren Zuwachs des Segments \textit{C} verwendet wurden. Nachdem alle exakte Nachbarpunkte von \textit{s\textsubscript{i}} markiert worden waren, wurde dieser Schritt für alle neue \textit{s\textsubscript{c}} wiederholt, bis es keiner Punkte mit dem Richtungsvektor von \textit{s\textsubscript{i}} übereinstimmten. Diese Schleife wurde zwecks der Wiederverwendbarkeit auf eine separate Methode namens \textit{GrowSegment} t. Danach wurde ein neuer unmarkierter initialer Seedpunkt \textit{s\textsubscript{i}} gewählt und das ganze Verfahren wiederholt. Im Anhang \ref{label} wird das Verfahren ausführlicher als Pseudocode angegeben. Die Abbildung \ref{segments_table} zeigt alle Segmente in unterschiedlichen Farben an, die durch Region-Growing Verfahren aus den Kanten des Tisches erkannt wurden.

\begin{figure}[h]
	\includegraphics[width=\textwidth]{Abbildungen/table_segments.png}
	\centering
	\caption{Diese Abbildung zeigt die, durch das Verfahren erkannte  Segmente des Tisches aus Abbildung \ref{edge_points_table}.}
	\label{segments_table}
\end{figure}

\subsection{Probleme bei der Reproduktion des AGPNs}
Bei der Reproduktion des AGPN Verfahrens tauchten ein paar Probleme auf, die entweder eine korrekte Erkennung der Kanten verhinderten oder die Performanz des Verfahrens beeinträchtigen. Das \textit{FindEdgePoints} Verfahren sollte möglichst schnell erfolgen, um für den Einsatz in der Schweißrobotik geeignet zu sein. Durch die Implementierung des Programmes in C++ erfolgte die Erkennung von Kanten schneller im Vergleich zu anderen Sprachen wie Python oder MATLAB. Allerdings wurde die Leistungsfähigkeit moderner Rechner und CPUs durch das Programm nicht völlig ausgeschöpft. Moderne Mehrkernprozessoren bieten die Funktionalität an, Aufgaben parallel auszuführen. Isolierte Rechenaufgaben, die möglichst homogen bleiben und wiederholt werden, lassen sich sehr gut parallelisieren. Die Schritte zur Bestimmung eines Randpunktes wurden in einer Schleife für jeden Punkt der Punktwolke wiederholt, weswegen sie sich zur Parallelisierung eigneten. Die Programmierschnittstelle OpenMP bot über Compiler-Befehle die Möglichkeit an, Prozesse in C, C++ und Fortran zu parallelisieren. Das Schleifenelement des Verfahrens wurde mittels OpenMP parallelisiert, da die Festlegung eines Punktes zum Randpunkt keinen Einfluss auf die Erkennung anderer Randpunkte hatte, und somit eine isolierte Aufgabe darstellte. Es musste auch sichergestellt werden, dass eine Variable in einer Speicheradresse nicht gleichzeitig durch zwei oder mehrere parallelen Instanzen der Schleife überschrieben wird. Auch Datenstrukturen mussten sorgfältig nach ihrer Eignung zur Parallelisierung gewählt werden, um Speicherlecks zu vermeiden. Auch die Berechnung des Winkels $\theta_i$ für jeden Punkt der RANSAC-Ebene \textit{E\textsubscript{N}} wurde parallelisiert. Durch diese nebenläufige Programmierung wurde auf einem Ryzen 5 3600 Prozessor \autocite{noauthor_amd_2022} mit sechs Kernen und 12 Threads und einer Basistaktrate von 3,6 GHz eine siebenfache Leistungsverbesserung beobachtet. Eine Punktwolke mit ca. 450.000 Punkte wurde vor der Parallelisierung innerhalb 162 Sekunden verarbeitet. Nach der Parallelisierung erfolgte die Kantenerkennung innerhalb 22,5 Sekunden. Es ließ sich postulieren, dass Kanten durch die Verwendung eines Prozessors mit mehr Kernen noch schneller erkannt werden könnten. Die Verwendung eines Grafikprozessors, die deutlich mehr Kernen besitzen, hätte die Leistung der Kantenerkennung für sehr großen Punktwolken enorm steigern können. Allerdings, durfte diese aufgrund der Software-Voraussetzungen in Abschnitt \ref{soft_voraus} nicht an einem Grafikprozessor delegiert werden.

Das zweite Problem bei der Reproduktion des AGPN Verfahrens tauchte auf, als das Programm auf zwei unterschiedlichen Versionen des Ubuntu Betriebssystems ausgeführt wurde. Die Kantenerkennung erfolgt auf die neuere Version des Betriebssystems - Ubuntu Jammy Jellyfish (Version 22.04) - reibungslos und lieferte sehr gute Ergebnisse. Die Wiederholung des Programms auf eine ältere Generation des Betriebssystems - Ubuntu Focal Fossa (Version 20.04) - lieferte im Gegensatz schlechtere Ergebnisse. Ein Fehler des Rechners wurde ausgeschlossen, indem der gleiche Rechner mit konstanten Spezifikationen für beide Betriebssysteme verwendet wurde. Auch der Einfluss fremder Softwarepakete auf dem Programm wurde ausgeschlossen, indem das Programm an Betriebssysteme nur mit den notwendigen Softwareabhängigkeiten ausführt wurde. Eine genauere Untersuchung lieferte den Hinweis, dass die Standardversion der PCL-Bibliothek für beide Betriebssysteme unterschiedlich war. Die PCL-Bibliotheksversion 1.10 wurde Standardweise mit Ubuntu Focal Fossa geliefert, wobei die Version 1.12 Standardweise mit Ubuntu Jammy Jellyfish geliefert wurde. Das Downsampling-Verfahren aus der Bibliotheksversion 1.10 konnte sehr dichte Punktwolken nicht korrekt verarbeiten. Dieses Fehler wurde allerdings in der neueren Version der Bibliothek behoben. Deswegen wurde für das Betriebssystem Ubuntu Focal Fossa die Standardversion der PCL-Bibliothek entfernt und die Version 1.12 installiert. Abbildung \ref{fig: pcl_version_comparision} zeigt die Randpunke und im weiteren Sinne die Kanten, die nach dem fehlerhaften Downsampling erkannt wurden.

\begin{figure}[h]
	\centering
	\begin{subfigure}[h]{0.95\textwidth}
		\includegraphics[width=\textwidth]{Abbildungen/blech_edges.png}
		\centering
		\caption{Randpunkte mit PCL 1.12}
		\label{fig: blech_edges}
	\end{subfigure}
	\hfill
	\begin{subfigure}[h]{0.95\textwidth}
		\includegraphics[width=\textwidth]{Abbildungen/blech_bad_edges.png}
		\centering
		\caption{Randpunkte mit PCL 1.10}
		\label{fig: bad_edges}
	\end{subfigure}
	\caption{Randpunkte, die durch beider Versionen von PCL erkannt wurden}
	\label{fig: pcl_version_comparision}
\end{figure}

Nachdem die Funktionsweise des AGPNs getestet wurde, erfolgte die Erweiterung des Verfahrens, um die Kantenerkennung und Segmentierung für wachsenden Punktwolken zu ermöglichen.

\section{Erweiterung des AGPNs - das IEFD}
\subsection{Erstellung der Testumgebung}
Um das Verhalten einer wachsenden Punktwolke zu simulieren und das modifizierte Verfahren zu testen wurde eine Testumgebung mittels das Softwarepaket ROS aufgebaut. Die Entscheidung für dieses Softwarepaket stammte aus der Tat, dass es bereits zur Kopplung unterschiedlicher Komponenten des Schweißroboters verwendet wurde \autocite[39]{savla_intelligente_2022}. In der Umgebung wurde ein ROS-Publisher zur Veröffentlichung der Punkten aus der Punktwolke sowie ein ROS-Subscriber zur Verarbeitung dieser Punkten entworfen. Das ROS-Publisher ließ eine Punktwolke-Datei ein, die durch einen Lasersensor aufgenommen wurde, und bereitete sie zur Veröffentlichung vor. Die gesamte Abtastung wurde so aufgeteilt, dass es die einzelnen Aufnahmen des Lasersensors entsprach. Die Einzelteile der gesamten Punktwolke wurden als Scan-Linien bezeichnet, da der Laserscanner eine streifenartige Aufnahme machte. Um den Einfluss der Robotergeschwindigkeit auch zu modellieren, wurden diese Scan-Linien mit einer Frequenz von 10 Hz publiziert. Daneben wurde der Richtungsvektor des Sensors auch ermittelt und zur Veröffentlichung bereitgestellt. Dieser Vektor gab an, in welcher Richtung der Sensor verlief und wurde für \textit{SegmentEdges} verwendet. Das ROS-Subscriber übernahm die Aufgabe der Kantenerkennung und -Segmentierung. Die freigegeben Daten aus dem ROS-Publisher wurden hier empfangen. Eine Anzahl \textit{n} der empfangenen Scan-Linien wurden gesammelt und in einer Punktwolke zusammengefasst, bevor die Kantenerkennung und Segmentierung erfolgte. Danach wurden die nächsten \textit{n} Scan-Linien verarbeitet. Diese Anzahl wurde dynamisch auf Basis des Abstands zwischen Punkten ermittelt. Die Sammlung \textit{n} Scan-Linien und deren Verarbeitung wird als eine Iteration referenziert. Innerhalb des ROS-Subscribers wurden die Methoden des reproduzierten AGPNs aufgerufen, um Kanten aus der Sammlung von Scan-Linien zu erkennen. Danach wurden die Kanten mittels des Region-Growing Verfahrens segmentiert. Diese Kanten und Segmente wurden intern gespeichert und mit neuen Kanten nach jeder Iteration erweitert. Zwecks der Datenvollständigkeit wurden die letzten \textit{k} Scan-Linien aufgespeichert und zu der nächsten Iteration hinzugefügt. Visualisiert wird das in der Abbildung \ref{fig: point_overlap}. Der Hintergrund dazu wird detaillierter in Abschnitt \ref{false_edges} erläutert.

\begin{figure}[h]
	\includegraphics[scale=0.75]{Abbildungen/points_overlap.png}
	\centering
	\caption{Diese Abbildung visualisiert die Überlappung aufeinanderfolgender Iterationen von Punkten. Zwecks der Übersichtlichkeit wurden nur Randpunkte abgebildet. Die vier Farben kennzeichnen Kanten aus jeweils unterschiedlichen Iterationen. Die roten Strichlinien weisen auf dem Ende einer Iteration hin.}
	\label{fig: point_overlap}
\end{figure}

\subsection{Anpassung der Erkennungs und Segmentierung-Verfahren}
Im folgenden werden die Änderungen zu dem AGPN-Verfahren aus Abschnitt \ref{agpn_reproduction} detailliert erläutert. Das angepasste Verfahren wird weiterhin als das \textit{IEFD}-Verfahren \textit{\(Iterative Edge and Feature Detection\)}refereiert. Die zwei Teile des AGPNs - Erkennung und Segmentierung - wurden zwecks der Anpassung als zwei getrennte Funktionen betrachtet. Die Kantenerkennung benötigte keine Änderungen, da der geometrischer Zusammenhang zwischen den Sammlungen von Scan-Linien unterschiedlicher Iterationen keine wichtige Rolle für diese Funktion spielte. Deswegen wurde entschlossen, die Funktion \textit{FindEdgePoints} ohne Änderungen für den Einsatz bei wachsenden Punktwolken zu verwenden. Im Gegensatz dazu, wurde der Bedarf an eine Anpassung des \textit{SegmentEdges} Verfahrens erkannt.

Bei \textit{SegmentEdges} spielte der geometrischer Zusammenhang zwischen Kanten der unterschiedlichen Iterationen eine wichtige Rolle. Um vorhandene Segmente mit neuen Kanten zu erweitern, waren die relativen Positionen aller Kanten zu einander wichtig. Methoden zur Hinzufügung neuer Randpunkte und im weiteren Sinne, Kanten, zu den Segmenten mussten auch bereitgestellt werden. Die Errechnung der Richtungsvektoren sowie die Bestimmung der exakten Nachbarschaft neuer Randpunkte dürfte ohne große Anpassung implementiert werden. Es musste sichergestellt werden, dass nur die neu hinzugefügten Punkte bei dieser Berechnung berücksichtigt werden. Für die Segmentierung mussten zuerst die Anzahl und Indizes der bereits erkannten Segmente für die nächsten Iterationen zur Verfügung gestellt werden. Es sollte bei jeder Iteration geprüft werden, ob neue Randpunkte zu bereits vorhandenen Segmenten hinzugefügt werden könnten. Der Bedarf an neuen Funktionalitäten hierfür wurde erkannt. Zur Erweiterung bestehender Segmente oder Cluster wurden grundsätzlich zwei Verfahren entwickelt. Beide Verfahren wurden zwecks der Übersichtlichkeit auf eine separate Methode \textit{ExtendSegment} verlagert.

Bei dem ersten Verfahren wurden neue Randpunkte darauf geprüft, ob sie vorhandene Segmente erweitern könnten. Während der Segmentierung wurden die exakten Nachbarpunkte eines Punktes \textit{p} durchgesucht, um deren Segmente zu finden. Falls einer der Nachbarpunkte bereits zu einem Segment \textit{C} gehörte, wurden die Richtungsvektoren des Segments sowie des Punktes \textit{p} verglichen. Der Richtungsvektor des Segments wurde aus der Summe der Richtungsvektoren aller Punkte des Segments ermittelt. Falls die beiden Richtungsvektoren eine hohe Kollinearität aufwiesen, wurde es versucht, den Segment \textit{C} mit diesem Punkt zu erweitern. Hierfür wurde die Methode \textit{GrowSegment} aus Abschnitt \ref{edge_segmentation} wiederverwendet. Der neue Randpunkt \textit{p} wurde als initialer Seedpunkt \textit{s\textsubscript{i}} innerhalb dieser Methode verwendet. Die Richtungsvektoren der unmarkierten exakten Nachbarpunkte von \textit{p} wurden mit dem Richtungsvektor des Segments verglichen. Die Nachbarpunkte mit übereinstimmenden Richtungsvektoren wurden zu dem Segment \textit{C} hinzugefügt und als neuer Seedpunkt \textit{s\textsubscript{C}} für die Erweiterung von \textit{C} verwendet. Falls der neue Randpunkt \textit{p} zu keinem vorhandenen Segment zugewiesen werde konnte, wurde mit ihm ein neues Segment mittels des Verfahrens nach Abschnitt \ref{edge_segmentation} erstellt. Diese Methode zur Erweiterung bestehender Segmente sollte mittels der Testumgebung überprüft werden, allerdings wurde das durch eine unerwartete Anomalie in den Randpunkten verhinderte. Diese Anomalie und die Lösung dazu wird in Abschnitt \ref{false_edges} detailliert behandelt.

Nachdem ein Verfahren zur Entfernung der anomalen Randpunkten implementiert wurde, wurde die erste Methode zur Erweiterung bestehender Segmente überprüft. Diese Methode lieferte unzureichende Ergebnisse, da neuer Randpunkte einer neuen Iteration immer zu einem neuen Segment zugewiesen wurden, obwohl sie offensichtlich zur Erweiterung älterer Segmente geeignet waren. Dieses wird in Abbildung \ref{fig: bad_segments} visualisiert. Dadurch entstand der Bedarf, das \textit{SegmentEdges} Verfahren nochmal anzupassen. Auch in dieser Variante wurde das Verfahren \textit{FindEdgePoints} nicht überarbeitet, sondern geschahen die größten Änderungen in der Methode \textit{ComputeVectors} sowie dem \textit{SegmentEdges} Verfahren. Die Aufhebung der wiederholten Randpunkte jeder Iteration war für diese zweite Anpassung beider Methoden wichtig. Die eindeutigen Indizes der wiederholten Randpunkten wurden in der Methode \textit{ComputeVectors} verwendet, um deren exakten Nachbarn sowie Richtungsvektoren erneut zu berechnen. Der Grund hierfür war es, die neuen Randpunkte aus der neuen Iteration bei der Berechnung miteinzubeziehen. Somit wurden die örtlichen Relationen zwischen den Randpunkten an den Grenzbereichen der neuen sowie älteren Iterationen etabliert. Diese geometrischen Beziehungen wurden später für die Segmentierung relevant.

\begin{figure}[h]
	\includegraphics[width = 1.0\textwidth]{Abbildungen/blech_segments_bad.png}
	\centering
	\caption{Die falsch segmentierten horizontalen Kanten}
	\label{fig: bad_segments}
\end{figure}

Das \textit{SegmentEdges} Verfahren wendete auch die wiederholten Randpunkte aus der vorherigen Iteration an, um vorhandene Segmente zu erweitern. Die Überprüfung jeder einzelnen neuen Randpunkt auf seine Eignung zur Erweiterung eines vorhandenen Segments aus der ersten Variation wurde zugunsten eines besseren und effizienteren Verfahrens ersetzt. In der neuen Variante wurden Kennzeichnungen aller bereits markierten Punkte in einer Datenstrukturkopie aufgespeichert. Die Kennzeichnung der wiederholten Randpunkte auf \textit{-1} zurückgesetzt. Auch die Kennzeichnungen aller exakten Nachbarpunkte dieser Randpunkte wurden auf \textit{-1} zurückgesetzt. Danach wurde jeder wiederholter Randpunkte \textit{p\textsubscript{r}} wiederverwendet, um sein älteres Segment zu erweitern. Hierbei wurde die \textit{GrowSegment} Methode wiederverwendet. Nach der Neusegmentierung der wiederholten Randpunkte wurden die Punkte aus der vorigen Iteration darauf überprüft, ob sie unmarkiert (mit \textit{-1} markiert). In diesem Fall wurde die originelle Kennzeichnung dieser Punkte aus der Datenstrukturkopie aufgerufen und den Punkten wieder zugewiesen. Danach wurden die übrigen unmarkierten Randpunkte aus der neuen Iteration verwendet, um neue Segmente zu wachsen. Diese Variante der Methode \textit{ExtendSegment} lieferte im Gegensatz zu der ersten Variante deutlich bessere Ergebnisse. Ausschlaggebend für diese guten Ergebnisse war auch die Behebung des Problems mit dem anomalen Verhalten der Randpunkte.

\subsection{Anomalie der falschen Kanten} \label{false_edges}
Aufgrund der Funktionsweise des \textit{FindEdgePoints} Verfahrens wurden fälschlicherweise Punkte zwischen zwei Iterationen als Kanten erkannt, obwohl sie im Kontext des Gesamtbildes keine Kanten seien sollten. Je nach Iteration gab es bis zu zwei Reihen von Punkten, die irrigerweise als Kanten markiert wurden. Die erste sowie die letzte Iterationen hatten am Ende beziehungsweise am Anfang eine Reihe Punkten, die markiert wurden. Iterationen  Der Grund hierfür lag daran, dass die Kantenerkennung nur im Umfang einer Iteration erfolgte und es keine Auskünfte über kommende Punkte verfügte. Diese falschen Kanten sind in Abbildung \ref{fig: false_edges} zu sehen. Diese falsch-markierten Kanten hatten auf das iterative \textit{SegmentEdges} Verfahren schlechte Auswirkungen und verhinderten die Erweiterung bestehender Segmente. Um diese Kanten und im weiteren Sinne, Randpunkte, zu entfernen wurden insgesamt drei Methoden entworfen, aus den letztlich eine Gewählt wurde.

\begin{figure}[h]
	\includegraphics[width = 1.0\textwidth]{Abbildungen/false_edges.png}
	\centering
	\caption{Diese Abbildung zeigt die Kanten, die durch das adaptierte Verfahren erkannt wurden. Die regelmäßigen durchquerenden Linien in der Mitte, die in Abbildung \ref{fig: blech_edges} abwesend sind, wurden fälschlicherweise als Kanten markiert. }
	\label{fig: false_edges}
\end{figure}

\subsubsection{Bounding-Box Methode}
Bei der ersten Methode wurde es versucht, mittels eines Rahmens - ein sogenanntes Bounding-Box - die falschen Kannten zu entfernen beziehungsweise zu markieren, sodass sie von dem \textit{SegmentEdges} Verfahren ausgeschlossen werden konnten. Zur Erstellung dieses Bounding-Boxes wurde das Verfahren nach \textcite{noauthor_find_2015} angewendet. Es wurde zuerst für alle neuen Punkte \textit{P} der Iteration Hauptkomponentenanalyse ausgeführt, um ihrer Eigenvektoren zu bestimmen. Diese Eigenvektoren wurden als ein referenzielles Koordinatensystem für die Punkten verwendet und wurden so transformiert, dass sie die Position und Orientierung des Weltkoordinatensystems entsprachen. Somit wurden auch die Punkte \textit{P} transformiert. Danach wurde der minimale Punkt \textit{p\textsubscript{min}} und maximale Punkt \textit{p\textsubscript{max}} der transformierten Punkten ermittelt, um eine Diagonale auszurechnen. Mittels der Diagonale, die Eigenvektoren und des Ursprung der Eigenvektoren wurde die translatorische Komponente der Rückwärtstransformation errechnet. Aus dem Verhältnis der Eigenvektoren und das Weltkoordinatensystem wurde die Orientierungskomponente der Rückwärtstransformation errechnet. Mit diesen Werten wurden \textit{p\textsubscript{min}} und \textit{p\textsubscript{max}} rückwärts transformiert, um das Minimum und Maximum des Bounding-Boxes zu ermitteln. Mittels dieser Informationen konnten alle Eckpunkte des Bounding-Boxes ermittelt werden. Abbildung \ref{fig: bounding_box} visualisiert das Bounding-Box um die Punkte einer Iteration herum.

\begin{figure}[h]
	\includegraphics[scale = 1.15]{Abbildungen/Bounding_box.png}
	\centering
	\caption{Diese Abbildung visualisiert das Verfahren zur Erkennung falschen Randpunkte bzw. Kanten mittels eines Bounding-Boxes. Die Randpunkte sind braun dargestellt. Das gelbe Rahmen um die Punkte bildet das Bounding-Box ab. Die roten gestrichelten Rechtecke stehen exemplarisch für die zweidimensionalen Regionen, die Bereiche mit falschen Kanten abbilden sollen.}
	\label{fig: bounding_box}
\end{figure}

Diese Eckpunkte wurden weiterhin zur Erkennung falschen Randpunkte verwendet. Auf Basis der Positionen der jeweiligen Ecken wurden zwei Bereiche oder Regionen definiert - am Anfang und am Ende des Bounding-Boxes. Für jeden Bereich wurden zwei zueinander vertikal stehender Punkte ausgewählt sowie die Scan-Richtung verwendet. Die Breite \textit{b} dieser Regionen durfte durch den Benutzer bestimmt werden. Um die Schritte zur Erstellung dieser Regionen einfacher zu erklären, wird die Abbildung \ref{fig: bounding_box} referenziert. Die Punkte \textit{P5} und \textit{P6} bildeten die Mittelpunkte der zwei kürzeren Seiten der Region am Anfang ab. Von diesen Punkten heraus wurden die Eckpunkte der Region in der Scan-Richtung ermittelt. Zusammen stellten die vier Eckpunkte eine zweidimensionale Region mit der Breite \textit{b} dar, die alle potenziell falschen Randpunkte umfasste. Gleichermaßen wurde mittels den Punkten \textit{P7} und \textit{P8} eine zweidimensionale Region am Ende des Bounding-Boxes errichtet. Damit standen zwei Regionen zur Verfügung, die zur Entfernung falscher Randpunkte verwendet werden konnten. Punkte, die innerhalb dieser Regionen lagen, konnten gelöscht werden. Der Benutzer durfte auch auswählen, welcher der beiden Regionen zur Entfernung Randpunkte aktiviert wurden. Um die irrtümliche Eliminierung korrekter Randpunkte zu verhindern, wurde die Scan-Richtung auch mitberücksichtigt. Punkte mit Richtungsvektoren, die dieser Richtung entsprachen, wurden nicht entfernt. 

Obwohl dieses Verfahren theoretisch funktionieren könnte, lieferte es unzureichende Ergebnisse. Die Ungenauigkeit des Bounding-Boxes wurde durch die sehr hohe Dichte der Punktwolke beziehungsweise den sehr kurzen Abstand zwischen Punkte amplifiziert. Bei einem mittleren Punktabstand von weniger als 0,1 mm führte die inhärente Ungenauigkeit des Bounding-Boxes dazu, dass übermäßig viele Randpunkte entfernt wurden. Darüber hinaus kosteten die Transformation der Punktwolke und weiteren Berechnungen zusätzliche Rechenleistung und eigneten sich zu zeitintensiven Operationen nicht, weil sie eine Zeitkomplexität deutlich höher als $O(n)$ aufwiesen. Deswegen wurde ein anderes Verfahren überlegt.

\subsubsection{Methode mit der Position des Scanners}
Bei diesem Verfahren wurde die Position sowie das Koordinatensystem des Sensors während der Abtastung ausgenutzt. Es sollten ähnlich wie das vorige Verfahren zwei Regionen definiert werden, die alle falschen Kanten umfasst. Es wurden die Positionen des Sensors am Anfang und am Ende jeder Iteration aufgespeichert, um die Positionen der falschen Randpunkte an den jeweiligen Seiten zu bestimmen. Das Koordinatensystem des Sensors wurde zur Orientierung der Regionen verwendet. Die x-Achse wurde als Stützvektor der Tiefe, die y-Achse als Stützvektor der Breite und die z-Achse als Stützvektor der Höhe verwendet. Die y-Achse des Sensors entsprach seinem Richtungsvektor. Zur Dimensionierung der Regionen wurden die Sensorspezifikationen verwendet. Der Lasersensor dieser Arbeit konnte 290 mm in der x-Richtung und 460 mm in der z-Richtung scannen. Auf diesen Werten wurde ein zusätzlicher Puffer addiert, sodass die Regionen möglichst alle falschen Randpunkte umfassten. Der Ausmaß für die Breite dieser Regionen durfte wie vorhin durch den Benutzer angegeben werden. Beim Testen dieses Verfahrens wurde jedoch enthüllt, dass die Sensorposition nicht immer Vertikal über das Bauteil lag, sondern meistens abgesetzt und in einer anderen Orientierung zu ihm lag. Als Lösungsansatz konnten die Randpunkte nach dem Koordinatensystem des Sensors transformiert werden, allerdings hätte diese wiederholte Berechnung für jeden Randpunkt einer Iteration zu einer Zeitkomplexität deutlich über $O(n)$ sowie zu einer Verlangsamung des gesamten Verfahrens geführt. Zugunsten eines einfacheren und präziseren Verfahren zur Entfernung falscher Randpunkte wurde dieses Verfahren nicht weiter entwickelt.

\subsubsection{Methode auf Basis der Reihenfolge der Scan-Linien}
Diese Methode bot die simpelste Variante an, falsche Randpunkte zu entfernen, indem die Reihenfolge der einzelnen Scan-Linien ausgenutzt wurde. Dank der Funktionalität von ROS-Publishers und Subscribers war es bewusst, dass die Reihenfolge der veröffentlichten Daten auch bei dem Empfang beibehalten wurde. Somit hatte die erste Scan-Linie eine Kante am Anfang der Iteration gebildet, während die letzte Scan-Linie der Iteration eine Kante am Ende bildete. Nachdem \textit{n} Scan-Linien gesammelt und zusammengefasst wurden, wurden die Indizes der Punkten aus den ersten und letzten paar Scan-Linien bemerkt. Die Anzahl der Scan-Linien, die zu den Ersten oder Letzten zählten, wurde dynamisch auf Basis des Abstands zwischen Punkten sowie die Größe der jeweiligen Iterationen bestimmt. Bei der Bestimmung von Randpunkten wurden diese Indizes der Methode übergeben. Nachdem die Methode alle Randpunkte erkannt hatte, wurde eine Vergleichsoperation ausgeführt. In dieser Operation wurden die Indizes der ersten oder letzten paar Scan-Linien mit den Indizes der ermittelten Randpunkte vergleicht, und bei einer Übereinstimmung der Indizes wurde der entsprechende Punkt aus der Liste Randpunkte entfernt. Dem Benutzer wurde die Auswahl ermöglicht, nur Randpunkte am Anfang, am Ende oder an beiden Seiten zu entfernen.

Diese Methode zur Entfernung falscher Randpunkte und im weiteren Sinne, Kanten, bot den besonderen Vorteil an, dass es die höchste Zeitkomplexität von nur $O(n)$ hat. Die Größe des Arrays hat einen Einfluss auf die Zeitkomplexität aufgrund der Verwendung von Schleifen. Die Operationen innerhalb der Schleifen bestanden aus Zugriffsoperationen auf Arrays sowie Suchoperationen auf Hashtabellen, die eine geringe Zeitkomplexität von $O(1)$ hatten. Daneben wies diese Methode auch eine sehr hohe Genauigkeit bei der Entfernung falscher Kanten auf. Aus diesen Gründen wurde final dieses Verfahren gewählt.

Zuerst wurden die Filterverfahren aus Abschnitt \ref{edge_detection_reprod} bei der Verwendung dieser Methode weggelassen, da die Komplexität dadurch gestiegen hätte. Da die Anzahl der Punkte durch die Filterverfahren sanken, galten die Indizes der Punkten vor der Filterung nicht mehr. Allerdings wurde festgestellt, dass ein unregelmäßiger Abstand zwischen Punkten die Kantenerkennung deutlich beeinträchtigte. Daneben stellten Punktwolken mit einer sehr hohen Dichte auch ein Problem vor, da diese durch das Verfahren zur Kantenerkennung nicht vernünftig verarbeitet werden konnten. Aus diesen Gründen mussten die Implementierungen beider Filterverfahren angepasst werden. Im nächsten Abschnitt werden die Schritte zur Integration dieser Verfahren aufgeleuchtet. 

\subsection{Implementierung von Filterverfahren}
Das Verfahren \textit{UniformSampling} verwendete ein Voxel-Grid um die Zentroide von Punkten eines Leafs zu bestimmen. Es wurde danach der Punkt mit dem kürzesten Abstand zum Zentroid bestimmt und der Reste an Punkte des Leafs verworfen. Bei der Methode \textit{StatisticalOutlierRemoval} wurde zu einem Punkt \textit{p} die \textit{N} nächsten Nachbarpunkte gefunden und deren mittleren Abstand zum \textit{p} errechnet. Danach wurden aus diesen Nachbarpunkten alle Punkte entfernt, die weiter als der mittlere Abstand plus eine \textit{x}-fache der Standardabweichung lagen. Bei der Implementierung dieser Verfahren wurden neben den Inliers der Verfahren auch die Ausreißer eingeholt. Zwecks der Übersichtlichkeit wurde die vorgefilterte Punktwolke \textit{C\textsubscript{vor}} und die gefilterte Punktwolke \textit{C\textsubscript{nach}} genannt. Nach Identifizierung der Ausreißer wurden alle Punkte von \textit{C\textsubscript{vor}} markiert, die entfernt wurden. Danach wurde errechnet, in welchem Zusammenhang die Indizes der Inliers von \textit{C\textsubscript{vor}} zu dem der Punkte von \textit{C\textsubscript{nach}} standen. Dieser Zusammenhang ergab sich aus der Differenz der Indizes von den Inliers beziehungsweise Punkten der beiden Punktwolken und wurde als der \textit{Punkt-Versatz} referiert. Das Algorithmus \ref{alg: mark_points} zeigt den Verlauf der Markierung und Berechnung der \textit{Punkt-Versätze}. Diese Markierung der Ausreißer sowie der \textit{Punkt-Versatz} wurden verwendet, um die Indizes der ersten paar Scan-Linien sowie der letzten paar Scan-Linien für die neue Punktwolke \textit{C\textsubscript{nach}} anzupassen. Darüber hinaus wurden auch die Indizes der \textit{k} wiederholten Punkten für die nächste Iteration korrigiert und für den Einsatz in dem \textit{SegmentEdges} Verfahren bereitgestellt. 

\begin{algorithm}
	\caption{Das Verfahren zum Korrigieren der Punktindizes}
	\label{alg: mark_points}
	\begin{algorithmic}[1]
		\Function{MarkPoints}{removed\textunderscore indices}
		\State $\textit{removed\textunderscore indices\textunderscore map} \gets \text{\{\}}$
		\State $\textit{removed\textunderscore indices\textunderscore map} \gets \text{\textbf{size of} \textit{C\textsubscript{vor}} \& \textbf{default} \textit{false}}$
		\For{\textit{index} \textbf{in} \textit{removed\textunderscore indices}}
		\State $\textit{removed\textunderscore indices\textunderscore map} \left[index\right] \gets \textit{true}$
		\EndFor
		\State $\textit{point\textunderscore shifts} \gets \text{\{\}}$
		\State $\textit{point\textunderscore shifts} \gets \text{\textbf{size of} \textit{C\textsubscript{vor}} \& \textbf{default} 0}$
		\State $\textbf{int} \textit{sum} \gets 0$
		\For{$i < \text{\textbf{size of} \textit{C\textsubscript{vor}}}$}
		\If{$\text{\textit{removed\textunderscore indices\textunderscore map}[i]} = \textit{false}$}
		\State $\textit{point\textunderscore shifts}\left[i\right] \gets \textit{sum}$
		\Else
		\State $\textit{sum} \gets \textit{sum} + 1$
		\EndIf
		\EndFor
		\EndFunction
	\end{algorithmic}
\end{algorithm}

Neben der Korrektur der Indizes der \textit{k} wiederholten Punkten nach dem Filtern mussten zusätzliche Korrekturmaßnahmen ergriffen werden. Die wiederholten Punkten mussten nach der Kantenerkennung wiederholt darauf geprüft werden, ob sie Teil der erkannten Randpunkte beziehungsweise Kanten waren. Danach mussten die Punktindizes der wiederholten Punkten angepasst werden, um die Punktwolke lediglich mit Kanten zu entsprechen, die am Ende der \textit{FindEdgePoints} Methode zurückgegeben wurde.

Somit wurden alle nötigen Schritte beendet, um ein funktionierendes Prototyp für die Erkennung und Segmentierung von Kanten in wachsenden Punktwolken zu erstellen. Um die Einsatzfähigkeit des IEFD-Verfahrens zu bewerten musste das Prototyp auf gewissen Kriterien überprüft werden. Die Überprüfung der Leistung des Prototyps nach diesen Kriterien erfolgt in dem nächsten Kapitel.