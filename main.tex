%\documentclass[]{report}
%\usepackage[ngerman]{babel}
%\usepackage{hyphenat}
%\hyphenation{Mathe-matik wieder-gewinnen}



% Title Page
%\title{Ingenieurmäßige Arbeit}
%\author{Eshan Savla}


%\begin{document}
%\maketitle

%\begin{abstract}
%\end{abstract}

%\tableofcontents
%\vspace{2cm}

%\chapter{Einleitung}
%Industrieroboter werden primär in der Industrie für die Aufgabe der Handhabung, Montage oder Verarbeitung von Werkstücken und Teile eingesetzt. Häufig finden Industrieroboter als Schweißroboter, Lackierroboter oder Montageroboter Anwendung. Diese Art der Roboter sind meistens nicht autonom und müssen bei Bedarf für ihre Aufgabenausführung an neuen Produkte oder Werkstücke angepasst und umprogrammiert werden.


%\end{document}          


\documentclass[
	fontsize=12pt, 		% Schriftgröße
	BCOR=8mm,			% Bindekorrektur (wird nicht benötigt, da Seitenrand definiert ist)
	DIV=calc,			% Satzspiegel automatisch berechnen
	ngerman, 			% für Umlaute, Silbentrennung etc.
	a4paper, 			% Papierformat
	oneside, 			% zweiseitiges Dokument (oneside - einseitig, z. B. für Bachelor-,Studien-, oder Masterarbeit)
	titlepage, 			% es wird eine Titelseite verwendet
	parskip=half, 		% Abstand zwischen Absätzen (halbe Zeile)
	headings=normal, 	% Größe der Überschriften verkleinern
	listof=totoc, 		% Verzeichnisse im Inhaltsverzeichnis aufführen
	bibliography=totoc, % Literaturverzeichnis im Inhaltsverzeichnis aufführen
	index=totoc, 		% Index im Inhaltsverzeichnis aufführen
	final, 				% Status des Dokuments (final/draft)
	numbers=noenddot,	% keine abschließenden Punkte bei Kapitelzahlen
	footheight=30pt,	% Höhe des Raumes für Fußnoten
	headheight=30pt,]{scrbook}
% PDF-einbinden ----------------------------------------------------------------
\usepackage{pdfpages} % FH-Dortmund Logo
\usepackage[ngerman]{babel} % für deutsche Sprache
\usepackage{csquotes} %Anführungszeichen
\usepackage[backend=biber,style=authoryear,sorting=nyt,autocite=inline, maxnames=2, minnames=2]{biblatex}%Literaturverzeichnis
\usepackage{amsmath} % matritzen
\usepackage{tabularx} % mehrzeilige Tabelle
\usepackage[a4paper]{geometry} %Seitengeometrie
\usepackage{changepage} %ändert Seiten Einstellung
\usepackage{lipsum}
\usepackage{lscape} %Querformat
\usepackage{acronym} % Abkürzungsverzeichnis
\usepackage{amssymb} % Symobl reelle Zahlen
\usepackage{placeins} %behalte Bilder in der zugehörigen Section
\usepackage{makecell} % Zeilenumbruch in Tabelle
\renewcommand{\cellalign}{tl}
\usepackage[T1]{fontenc}% wichtig für Trennung von Wörtern mit Umlauten
\usepackage{microtype}% verbesserter Randausgleich
\usepackage{booktabs} % schönere Tabellen
\usepackage{caption} % Bild aus Abbildungsverzeichnis entfernen
\usepackage{subcaption}
\usepackage{lmodern}
\usepackage{graphicx}
\usepackage{blindtext}
\usepackage{siunitx}
\usepackage{float}
\usepackage{listings}
\usepackage{xcolor}
\usepackage{algorithm}
\usepackage[noend]{algpseudocode}
%\usepackage{abstract}
%\usepackage[authoryear]{natbib}
% Meta-Informationen -----------------------------------------------------------
%   Definition von globalen Parametern, die im gesamten Dokument verwendet
%   werden können (z.B auf der Titelseite etc.).
%          
% ------------------------------------------------------------------------------
\newcommand{\iatitle}{Die intelligente Schweißrobotik}
\newcommand{\engltitle}{Intelligent welding robots}
\newcommand{\beschreibung}{Eine Untersuchung der Anforderungen und Funktionsweise eines autonomen Schweißroboters}
\newcommand{\autor}{Eshan Savla}
\newcommand{\Matrikelnr}{7203288}
\newcommand{\Erstpruefer}{Prof. Dr.-Ing. Dennis Ziegler}
\newcommand{\Zweitpruefer}{Max Daiber-Huppert, M.Sc.}
%\bibliography{Literatur/Quellen.bib}
\addbibresource{Literatur/Quellen.bib}
\DefineBibliographyStrings{german}{andothers = {et al.}}
\setlength\bibitemsep{1.5\itemsep}  % Spacing between bib entries

\hypersetup{
	pdftitle={Bachelorarbeit-Savla},
	pdfauthor={Eshan Savla},
	colorlinks=false,
	}

\algrenewcommand\textproc{}% Used to be \textsc

%%colour and style settings code blocks
%\definecolor{codegreen}{rgb}{0,0.6,0}
%\definecolor{codeteal}{rgb}{0,0.5,0.5}
%\definecolor{codeyellow}{rgb}{0,1,1}
%\definecolor{codedandelion}{rgb}{1,0.84,0.4}
%\definecolor{codepurple}{rgb}{0.79,0.17,0.57}
%\definecolor{codeblue}{blue}{0,0.43,0.72}
%
%\lstset{emph={%
%		class, def%
%	 	}
%}
%
%\lstdefinestyle{pythonstyle}{
%	commentstyle=\color{codegreen},
%	keywordstyle=\color{codeblue},
%	stringstyle=\color{color}
%	
%}
\geometry{paper=a4paper,left=30mm,right=20mm,top=30mm,bottom=30mm}
\onehalfspacing
\RedeclareSectionCommand[afterindent=false,beforeskip=-\topskip]{chapter}

%Worttrennung
\babelprovide[hyphenrules=ngerman-x-latest]{ngerman}
\hyphenation{Nach-bar-punkte fälsch-lich-er-weise falsch-mark-ierten Stand-ard-ab-weichung vor-ge-filterte Proto-typen-phase Exakti-fizierung Kanten-erkennung Kanten-segmentierung Pro-gramm-ier-schnitt-stelle Winkel-abstand Winkel-abstände Winkel-abstands Rand-punkten Rand-punkt Rand-punkte Hard-ware-beschleunigung Punkte-abstand Punkte-abstände Punkt-abständen}

\begin{document}
	\begin{titlepage}
% Titelseite ohne Seitenzahl
\vspace{2cm}

\begin{center}
\large\textbf{Fachhochschule Dortmund \\}
\vspace{2cm}
\large{Fachbereich: Maschinenbau \\}
\large{Studiengang: Maschinenbau (B.Eng.)}
\vspace{2cm}

\includegraphics[width=10cm]{"Abbildungen/FH_Dortmund-logo".png}  %Logo der FH einfügen
\vspace{1cm}

\large{\textbf{\iatitle}}\\
\vspace{0.5cm}
\large{\beschreibung}

\vspace{1.5cm}

\begin{tabular}{ll}
Erstprüfer: & \Erstpruefer \\
Zweitprüfer: & \Zweitpruefer \\
\end{tabular}

\vspace{1.5cm} 

\large{Vorgelegt von\\
\textbf{\autor}\\
Matrikelnummer: \Matrikelnr\\
am \today}
\end{center}
\end{titlepage}
	\pagenumbering{Roman}
	\setcounter{page}{2}
	%\chapter*{Eidesstattliche Erklärung}

Hiermit erkläre ich, dass ich die vorliegende Arbeit eigenständig und ohne fremde Hilfe angefertigt habe. Textpassagen, die wörtlich oder dem Sinn nach auf Publikationen oder Vorträgen anderer Autoren beruhen, sind als solche kenntlich gemacht. Die Arbeit wurde bisher keiner anderen Prüfungsbehörde vorgelegt und auch noch nicht veröffentlicht.
\vspace{0.5cm}

Stuttgart, \today

\vspace{0.5cm}

\_\_\_\_\_\_\_\_\_\_\_\_\_\_\_\_

Eshan Savla


%	\section*{\centering Zusammenfassung}
Die Erkennung von geometrischen Merkmalen eines Objektes, während es durch einen optischen Sensor abgetastet wird, hat für die sensor-basierte Programmierung von Robotern eine hohe Relevanz. In dieser Arbeit wird ein numerisches Verfahren für diesen Zweck vorgestellt. Das Ziel dieser Arbeit ist die Bewertung der Effektivität des Verfahrens. Dazu ergänzend wird diese Forschungsfrage gestellt: Wie Effektiv ist ein numerisches Verfahren bei der Kantenerkennung und Segmentierung von wachsenden Punktwolken? Zuerst wurde das AGPN-Verfahren aus der Literatur reproduziert sowie für den Einsatzzweck angepasst und erweitert. In dieser Arbeit werden die Schritte zur Erweiterung des Verfahrens detailliert beschrieben sowie auf Methoden zur Entfernung falsch-detektierten Kanten eingegangen. Zur Beantwortung der Forschungsfrage wurden drei Untersuchungen konzipiert. Konkret wurde die Genauigkeit des Verfahrens unter verschiedenen Bedingungen überprüft und auf Basis der erkannten und segmentierten Kanten bewertet. Hierfür wurden reelle sowie synthetische Objekte verwendet. Diese Arbeit lieferte das Ergebnis, dass das Verfahren zu einer hohen Genauigkeit Kanten von geometrischen Merkmalen erkennen und segmentieren kann. Daneben wurden auch wichtige Erkenntnisse über die Schwachstellen und Grenzen des Verfahrens gewonnen. Schließlich werden aufbauende Forschungsmöglichkeiten im Bereich der Robotik und Automatisierung besprochen.

\section*{\centering Abstract}
The recognition of geometric features of an object while being scanned by an optical sensor is of high relevance for sensor-based robotics programming. This paper presents a numerical method for this purpose and aims to evaluate its effectiveness. The research question posed is: How effective is a numerical method for edge detection and segmentation in growing point clouds? Initially, the AGPN Method from current literature was reproduced before being adapted and extended for the specific application. The steps taken to extend the method are detailed in this paper, including methods for removing falsely detected edges. Three studies were designed to answer the research question, where the accuracy of the method was tested under different conditions and evaluated based on detected and segmented edges. Real and synthetic objects were used for this purpose. The result showed that the method can detect and segment edges of geometric features with high accuracy, while also highlighting weaknesses and limitations. Finally, future research opportunities on the basis of this method in the field of robotics and automation are discussed.

	\newpage
\setuptoc{toc}{totoc} %Inhaltsverzeichnis im Inhaltsverzeichnis
\tableofcontents{}
\newpage
%\addchap{Abkürzungsverzeichnis}
%Führe lieber keine Abkürzungen ein, wenn du sie nicht mindestens dreimal verwendest. Das gilt allerdings nicht für Fachausdrücke, die standardmäßig abgekürzt werden.

%Deine nicht allgemein gebräuchlichen Abkürzungen solltest du im Abkürzungsverzeichnis einführen.
%\begin{acronym}[Grundlagen]
%\acro{tcp}[TCP]{Tool-Center-Point}
%\acro{cobots}[Cobot]{kollaborierender Roboter}
%\acro{ros}[ROS]{Robot Operating System}
%\acro{ur}[UR]{Universal Robots}
%\end{acronym}
\newpage
%\listoffigures
%\listoftables
	
	\pagenumbering{arabic}
	\chapter{Einleitung}
Moderne Technologien wie das maschinelle Sehen finden in der heutigen Ära weit und breit in diversen industriellen und technischen Sektoren eine Anwendung. Optische Sensoren werden in diversen Prozessen eingesetzt, um Informationen aufzunehmen und zu verarbeiten, sodass darauf basierend (autonom) Entscheidungen über die Prozessreglung getroffen werden können. Einer der häufigsten Anwendungsfälle für optischen Sensoren und das maschinelle Sehen ist die Qualitätskontrolle. Hierzu werden Bilder aufgenommen und durch spezielle Verfahren verarbeitet, um beispielsweise Defekte zu erkennen. \autocite[3-11]{beyerer_machine_2015}

Maschinelles Sehen wird nicht nur auf die Auswertung und Verarbeitung zweidimensionaler Bilder beschränkt, sondern lässt es sich auch auf dreidimensionale Abbildungen anwenden \autocite{biegelbauer_model-based_2010}. Um solche Abbildungen zu erstellen werden Objekte häufig mit bestimmten Sensoren wie Lasersensoren abgetastet, um ihre Oberflächen in einem dreidimensionalen virtuellen Raum abzubilden \autocite[20-22]{savla_intelligente_2022}. Diese Aufnahmen werden in Punktwolken gespeichert, die alle Oberflächen des Objektes mit einer Vielzahl an dreidimensionalen Punkten modellieren. \Textcite{lougheed_3-d_1988} stellen Abbildungssysteme und Verarbeitungsverfahren für dreidimensionalen Abbildungen vor, die in der Robotik angewendet werden können. Es handelt sich dabei um die Entwicklung eines Verfahrens zur Erkennung von Flächen, wo Gegenstände durch den Roboter gegriffen werden sollten. 

Neben Flächen setzen sich Objekte aus mehr dreidimensionalen Merkmalen zusammen und die Erkennung solcher Merkmale eines Objektes hat viele relevante Anwendungen in der Industrie. Eine Teildisziplin des dreidimensionalen maschinellen Sehens beschäftigt sich mit der Erkennung von Kanten. Zum heutigen Stand sind in der Literatur diverse Beiträge zu finden, die unterschiedlichen Verfahren zur Erkennung dieser Kanten in Punktwolken vorstellen.

\section{Stand der Technik} \label{Stand_der_Technik}
Grundsätzlich bestehen zwei Gattungen von Verfahren zur Kantenerkennung, die unterschiedliche Stärken und Schwächen aufweisen. Die meisten Vorschläge für die Kantenerkennung machen sich entweder numerischen Methoden oder neuronalen Netzen zunutze.

\Textcite{hu_jsenet_2020} schlagen ein neuartiges Verfahren zur Kantenerkennung sowie Oberflächensegmentierung vor. Es wird als Eingabe eine Punktwolke vorausgesetzt und gleichzeitig durch zwei unterschiedlichen Operationen verarbeitet. Bei der ersten Operation werden mittels eines faltenden neuronalen Netzwerk (englisch: Convolutional Neural Network oder CNN) die Oberflächen der Punktwolke segmentiert. Bei der Segmentierung handelt es sich um die Gruppierung aller gleichartigen Punkte. Im zweiten Datenstrom werden die Kanten erkannt, die eine Punktwolke repräsentieren können. Schließlich werden beide Datenströme zusammengeführt, um die Informationen beider Ströme zu kombinieren. \Textcite{bazazian_edc-net_2021} schlagen auch ein Verfahren vor, das mittels eines neuronalen Netzes alle Randpunkte der Punktwolke klassifiziert. \Textcite{himeur_pcednet_2021} schlagen ähnlich ein Verfahren vor, welches auch ein CNN anwendet, um Kanten zu erkennen. 

\Textcite{choi_rgb-d_2013} schlagen in ihrem Werk ein neuartiges Verfahren zur Kantenerkennung in dreidimensionalen Farbbilder oder RGB-D Bilder vor. Diese Bilder haben neben den drei gewöhnlichen Farbkanäle auch einen vierten Kanal, welcher die Entfernung jedes Pixels von dem Sensorursprung angibt. Zur Erkennung der Kanten werden große Sprünge oder Unterbrechungen in der Tiefe des Bildes gesucht und die entsprechenden Pixel als Kanten erkannt. Darüber hinaus werden zweidimensionale Kantenerkennungsverfahren auf die Bildkomponente angewendet, um Kanten zwischen Flächen mit einer starken Krümmung zu erkennen. 

\Textcite{mineo_novel_2019} schlagen ein neuartiges Verfahren zur Kantenerkennung in unorganisierten Punktwolken, deren Punkte im Vergleich zu organisierten Punktwolken wie RGB-D Bilder nicht in einer vordefinierten Anordnung zu einander gespeichert werden. Das vorgeschlagene Verfahren bestimmt Randpunkte mittels der Analyse einer Nachbarschaft von Punkten innerhalb einer Sphäre. Das Verfahren ist in der Lage, konvexe und konkave Kanten zu erkennen, sowie den Umfang des Objekts nahtlos nachzuzeichnen.

\Textcite{lu_fast_2019} präsentieren ein Verfahren zur Kantenerkennung, welches etablierte Methoden wie Segmentierung, kleinste Quadrate (englisch: Least Mean Squares) und Projektion verwendet. In diesem Verfahren werden zuerst die Oberflächen der Punktwolke segmentiert. Danach werden alle Punkte der jeweiligen Oberflächen auf eine Ebene projiziert, um mittels der kleinsten Quadrate Methode die Kanten zu erkennen. Diese Kanten werden wieder zurück in die dritte Dimension projiziert.

\Textcite{ahmed_edge_2018} verwenden ein statistisches Verfahren zur Ermittlung eines mittleren Punktes oder eines Centroids einer Nachbarschaft von Punkten. Liegt der Abstand eines Punktes der Nachbarschaft über einen statistisch bestimmten Schwellwert zu dem Centroid, wird der Punkt als einen Randpunkt klassifiziert. So ermittelt das Verfahren nach den Autoren statistisch die Kanten. Es wird ein zusätzliches Verfahren zur Erkennung von Eckpunkten aus Randpunkten auf Basis der Krümmung der Randpunkte vorgeschlagen.

\Textcite{ni_edge_2016} legen ein Verfahren zur Erkennung und Segmentierung von Kanten dar. Die Kantenerkennung erfolgt durch die Bestimmung lokaler Nachbarschaften von Punkten und die Analyse der Winkelabstände zwischen den zugehörigen Punkte. Die Kantensegmentierung erweitert das Verfahren durch das sinnvolle Clustern von Randpunkte, die eine hohe Kollinearität aufweisen und lückenlos an einander anschließen. 

\section{Motivation und Zielsetzung} \label{Motivation}
Bemerkenswert ist die Tatsache, dass sehr viele Beiträge in der Literatur verschiedene neuartige Methoden zur Kantenerkennung und Segmentierung für offline Fällen präsentieren, während der Bereich der Kantenerkennung in online Fällen nach aktuellem Kenntnisstand weitgehend unerforscht blieb. Die Gewinnung von Informationen über Kanten in Punktwolken, während sie nebenbei wachsen, hat für viele Anwendungsgebiete eine große Bedeutung. Die Arbeit von \textcite{savla_intelligente_2022} untersucht eine Implementierung eines \textit{kognitiven} Schweißroboters, der die Schweißnaht des Bauteils während des Schweißvorgangs mittels eines Lasersensors bestimmt und verfolgt. Eine aktuelle Limitation dieses Roboters ist die fehlende Wahrnehmung der Bauteilgeometrie während des Schweißvorgangs. Mittels einer online Kantenerkennung wäre der Schweißroboter nicht nur in der Lage, Erkenntnisse über die Bauteilgeometrie zu sammeln, sondern auch die Laufbahnplanung und Prozessparameter dementsprechend anzupassen.

Für den Einsatz in einem automatischen Schweißvorgang des obigen Falls muss die Kantenerkennung einige Voraussetzungen erfüllen. Das wichtigste Kriterium für die Kantenerkennung ist die hohe Genauigkeit und Robustheit des Verfahrens. Das Verfahren soll unter reellen Bedingungen zuverlässige Ergebnisse liefern, um möglichst genau die Schweißnaht zu bestimmen sowie Hindernisse zu erkennen. Zur korrekten Erkennung von Kehlnähte soll das Verfahren auch Innen- sowie Außenkanten erkennen können. Die Anwendung des Verfahrens darf nicht nur auf organisierten Punktwolken beschränkt sein, sondern auch unorganisierten Punktwolken. Die online Funktionalitäten des Roboters setzen voraus, dass das Verfahren performant und in einer kurzen Zeitspanne die Kanten erkennen soll. Letztlich wird die Hardwarebeschleunigung mit einem Grafikprozessor ausgeschlossen, um Konflikte mit dem Echtzeitkernel der Robotersoftware zu verhindern. 

Im Rahmen dieser Arbeit wird angestrebt, ein Verfahren zur Erkennung der Kanten geometrischer Merkmale in wachsenden Punktwolken unter Einhaltung der obigen Anforderungen vorzulegen. Aus der Vielzahl der Verfahren aus Abschnitt~\ref{Stand_der_Technik} wird das AGPN-Verfahren (englisch: Analysis of Geometric Properties of Neighborhoods) nach \textcite{ni_edge_2016} als Grundlage für die online Kantenerkennung gewählt und mit weiteren Funktionalitäten für die Online-Erkennung erweitert. Für ihre Wahl über anderen Verfahren der Literatur hat die Methode ihre hohen Genauigkeit zu verdanken, insbesondere bei der Trennung zwei naheliegender Kanten. Darüber hinaus bietet das Verfahren der Autoren zur Kantensegmentierung eine elegante Möglichkeit an, zwischen den Kanten eines Objektes zu unterscheiden. Da das Verfahren auf numerischen Methoden zurückgreift und keine neuronale Netze verwendet, muss kein rechenaufwändiges Training mit einem großen Datensatz auf einem Grafikprozessor durchgeführt werden.  

Es wird im Laufe dieser Arbeit der Frage nachgegangen, wie ein nummerisches Verfahren aus der Literatur für die Kantenerkennung und Segmentierung für eine Anwendung auf wachsenden Punktwolken in der Online-Erkennung erweitert werden kann. Dabei ist das Ziel dieser Arbeit die Überprüfung der Effektivität eines solchen Verfahrens. Um die vagen Rahmen dieser Forschungsfrage zu konkretisieren wird sie in drei weiteren Forschungsfragen unterteilt. Im Umfang dieser Frage wird hinterfragt, zu welcher Genauigkeit das Verfahren Kanten erkennen kann, welcher Einfluss die Punktdichte auf die Genauigkeit hat und wie Robust das Verfahren gegen Unregelmäßigkeiten ist. Kurzgefasst sind dies die Forschungsfragen:

\begin{itemize}
	\item Wie Effektiv ist ein numerisches Verfahren bei der Kantenerkennung und Segmentierung von wachsenden Punktwolken?
	\begin{itemize}
		\item Zu welcher Genauigkeit kann ein solches Verfahren Kanten erkennen und segmentieren?
		\item Welcher Einfluss hat die Punktdichte auf die Genauigkeit?
		\item Wie Robust ist das Verfahren gegen Unregelmäßigkeiten?
	\end{itemize}
\end{itemize}

Anhand quantitativen Metriken wird das Verfahren nach den Richtlinien der drei Forschungsfragen ausgewertet. Dabei werden sowohl künstlich erzeugte Ground-Truth Dateien verwendet, als auch reelle Aufnahmen von echten Bauteilen. Durch die Verwendung eines Ground-Truths können die Ergebnisse der jeweiligen Untersuchungen mit höchster Zuversicht ausgewertet werden, während die reellen Aufnahmen eine Beurteilung der \textit{tatsächlichen} Effektivität bei der Anwendung unter reellen Bedingungen ermöglichen werden. Die quantitativen Ergebnisse der Untersuchungen ermöglichen auch eine vernünftige Gegenüberstellung des Verfahrens, um seine Genauigkeit mit anderen Verfahren aus der Literatur zu vergleichen. 

\section{Aufbau und Struktur dieser Arbeit}
Im folgenden Kapitel werden die nötigen theoretischen Grundlagen zur Methodik besprochen. Hierbei werden wichtige Kenntnisse über allgemeine Datenstrukturen und Algorithmen vermittelt. Darüber hinaus werden besondere Datenstrukturen und Algorithmen detaillierter behandelt, die in der Methodik eine Anwendung finden. 

In der Methodik wird darauf eingegangen, wie auf einem aktuell vorhandenen Verfahren aufgebaut wird, um sie mit neuen Funktionalitäten zur Kantenerkennung in wachsenden Punktwolken zu bereichern. Dabei wird zuerst behandelt, wie das Verfahren aus der Literatur nachgestellt wird. Danach werden die konkreten Schritte zur Erweiterung des Verfahrens genannt. Dabei werden konkrete Maßnahmen vorgestellt, die die Leistung oder Genauigkeit des Verfahrens verbessert haben. 

Im nächsten Kapitel wird die Genauigkeit und Einsatzfähigkeit des Verfahrens überprüft. Hierzu werden anhand der Forschungsfrage sowie der drei Teilforschungsfragen Tests konzipiert, die gezielt die Genauigkeit und Robustheit des Verfahrens ausmessen und darlegen. Neben der Beschreibung der drei Tests werden auch Testdateien und Auswertungsmetriken vorgestellt, die für die Tests verwendet werden. Unter den Testdaten wird eine künstliche Ground-Truth Datei sowie vier Aufnahmen reeller Bauteile vorgestellt. 

Als Vorletztes werden die Methodik und Ergebnisse dieser Arbeit diskutiert. Zuerst werden diese zusammengefasst und die wichtigsten Erkenntnisse aus den jeweiligen Kapiteln nochmals genannt. Darüber hinaus werden besondere Erkenntnisse der Methodik genannt, welcher zu einer Leistungsverbesserung geführt haben. Es werden die Ergebnisse der drei Tests interpretiert. Letztlich werden Limitationen dieser Arbeit diskutiert, die die Entwicklung des Verfahrens sowie seine Auswertung beeinflusst haben. 

Zuletzt wird der Beitrag dieser Arbeit zur Forschung zusammengefasst und über die Zukunftspotentiale hinsichtlich weiterer Untersuchungen sowie Erweiterungsmöglichkeiten diskutiert.
	%Datenstrukturen und Algorithmen
%	Was sind Algorithmen und Datenstrukturen
% 	Arten von Basis Datenstrukturen
%		Vector/Liste
%		Queue(Warteschlange)
%		Stack??
%		map und hash tabellen
%	besondere Datenstruktur - PCL Punktwolke
%	
%	Komplexität und Notation von Algorithmen
%	Arten von Basis Algorithmen
%		
% 		
%	besondere Algorithmen:
%		kd-tree
%		octree(vielleicht)
%		RANSAC
%		Segmentierung

\chapter{Theorie}
Im Kern dieser Arbeit steht eine Rechenaufgabe an. Die positionellen Informationen über Objekte und Bauteile müssen sinnvoll verarbeitet werden, um die Lage und Form der geometrischen Merkmale des Objektes zu bestimmen. Bei der Entwicklung eines allgemeinen Verfahrens zur Erkennung der geometrischen Merkmale in Abschnitt~\ref{Methodik} werden Algorithmen und Datenstrukturen verwendet. 

\section{Datenstrukturen} \label{Datenstrukturen}

Datenstrukturen dienen der Organisation und Speicherung von Daten sodass, die Beziehung zwischen einzelnen Elemente auch bewahrt wird. In einer Datenstruktur werden darüber hinaus auch Zugriffsmethoden für den Zugriff auf die gespeicherten Daten definiert sowie Angaben über Möglichkeiten zur Verarbeitung der Daten gemacht. Eine gute Datenstruktur setzt voraus, dass die Beziehung zwischen der Daten bewahrt und gut definiert wird sowie die Verarbeitung der Daten leicht gemacht wird. Eine Datenstruktur soll auch bestimmte Operationen auf die Daten ermögliche, beispielsweise die Hinzufügung oder Entfernung von Datenpunkte, die Zusammenführung oder Sortierung der Daten sowie das Durchqueren der Datenstruktur nach bestimmten Daten. In der Informatik gibt es bereits etablierte Datenstrukturen, die sich nach unterschiedlichen Einsatzzwecken richten und eine sehr breite Anwendung finden. Diese lassen sich nach Abbildung~\ref{fig: datastructures} nach lineare und nichtlineare Datenstrukturen unterteilen. \autocite[1-2]{mohanty_data_2021}

\begin{figure}[h]
	\includegraphics[width=\textwidth]{Abbildungen/Datenstruktur_arten.png}
	\centering
	\caption[Datenstrukturarten]{Die unterschiedlichen Arten von linearen und nichtlinearen Datenstrukturen nach \textcite[2]{mohanty_data_2021}}
	\label{fig: datastructures}
\end{figure}

\subsection{Lineare Datenstrukturen}
Eine lineare Datenstruktur ist eine Datenstruktur, bei der die Elemente in einer sequentiellen Reihenfolge angeordnet sind. Dies bedeutet, dass jedes Element genau einen Vorgänger und einen Nachfolger hat, außer dem ersten und letzten Element. Lineare Datenstrukturen können als tabellarische Liste oder als verkettete Liste implementiert werden. \autocite[314-315]{hoffmann_einfuhrung_2011}

Die Operationen, die auf einer linearen Datenstruktur ausgeführt werden können, sind in der Regel das Initialisieren der Datenstruktur als leere Menge, das Einfügen eines Elements in die Datenstruktur und das Entfernen eines Elements aus der Datenstruktur. Es ist auch möglich, andere Operationen auszuführen, die nicht unbedingt auf der Ordnungsbeziehung zwischen den Elementen basieren. Unter diesen Operationen zählen beispielsweise das Suchen nach einem Element oder das Ersetzen eines Elements in der Datenstruktur. \autocite[314-315]{hoffmann_einfuhrung_2011}

Die Implementierung einer linearen Datenstruktur kann je nach Anforderungen und verfügbaren Ressourcen variieren. Es ist jedoch wichtig sicherzustellen, dass die Datenstruktur korrekt implementiert ist und dass alle Operationen den Zustand der Datenstruktur ordnungsgemäß ändern. \autocite[314-315]{hoffmann_einfuhrung_2011}

Unter den linearen Datenstrukturen finden hauptsächlich drei Datenstrukturen eine breite Anwendung, nämlich Felder (Arrays), Queues (Warteschlangen) und Stapel (Stacks).

\subsubsection{Felder}

Das Feld ist einer der einfachsten linearen Datenstrukturen. Ein Feld besteht aus mehreren Daten des gleichen Formats oder Datentyps, die je nach Implementierung in aufeinanderfolgenden Speicherorten gespeichert werden. Diese werden sequenziell hintereinander angeordnet und zusammen gespeichert. Elemente eines Feldes dürfen eindimensional oder auch mehrdimensional gespeichert werden, wodurch diese Datenstruktur mit Matrizen oder Vektoren aus der Mathematik verglichen werden kann. Die Größe oder Dimension des Feldes wird immer vorgegeben und bleibt in der Regle statisch. Jedes Element eines Feldes besitzt einen sogenannten Index, welcher auf die Position des Elements in dem Feld deutet. Im Falle eines zwei Dimensionalen Feldes besitzt jedes Element der Datenstruktur zwei Indizes. In der regel deutet der erste Index auf die Zeile und der zweite Index auf die Spalte des Datenfeldes hin, allerdings kann diese Regel von Implementierung zu Implementierung variieren. Felder dürfen eine beliebig Anzahl \textit{n} Dimensionen besitzen, allerdings steigt somit auch die Anzahl der Indizes aller Elemente. Datenfelder über vier Dimensionen können sogar räumlich nicht vorgestellt werden, jedoch stellt dies für einen Rechner keine Probleme dar. Zugriffsoperationen auf Elemente sowie Operationen zur Einfügung und Entfernung dieser Elemente verwenden ihre Indizes, um den Eintrag an einer bestimmten Position des Feldes aufzurufen oder zu manipulieren. \autocite[35-36]{ollmert_datenstrukturen_2020}

Eine andere Variante der Felder ist die sogenannte lineare Liste. Diese Liste unterscheidet sich von Felder, indem sie dynamisch initialisiert werden darf. Im Gegensatz zu der statischen Größe oder Dimension eines Feldes, darf die Größe einer linearen Liste beliebig geändert werden. Während ein Feld mit maximale Größe \textit{n} und \textit{n} Elemente nicht um ein weiteres Element \textit{k} erweitert werden darf, kann eine lineare Liste der gleichen Größe mit der gleichen Anzahl an Elementen um das \textit{k-te} Element erweitert werden. Das Element darf auch an einer beliebigen Stelle der Liste eingefügt werden. Die Reihenfolge beziehungsweise die Positionen der anderen Elemente werden automatisch angepasst. Dies könnte auch dazu führen, dass ein Element Z, welches zum Zeitpunkt t\textsubscript{1} vor der Einfügung eines neuen Elements einen bestimmten Index \textit{i} besaß, zu einem Zeitpunkt t\textsubscript{2} nach der Einfügung nicht mehr an der gleichen Position zu finden ist. Das gleiche kann auch durch das Entfernen von Elementen an beliebigen Positionen geschehen. Die Positionen und somit die Indizes aller Elemente dürfen auch geändert werden, indem sie Beispielsweise nach einem Kriterium sortiert werden. \autocite[40-42]{ollmert_datenstrukturen_2020}

Lineare Listen bieten viele ähnlichen Operationen wie die von Feldern an, um mit den gespeicherten Elementen zu interagieren. Dazu gehört das Abrufen eines bestimmten Elements, das Einfügen eines neuen Elements zwischen zwei benachbarten Elementen (sowie Spezialfälle für das Hinzufügen eines neuen ersten oder letzten Elements), das Entfernen eines bestimmten Elements, das Bestimmen der aktuellen Länge der Liste anhand der Elementanzahl, die Suche nach einem Element mit einem bestimmten Wert, das Zusammenführen von zwei linearen Listen und das Aufteilen einer Liste in zwei Teillisten.\autocite[42-43]{ollmert_datenstrukturen_2020}


Verkettete lineare Listen sind eine bestimmte Art der Implementierung von linearen Listen. Eine solche Kette unterscheidet sich grundsätzlich nach ihrem Aufbau und Speicherverfahren von gewöhnlichen Listen und Feldern. Die Elemente einer verketteten linearen Liste werden nicht aufeinanderfolgend gespeichert. Stattdessen wird mit jedem Element dieser Liste auch ein Zeiger beigegeben, der entweder die Speicheradresse des nächsten Elements angibt oder auf das Ende der Liste hinweist. Der Anker ist ein besonderer Zeiger, der auf den Anfang der Liste deutet. Die Struktur und Operationen einer linearen Liste ist in Abbildung~\ref{fig:linear_list} abgebildet. Bei der doppelt-verketteten linearen Listen werden mit jedem Element nicht nur einen Zeiger zu dem nächsten Element beigegeben, sondern auch ein Zeiger zu dem vorigen Element. Bei dieser Art der verketteten Liste werden zwei Anker geliefert - der Vorwärtsanker für den Anfang und der Rückwärtsanker für das Ende der Liste. Die Verarbeitung mancher linearen Listen stellt eine Herausforderung dar, insbesondere wenn sie gleichzeitig durch mehrere Programme verarbeitet werden. In solchen Fällen könnte es beispielsweise dazu führen, dass ein Programm auf ein bestimmtes Element mittels seines Index zugreift, während ein anderes Programm ein Element davor hinzufügt oder entfernt. \autocite[43-44]{ollmert_datenstrukturen_2020}

\begin{figure}[t]
	\includegraphics[width=\textwidth]{Abbildungen/Verkettete_lineare_liste.png}
	\centering
	\caption[Lineare Liste]{Die Struktur und Operationen einer linearen Liste. Die Großbuchstaben deuten auf die Elemente der Liste während die \textit{z}-Buchstaben die Zeiger repräsentieren. Die Kopfzelle steht in dieser Abbildung für den Anker.\autocite[611]{ernst_grundkurs_2020}}
	\label{fig:linear_list}
\end{figure}

Für alle gängigen Implementierungen von linearen Listen ist die Implementierung einer sequentiellen Zugriffsfunktion, die das nächste Element in der Liste ausgehend von einem bestimmten Element bereitstellt, einfach. Die Implementierung einer direkten Zugriffsfunktion, die das \textit{i}-te Element in der Liste bereitstellt, ist zwar ebenfalls einfach, aber die benötigte Zeit hängt von der Position des Elements in der Liste ab und nimmt mit zunehmender Länge der Liste zu.\autocite[45]{ollmert_datenstrukturen_2020}

\subsubsection{Queues}
Queues sind besondere sequentielle Datenstrukturen, die Daten nur in einer bestimmten Reihenfolge speichern. Dadurch wird auch die Entnahme und Einfügung der Daten geregelt. Queues folgen dem Prinzip nach \textit{First-In-First-Out} (FIFO), also werden die Elemente zur Verfügung gestellt, die zuerst zu der Datenstruktur hinzugefügt wurden. Elemente dürfen in der Regel nur am Ende der Queue eingefügt und vom Anfang der Queue entnommen werden. \autocite[371]{gumm_band_2016}

Das Prinzip nach \textit{FIFO} wird üblicherweise mittels verketteten linearen Listen ermöglicht. Der Anfang sowie das Ende der Queue werden mittels Zeiger gekennzeichnet. Jedes Element, welches zu der Queue hinzugefügt wird, erhält je nach Art der verketteten Liste einen Zeiger zu dem Element davor oder danach. Im Falle des ersten oder letzten Elements wird ihm ein besonderer Zeiger - der Nullzeiger - zugewiesen. Dieses deutet darauf hin, dass es keine weiteren Elemente zu finden sind. Der Zeiger \textit{anfang} deutet somit auf das vorderste und älteste Element der Queue während der Zeiger \textit{ende} auf das hinterste und neuste Element deutet. Sobald der erste Eintrag der Queue gelesen oder entnommen wird, wird der Zeiger \textit{anfang} inkrementiert, sodass es auf das nächste Element zeigt. Wenn ein neues Element zu der Queue hinzugefügt wird, wird der Zeiger \textit{ende} inkrementiert, sodass es auf den letzten Eintrag der Queue deutet. \autocite[48-49]{ollmert_datenstrukturen_2020} \autocite[371]{gumm_band_2016}

Queues bieten ähnliche Operationen wie Felder und Listen an. Im Gegensatz zu Listen oder Felder, wo Elemente mit einem bestimmten Funktionsaufruf an beliebige Positionen innerhalb der Datenstruktur platziert werden dürfen, dürfen Elemente aus einer Queue nur am Ende eingefügt werden. Abhängig von der Implementierung wird eine Methode bereitgestellt, die das Einfügen eines Elements ermöglicht. Allgemein wird das Einfügen einer Datei in einer Queue als \textit{enqueue} benannt. Felder und Liste ermöglichen mittels bestimmter Funktionen einen Zugriff auf beliebige Elemente innerhalb der Datenstruktur in einer beliebigen Reihenfolge. Bei Queues darf nur das erste Element aus der Datenstruktur entnommen werden, welches am längsten in der Datenstruktur enthalten war. Auch wird der Zugriff auf dieses Element abhängig von der Implementierung durch eine bestimmte Methode ermöglicht. Allgemein wird das entfernen oder auslesen eines Elements einer Queue als \textit{dequeue} bezeichnet. Die Anzahl der Elemente \textit{count} in einer Queue lässt sich bestimmen, indem die Differenz zwischen den Zeigern \textit{anfang} und \textit{ende} berechnet wird. In den meisten Implementierungen werden diese Schritte innerhalb einer Methode zur Bestimmung der Länge eingekapselt. Diese Operationen zusammen mit den zur Erstellung und Vernichtung von Queues sind die Einzigen Methoden, die für diese Datenstruktur in der Regel bereitgestellt werden. \autocite[71-72]{hubwieser_fundamente_2015} \autocite[371]{gumm_band_2016}

Die Ausführung der Prozeduren \textit{enqueue} und \textit{dequeue} erfordert bei einer Queue einen konstanten Zeitaufwand, der unabhängig von der Anzahl der Elemente in der Datenstruktur ist. Die Suche nach einem Bestimmten Element in einer Queue erfordert jedoch einen Zeitaufwand, der linear von der Anzahl der aktuellen Elemente in der Datenstruktur abhängt. Zur Überprüfung, ob ein bestimmtes Element in einer Queue vorhanden ist, muss im Durchschnitt die Hälfte der Queue durchsucht werden. \autocite[318]{hoffmann_einfuhrung_2011}

\begin{figure}[t]
	\includegraphics[width=\linewidth]{Abbildungen/Queue.png}
	\centering
	\caption[Eine Queue]{Eine Queue mit den Operationen zum Einfügen und Entfernen von Elementen dargestellt \autocite[371]{gumm_band_2016}}
	\label{fig: queue}
\end{figure}

\subsubsection{Stapel}
Die Datenstruktur des Stapels, auch als Stack bekannt, ist eine homogene, sequentielle Struktur, die nur das Einfügen und Lesen von Elementen am Anfang der Struktur erlaubt. Beim Lesen eines Elements wird dieses gleichzeitig entfernt, so dass das folgende Element an den Anfang rückt. Stapeln folgen das Prinzip nach \textit{First-In-Last-Out (FILO)} oder \textit{zuerst-rein-zuletzt-raus}. Die Anzahl der Speicherplätze des Stacks ist einseitig potenziell unbegrenzt, so dass der Stack dynamisch wachsen und schrumpfen kann. \autocite[614]{ernst_grundkurs_2020}

Stapel werden üblicherweise als verketteten linearen Listen implementiert, die nur in eine Richtung expandieren oder reduzieren dürfen. Somit erfolgt das Einfügen sowie Entnehmen von Elementen im Gegensatz zu Queues nur an einem Ende des Stapels. Beim Einfügen eines neuen Elements auf die vorhandenen Elemente des Stapels platziert und dem neuen Element wird ein Zeiger zum vorigen Element beigegeben. Im Falle des ersten Elements wird ihm ähnlich wie zuvor ein Nullzeiger zugewiesen, der auf das Ende des Stapels hinweist. Elemente des Stapels dürfen nur in der umgekehrten Reihenfolge ausgelesen werden, in der sie eingefügt wurden. Somit wird das allererste Element nur als allerletztes Element ausgelesen. \autocite[363]{gumm_band_2016}

Ähnlich wie Queues bieten Stapel im Vergleich zu Felder und Listen nur begrenzte Funktionalitäten an. Elemente eines Stapels dürfen nicht an beliebigen Stellen zwischen vorhandenen Elementen des Stapels platziert werden, sondern nur am \textit{Kopf} des Stapels. Dies wird durch die Operation \textit{push} ermöglicht, welches ein neues Element an der obersten Ebene platziert und den Zeiger für den Kopf nach oben nachrückt. Genau wie das Speichern darf das Auslesen eines Elementes nicht an einer beliebigen Stelle des Stapels erfolgen, sondern nur am Kopf des Stapels geschehen. Die Operation \textit{pop} ist dafür zuständig, das erste beziehungsweise oberste Element des Stapels zu entfernen und für das Auslesen bereitzustellen. Gleichzeitig wird der Zeiger für den Kopf nach hinter gerückt, sodass es auf das vorige Element zeigt. Neben dieser Operationen werden auch zusätzliche Operationen bereitgestellt, die Stapel erzeugen, vernichten oder seine Größe bestimmen. Wie diese Operationen definiert werden, hängt allerdings von der Implementierung ab. \autocite[614]{ernst_grundkurs_2020} \autocite[45-46]{ollmert_datenstrukturen_2020}

Im Prinzip funktioniert ein Stapel ähnlich zu einer Queue, indem er das Auslesen oder Einfügen von Elementen nur an bestimmten Stellen erlaubt. \textit{Push} und \textit{pop} sind beide Operationen, die ähnlich wie \textit{enqueue} und \textit{dequeue} funktionieren und unabhängig von der Größe der Datenstruktur sind. Somit sollte der Zeitaufwand dieser Operationen auch konstant bleiben. Die Suche nach einem bestimmten Element in einem Stapel sollte ähnlich wie bei Queues direkt von der Anzahl der Elemente abhängen.

\begin{figure}[t]
	\includegraphics[width=\textwidth]{Abbildungen/Stack.png}
	\centering
	\caption[Stapel]{Ein Stapel mit den Operationen zur Einfügung und Entfernung von Elementen dargestellt \autocite[371]{gumm_band_2016}}
	\label{fig: stack}
\end{figure}

\subsection{Nichtlineare Datenstrukturen} \label{nicht_lineare_datenstrukturen}
Bei nichtlinearen Datenstrukturen werden im Gegensatz zu linearen Datenstrukturen die einzelnen Elemente nicht in einer sequentiellen Reihenfolge zueinander stehen \autocite[321]{hoffmann_einfuhrung_2011}. 

Die Funktionsweise sowie die Operationen nichtlinearer Datenstrukturen lassen sich nicht verallgemeinern, sondern können von Implementierung zu Implementierung abweichen. Unter nichtlinearen Datenstrukturen finden hauptsächlich vier Arten dieser eine breite Anwendung, nämlich Bäume (Tree), Graphen (Graph), Tabellen (Tables) und Mengen (Sets). Im Rahmen dieser Arbeit werden nur Bäume und Graphen behandelt.

\subsubsection{Bäume}
Bäume stellen eine grundlegende Datenstruktur in der Informatik dar, die eine zweidimensionale Verallgemeinerung von Listen darstellen. Im Gegensatz zu Listen erlauben Bäume die Speicherung von Daten sowie die relevanten Beziehungen zwischen diesen Daten, wie beispielsweise Ordnungs- oder hierarchische Beziehungen. Daher sind Bäume besonders gut geeignet, um gesuchte Daten schnell wiederzufinden. Ein Baum besteht aus einer Menge von Knoten, die miteinander durch Kanten verbunden sind. Wenn eine Kante von Knoten A zu Knoten B führt, wird dies als A$\rightarrow$B notiert und A wird als Vater von B oder B als Kind von A bezeichnet. Ein Knoten ohne Kinder wird als Blatt bezeichnet, während alle anderen Knoten als innere Knoten bezeichnet werden. \autocite[389]{gumm_band_2016}

Ein Pfad von A nach B führt als Folge von Knoten und Pfeilen von A nach B, wobei die Knoten durch Pfeile verbunden sind. Dieser Pfad wird als A $\rightarrow$ X1 $\rightarrow$ X2 $\rightarrow$ $\cdots$ $\rightarrow$ B notiert, wobei die Länge des Pfades durch die Anzahl der Knoten bestimmt wird. Diese Anzahl kann 0 oder mehr betragen. Wenn es einen Pfad von A nach B gibt, so gilt B als Nachkomme von A, während A als Vorfahre von B bezeichnet wird. \autocite[389]{gumm_band_2016}

Ein Baum erfüllt verschiedene Axiome, die ihn von anderen Datenstrukturen unterscheiden. So gibt es genau einen Knoten, der als Wurzel bezeichnet wird und keinen Vater hat. Jeder andere Knoten hat genau einen Vater und ist ein Nachkomme der Wurzel. Zudem darf es keine zyklischen Pfade im Baum geben, da dies zu Widersprüchen führen würde. Ein weiteres Axiom besagt, dass es von der Wurzel zu jedem anderen Knoten genau einen Pfad gibt. Das bedeutet, dass jeder Knoten eindeutig bestimmt werden kann und es keine mehrdeutigen Wege durch den Baum gibt. Schließlich bilden die Nachkommen eines beliebigen Knotens K zusammen mit allen ererbten Kanten einen zusätzlichen Baum mit K als Wurzel. Dieser Baum wird als Unterbaum mit Wurzel K bezeichnet und erfüllt ebenfalls alle Axiome eines Baumes. Aufgrund dieser Eigenschaften sind Bäume besonders nützlich, hierarchische Strukturen abzubilden und Daten effizient abzuspeichern und wiederzufinden. \autocite[389]{gumm_band_2016}

Aufgrund der hierarchischen Aufbau und Form dieser Datenstruktur besitzt sie eine besondere Eigenschaft - die Tiefe. Die Tiefe wird anhand der Anzahl der Ebenen eines Baums bestimmt. Zur Bestimmung der Tiefe eines Baums wird der kürzeste Pfad zu der tiefsten Ebene des Baums verfolgt und dabei die Anzahl der Knoten aufgezählt. Diese Anzahl repräsentiert die Tiefe eines Baums. \autocite[390]{gumm_band_2016}

Eine exemplarische Abbildung eines Baums ist in Abbildung~\ref{fig: tree} zu sehen.

\begin{figure}[t]
	\includegraphics[width=\linewidth]{Abbildungen/Tree.png}
	\centering
	\caption[Binärbaum]{Ein Baum mit einer Tiefe von Fünf sowie die Darstellung der Beziehungen zwischen einzelnen Knoten \autocite[390]{gumm_band_2016}}-
	\label{fig: tree}
\end{figure}

Ein binär Baum ist die Bezeichnung einer spezifischen Implementierung dieser Datenstruktur. Sie dient als einer der wichtigsten dynamischen Datenstrukturen. Binäre Bäume ermöglichen das Einfügen und Entfernen von Elementen in der gleichen Geschwindigkeit wie Felder und einfachen linearen Listen. Darüber hinaus bieten Sie auch eine Möglichkeit an, die Suche nach bestimmten Elementen sowie das Durchsuchen innerhalb der Datenstruktur im Vergleich zu konventionellen sequentiellen Datenstrukturen zu beschleunigen. \autocite[617]{ernst_grundkurs_2020}

Alle Knoten eines binären Baums ausschließlich der Blätter haben genau zwei Nachkommen oder Kinder. Darüber hinaus darf ein Vorfahre nicht zwei leere, inhaltslose Kinder haben, sondern darf maximal nur ein Kind keinen Wert haben. Durch die Festlegung der ersten Bedingung wird erzielt, dass jeder Vorfahre immer zu zwei Unterbäumen führt. Die zweite Bedingung verhindert das Aufblasen eines Baums, indem keine unnötigen leere Unterbäume entstehen. Binäre Bäume sind für die Aufgabe des Suchens besonders geeignet. Sie dienen der effizienten binären (Ja/Nein) Auswertung von Aussagen. Bäume für diesen Zweck werden oft als binäre Suchbäume bezeichnet. Ein binärer Suchbaum ist genau so wie ein allgemeiner binärer Baum aufgebaut, indem jeder Knotenpunkt einen linken und rechten Unterbaum besitzt. Jeder Knoten enthält einen Schlüsselwert, der ihm zugeordnet ist. Eine wichtige Eigenschaft von binären Suchbäumen ist, dass der Schlüsselwert eines jeden Knotens größer ist als alle Schlüsselwerte im linken Unterbaum des Knotens und kleiner als alle Schlüsselwerte im rechten Unterbaum des Knotens. \autocite[94-95]{ollmert_datenstrukturen_2020}

Um nach einem Element in einem binären Suchbaum zu suchen, wird zuerst der Wert der Wurzel überprüft. Falls dieser nicht dem Element entspricht, werden die nächsten Nachkommen der Wurzel überprüft. Hierbei werden allerdings nicht beide Kinder der Wurzel mit dem Element verglichen, sondern der Wert des Elements zuerst mit dem der Wurzel verglichen. Ist das Element kleiner als die Wurzel, wird nur das linke Kind beziehungsweise Unterbaum weiter untersucht. Ist es hingegen größer, wird der rechte Unterbaum untersucht. Dieses wird rekursiv wiederholt, bis ein Konten mit dem Wert des Elements gefunden wird. Bei den meisten Implementierungen der Suchfunktion erfolgt das allerdings iterativ und nicht rekursiv. Dieser Vergleich wird solange wiederholt, bis das Element gefunden wird. \autocite[139-140]{knebl_algorithmen_2021}

Um ein Element in dem Suchbaum einzufügen erfolgen die folgenden Schritte. Mittels der Suchfunktion wird zuerst überprüft, ob das Element bereits in dem Suchbaum vorhanden ist. Im Falle des Vorhandenseins wird die Funktion abgebrochen. Ansonsten wird der Wert des Blattes aufgerufen, wo die Suchfunktion aufgehört hat. Dank der Funktionsweise eines binären Baums wird der Wert des neuen Elements ähnlich groß wie der des Blattes sein. Das Blatt wird zu einem inneren Knoten eines Unterbaums gewandelt. Falls der Wert des neuen Elements kleiner als der des Blattes ist, wird es links im Unterbaum gespeichert. Ansonsten wird es rechts im Unterbaum gespeichert. \autocite[140]{knebl_algorithmen_2021}

Beim Löschen eines Elements wird der gleiche erst Schritt wie beim Einfügen verwendet. Mittels der Suchfunktion wird sichergestellt, dass das Element in dem Baum vorhanden ist. Falls das Element in einem Blatt gespeichert ist, erfolgt die Löschung sehr einfach, indem das Blatt vom Baum entfernt wird und der Vorfahre aktualisiert wird, somit dieser nicht auf das leere Blatt zeigt. Die Entfernung des Elements erfolgt auch relativ simpel im Falle, dass der Knoten \textit{v} des Elements nur einen Nachkommen hat. In diesem Fall wird der Vorfahre des Knotens \textit{v} geändert, sodass es auf den Nachkommen des Knotens \textit{v} zeigt. Danach wird der Knoten \textit{v} gelöscht. Problematischer wird die Löschung wenn der Knoten \textit{v} des Elements zwei Nachkommen besitzt. Hierfür wird in dem linken Unterbaum von \textit{v} möglichst tief nach dem größten Wert \textit{$\overline{e}$} gesucht. Dieser Wert wird möglichst tief und rechts liegen. Dieser Wert wird sowohl größer als andere Werte im linken Unterbaum sein als auch kleiner als das Element. Dieser Wert darf maximal einen linken Nachkommen besitzen. Sobald \textit{$\overline{e}$} bestimmt wird, wird dieser Wert anstelle des Elements in \textit{v} eingefügt. Der linke Nachkomme von \textit{$\overline{e}$} wird somit der rechte Nachkomme von dem vorigen Vorfahre von \textit{$\overline{e}$}. Somit wird erzielt, das Element aus dem Baum zu löschen und die binäre Eigenschaft des Suchbaums aufrechtzuerhalten. Die Abbildung~\ref{fig: tree_delete} visualisiert dieses Verfahrens. \autocite[140-141]{knebl_algorithmen_2021}

\begin{figure}[!b]
	\includegraphics[width=\linewidth]{Abbildungen/tree_delete.png}
	\centering
	\caption[Löschvorgang binären Baums]{Eine Veranschaulichung des Löschvorgangs in einem binären Suchbaum \autocite[140]{knebl_algorithmen_2021}}
	\label{fig: tree_delete}
\end{figure}

Neben dem Suchbaum gibt es auch andere Varianten des binären Baums, die unterschiedlichen Eigenschaften aufweisen und besondere Funktionalitäten anbieten. Unter diesen sind B-Bäume und AVL-Bäume einer der gängigsten Varianten, die häufig eine Anwendung finden. B-Bäume eignen sich insbesondere für das Speichern und effiziente Verwalten von sehr großen Datenmengen während AVL-Bäume, aufgrund ihrer bilanzierten Struktur, Suchfunktionen besonders schnell ausführen können. Die genauere Untersuchung der Struktur und Funktionsweise dieser Varianten liegt allerdings außerhalb des Umfangs dieser Arbeit \autocite[407-412]{gumm_band_2016}.

\subsubsection{Graphen}
Diese Datenstruktur findet ihre Herkunft in dem Teilgebiet des diskreten Mathematik in der sogenannten Graphentheorie. Graphen werden anschaulich als eine Menge von Knoten dargestellt, die durch Kanten miteinander verbunden sind. Sie umfassen eine sehr allgemeine Klasse von Datenstrukturen, welche andere Strukturen wie Bäume und lineare Listen als Teilmenge enthalten. In der Praxis haben Graphen eine große Bedeutung, da sich sehr viele statische und dynamische Strukturen der realen Welt darauf abbilden lassen. Straßenverbindungen, Kommunikations- und Rechnernetze, Flussdiagramme, Automaten, elektronische Schaltpläne sind Beispiele dafür. \autocite[215]{knebl_algorithmen_2021} \autocite[654]{ernst_grundkurs_2020}

Durch die Verbindung von Knoten mit Kanten in einem Graph werden die Relationen zwischen den Datenelemente dargestellt. Die Kanten verbinden jeweils zwei Knoten miteinander und können ungerichtet oder gerichtet sein. Ungerichtete Kanten stellen eine Verbindung zwischen zwei Knoten dar, ohne eine bestimmte Richtung anzugeben. Gerichtete Kanten hingegen haben eine bestimmte Richtung und stellen eine Verbindung von einem Knoten zu einem anderen dar. \autocite[221-222]{knebl_algorithmen_2021}

Für die Suche nach einem Datenelement innerhalb des Graphen müssen Graphen durchquert werden. Die Breitensuche und Tiefensuche sind zwei etablierte Verfahren, die dazu dienen. Bei der Breitensuche (Breadth-First Search, BFS) wird der Graph schichtenweise durchsucht. Zu Beginn wird ein Startknoten gewählt und seine direkten verbundenen benachbarten Knoten (Nachbarn) werden besucht. Anschließend werden die Nachbarn der Nachbarn sequentiell weiterhin besucht, bis alle Knoten besucht wurden oder ein bestimmter Zielknoten gefunden wurde. Das besondere an diesem Verfahren ist, dass ein Knoten nur dann besucht, wenn er noch nicht besucht wird. Diese Tatsache verleiht diesem Verfahren seine besonders hohe Geschwindigkeit bei einem Suchvorgang. Um zu verhindern, dass der Algorithmus in einer Endlosschleife landet, werden die besuchten Knoten in einem Stapel gespeichert. \autocite[227-228]{knebl_algorithmen_2021} \autocite[666]{ernst_grundkurs_2020}

Bei der Tiefensuche (Depth-First Search, DFS) wird der Graph hingegen rekursiv durchsucht. Auch hier wird ein Startknoten gewählt, dessen erster direkter Nachbar besucht wird. Nachdem der erster Nachbar besucht wurde, wird dessen direkter Nachbarn wieder besucht. So werden sequentiell die Nachbarknoten von Nachbarknoten durchquert, bis ein Endknoten erreicht wird. Danach wird der vorherige Knoten wieder ausgewählt und sein nächster Nachbar besucht. Das Verfahren wird solange fortgesetzt, bis alle Knoten besucht wurden. Auch hier werden die besuchten Knoten markiert, um zu verhindern, dass der Algorithmus bereits besuchte Knoten wieder besucht und in einer Endlosschleife landet. Zur Aufspeicherung der bereits besuchten Knoten wird an der Stelle eine Queue verwendet. \autocite[231-232]{knebl_algorithmen_2021} \autocite[666]{ernst_grundkurs_2020}

\begin{figure}[t]
	\centering
	\begin{subfigure}[h]{0.49\textwidth}
		\includegraphics[width=\linewidth]{Abbildungen/Breitensuche.png}
		\centering
		\caption[Breitensuche in Graphen]{Eine Visualisierung der Breitensuche nach \textcite[228]{knebl_algorithmen_2021}}
		\label{fig: breitensuche}
	\end{subfigure}
	\hfill
	\begin{subfigure}[h]{0.49\textwidth}
		\includegraphics[width=\linewidth]{Abbildungen/Tiefensuche.png}
		\centering
		\caption[Tiefensuche in Graphen]{Eine Visualisierung der Tiefensuche nach \textcite[232]{knebl_algorithmen_2021}}
		\label{fig: tiefensuche}
	\end{subfigure}
	\caption[Suchverfahren von Graphen]{Eine Visualisierung beider Suchverfahren für Graphen}
	\label{fig: graph_search_functions}
\end{figure}

Wie bereits erwähnt, Graphen eignen sich zur Speicherung von statischen und dynamischen Strukturen der reellen Welt. Mittels Algorithmen wie Dijkstras Algorithmus können gewichtete Graphen kostengünstig durchquert werden. Die Kanten eines gewichteten Graphen erhalten auf Basis der Kosten (Beispielsweise Zeit oder Streckenlänge) zur Durchquerung über der Kante eine Gewichtung. Die kürzesten oder schnellsten Pfade zwischen Knoten können somit gefunden werden. Andere Algorithmen wie Prims Algorithmus dienen der Ermittelung des minimalen Spannbaums eines Graphen. Ein Spannbaum könnte als ein Teilgraph \textit{T} definiert werden, der alle Knoten eines Graphen \textit{G} enthält, allerdings weniger Kanten. Trotzdem sind alle Knoten eines Spannbaums vollständig vernetzt. Ein minimaler Spannbaum ist ein Teilgraph eines gewichteten Graphen, dessen Summe der Gewichte weniger als alle andere möglichen Spannbäume ist. Prims Algorithmus lässt sich besonders gut für Routing-Probleme in lokalen Rechnernetze verwenden. \autocite[277-282]{hubwieser_fundamente_2015}

\subsection{Datenstruktur zur Speicherung mehrdimensionalen Daten}

Ein k-dimensionaler Baum, ein kd-Baum, ist eine geometrische Datenstruktur, die dazu dient, mehrdimensionale Punkte in einem effizientem binären Suchbaum zu organisieren. Dabei werden raumbezogene Suchanfragen bearbeitet, indem über Dimensionen abwechselnd gesucht wird. \autocite[92]{saha_advanced_2019} \autocite{bentley_fast_1978}

Der kd-Baum ist die gängigste räumliche Datenstruktur für die Durchführung von Bereichs- und nächsten Nachbarschafts-Suchanfragen. In jeder Ebene des Baums werden alle Kinder entlang einer bestimmten Dimension partitioniert. Danach wird in der nächsten Ebene des Baums die nächste Dimension der Datenstruktur verwendet. Zu Beginn des Baums werden alle Punkte basierend auf der ersten Dimension des Wurzelknotens partitioniert, sodass alle Punkte des rechten Teilbaums eine größere erste Dimension als die Wurzel haben. Im Umkehrschluss ist der Wert der ersten Dimension aller Punkte im linken Teilbaum kleiner. Jede Ebene des Baums partitioniert den Suchraum basierend auf der nächsten Dimension und kehrt periodisch zur ersten Dimension zurück, wenn die letzte abgeschlossen ist. Der effizienteste Weg zur Erstellung eines statischen KD-Baums besteht darin, eine Partitionierungsmethode zu verwenden, die den Medianwert der Raumpunkte verwendet. Dabei bewegen sich Daten mit kleineren eindimensionalen Werten nach links und Daten mit größeren Werten nach rechts. Dieser Vorgang wird dann rekursiv auf beide Teilbäume wiederholt, bis nur noch ein Element übrig bleibt. Dieses Element wird als Blatt des Baumes gespeichert. \autocite[92]{saha_advanced_2019}

Die Erstellung eines statischen kd-Baums für \textit{n} Raumpunkte in \textit{d} Dimensionen erfolgt nach den folgenden Schritten. Zuerst wird der Median der Randpunkte ermittelt und in dem Wurzel des Baums platziert. Danach wird die erste Dimension der \textit{d} Dimensionen verwendet, um eine Hyperebene zu erstellen. Falls \textit{d} gleich zwei ist, wird eine Linie erstellt. Falls \textit{d} drei beträgt, wird eine zweidimensionale Ebene erstellt. Diese Hyperebene teilt die Raumpunkte gleichmäßig auf. Alle Punkte mit einer kleineren ersten Dimension befinden sich somit links zur Hyperebene und Punkte mit einer größeren ersten Dimensionen an der rechten Seite. Die linken und rechten Seiten der Hyperebenen werden in Teilbäume aufgespeichert. Danach werden die Medianwerte der Punkte der jeweiligen Teilbäume ermittelt und auf Basis der nächsten Dimension wieder mittels einer Hyperebene aufgeteilt. Diese Schritte werden rekursive für Teilbäume wiederholt, wobei die Dimension zur Erstellung der Hyperebene für jede sukzessive Ebene zyklisch gewechselt wird. Bis zum Ende dieses Zyklus werden alle Punkte so partitioniert, dass am Ende des Baums nur Blätter vorhanden sind. Abbildung~\ref{fig: kd-tree_creation} bildet einen kd-Baum ab und visualisiert seine Erstellung. \autocite[93-94]{saha_advanced_2019} 

% Die Erstellung eines kd-Baums erfolgt mit einer Komplexität von $\mathcal{O}(n\log n)$, wobei \textit{n} die Anzahl der Punkte beträgt.

\begin{figure}[!b]
	\includegraphics[width=\textwidth]{Abbildungen/2d_kd-tree.png}
	\centering
	\caption[kd-Baum einer zweidimensionaler Datenstruktur]{Ein kd-Baum mit für eine Datenstruktur mit zwei Dimensionen sowie der Prozesses zur Erstellung dieses Baums \autocite[60]{garcia-garcia_alberto_towards_2015}}
	\label{fig: kd-tree_creation}
\end{figure}

Da das kd-Baum einen binären Baum als die grundlegende Datenstruktur verwendet, können alle Operationen eines allgemeinen binären Baums auf kd-Bäume angewendet werden. Mittels einer Suchoperation wird die Suche nach einem bestimmten Raumpunkt innerhalb der Baumstruktur ermöglicht. Bei der Suche nach einem bestimmten Punkt \textit{p} wird die erste Dimension mit der des Wurzels verglichen. Falls der Punkt kleiner als der Median ist, wird im linken Teilbaum weitergesucht. Ansonsten gelangt das Verfahren zum rechten Teilbaum. Danach wird die zweite Dimension von \textit{p} mit der zweiten Dimension der Wurzel des Teilbaums verglichen, um den nächsten Nachkommen zu bestimmen. Diese Schritte werden rekursiv wiederholt, bis der Punkt \textit{p} gefunden wird. Das Einfügen sowie das Entfernen von Elementen aus einem kd-Baum erfolgt analog zu den allgemeinen Verfahren für binären Bäume aus Abschnitt~\ref{nicht_lineare_datenstrukturen}. \autocite[94]{saha_advanced_2019}

Eine besondere Eigenschaft des kd-Baums ist die Möglichkeit schnell und effizient nach einer bestimmten Anzahl von Nachbarpunkte eines Raumpunktes zu suchen. Dies erfolgt nach zwei Verfahren - die Bereichssuche und die Suche der nächsten Nachbarschaft. Gewöhnlich wird die Bereichssuche von kd-Bäume eingesetzt, um alle Punkte innerhalb eines vorbestimmten sphärischen Radius \textit{R} zu suchen. Hierfür wird ein Stapel verwendet. Zuerst wird der Mittelpunkt \textit{m} des sphärischen Bereiches festgelegt, der häufig ein bereits vorhandener Raumpunkt des kd-Baums ist. Die Suchfunktion des Baums beginnt bei dem Wurzel und quert durch die Unterbäume durch. Dabei werden die Dimension des angefragten Punktes abwechselnden mit dem Knoten verglichen, um den nächsten Unterbaum zu bestimmen. Bei dem Suchverlauf werden alle durchquerten Knoten sowie deren Nachkommen zu dem Stapel hinzugefügt. Wenn ein Blatt des Baumes erreicht wird, werden die eingespeicherten Punkte in dem Stapel der Reihe nach ausgelesen und die Abstände zwischen dem Punkt und \textit{m} bestimmt. Liegt der Abstand eindeutig unter dem Wert \textit{R}, wird der Punkt akzeptiert und abgespeichert. Befindet sich ein Punkt auf den Umfang der Sphäre, werden die Abstände seiner Kinder zu \textit{m} bestimmt. Dabei werden alle Kinder ausgeschlossen, die außerhalb des Radius \textit{R} liegen. Abhängig von der Implementierung ergibt sich auch die Möglichkeit, die Suche auf die ersten \textit{n} Punkte zu beschränken. \autocite[95]{saha_advanced_2019}

Bei einer Suche nach dem nächsten Nachbar in einem kd-Baum wird ähnlich wie bei einer Bereichssuche vorgegangen, nur dass der Radius \textit{R} als aktuelles Minimum und Ablehnungskriterium anstatt als akzeptierte Bedingung behandelt wird. Die Suche nach dem nächsten Nachbar wird verwendet, um den nächsten Punkt in unmittelbarer Nähe zu einem bestimmten Punkt \textit{p} zu finden. Auch für diese Suche wird ein Stapel verwendet. Mit der Suchfunktion wird ein binärer Vergleich zwischen dem Wert von \textit{p} und einen Knoten (am Anfang die Wurzel) ausgeführt und abhängig davon durch den linken oder rechten Teilbaum durchquert. Dabei werden alle besuchten Knoten auf dem Suchpfad und deren Kinder zu dem Stack so lange eingefügt, bis ein Blatt des Baums erreicht wird. Die Entfernung dieses Blatts zu dem Abfragepunkt \textit{p} wird ermittelt und als aktuelles Minimum \textit{R} betrachtet. Danach werden die Punkte aus dem Stapel sukzessiv ausgelesen und der Abstand zwischen ihnen und \textit{p} berechnet. Sei dieser Abstand kleiner als \textit{R}, wird der Wert von \textit{R} aktualisiert, um den neuen kürzesten Abstand zu entsprechen. Wenn der Stapel leer ist, wird der Punkt mit dem kürzesten Abstand zu \textit{p} als der nächste Nachbar zurückgegeben. Abhängig von der Implementierung wird es auch ermöglicht, die ersten \textit{k} nächsten Nachbarn von \textit{p} zu bestimmen. \autocite[96]{saha_advanced_2019}

Da kd-Bäume in der Regel sehr hohen Datenmengen mit einer hohen Dimensionalität speichern, ist es von hohem Wert, wenn die Laufzeiten der einzelnen Operationen eines kd-Bäums geschätzt werden könnten. Hierfür wird die O-Notation verwendet, die in Abschnitt~\ref{o-notation} detaillierter behandelt wird. Die Zeitkomplexität des Initialisierungs- und Erstellungsverfahrens beträgt $\mathcal{O}(n\log n)$. Die Suchfunktionen für einen bestimmten Punkt in dem Baum hat im Gegensatz dazu eine niedrigere Zeitkomplexität von $\mathcal{O}(\log n)$. Das Einfügen und Entfernen eines Elements sowie die Bereichssuche und Suche nach nächsten Nachbarn hat normalerweise auch die gleiche Zeitkomplexität. Allerdings sei es mit einer sehr hohen Anzahl an Dimensionen sowie einem unsymmetrischen  kd-Baum möglich, dass die Zeitkomplexität der Bereichssuche sowie der Suche des nächsten Nachbars auf $\mathcal{O}(k \cdot n^{1-\frac{1}{k}})$ steigt. \autocite[104-105]{bentley_fast_1978} \autocite[94-96]{saha_advanced_2019}

\section{Algorithmen}
Ein Algorithmus ist ein Verfahren, welcher zur Bestimmung einer oder mehrerer Lösungen eines bestimmten Rechenproblems verwendet wird \autocite[1]{knebl_algorithmen_2021}. In einem Algorithmus wird ein Lösungsansatz möglichst präzise ausformuliert, indem kleine, isolierte und klar definierte Verarbeitungsschritte definiert werden. Auch ein simples Verfahren zur Summierung zweier Zahlen lässt sich als ein Algorithmus nennen. \autocite[9-10]{hubwieser_fundamente_2015}

Gewisse strukturelle Gemeinsamkeiten sind bei allen Algorithmen vorhanden, auch wenn sie verschiedene Darstellungsarten haben können. Laut \textcite[12-14]{hubwieser_fundamente_2015} können Algorithmen durch elementare Verarbeitungsschritten sowie Sequenzen, bedingte Verarbeitungsschritten und Wiederholungen von elementaren Verarbeitungsschritten. Diese Bausteine werden auch als Strukturelemente von Algorithmen bezeichnet.

Es gibt simple, nicht-zusammengesetzte  Verarbeitungsschritte, die zwingend ausgeführt werden müssen. Zusätzlich zu diesen elementaren Verarbeitungsschritten sind noch drei Arten von zusammengesetzten Verarbeitungsschritten erforderlich, um Abläufe zu beschreiben: Sequenzen, bedingte Verarbeitungsschritte und Wiederholungen. 

Elementare Verarbeitungsschritte können zu Sequenzen zusammengefasst werden, um hintereinander auszuführende Schritte darzustellen. Jeder Schritt übernimmt dabei das Ergebnis seines Vorgängers. Zur Trennung der einzelnen Komponenten einer Sequenz wird ein festes Trennzeichen festgelegt, wie beispielsweise ein Strichpunkt oder ein Zeilenwechsel. In manchen Programmiersprachen können Sequenzen auch mit Begrenzungssymbolen (beispielsweise geschweiften Klammern) zu einem Block zusammengefasst werden.

Bei einigen Verarbeitungsschritten ist es notwendig, dass sie nur unter bestimmten Bedingungen ausgeführt werden. In solchen Fällen kann es auch erforderlich sein, alternative Verarbeitungswege festzulegen, die ausgeführt werden, wenn die Bedingung nicht erfüllt ist. Diese Art eines Verarbeitungsschrittes wird als ein bedingter Verarbeitungsschritt bezeichnet. Bei bedingten Verarbeitungsschritten dürfen sowohl elementare als auch zusammengesetzte Verarbeitungsschritte ausgeführt werden.

Es ist häufig notwendig, dass Sequenzen von Verarbeitungsschritten mehrfach ausgeführt werden. Dabei werden bestimmte Bedingungen definiert, um die Anzahl der Wiederholungen zu regeln. Es ist möglich, sowohl elementare Verarbeitungsschritte als auch zusammengesetzte Verarbeitungsschritte zu wiederholen. Dabei gibt es einen Unterschied zwischen zwei Arten von Wiederholungen: solchen mit einer vorgegebenen Anzahl von Wiederholungen und solchen mit einer Bedingung, die zu Beginn oder am Ende der Wiederholung geprüft wird. Die letztere Art ist sicherer, während die erstere flexibler ist und eine größere Vielfalt von Funktionen ermöglicht.

\subsection{Bestimmung der Komplexität eines Algorithmus} \label{o-notation}
Die asymptotische Analyse ist eine Disziplin der Informatik, die sich mit der Bestimmung des Aufwands von Algorithmen zur Lösung von Problemen beschäftigt. Dabei werden insbesondere die Zeitkomplexität und die Speicherkomplexität betrachtet. In diesem Zusammenhang soll im Folgenden nur auf die zeitliche Komplexität eingegangen werden, da der verfügbare Speicherplatz in der heutigen Zeit in der Regel ausreichend ist und zudem immer preisgünstiger wird. Dennoch lassen sich die angewandten Analyse-Techniken auch auf die Untersuchung des Speicherbedarfs übertragen. \autocite[201]{hubwieser_fundamente_2015}

Eine Möglichkeit, die Laufzeit eines Algorithmus zu bestimmen, wäre die Durchführung von Tests mit verschiedenen Eingabegrößen. Dieses Vorgehen ist jedoch häufig unpraktikabel, da manche Algorithmen für größere Eingaben so lange laufen, dass eine Messung nicht durchführbar ist. Zudem würde die zum Testen verwendete Hardware das Ergebnis beeinflussen, was für eine aussagekräftige Effizienzbewertung unerwünscht ist.\autocite[201]{hubwieser_fundamente_2015}

Aus diesen Gründen werden in der asymptotischen Analyse Techniken verwendet, die unabhängig von der Hardware sind und auf mathematischen Überlegungen basieren. Dabei wird der Aufwand eines Algorithmus in Abhängigkeit von der Größe der Eingabe analysiert. Das Ergebnis dieser Analyse sind sogenannte Laufzeitfunktionen, die das Wachstum des Algorithmus in Bezug auf die Größe der Eingabe beschreiben. Durch die Verwendung von Notationsmethoden wie der O-Notation können die Laufzeitfunktionen kompakt und übersichtlich dargestellt werden. Dies ermöglicht es, Algorithmen auf ihre Effizienz hin zu vergleichen und geeignete Algorithmen für spezifische Problemstellungen auszuwählen.\autocite[201-203]{hubwieser_fundamente_2015}

Es hat sich bei der asymptotischen Analyse etabliert, drei verschiedenen Maße für die Zeitkomplexität zu verwenden. Bei dem besten Fall (englisch: best case) T\textsubscript{best} wird der Fall des Algorithmus betrachtet, der am schnellsten gelaufen ist. Bei dem schlimmsten Fall (englisch: worst case) T\textsubscript{worst} wird die langsamste Berechnungsdauer eines Algorithmus in Betracht gezogen. Um die durchschnittliche Dauer der Berechnung anzugeben, wird der Durchschnittsfall (englisch: average case) T\textsubscript{avg} verwendet. Der beste Fall eines Algorithmus ist in den meisten Fällen irrelevant, da Programme meistens gegen dem schlimmsten Fall gesichert werden müssen. Hier sind die Maße T\textsubscript{worst} oder T\textsubscript{avg} von höherer Bedeutung. \autocite[202]{hubwieser_fundamente_2015}

\begin{table}[b]
	\centering
	\begin{tabular}{{l}{r}}
		\hline
		\textbf{Notation} & \textbf{Sprechweise} \\
		\hline
		$\mathcal{O}(1)$ & konstant \\
		$\mathcal{O}(\log (n))$ & logarithmisch \\
		$\mathcal{O}(n)$ & linear \\
		$\mathcal{O}(n\log (n))$ & log-linear \\
		$\mathcal{O}(n^2)$ & quadratisch \\
		$\mathcal{O}(n^k, k > 2)$ & polynomial \\
		$\mathcal{O}(k^n), k \geq 2$ & exponentiell \\
		\hline
	\end{tabular}
	\caption[Notationen der Zeitkomplexität]{Die gängigen Varianten der O-Notation, sortiert nach steigender Zeitkomplexität \autocite[205]{hubwieser_fundamente_2015}}.
	\label{table: o-notation}
\end{table}

Die Angabe der Zeitkomplexität mittels Einheiten der Zeit wie Sekunden lässt das Treffen einer allgemeinen Aussage über die Effizienz des Algorithmus nicht zu. Vielmehr hilft die Modellierung der Zeitkomplexität als eine Funktion der Eingabegröße, da sie eine Abstrahierung der Effizienz ohne den Einfluss von Faktoren wie Hardware ermöglicht. Zur Angabe der Zeitkomplexität wird am gängigsten das Landau-Symbol oder die \textit{O-Notation} verwendet. Das Wachstumsverhalten der benötigten Berechnungsdauer eines Algorithmus wird gegen eine unendliche große Eingabegröße aufgezeichnet, indem die Eingabegröße möglichst hohe Werte annimmt. Zeigt die Zeitkomplexität eine polynomialen Trend, wird zur Darstellung der Zeitkomplexität der höchste Grad der Funktion verwendet. Konstante werden bei der O-Notation auch nie berücksichtigt. Der Verlauf der Zeitkomplexität gegen steigender Eingabegröße kann auch einen logarithmischen Trend aufweisen. Die Tabelle~\ref{table: o-notation} fasst alle gängige Komplexitätsgrade zusammen, die häufig verwendet werden. \autocite[203]{hubwieser_fundamente_2015}

\begin{figure}[t]
	\centering
	\includegraphics[scale=0.8]{Abbildungen/Time_complexities_lower.png}
	\caption[Zeitkomplexität von Algorithmen]{Visualisierung der unterschiedlichen Größen der Zeitkomplexität}
	\label{fig: time_complexity}
\end{figure}

\subsection{Grundlegende Algorithmen}
Viele alltäglichen Rechenaufgaben haben einen häufigen Bedarf an Algorithmen, die bestimmte Aufgaben ausführen. Hierunter zählen das Sortieren von Elementen oder das Suchen nach bestimmten Elementen innerhalb einer Datenstruktur. Im folgenden werden Algorithmen diskutiert, die standardisierte und effiziente Methoden zur Ausführung dieser Aufgaben anbieten.

\subsubsection{Sortieren}
Sortieren wird als das Arrangieren der Elemente oder einer Teilmenge der Elemente einer Datenstruktur in einer bestimmten Reihenfolge verstanden. Sortieren findet häufig als eine Vorverarbeitungsmaßnahme für komplexere Algorithmen eine Anwendung, da sie sich sowohl menschlich als auch maschinell deutlich schneller verarbeiten lässt. Die Suche nach einem bestimmten Element sowie bestimmten Elementen innerhalb eines Bereiches wird durch die Sortierung effizienter gestaltet. Auch die Gruppierung gleicher oder ähnlicher Elemente in einer Datenstruktur erfolgt deutlich schneller, da diese durch eine Sortierung näher aneinander gebracht werden. \autocite[153-154]{sanders_sequential_2019}

Die einfachsten Verfahren zur Sortierung linearer Datenstrukturen erfolgen nach dem Prinzip des Auswählens und Einfügens (Insertion). Bei der Sortierung durch Auswählen wird in jeder Iteration das kleinste oder größte Element ausgelesen, entfernt und am Ende des Ausgabefeldes eingefügt. Am Ende stehen die Elemente in einer aufsteigenden beziehungsweise absteigenden Reihenfolge in dem Ausgabefeld. Die Elemente dürfen auch rekursiv innerhalb des Eingabefelds ohne den Bedarf an einem zusätzlichen Ausgabefeld sortiert werden. Konsequenterweise ist diese rekursive Sortierung besonders effizient bei der Speicherverwendung, da die Feldgröße konstant bleibt. Die Zeitkomplexität bei dieser Art der Sortierung lässt sich anhand der Anzahl der Elemente bestimmen. Da diese Variante der Sortierung durch \textit{n} Elemente des Felds iteriert, wird dies genau \textit{n} mal ausgeführt. In jedem Durchlauf der Schleife werden alle verbliebenen Elemente des Eingabefeld nochmal durchquert, um den minimalen oder maximalen Wert zu bestimmen. Dies resultiert in zusätzlichen Kosten von $n\cdot c$ für jeden Durchlauf. Insgesamt hat die Sortierung durch Auswählen eine Zeitkomplexität von $\mathcal{O}(n^2)$, wenn die Konstanten fallen gelassen werden. \autocite[156]{sanders_sequential_2019} \autocite[212]{hubwieser_fundamente_2015}

Bei der Sortierung durch das Einfügen werden die Elemente aus dem Eingabefeld der Reihe nach ausgelesen und, basierend  auf ihren Werten in einem neuen Ausgabefeld eingefügt. Beim Einfügen wird der Wert des ausgelesenen Elements mit den Werten der bereits sortierten Elemente in dem Ausgabefeld verglichen. Abhängig von dem Kriterium der Sortierung wird das ausgelesene Element zwischen zwei Elementen eingefügt, die jeweils weniger oder mehr betragen. Im Falle des ersten Elements wird es am Anfang des Feldes platziert. Eine Sortierung durch Einfügen ist auch ohne die Verwendung eines Ausgabefelds möglich. Dabei werden die unsortierten Elemente um eine Stelle nach hinten gerückt, um Platz für das ausgelesene Element zu schaffen. Danach wird der Wert des ausgelesenen Elements mit den der bereits sortierten Werte am Anfang des Eingabefelds verglichen, um dessen neue Position zu bestimmen. Auch bei dieser Variante der Sortierung werden für \textit{n} Elemente genau \textit{n} Iterationen ausgeführt. Beim Einfügen eines Elementes werden zur Bestimmung seiner neuen Position alle bereits sortierten Elemente durchquert. Dieser Wertvergleich zusammen mit der Einrückung der unsortierten Punkten führt dazu, dass höchstens \textit{$n-1$} Elemente in jeder Iteration durchquert werden. Somit ergibt sich eine Zeitkomplexität für die Sortierung durch Einfügen von $\mathcal{O}(n^2)$. \autocite[157]{sanders_sequential_2019} \autocite[210-211]{hubwieser_fundamente_2015}

Zur Sortierung eines Feldes bietet sich eine andere Möglichkeit im Form des Quicksorts (deutsch: schnelle Sortierung) an. Bei diesem Algorithmus wird die Strategie des \enquote{Teilen und Herrschens} (englisch: Divide-and-Conquer) angewendet. Im Grunde wird das zu sortierende Feld rekursiv aufgeteilt, um die Sortierung durch die Lösung kleinerer Teilprobleme auszuführen. Diese Teilprobleme können möglicherweise auch rekursiv durch das Quicksort-Verfahren gelöst werden. Schließlich werden alle Teillösungen zur Lösung des Gesamtproblems zusammengeführt, um eine sortiertes Ausgabefeld zurückzugeben. Zur Aufteilung eines Feldes \textit{F} wird ein zufälliges Pivotelement \textit{x} aus der Datenstruktur gewählt. Alle Elemente mit einem kleineren Wert als \textit{x} werden zum ersten Teilfeld \textit{F\textsubscript{1}} hinzugefügt, während die restlichen Elemente zum zweiten Teilfeld \textit{F\textsubscript{2}} hinzugefügt werden. Danach wird das Quicksort Verfahren rekursiv auf die Teilfelder angewandt, um weitere Teilfelder zu erzeugen. In jedem Teilfeld sind alle Werte entweder größer oder kleiner als das Pivotelement. Durch die rekursive Zerlegung der Teilfelder werden die Teilprobleme immer kleiner, bis jedes Teilproblem nur ein Element enthält. Dank der Teilsortierung in jeder Rekursion stehen auch alle Teilprobleme am Ende sortiert und können zusammengefügt werden. Diese Schritte werden anhand des Beispiels einer Zahlenfolge in Tabelle~\ref{table: quick_sort} verdeutlicht. \autocite[76-77]{knebl_algorithmen_2021}

Zur Bestimmung der Zeitkomplexität des Quicksort-Verfahrens hilft es, die einzelnen Teilprobleme in einer Baumstruktur sich vorzustellen. Jeder Knoten des Baums besitzt maximal zwei Kinder aber mindestens ein Kind. Ausgeschlossen sind von dieser Verallgemeinerung sind die Blätter des Baums. Unter Berücksichtigung dieser Eigenschaft kann die Höhe des Baums mit $|\log_{2}(n)|$ bestimmt werden, wobei \textit{n} die Anzahl der Elemente beträgt. In jeder Ebene des Baums finden bis zu \textit{n} Vergleiche mit dem Pivotelement statt. Somit ergibt sich eine Zeitkomplexität von $\mathcal{O}(n\log_{2}(n))$, die sich als $\mathcal{O}(n\log(n))$ verallgemeinern lässt. Für den schlimmsten Fall, wo jedes Teilproblem nur ein Kind hat, beträgt die Zeitkomplexität \textit{T\textsubscript{worst}} $\mathcal{O}(n^2)$. \autocite[216-217]{hubwieser_fundamente_2015} \autocite[79]{knebl_algorithmen_2021} \autocite[169-170]{sanders_sequential_2019}

\begin{table}[b]
	\centering
	\begin{tabular}{l *{8}{c}}
		\hline
		Array: & 67 & 56 & 10 & 41 & 95 & 18 & 6 & \textit{\underline{42}} \\
		Pivotelement: & & & & & \textit{\underline{42}} & & & \\
		Aufteilung: & 6 & 18 & 10 & \textit{\underline{41}} & & 56 & 67 & \textit{\underline{95}} \\
		Pivotelemente: & & & & \textit{\underline{41}} & & & & \textit{\underline{95}} \\
		Aufteilung:  & 6 & 18 & \textit{\underline{10}} & & & 56 & \textit{\underline{67}} & \\
		Pivotelemente: & & \textit{\underline{10}}  & & & & & \textit{\underline{67}} & \\
		Aufteilung: & 6 & & 18 & & & 56 & & \\
		Sortiertes Array: & 6 & 10 & 18 & 41 & 42 & 56 & 67 & 95 \\
		\hline
	\end{tabular}
	\caption[Beispiel eines Sortieralgorithmus]{Die schrittweise Sortierung eines Feldes mit dem Quicksort Algorithmus}
	\label{table: quick_sort}
\end{table} 

\Textcite[213-214]{hubwieser_fundamente_2015} und \textcite[582-585]{ernst_grundkurs_2020} beschreiben einen weiteren Algorithmus namens Bubblesort. Allerdings wird im Umfang dieser Arbeit dieses Algorithmus nicht detaillierter behandelt, da es zugunsten einer niedrigeren Zeitkomplexität eine inferiore Korrektheit anbietet. Sortieralgorithmen wie Heapsort oder Shellsort können auch laut den Autoren eine hohe Korrektheit mit einer ähnlichen oder niedrigeren Zeitkomplexität leisten, indem sie besondere Datenstrukturen verwenden. Da diese Algorithmen allerdings in dieser Arbeit keine Anwendung finden, werden sie nicht ausführlicher behandelt. 

\subsubsection{Suchen}
Beim Suchen handelt es sich um das Finden eines bestimmten Elements innerhalb einer Datenstruktur. Hierzu wurden bereits ein paar Verfahren in Abschnitt~\ref{Datenstrukturen} erwähnt. In diesem Abschnitt werden sie zusammengefasst und ihre Zeitkomplexität ermittelt. 

Die sequentielle Suche ist das einfachste Verfahren, das zum Finden eines Elements in linearen Datenstrukturen verwendet werden kann. Bei der sequentiellen Suche wird durch die lineare Datenstruktur iteriert und der Wert jedes Elements mit dem des angefragten Elements verglichen. Wird das Element gefunden, liefert das Verfahren eine positive Rückmeldung. Im konversen Fall gelangt die Suche zum Ende der Datenstruktur und findet das angefragte Element nicht. In diesem Fall liefert der Algorithmus eine negative Rückmeldung. Da es bei diesem Verfahren durch die Datenstruktur durchquert wird, müssen in der Regel \textit{n} Elemente verglichen werden. Somit beträgt die Zeitkomplexität der sequentiellen Suche $\mathcal{O}(n)$. \autocite[224]{hubwieser_fundamente_2015}

Die binäre Suche erfolgt ähnlich wie das Quicksort Algorithmus nach der Strategie des \enquote{Teilen und Herrschens}. Eine wichtige Voraussetzung der binären Suche ist die aufsteigende Sortierung der Datenstruktur. In diesem Verfahren wird zuerst ein Element \textit{m} an der mittleren Position der Datenstruktur, in diesem Fall eine Liste, gewählt. Sei das angefragte Element kleiner als \textit{m}, wird in der linken Teilliste weitergesucht. Sei das angefragte Element größer als \textit{m}, wird in der rechten Teilliste weitergesucht. Beträgt das angefragte Element zufälligerweise gleich \textit{m}, wird die Suche erfolgreich beendet. Ansonsten werden die obigen Schritte auf die Teilliste rekursiv angewendet, bis das Element gefunden wird. Zur Bestimmung der Laufzeit dieses Verfahrens soll bemerkt werden, dass die Anzahl der zu durchsuchenden Elemente in jeder Teilliste halbieren. Diese Anzahl steigt mit jeder weiteren Teilliste logarithmisch ab, wodurch es sich eine Zeitkomplexität von $\mathcal{O}(\log(n))$ ergibt. \autocite[224-226]{hubwieser_fundamente_2015}

Suchfunktionen werden mit den meisten nichtlinearen Datenstrukturen wie Bäume gebündelt geliefert. Binärer Suchbäume dienen dem Zweck des Suchens, indem das angefragte Element rekursiv mit der Wurzel sowie dem Knoten jedes Teil- oder Unterbaums verglichen wird. Die genaue Funktionsweise des binären Suchbaums wurde in Abschnitt~\ref{nicht_lineare_datenstrukturen} detaillierter behandelt. Bei einem bilanzierten oder symmetrischen Baum wird die Anzahl der zu untersuchenden Elemente auf jeder Ebene halbiert und folgt einen logarithmischen Trend. Somit ergibt sich im besten Fall eine Zeitkomplexität T\textsubscript{best} von $\mathcal{O}(\log(n))$. Im Schlimmsten Fall T\textsubscript{worst} wird durch einen degenerierten Baum durchquert, der wie eine lineare Liste strukturiert ist. So werden \textit{n} Elemente des Baums durchsucht, weswegen die Zeitkomplexität $\mathcal{O}(n)$ beträgt. \autocite[226-228]{hubwieser_fundamente_2015}

\subsubsection{Hashing}

Hashing Algorithmen dienen dem effizienten Einfügen, Löschen und Suchen von Elementen innerhalb einer Datenstruktur. Gewöhnlich werden für diese drei Operationen Vergleiche der Werte oder der Schlüssel ausgeführt, die in der Regel kostspielig sind, da mehrere Elemente der Datenstruktur dabei durchquert werden, bevor die Operationen enden. Hashing Algorithmen machen von einzigartigen Schlüsseln für jedes Element einer Datenstruktur Gebrauch. Beim Hashing wird der eindeutige Schlüssel eines Elements verwendet, um es zu finden. \autocite[229]{hubwieser_fundamente_2015}

Hashing lässt sich natürlich nur auf Datenstrukturen verwenden, deren Elemente mit einzigartigen Schlüssel verknüpft sind. Bei Hashing werden besondere Datenstrukturen namens Hashtabellen verwendet, die in zwei assoziierte Felder einen Wert mit einem Schlüssel speichern. Dies heißt, dass jedes Element seinen eindeutigen Schlüssel hat. Zur Ermittelung der Position eines Elementes aus seinem Schlüssel wird eine sogenannte Hashfunktion verwendet. Bei einer Hashfunktion wird auf mathematischer Basis die Speicheradresse oder Position eines Elements anhand des Schlüssels berechnet. Nachdem die Position des Elements ermittelt wird, kann das dort gespeicherte Element abhängig von der Operation ausgelesen oder entfernt werden sowie ein neues Element eingefügt werden. Somit werden alle Operationen auf Elemente einer Datenstruktur auf das Aufrufen des Elements reduziert. Die Länge der Datenstruktur verliert somit ihre Einfluss auf die Zeitkomplexität der Operationen und ermöglicht theoretisch eine Zeitkomplexität von $\mathcal{O}(1)$. Um den Einfluss der Hashfunktion auf die Zeitkomplexität möglichst gering zu halten, sollte die Berechnung der Speicheradresse aus dem Schlüssel möglichst effizient erfolgen und dabei auch Kollisionen vermeiden. Kollisionen entstehen dann, wenn zwei eindeutige Schlüssel für unterschiedlichen Elemente auf die gleiche Position innerhalb der Datenstruktur zeigen. \autocite[230-231]{hubwieser_fundamente_2015}

Eine der einfachsten Hashfunktionen ist die Divisions-Rest-Methode oder die Modulo-Methode. In dieser Funktion wird die Position eines Elements ermittelt, indem der Restwert nach einer Division des Schlüssels durch eine konstante Größe \textit{T} berechnet wird. Eine gute Auswahl des Teilers \textit{T} ist für die Kollisionsvermeidung von höchster Bedeutung. Primzahlen, die weit von einer Zweierpotenz liegen, eignen sich für die Verwendung als den Teiler \textit{t} besonders gut. Herkömmlicherweise wird die Größe \textit{T} durch die Anzahl der verfügbaren Schlüssel bestimmt, also muss zur Manipulierung des Teilers die Anzahl der Schlüssel angepasst werden. \autocite[230-231]{hubwieser_fundamente_2015}

\Textcite[232]{hubwieser_fundamente_2015} stellen die perfekte Hashfunktion oder "Perfektes Hashing" vor: ein theoretisches Schema zur kollisionsfreien Ermittlung der Position eines Elementes. Dabei wird die Voraussetzung gesetzt, dass die Anzahl der Elemente nicht die Anzahl der verfügbaren Schlüsseln übersteigt. Mathematisch ausgedrückt sei die Anzahl der Elemente $|K| \leq t$. Grundsätzlich wird die Reihenfolge der Schlüssel nach ihrer Anordnung verwendet, um eindeutig die Positionen von Elementen zu bestimmen. Die genauere Funktionsweise sowie eine Untersuchung der Kollisionsvermeidung liegt außerhalb des Umfangs dieser Arbeit und kann in \textcite[92-94]{mehlhorn_algorithms_2008} nachgelesen werden.

Eine universelle Hashfunktion ist eine spezielle Art von Hashfunktion, die so konstruiert ist, dass sie eine geringe Wahrscheinlichkeit von Kollisionen aufweist. Im Gegensatz zu anderen Hashfunktionen ist eine universelle Hashfunktionen nicht speziell auf einen bestimmten Datentyp oder eine bestimmte Anwendung zugeschnitten, sondern kann auf verschiedene Datentypen und Anwendungen angewendet werden. Die Idee hinter einer universellen Hashfunktionen ist, dass sie zufällig ausgewählt wird, wodurch die Wahrscheinlichkeit von Kollisionen reduziert wird. Eine universelle Hashfunktionen ist im Allgemeinen eine Familie von Hashfunktionen, die auf eine bestimmte Größe von Hashtabellen abgestimmt sind. Um eine universelle Hashfunktionen auszuwählen, wird in der Regel eine zufällige Hashfunktionen aus der Familie ausgewählt, um eine gute Streuung der Schlüssel in der Tabelle zu erreichen. \autocite[232-234]{hubwieser_fundamente_2015} \autocite[114-116]{knebl_algorithmen_2021}

Je nach Auswahl einer Hashfunktion besteht die Gefahr, dass für zwei unterschiedliche Schlüssel der gleiche Wert zurückgegeben wird. Dies fordert eine Methode zur Kollisionsbehandlung auf. Dieses Gefahr besteht insbesondere beim Einfügen eines neuen Elements in der Hashtabelle. Hierzu schlagen \textcite[568-572]{ernst_grundkurs_2020} Methoden vor, die zur Behebung dieser Konflikte verwendet werden. 

\subsection{Verfahren zur datenbasierten Anpassung eines Modells}

\subsubsection{RANSAC}
Der RANSAC-Algorithmus ist eine Methode zur Schätzung von Parametern eines mathematischen Modells aus einer Menge von beobachteten Datenpunkten, die mit Rauschen überlagert sind. Der Name steht für "Random Sample Consensus", was auf die Tatsache zurückzuführen ist, dass der Algorithmus zufällige Stichproben aus den Daten verwendet, um ein Modell zu schätzen, das am besten zu den Daten passt. Das Ziel des RANSAC-Algorithmus besteht darin, ein Modell zu finden, das die meisten Datenpunkte erklären kann, wobei die Ausreißer berücksichtigt werden. Ausreißer sind Datenpunkte, die nicht zum Modell passen und aufgrund von Fehlern in der Messung, unerwarteten Ereignissen oder anderen Faktoren auftreten können. \autocite[381-383]{fischler_random_1981}

Der Algorithmus beginnt damit, eine zufällige Stichprobe von Datenpunkten aus der Gesamtmenge auszuwählen. Diese zufällige Teilmenge wird verwendet, um eine vorläufige Schätzung der Parameter zu erstellen, die die Daten am besten beschreiben. Als nächstes wird jeder Datenpunkt in der Gesamtmenge überprüft, um festzustellen, ob er innerhalb einer vordefinierten Toleranzgrenze der vorläufigen Schätzung liegt. Wenn ein Datenpunkt innerhalb dieser Toleranzgrenze liegt, wird er als \textit{konsistent} oder \textit{Inlier} betrachtet. Andernfalls wird er als \textit{inkonsistent} oder \textit{Ausreißer} betrachtet. Die konsistenten Datenpunkte werden verwendet, um eine neue Schätzung der Parameter zu erstellen. Dieser Prozess wird iterativ wiederholt, indem in jeder Iteration eine neue zufällige Stichprobe von Datenpunkten ausgewählt wird. Dadurch gelingt es dem Verfahren, eine neue vorläufige Schätzung der Parameter zu erstellen. Die Anzahl der zufälligen Stichproben und die Toleranzgrenze werden so gewählt, dass eine hinreichende Wahrscheinlichkeit besteht, dass der Algorithmus ein sinnvolles Modell findet. Am Ende des Algorithmus wird die Schätzung mit den meisten konsistenten Datenpunkten als endgültige Schätzung der Parameter verwendet. Diese endgültige Schätzung minimiert die Auswirkungen von Rauschen und Ausreißern und ist daher robust gegenüber unvorhergesehene Störungen in den Daten. \autocite[383-384]{fischler_random_1981} \autocite[3]{martinez-otzeta_ransac_2023}

Die Funktionsweise des Verfahrens lässt sich besser anhand eines Beispiels verstehen. Angenommen, dass es eine Menge von \textit{n} Punkten auf einer zweidimensionalen Fläche verteilt sind. Zu diesen Punkten gehören auch Rauschen oder Ausreißer. Mit dem RANSAC-Algorithmus wird angestrebt, eine Gerade zu schätzen, die möglichst gut durch die Punkte verläuft. Zuerst werden zwei Punkte aus \textit{n} zufällig gewählt, um eine vorläufige Gerade zu definieren. Danach werden alle Punkte identifiziert, die innerhalb einer gewissen Toleranz zu der Gerade liegen und somit als \textit{Inliers} der Gerade zählen. Diese Anzahl der Inliers wird gespeichert und zur späteren Bewertung dieser Iteration verwendet. Anhand der \textit{Inliers} wird eine Geradengleichung für diese Iteration bestimmt. Danach werden zwei neue zufällige Punkte gewählt, und die vorigen Schritte werden für eine vordefinierte Anzahl von Iterationen wiederholt. Schließlich wird die Geradengleichung ausgewählt, mit der die höchsten Anzahl an Inliers zustimmten. 

\begin{figure}[!b]
	\centering
	\begin{subfigure}{0.43\textwidth}
		\includegraphics[width=\textwidth]{Abbildungen/Line_with_outliers.png}
		\centering
		\caption[Linie mit Ausreißer]{Punkte auf einer zwei dimensionalen Linie mit Ausreißern}
		\label{fig: line_with_outliers}
	\end{subfigure}
	\hfill
	\begin{subfigure}{0.43\textwidth}
		\includegraphics[width=\textwidth]{Abbildungen/Fitted_line.png}
		\centering
		\caption[Linienmodell für eine Linie]{Ein angepasstes Linienmodell auf die Punkte}
		\label{fig: fitted_line}
	\end{subfigure}
	\caption[Visualisierung des RANSAC-Verfahrens]{Eine durch RANSAC erkannte Gerade durch Punkte. Die blauen Punkte stellen Inliers und roten Punkte die Outliers dar.}
	\label{fig: ransac_line}
\end{figure}

Insgesamt ist der RANSAC-Algorithmus eine robuste Methode zur Schätzung von Parametern aus beobachteten Datenpunkten, die durch Rauschen beeinträchtigt sind. Aufbauend auf dem nach \textcite{fischler_random_1981} wurden zusätzliche Abzweigungen des Verfahrens konzipiert, die die Robustheit oder Leistung des Verfahrens verbessern. Das RANSAC Verfahren findet in diversen Bereichen wie maschinelles Sehen (Computer Vision), Robotik und Geodäsie eine sehr breite Anwendung \autocite[2]{martinez-otzeta_ransac_2023}.

\subsubsection{Region-Growing}
Region-Growing-Segmentierung ist ein Verfahren der Bildsegmentierung, welches auf der Idee basiert, dass Pixel, die ähnliche Eigenschaften aufweisen, zu einer zusammenhängenden Region zusammengefasst werden können. Das Ziel dabei ist es, die relevanten Objekte im Bild zu identifizieren und von der Hintergrundinformation zu trennen. \autocite[641]{adams_seeded_1994}

Das Verfahren beginnt mit einem ausgewählten Startpixel, welches als Samen oder Seed bezeichnet wird. Anschließend werden benachbarte Pixel des Seeds analysiert und geprüft, ob sie ähnliche Eigenschaften aufweisen wie der Seed. Wenn dies der Fall ist, werden diese Pixel zur wachsenden Region hinzugefügt. Der Prozess wird wiederholt, bis keine weiteren Pixel gefunden werden, die zur wachsenden Region hinzugefügt werden können.\autocite[641-642]{adams_seeded_1994}

Ein wichtiger Faktor bei der Region-Growing-Segmentierung ist die Wahl der Kriterien für die Ähnlichkeit von Pixeln. Es gibt verschiedene Methoden zur Bestimmung von Ähnlichkeitskriterien, darunter Intensität, Farbe, Textur und Form. Die Wahl hängt von der Art des zu segmentierenden Bildes und den gewünschten Ergebnissen ab. Darüber hinaus spielt die Wahl des Seeds auch eine wichtige Rolle. In Daten mit wenigen Ausreißern hat eine falsche Wahl des Seeds keinen wesentlichen Effekt auf die Korrektheit des Algorithmus, allerdings können viele Ausreißer den Algorithmus stören. In diesem Fall ist es wichtig, dass kein Ausreißer als Seed gewählt wird, um die Erzeugung falscher Segmente zu vermeiden. \autocite[641-643]{adams_seeded_1994}

Region-Growing-Segmentierung lässt sich auf eine breite Auswahl an Bildern anwenden. Dies ermöglicht zahlreiche Anwendungen dieses Algorithmus in der medizinischen Bildgebung, z.B. bei der Segmentierung von Tumoren oder Organen. Diese hohe Anwendbarkeit des Algorithmus lässt auch seine Anwendung in der industriellen Bildverarbeitung und der Robotik zu.\autocite[646]{adams_seeded_1994}
	%Struktur
% -Festlegung von Software-Kriterien
% -Auswahl des Verfahrens
% -Reproduktion des Verfahrens aus der Literatur
% 	-Kantenerkennung
% 	-Segmentierung
%	-Probleme & besondere Erkenntnisse
% -Erweiterung des Verfahrens
%	-Grundlegende Änderungen des Verfahrens
%	-False edge removal
%	-Point marking and cloud filtering
%	-misc.


\chapter{Entwicklungsprozess der Software}
Die Rücksichtnahme des Einsatzzwecks bei der Design und Entwurf des Verfahrens sowie die Entwicklung der Software war erforderlich, um die gewünschte Funktionalitäten gewährleisten zu können. Das Verfahren soll Kanten und Geometrien nicht nur in vollständig generierten Punktwolken erkennen, sondern auch in unvollständige Punktwolken, die iterativ wachsen. Hierbei wird ein Laserliniensensor eine Kante eines Werkstücks oder Objektes entlang geführt und somit sequentiell abgetastet. Deswegen wird die räumliche Struktur des Objektes nicht in einer einzigen Aufnahme abgebildet, sondern durch mehrere kleine Einzelaufnahmen. Der intelligente Schweißroboter, der durch das Fraunhofer Institut für Produktions- und Automatisierungstechnik entwickelt wird, verwendet ein solches Verfahren zum Scannen eines Werkstückes und zur Erkennung Schweißnähte \autocite[39]{savla_intelligente_2022}. Mittels eines Lasersensors wird die Oberfläche des Werkstückes dreidimensional abgebildet. Aktuell wird eine Schweißkegelnaht durch die Erkennung der Schnittlinie zwei Ebenen markiert, die mittels RANSAC-Algorithmen auf die Punktwolke des Werkstückes gefittet werden. Dieses Verfahren zur Erkennung der Schweißnaht bietet allerdings kaum detaillierte Informationen über die Geometrie des Werkstückes an.\autocite[39-52]{savla_intelligente_2022}. Das, in dieser Arbeit entwickelte Verfahren soll das bestehende Verfahren ersetzen und somit seine Limitationen überwinden.

\begin{figure}[h]
	\includegraphics[width = \textwidth]{Abbildungen/collage.jpg}
	\centering
	\caption{Der Laserliniensensor} 
\end{figure}

\section{Vorbereitungen}
\subsection{Software-Voraussetzungen}\label{soft_voraus}
Bei der Auswahl eines geeigneten Verfahrens zur Detektierung Kanten in einer Punktwolke wurden einige Voraussetzungen festgelegt. Die Methode sollte in der Lage sein, nicht nur Außenkanten zu erkennen, sondern auch Innenkanten beziehungsweise Faltungen. Neben dem originellen Einsatzzweck sollte das Verfahren möglichst breit anwendbar sein und eine hohe Modularität aufweisen. Die Funktionen der Kantendetektierung und Punktesegmentierung sollten unabhängig von einander aufrufbar gestaltet werden, um dem Benutzer eine möglichst hohe Flexibilität anzubieten. Die Kantenerkennung sollte performant erfolgen und Punktwolken innerhalb eines praktischen Zeitraums verarbeiten. Letztlich soll das Programm in dem bestehenden Programmpaket des Schweißroboters integrierbar sein. Die Hardwarebeschleunigung des Verfahrens mittels eines Grafikprozessors wurde ausgeschlossen, da ihrer Verwendung mit dem Echtzeitkernels des Programmpakets zur Konflikte führt. 

\subsection{Auswahl eines Verfahrens}
Eine Literatursuche nach Verfahren zur adaptiven Erkennung von Kanten in wachsenden 3D Punktwolken für den Einsatzzweck ergab nichts. Die meisten Verfahren eigneten sich für die Kantenerkennung nur in vollständigen Punktwolken. Aus diesem Grund wurde die Entscheidung getroffen, ein vorhandenes Verfahren aus der Literatur zu wählen und es für den Einsatzzweck anzupassen. Drei unterschiedlichen Verfahren nach \textcite{bazazian_edc-net_2021}, \textcite{himeur_pcednet_2021} und \textcite{rachmadi_road_2017} zeigten viel versprechende Ergebnisse. Allerdings wurden neuronale Netze in dieser Verfahren verwendet, welches zu zwei Problemen geführt hätte. Aufgrund der Funktionsweise neuronaler Netze wäre es schwierig gewesen, diese für den Einsatzzweck ohne eine umständliche Anpassung des neuronalen Netzes anzupassen. Das zweite Hindernis entsteht durch die Einschränkung bei der Verwendung von Grafikprozessoren. Diese Prozessoren hätten die Rechenzeit neuronaler Netze sehr stark verringert und die schnelle Performanz des Verfahrens gewährleistet \autocite[625]{luo_artificial_2005}. Das numerische Verfahren nach \textcite{choi_rgb-d_2013} war auch für den Einsatzzweck ungeeignet, da es als Eingangsparameter eine RGB-D Datei erfordert. Somit wäre das Verfahren nur für eine Anwendung auf organisierten, gefärbten Punktwolken eingeschränkt. Es wurden zwei weitere Verfahren gefunden, die sich zur Erkennung Kanten in organisierten sowie unorganisierten Punktwolken eignen würden. \textcite{mineo_novel_2019} stellten eine numerische Methoden vor, welche zu einer hohen Genauigkeit Kanten erkennen konnte. Allerdings wurden keine Angaben über die Erkennung Innenkanten in dieser Arbeit gemacht. \textcite{ni_edge_2016} schlagen im Gegensatz eine Methode namens AGPN vor, die nicht nur Außen- sowie Innenkanten und Faltungen erkennt, sondern die erkannten Randpunkte zusammen clustert, um Kannten voneinander zu trennen. Diese Studie präsentierte ein Verfahren mit einer hohen Genauigkeit sowie eine Möglichkeit, die Randpunkte sinnvoll zusammen zu gruppieren. Aus diesem Grund wurde dieses Verfahren als Grundlage für das adaptive Verfahren dieser Arbeit gewählt.

\section{Reproduktion des AGPNs}
Bevor das Verfahren für den Einsatzzweck angepasst wurde, wurde es zuerst zwecks einer Überprüfung unverändert implementiert. Es sollte sichergestellt werden, dass das Verfahren für die Erkennung Innenkanten und potenzielle Schweißnähte geeignet ist. Da die Autoren das Quellcode ihres Verfahrens nicht öffentlich zugängig gemacht haben, musste das Programm händisch reproduziert werden. Die Reproduktion des Programms erfolgte in zwei Schritten - die Reproduktion des Verfahrens zur Kantenerkennung und dessen zur Kantensegmentierung. Obwohl andere Skriptsprachen wie Python und MATLAB hinsichtlich des Prototypings Vorteile anbieten, wurde das Programm in C++ wegen seiner besseren Leistungsfähigkeit implementiert \autocite{svensson_performance_2021}. Viele Funktionalitäten der PCL-Bibliothek \autocite{rusu_3d_2011} wurden auch zum Entwurf des Verfahrens verwendet.

\subsection{Verfahren zur Erkennung Randpunkte}
Während Randelemente in zweidimensionale Bilder als eine klare Definition haben, fehlt eine solche Definition für Randelemente und Kanten in 3D-Punktwolken. In diesem Verfahren wurden die geometrischen Eigenschaften einer Kollektion von Punkten zur Erkennung Randpunkte berücksichtigt. Randpunkte weisen eine besondere geometrische Eigenschaft auf - der Winkelabstand zwischen benachbarten Randpunkte ist im Vergleich zu anderen benachbarten Punkten deutlich größer. Faltungen stellen den Grenzbereich zwischen zwei angrenzenden Ebenen dar, deren Normale in unterschiedlichen Richtungen zeigen. Diese geometrischen Eigenschaften wurden zur Erkennung Randpunkte verwendet. \autocite[1-2]{ni_edge_2016}

Im folgenden wird das Verfahren zur Erkennung Randpunkte detaillierter erläutert. Für einen Punkt \textit{o} wurde eine Sammlung von \textit{K} benachbarten Punkten mittels eines kd-trees erstellt. Diese Sammlung wird als eine Nachbarschaft \textit{N\textsubscript{o}} referiert. Danach wurde mittels eines RANSAC-Algorithmus eine Ebene \textit{E\textsubscript{N}} auf diese Nachbarschaft gefittet, um Ausreißer herauszufiltern und zwei angrenzenden Flächen innerhalb der Nachbarschaft voneinander zu trennen. Danach wurde Überprüft, ober der Punkt \textit{o} auf der RANSAC-Ebene lag. Falls dieser Punkt ein Ausreißer der Ebene \textit{E\textsubscript{N}} war, wurde er nicht als einen Randpunkt markiert. Ansonsten wurden weiterhin die geometrischen Eigenschaften der Nachbarschaft überprüft. Abbildung \ref{RANSAC-Ebene} visualisiert die Trennung zwischen unterschiedlichen Flächen einer Punktwolke mittels des RANSAC-Verfahrens. 

\begin{figure}[t]
	\includegraphics[width=\textwidth]{Abbildungen/RANSAC-Ebene.png}
	\centering
	\caption{Eine lokale RANSAC-Ebene (rot dargestellt) neben anderen Oberflächen (blau dargestellt). In \textbf{a} sind drei Ebenen zu sehen, wobei in \textbf{b} nur zwei zu sehen sind. \autocite{ni_edge_2016}}
	\label{RANSAC-Ebene}
\end{figure} 

Im Falle, dass der Punkt \textit{o} ein Inlier war und zu der Ebene \textit{E\textsubscript{N\textsubscript{o}}} gehörte, fang die tatsächliche Überprüfung der geometrischen Eigenschaften der Nachbarschaft \textit{N\textsubscript{o}} an. Um die Ebenengleichung von \textit{E\textsubscript{N}} näherungsweise zu schätzen, wurde zuerst die Normale \textit{$\vec{n}$} der Ebene geschätzt. In einer effizienten Weise wurde die Ebenengleichung durch das RANSAC-Verfahren näherungsweise geschätzt. Diese Gleichung wurde weiterhin auf die RANSAC-Inliers optimiert und daraus die Normale \textins{$\vec{n}$} ermittelt. Danach erfolgte die Errechnung des Winkelabstands zwischen den jeweiligen Punkten von \textit{N\textsubscript{o}}. Hierfür wurden für die Ebene die jeweiligen Eigenvektoren \textit{E\textsubscript{N}} $\vec{u}$ und $\vec{v}$ aus der Normale $\vec{n}$ errechnet. Das fertige Verfahren von PCL zur Ausrechnung der Eigenvektoren lieferte ungenaue Ergebnisse. Stattdessen wurden zur Ermittelung \textit{$\vec{u}$} zwei zufällig gewählte Punkten aus der Inliers verwendet. Es wurde dabei sichergestellt, dass keiner der Punkten den Punkt \textit{o} entsprachen. Zur Errechnung des Winkelabstands wurden zuerst die Winkel aller Punkte der lokalen Ebene \textit{E\textsubscript{N}} errechnet. Mit dem Punkt \textit{o} als Ursprung wurde für jeden Punkt \textit{p\textsubscript{i}}, aus \textit{N\textsubscript{r}} Punkten, der Winkel \textit{$\theta_i$} zu einer Nulllinie errechnet. Danach wurde die Differenz zwischen zwei konsekutiver Punktwinkel $\theta_i$ und $\theta_{i+1}$ errechnet, welcher den Winkelabstand \textit{G\textsubscript{$\theta$}} zwischen zwei Punkten \textit{p\textsubscript{i}} und \textit{p\textsubscript{i+1}} betrug. Abbildung \ref{edge_boundary} zeigt, wie der Winkelabstand zwischen Punkten am Rand der Punktwolke aussieht.

\begin{figure}[h]
	\includegraphics[width=0.5\textwidth]{Abbildungen/angular_gap_boundary}
	\centering
	\caption{Der Winkelabstand \textit{G\textsubscript{$\theta$}} zwischen Punkten am Rand der Punktwolke. \textbf{\(a\)} zeigt ein interner Punkt \textit{o}und ein Nachbarpunkt \textit{p\textsubscript{i}}. Im Vergleich dazu zeigt \textbf{\(b\)} \textit{o} am Rand und den großen Winkelabstand \textit{G\textsubscript{$\theta$}} zwischen Punkte \textit{p\textsubscript{i}} und \textit{p\textsubscript{i + 1}}. \autocite{ni_edge_2016}}
	\label{edge_boundary}
\end{figure}

Auch die Erkennung von Punkten in Innen- und Außenkanten war durch diese Berechnungen möglich. Wie bereits erwähnt, wurden zwei angrenzenden Flächen mittels das RANSAC-Verfahren voneinander getrennt. Falls der Punkt \textit{o} auf der Schnittlinie beider Flächen sowie auf der lokalen RANSAC-Ebene \textit{E\textsubscript{N}} liegt, dann gehört es zum lokalen Rand der Ebene. Falls der Punkt \textit{o} auf der Schnittlinie beider Flächen liegt, aber nicht zu der RANSAC-Ebene gehört, wird es automatisch nicht als einen Randpunkt gemerkt. \textit{E\textsubscript{N}}. Die Abbildung \ref{edge_fold} zeigt, wie der Winkelabstand zwischen Punkten auf einer Schnittlinie zwischen zwei Flächen der Punktwolke aussieht. Die Errechnungen des Winkelabstands erfolgte nach den Gleichungen \ref{first_equation} - \ref{last_equation}.

\begin{figure}[h]
	\includegraphics[width=\textwidth]{Abbildungen/angular_gap_fold}
	\centering
	\caption{Der Winkelabstand zwischen Punkten auf einer Schnittlinie zwei angrenzender Flächen. \textbf{\(a\)} zeigt ein interner Punkt \textit{o} der RANSAC-Ebene. \textbf{\(b\)} zeigt \textit{o} am lokalen Rand der RANSAC-Ebene und den Winkelabstand \textit{G\textsubscript{$\theta$}} zwischen Punkte \textit{p\textsubscript{i}} und \textit{p\textsubscript{i + 1}}. \textbf{\(c\)} zeigt \textit{o} als ein Ausreißer der RANSAC-Ebene. \autocite{ni_edge_2016}}
	\label{edge_fold}
\end{figure}

\begin{equation}
\label{first_equation}
d_i^u = \vec{{op}_i} \cdot \vec{u}
\end{equation}
\begin{equation}
d_i^v = \vec{{op}_i} \cdot \vec{v}
\end{equation}
\begin{equation}
\theta_i = \arctan{\frac{d_i^u}{d_i^v}}
\end{equation}
\begin{equation}
G_\theta = \max(\theta_{i + 1} - \theta_i), i \in \{1, \ldots, N_r\},
\label{last_equation}
\end{equation}



Zur Korrekten Ausrechnung des maximalen Winkelabstands einer Nachbarschaft \textit{G\textsubscript{$\theta$}} war eine aufsteigende Sortierung der Winkel \textit{$\theta_i$} notwendig. Diese Sortierung entsprach eine Sortierung der Punkte \textit{p\textsubscript{i}} nach ihrer aufsteigenden polaren Entfernung von der Nulllinie. Die Abbildung \ref{} dient zur Visualisierung der Methode zur Ausrechnung von $\theta_i$.

	\chapter{Evaluierung und Ergebnisse}
Das IEFD-Verfahren dieser Arbeit ist noch in der Prototypenphase. Wie in Abschnitt~\ref{Motivation} erwähnt, sollte dieses Verfahren unter anderem, eine Anwendung bei der Online-Erkennung von Schweißnähten finden. Um dieses zu erreichen, wurde das Verfahren unter bestimmten Kriterien evaluiert. Zur Evaluierung des Verfahrens wurden Anhand der Forschungsfrage sowie den drei Teilforschungsfragen Tests entworfen. Der erste Test sollte die Genauigkeit des IEFD-Verfahrens überprüfen. Mittels des zweiten Tests wurde der Einfluss der Punktdichte auf das Verfahren überprüft. Der letzte Test überprüfte die Robustheit des IEFD-Verfahrens gegen Objekte mit unterschiedlichen geometrischen Eigenschaften.

\section{Testdaten und Evaluierungsmetriken}
\subsubsection{Testdaten}\label{test_data}
Für alle drei Tests wurden Testdaten erstellt. Nach Bedarf der Tests wurden hierzu Punktwolken künstlich entworfen oder mit einem Lasersensor aufgenommen. Bei der Aufnahme wurden unterschiedliche Werkstücke mit eindeutigen Kanten sowie Geometrien und/oder sichtbaren Knicken ausgesucht, um eine möglichst vielfältige Datenbasis zu erzeugen. Für das Scannen wurde ein Laserliniensensor von \textit{scanControl} verwendet. Die Testdaten entstanden aus Aufnahmen vier reeller Bauteile mit verschiedenen geometrischen Merkmalen. Außerdem ersten wurden die restlichen drei Bauteile auf einem Schweißtisch (PROFIPlusLINE D16) mit Löchern von einem Durchmesser von ca. 15\,mm gespannt. Das erste Bauteil bestand aus einem Blech mit einer kreisförmigen Aussparung sowie zwei runden schmalen Nuten mit einer durchschnittlichen Breite von ca. 3,5\,mm sowie einer durchschnittlichen Tiefe von ca. 3\,mm. Über dem Blech befand sich auch ein halbkreisförmiges Objekt, welches auch aufgenommen wurde. Die nächsten zwei Bauteile bestanden aus zwei senkrecht zueinanderstehenden Bleche mit rechteckigen Aussparungen. Das letzte Bauteil war geometrisch sehr ähnlich zu den letzten zwei Bauteile, allerdings hat es keine rechteckige Aussparungen. Die Innenkante des zweiten Bauteils war rund. Auf dem dritten sowie des vierten Bauteil befanden sich rechteckige Aussparungen unterschiedlicher Dimensionen. Darüber hinaus hatte die Innenkante dieser beiden Bauteile ein stufenartiges Form. Teile des Schweißtisches sowie dessen Löcher wurden bei dem Scan der letzten drei Bauteile mitaufgenommen. Obwohl die Entfernung des Lasersensors von der Werkstückoberfläche über die gesamte Abtastung möglichst konstant gehalten wurde, waren die Punktabstände der Punktwolken manchmal leicht unregelmäßig. Laut \textcite[9]{ni_edge_2016} hatte der Punktabstand den größten Einfluss auf die Leistung von AGPN. Es wurde behauptet, dass das \textit{UniformSampling} diese Verzerrungen der Punktabstände ausgleichen würde. 

Die künstliche Punktwolke wurde auf Basis vorgegebener Parameter mittels eines Python-Skriptes erstellt. Der Zweck dieser Punktwolke war es, ein Objekt mit ein paar einfachen geometrischen Merkmalen zu simulieren. Dieses Objekt wurde dann als ein Ground-Truth verwendet. Unter Berücksichtigung des Einsatzzwecks wurde eine treppenförmige Oberfläche erzeugt, welche eine Innenkante sowie eine Außenkante vorhanden war. Diese Kanten sollten Schweißnähte nachahmen, entlang der durch den Schweißroboter geschweißt werden sollte. Das Weltkoordinatensystem wurde als Referenz für die Erstellung verwendet. Die \textit{x}, \textit{y} und \textit{z} Dimensionen entsprachen den Werten \textit{0,1\,m}, \textit{0,2\,m} beziehungsweise \textit{0,05\,m}. Als den Start wurde der Punkt $\left(\begin{smallmatrix}
	0,01\,m & 0,01\,m & 0,01\,m
\end{smallmatrix}\right)$ gewählt. Der erste Teil der Punktwolke bestand aus eine Oberfläche auf der \textit{xy}-Ebene mit der Dimension $\left(\begin{smallmatrix}
0,05m & 0,2m
\end{smallmatrix}\right)$. Der zweite Teil bestand aus einer Oberfläche auf der \textit{yz}-Ebene mit der Dimension $\left(\begin{smallmatrix}
0,2m & 0,05m
\end{smallmatrix}\right)$, die am Ende des ersten Teils begann. Die Schnittlinie der ersten beiden Teile bildete die Innenkante ab. Der letzte Teil bestand aus einer Oberfläche auf der \textit{xy}-Ebene mit der gleichen Dimension wie der erste Teil. Diese Oberfläche schloss sich am Ende des zweiten Teils an und die daraus entstandene Schnittlinie bildete die Außenkante. Somit ergab sich eine Punktwolke von einer konstanten Größe. Der Punkteabstand dieser Punktwolke wurde für alle Tests unterschiedlich gewählt. Für den ersten und dritte Test wurde ein konstanter und einheitlicher Punkteabstand zwischen allen benachbarten Punkten festgelegt. Für den zweiten Test wurde allerdings der Punkteabstand variiert und wird näher im Abschnitt~\ref{test_2} behandelt. Abbildung~\ref{fig: ground_truth} zeigt die künstlich erstellte Punktwolke, die weiterhin als Ground-Truth benannt wird. 

\begin{figure}[t]
	\includegraphics[scale = 0.5]{Abbildungen/ground_truth.png}
	\centering
	\caption[ground truth]{Die Testdatei beziehungsweise das Ground-Truth}
	\label{fig: ground_truth}
\end{figure}

\subsubsection{Evaluierungsmetriken} \label{evaluations_metrics}
Zur Evaluierung der Leistung des IEFD-Verfahrens wurden ein paar quantitative Metriken nach \textcite[10]{ni_edge_2016} entwickelt. Diese sollten die Genauigkeit der Kantenerkennung sowie die Genauigkeit der Segmentierung überprüfen. Wie in Abschnitt~\ref{edge_detection_reprod} erwähnt, sollte die Differenz der Winkelabstände \textit{G\textsubscript{$\theta$}} Randpunkte zu einem anderen Randpunkt größer oder gleich dem Schwellwert $\alpha$ sein, welcher für diesen Test $\frac{\pi}{2}$ betrug. Die Metrik \textit{p\textsubscript{dc}} wurde zur Angabe der Genauigkeit des IEFD-Verfahrens bei der Kantenerkennung verwendet. Es wurden die Anzahl der korrekt erkannten Kanten \textit{N\textsubscript{dc}} mit der Anzahl der tatsächlich vorhanden Kanten \textit{N\textsubscript{gc}} verglichen. Gleichzeitig wurde auch die Ungenauigkeit \textit{p\textsubscript{mj}} des Verfahrens geprüft, indem die Anzahl der fälschlicherweise markierten Kanten \textit{N\textsubscript{mj}} mit der Anzahl der tatsächlich vorhanden Kanten \textit{N\textsubscript{gc}} verglichen wurden. Wie in Abschnitt~\ref{edge_segmentation} erwähnt, dürfen die Kanten zusammengefasst werden, deren Randpunkte in unmittelbarer Nähe zu einander stehen und deren Richtungsvektoren nicht zu sehr voneinander abweichen. Die Metrik \textit{p\textsubscript{dct}} wurde zur Überprüfung der Genauigkeit der Kantensegmentierung entworfen. Hierbei wurde die Anzahl der korrekt erzeugten Segmente \textit{N\textsubscript{tc}} mit der Anzahl der korrekt erkannten Kanten \textit{N\textsubscript{dc}} sowie der Anzahl der fälschlicherweise erkannten Kanten \textit{N\textsubscript{mj}} verglichen. Gleichzeitig wurde auch die Ungenauigkeit \textit{p\textsubscript{mjt}} überprüft, indem die Anzahl der fälschlicherweise erzeugten Segmente \textit{N\textsubscript{mjt}} mit der Anzahl der korrekt erkannten Kanten \textit{N\textsubscript{dc}} sowie der Anzahl der fälschlicherweise erkannten Kanten \textit{N\textsubscript{mj}} verglichen wurde. Falls \textit{N\textsubscript{mjt}} größer als \textit{N\textsubscript{dc}} + \textit{N\textsubscript{mj}} betrug, wurde \textit{N\textsubscript{mjt}} mit der gesamten Anzahl der erkannten Segmente $\textit{N\textsubscript{mjt}} + \textit{N\textsubscript{tc}}$ verglichen. Die fälschlicherweise erkannten Kanten durch die \textit{FindEdgePoints}-Methode wurden bei der Metrik \textit{p\textsubscript{dct}} sowie \textit{p\textsubscript{mj}} auch mitbetrachtet, da diese vor der Kantensegmentierung nicht verworfen werden. Die Gleichungen ~\ref{pdc} bis ~\ref{pmjt} zeigen die vier Metriken.

\begin{equation}
	\label{pdc}
	p_{dc} = \frac{N_{dc}}{N_{gc}}
\end{equation}
\begin{equation}
	\label{pmj}
	p_{mj} = \frac{N_{mj}}{N_{gc}}
\end{equation}
\begin{equation}
	\label{pdct}
	p_{dct} = \frac{N_{tc}}{N_{dc} + N_{mj}}
\end{equation}
\begin{equation}
	\label{pmjt}
	p_{mjt} =
	\begin{cases}
		\frac{N_{mjt}}{N_{dc} + N_{mj}} & wenn\ N_{mjt} \leq N_{dc} + N_{mj}\\
		\frac{N_{mjt}}{N_{tc} + N_{mjt}} & sonst\ N_{mjt} > N_{dc} + N_{mj}
	\end{cases}
\end{equation}

\section{Erprobung des Verfahrens}
\subsection{Überprüfung der allgemeinen Genauigkeit} \label{test_1}
Die erste Forschungsfrage stellt die Genauigkeit des Verfahrens in Frage. Hierbei sollte geprüft werden wie viele Kanten durch das Verfahren richtig sowie falsch erkannt wurden, und wie viele davon korrekt oder inkorrekt segmentiert wurden. Es wurde postuliert, dass das IEFD-Verfahren zu einer hohen Genauigkeit dank der Einheitlichkeit der Ground-Truth alle Randpunkte erkennen würde. Für diesen Test wurden die beide Verfahren - das AGPN und IEFD - auf Basis des Ground-Truth geprüft. Hierzu wurde ein Punkteabstand von genau 0,00025\,\si{\metre} festgelegt. Die Überprüfung der Genauigkeit erfolgte durch zwei Methoden. Es wurde die Anzahl der, durch das Verfahren erkannten, Randpunkte und Segmente gezählt und mit der Anzahl der tatsächlich vorhandenen Randpunkte beziehungsweise Segmente verglichen. Für die zweite Methode wurden die Metriken aus Abschnitt ~\ref{evaluations_metrics} verwendet, um die Verhältnisse der richtig sowie falsch erkannten Kanten sowie Segmente zu bestimmen. Bei der Festlegung der Parameter wurden die Erkenntnisse aus der Literaturquelle betrachtet. Es wurde erkannt, dass die Parameter \textit{K\textsubscript{1}} und \textit{K\textsubscript{2}} nahezu keinen großen Einfluss auf die Genauigkeit des AGPNs hatten. Der Schwellwert beziehungsweise Glättungsfaktor $\phi$ bestimmte die maximale Abweichung zwischen zwei fugenlosen Segmente. Die Grenzwerte \textit{d\textsubscript{t1}} und \textit{d\textsubscript{t2}} für den Punkteabstand der RANSAC-Verfahren aus der Kantenerkennung und Kantensegmentierung, die die Bestimmung von \textit{Inliers} der Verfahren regeln, hatten die größten Einflüsse auf die Genauigkeit. Es wurde wie in dem Referenzwerk einen Wert für \textit{d\textsubscript{t1}} gleich dem durchschnittlichen Punkteabstand gewählt. Der Wert für \textit{d\textsubscript{t2}} wurde größer gewählt und betrug das dreifache des Punkteabstandes. \autocite[10-11]{ni_edge_2016}

Unter Berücksichtigung der Erkenntnisse aus dem Referenzwerk wurden die restlichen Parameter für das Ground-Truth festgelegt. Die Größe einer Iteration wurde auf 0,01m festgelegt. Die letzten 0,001\,\si{\m} einer Iteration werden wiederholt sowie einen Bereich von 0,0002\,\si{\m} zur Entfernung falscher Kanten festgelegt. Die Tabelle ~\ref{table: parameters_test1} listet die Parameter für das IEFD auf. Aufgrund der relativ hohen Punktdichte wurde festgelegt, dass Kanten und Randpunkte, die weniger als ein drittel Millimeter voneinander entfernt sind, zusammen segmentiert werden durften. Deswegen wurde ein \textit{d\textsubscript{t2}} von 0,0003\,\si{\m} gewählt. Im Gegenzug wurde zwecks einer hohen Genauigkeitsanforderung ein \textit{d\textsubscript{t1}} von 0,0001\,\si{\m} für die Kantenerkennung gewählt.

\begin{table}
	\centering
	\begin{tabular}[width=\textwidth]{l *{8}{c}}
		\hline
		\multirow{2}{*}{\textbf{Datei}}&\multirow{2}{*}{\textbf{Punktezahl}}&\multirow{2}{*}{\textbf{Punkteabstand}}&\multicolumn{6}{c}{\textbf{Parameter}}\\
		& & & \textbf{K\textsubscript{1}} & \textbf{d\textsubscript{t1}} & \textbf{$\alpha$} & \textbf{K\textsubscript{2}} & \textbf{d\textsubscript{t2}} & \textbf{$\phi$} \\
		\hline
		Ground-Truth & 12000000 & 0,0001 & 200 & 0,0001 & $\frac{\pi}{2}$ & 30 & 0,0003 & 0,2 \\
		\hline
	\end{tabular}
	\caption{Parameter für den Test auf die Ground-Truth}
	\label{table: parameters_test1}
\end{table}

Die Überprüfung der Genauigkeit der Kantenerkennung und Segmentierung nach beiden obigen Methoden wurde auch für das AGPN ausgeführt, um Vergleichswerte zu erzeugen. Dank der Einheitlichkeit des Ground-Truths war bekannt, dass insgesamt 14.000 Punkte auf den äußeren Rändern sowie 8.000 auf die Innen- und Außenkanten gaben. Daneben war auch bekannt, dass es insgesamt acht Außenränder sowie zwei Innen- beziehungsweise Außenkanten gibt. Somit ergab sich der Wert von \textit{N\textsubscript{gc}} als 10. Auf Basis dieser Größen wurde die Genauigkeit der IEFD- und AGPN-Verfahren überprüft.

Um die Einflüsse von Ausreißer aus den Ergebnissen zu reduzieren wurden die AGPN und IEFD Verfahren jeweils sechs Mal auf das Ground-Truth ausgeführt. Die Ergebnisse daraus wurden gemittelt und präsentiert. Das IEFD-Verfahren konnte durchschnittlich 13.100 Randpunkte erkennen. Dies entsprach eine Erfolgsrate von $59,54\%$. Die erkannten Randpunkte befanden sich über alle Außenränder sowie Innen- und Außenkanten verteilt. Jede Kante der Ground-Truth wurde durch das Verfahren erkannt und eindeutig bestimmt. Die Kantensegmentierung des IEFD-Verfahrens lieferte sehr gute Ergebnisse. Es wurden durchschnittlich $97,29\%$ der erkannten Randpunkte richtig segmentiert. Bis auf einen Fall wurden in den restlichen fünf Ausführungen des Verfahrens alle 10 Kanten der Ground-Truth erkannt. Im Falle des Ausreißers wurde die Innenkante nicht vollständig segmentiert, sondern wurden zwei zusätzliche Segmente auf der Kante erzeugt. Das AGPN-Verfahren konnte im Gegensatz durchschnittlich 6.900 Randpunkte erkennen und entsprach eine Erfolgsrate von $31,36\%$. Die erkannten Randpunkte befanden sich lediglich auf den Außenränder verteilt. Die Innen- sowie Außenkante wurden durch das Verfahren überhaupt nicht erkannt. Die Kantensegmentierung des AGPN-Verfahrens lieferte im Gegensatz sehr gute Ergebnisse. Hierbei ist zu bemerken, dass die Kantensegmentierung nur auf Basis der erkannten Kanten bewertet wurde, und die fehlende Innen- sowie Außenkante nicht in Betracht gezogen wurden. Es wurden durchschnittlich $99,27\%$ der Randpunkte korrekt segmentiert. Anders als im Falle des IEFD-Verfahrens wurden in keiner der Ausführungen unvollständige Segmente erzeugt. Da die Anzahl der korrekt erkannten und segmentierten Randpunkte nicht im Umfang dieser Arbeit wichtig waren, wurden die Metriken nach Abschnitt ~\ref{evaluations_metrics} zur Bewertung der Verfahren weiterhin verwendet. 

%Obwohl die Genauigkeitsraten der Kantenerkennung beider Verfahren nicht vielversprechend sind, bildeten die erkannten Randpunkte der Verfahren alle Kanten der Testdatei zu einer hohen positionellen Genauigkeit ab. Die fehlenden Randpunkte verursachten keine Lücken in den Kanten und beeinträchtigen die Glätte der erkannten Kanten nicht. Nur für einen Anwendungsfall mit sehr hohen Genauigkeitsanforderung, wo jeder einzelner Randpunkt richtig erkannt werden soll, würden die AGPN- und IEFD-Verfahren unzureichend sein. SOLL IN DISKUSSION

Die Ergebnisse aus den sechs Ausführungen wurden für die Auswertung nach Abschnitt ~\ref{evaluations_metrics} wiederverwendet. Das IEFD-Verfahren hatte einen durchschnittlichen \textit{p\textsubscript{dc}} Wert von 1,00, da in allen sechs Ausführungen alle Kanten der Ground-Truth erkannt wurden. Spiegelbildlich zu dieser Metrik hatte die Metrik \textit{p\textsubscript{mj}} einen Wert von 0,00, da das Verfahren keine Kanten fälschlicherweise erkannt hat. In fünf der Ausführungen wuren alle erkannten Kanten richtig und vollständig segmentiert. Im Falle des einen Ausreißers wurde die Innenkante teilweise richtig segmentiert, sowie zwei weitere Segmente auf der Kante erzeugt. Somit ergibt sich einen durchschnittlichen Wert von 0,98 für die Metrik \textit{p\textsubscript{dct}} allerdings auch einen Wert von 0,02 für die Metrik \textit{p\textsubscript{mjt}}. Insgesamt funktionierte das IEFD-Verfahren mit einer Genauigkeit von circa 98,61\%. Das AGPN-Verfahren hatte einen durchschnittlichen \textit{p\textsubscript{dc}} Wert von 0,80, da die Innen- und Außenkante in keiner der sechs Ausführungen erkannt wurden. Das AGPN-Verfahren hat zwei Kanten nicht erkannt und erzielte somit eine Bewertung von 0,2 für die Metrik \textit{p\textsubscript{mj}}. Die Kantensegmentierung erfolgte in allen sechs Ausführungen fehlerfrei, abgesehen davon, dass die Innen- und Außenkanten nicht erkannt wurden. Das AGPN-Verfahren erzielte eine Bewertung von 1,00 für die Metrik \textit{p\textsubscript{dct}} und 0,00 für die Metrik \textit{p\textsubscript{mjt}}. Insgesamt konnte das AGPN-Verfahren Kanten mit einer Genauigkeit von 80,00\% erkennen und segmentieren. Die Tabelle ~\ref{table: metric_values} fasst diese Ergebnisse zusammen.

\begin{table}[h]
	\centering
	\begin{tabular}{l *{9}{c}}
		\hline
		\textbf{Verfahren} & \textbf{N\textsubscript{dc}} & \textbf{N\textsubscript{mj}} & \textbf{N\textsubscript{gc}} & \textbf{p\textsubscript{dc}} & \textbf{p\textsubscript{mj}} & \textbf{N\textsubscript{tc}} & \textbf{N\textsubscript{mjt}} & \textbf{p\textsubscript{dct}} & \textbf{p\textsubscript{mjt}} \\
		\hline
		AGPN & 8 & 0 & 10 & 0,80 & 0 & 8 & 0 & 1,00 & 0 \\
		IEFD & 10 & 0 & 10 & 1,00 & 0 & 9 & 2 & 0,9 & 0,2 \\
		\hline
	\end{tabular}
	\caption{Die schlechtesten Werte der jeweiligen Metriken aus sechs Wiederholungen}
	\label{table: metric_values}
\end{table}

Die Abbildung ~\ref{fig: segments_comparision_grnd_trth} visualisiert die Qualität der erkannten und segmentierten Kanten und vergleicht die Ergebnisse aus beiden Verfahren. Wie es zu sehen ist, wurden durch das AGPN-Verfahren nach Abbildung ~\ref{fig: agpn_segments_grnd_trth} die Innen- und Außenkanten nicht erkannt. Im Gegensatz dazu ist lauf Abbildung ~\ref{fig: iefd_segments_grnd_trth} zu sehen, dass das IEFD-Verfahren alle Ränder sowie Innen- und Außenkanten erkennen konnte. 
%Wie es in dieser Abbildung zu sehen ist, verlaufen die erkannten Kanten nahezu lückenlos. Trotz der fehlenden Randpunkte sind alle Kanten und somit die Struktur und das Rahmen der Testdatei eindeutig zu erkennen. Die Ecken dieser Kanten stellen Bereiche dar, wo ein paar Randpunkte zu keiner der angrenzenden Kanten zugewiesen werden konnten. Allerdings sind diese Randpunkte so niedrig in der Anzahl, dass es keinen großen Ausmaß machte. SOLL IN DISKUSSION

\begin{figure}[h]
	\centering
	\begin{subfigure}[h]{0.49\textwidth}
		\includegraphics[width=\textwidth]{Abbildungen/ground_truth_segments_agpn.png}
		\centering
		\caption{Durch AGPN erkannten Segmente}
		\label{fig: agpn_segments_grnd_trth}
	\end{subfigure}
	\hfil
	\begin{subfigure}[h]{0.49\textwidth}
		\includegraphics[width=\textwidth]{Abbildungen/ground_truth_segments_iefd.png}
		\centering
		\caption{Durch IEFD erkannten Segmente}
		\label{fig: iefd_segments_grnd_trth}
	\end{subfigure}
	\caption{Die Segmente der Ground-Truth, die durch beide Verfahren erkannt werden}
	\label{fig: segments_comparision_grnd_trth}
\end{figure}

Somit ließ sich die erste Forschungsfrage beantworten - das IEFD-Verfahren bietet eine vergleichbare hohe Genauigkeit wie das AGPN-Verfahren und kann sogar Kanten besser erkennen. Um dieses Verfahren weiterhin auf seine Einsatzfähigkeit zu überprüfen erfolgten die nächsten Tests.

\subsection{Überprüfung des Einflusses der Punktedichte} \label{test_2}
Das Referenzwerk \autocite{ni_edge_2016} deutete auf den Einfluss der Punktedichte auf die Genauigkeit des Algorithmus hin. Die Punktedichte einer Punktwolke korreliert mit schärfer umrissenen Abbildungen, da es eine höhere Anzahl von Punkten auf die Kanten und andere geometrischen Merkmale vorhanden sind. Somit sollten diese Merkmale und deren Kanten einfacher erkannt werden. Deswegen wurde die Hypothese aufgestellt, dass sich die Genauigkeit des IEFD-Verfahrens proportional zu der Punktedichte verhalten würde - eine Abnahme der Punktdichte sollte eine Verringerung der Genauigkeit entsprechen. Um diese Hypothese auszutesten wurde auch das Ground-Truth verwendet. Um unterschiedlichen Punktedichten zu simulieren wurde der Punkteabstand der Ground-Truth zwischen einen Bereich von 0,00005\,\si{\m} zu 0,005\,\si{\m}m diskret variiert. Die Schrittgröße lässt sich aus der Tabelle~\ref{table: test_2_results} ermitteln. Somit ergaben sich 9 unterschiedlichen Varianten der Ground-Truth mit verschiedenen Punktedichten. Die Dimensionen aller Varianten der Ground-Truth blieben dabei konstant. Abbildung~\ref{fig: testdata_pointdensity_comparision} zeigt den Unterschied der Punktedichte zwischen der ersten Variante der Ground-Truth mit einem Punkteabstand von 0,00005\,\si{\m}m und der neunten Variante der Ground-Truth mit einem Punkteabstand von 0,005\,\si{\m}m.

\begin{figure}[h]
	\centering
	\begin{subfigure}{0.49\textwidth}
		\includegraphics[width=\textwidth]{Abbildungen/ground_truth_0,00005.png}
		\centering
		\caption{Die erste Variante der Ground-Truth mit Punkteabstand 0,00005m}
		\label{fig: testdata_0,00005m}
	\end{subfigure}
	\hfill
	\begin{subfigure}{0.49\textwidth}
		\includegraphics[width=\textwidth]{Abbildungen/ground_truth_0,005.png}
		\centering
		\caption{Die neunte Variante der Ground-Truth mit Punkteabstand 0,005m}
		\label{fig: testdata_0,005m}
	\end{subfigure}
	\caption{Vergleich zwischen der Varianten der Ground-Truth mit der höchsten und niedrigsten Punktedichte}
	\label{fig: testdata_pointdensity_comparision}
\end{figure}

Da der Punkteabstand in diesem Test variiert wurde, durfte die Anzahl der Scan-Linien \textit{n} in einer Iteration nicht auf Basis der Fensterbreite von 0,4\,\si{\mm} bestimmt werden. Stattdessen wurde \textit{n} auf die fest Zahl von 80 festgelegt. Bei der Variante mit einem Punkteabstand von 0,001\,\si{\m} wurde \textit{n} auf 50, bei 0,0025\,\si{\m} auf 20 und bei 0,005\,\si{\m} auf 10 . Dabei wurde es versucht, dass mindestens fünf Iterationen ausgeführt wurden. Auch die Anzahl \textit{k} der wiederholten Scan-Linien aus Abschnitt~\ref{Testumgebung} wurde so festgelegt, dass immer 10 Scan-Linien wiederholt wurden. Für die Entfernung falscher Kanten wurde ein Bereich definiert, der maximal zwei Scan-Linien am Anfang oder am Ende der Iteration entfernt hat. Außer den Parametern \textit{d\textsubscript{t1}} und \textit{d\textsubscript{t2}} wurden alle Parameter konstant gehalten und mit den gleichen Werten aus Abschnitt ~\ref{test_1} verwendet. Die Parameter \textit{d\textsubscript{t1}} und \textit{d\textsubscript{t2}} wurden wieder abhängig von dem Punkteabstand gewählt. \textit{d\textsubscript{t1}} wurde dem Punkteabstand gleich groß gewählt und \textit{d\textsubscript{t2}} betrug das dreifache von \textit{d\textsubscript{t1}}. Tabelle ~\ref{table: test_2_results} stellt die Ergebnisse dieses Tests dar.

\begin{table}[b]
	\centering
	\begin{tabular}[width = \textwidth]{l *{8}{c}}
		\hline
		\multirow{2}{2em}{\textbf{Id.}} & \multirow{2}{3em}{\textbf{Punkte-zahl}} & \multirow{2}{3em}{\textbf{Punkte-abstand}} & \multicolumn{2}{c}{\textbf{Parameter}} & \multirow{2}{*}{\textbf{p\textsubscript{dc}}} & \multirow{2}{*}{\textbf{p\textsubscript{mj}}} & \multirow{2}{*}{\textbf{p\textsubscript{dct}}} & \multirow{2}{*}{\textbf{p\textsubscript{mjt}}} \\
		& & & \textbf{d\textsubscript{t1}} &\textbf{d\textsubscript{t2}} & & & & \\
		\hline
		1 & 48.000.000 & 0,00005 & 0,00005 & 0,00015 & 0,8 & 0 & 1 & 0 \\
		2 & 21.326.667 & 0,000075 & 0,000075 & 0,000225 & 1 & 0 & 1 & 0 \\
		3 & 12.000.000 & 0,0001 & 0,0001 & 0,0003 & 1 & 0 & 1 & 0 \\
		4 & 1.920.000 & 0,00025 & 0,00025 & 0,00075 & 1 & 0 & 1 & 0 \\
		5 & 480.000 & 0,0005 & 0,0005 & 0,0015 & 1 & 0 & 1 & 0 \\
		6 & 212.667 & 0,00075 & 0,00075 & 0,00225 & 1 & 0 & 1 & 0 \\
		7 & 120.000 & 0,001	& 0,001 & 0,003 & 1 & 0 & 1 & 0 \\
		8 & 19.200 & 0,0025 & 0,0025 & 0,0075 & 1 & 0 & 0,9 & 0,1 \\
		9 & 4.800 & 0,005 & 0,005 & 0,015 & 0,7 & 0 & 0,86 & 0,14 \\
		\hline 
	\end{tabular}
	\caption{Ergebnisse des zweiten Tests}
	\label{table: test_2_results}
\end{table}

Die Varianten \textit{2-7} liefern perfekte Ergebnisse. Das IEFD-Verfahren leistet trotz einer absteigende Punktedichte für die Varianten eine sehr hohe Genauigkeit. Bei der achten Variante wurden trotz der höheren Punktedichte alle Kanten der Ground-Truth richtig erkannt. Allerdings war die Kantensegmentierung nicht genau so erfolgreich. Die Außenkante, die in der y-Richtung und somit der Scan-Richtung zeigte, wurde nicht vollständig segmentiert. Stattdessen wurden durch das Verfahren zwei Segmente auf der Kante erzeugt. Somit erzielte der Testdurchlauf der achten Variante für \textit{p\textsubscript{dct}} eine Bewertung von \textit{0,9} sowie für die Metrik \textit{p\textsubscript{mjt}} einen Wert von \textit{0,1}. Bei der neunten Variante der Ground-Truth konnten weder die Kantenerkennung noch die Kantensegmentierung vergleichsweise gut abschneiden. Es wurden bei dieser Variante der Ground-Truth drei der sechs Seitenkanten nicht vollständig erkannt. Diese Kanten lagen auf der ersten sowie der dritten \textit{xy}-Ebene und zeigten in der x-Richtung. Die Randpunkte auf die, durch das Verfahren erkannten, Kanten waren dürftig verteilt, wodurch die Kanten nicht lückenlos erschienen. Allerdings wurden keine zusätzliche falsche Kanten durch as Verfahren erkannt. Aus diesem Grund erlangte der Testdurchlauf der neunten Variante einen \textit{p\textsubscript{dc}}-Wert von \textit{0,7} und sowie einen \textit{p\textsubscript{mj}}-Wert von \textit{0}. Die Kantensegmentierung konnte aus den sieben erkannten Kanten nur fünf richtig segmentieren. Hierbei wurde eine Seitenkante der dritten \textit{xy}-Ebene nicht von der Längskante unterschieden und wurde somit falsch segmentiert. Auch die Außenkante konnte nicht vollständig segmentiert werden, sondern wurden auf die Kante zwei Segmente erzeugt. Der Wert für \textit{p\textsubscript{dct}} betrug somit \textit{0,86} und \textit{0,14} für die Metrik \textit{p\textsubscript{mjt}}. Die erste Variante der Ground-Truth mit einem Punkteabstand von 0,00005m lieferte auch unerwartete Ergebnisse. Trotz der sehr hohen Punktedichte wurden die Innen- und Außenkanten von Ground-Truth durch das Verfahren nicht erkannt, weswegen dieser Testdurchlauf eine \textit{p\textsubscript{dc}} Bewertung von \textit{0,8} sowie eine \textit{p\textsubscript{mj}} Bewertung von \textit{0,2} erhielt. Diese acht Randelemente wurden korrekt und vollständig durch das Verfahren segmentiert und erhielten eine perfekte Bewertung \textit{1,0} für die Metrik \textit{p\textsubscript{dct}}. Abbildung ~\ref{fig: point_density_bar_chart} visualisiert die Leistung des IEFD-Verfahrens.

\begin{figure}[t]
	\includegraphics[width=\textwidth]{Abbildungen/point_density_influence_bar.png}
	\centering
	\caption{Vergleich der Genauigkeit des IEFD-Verfahrens bei einer Änderung der Punktedichte}
	\label{fig: point_density_bar_chart}
\end{figure}

Nachdem der Einfluss der Punktedichte auf die Genauigkeit des Verfahrens überprüft wurde, wurde der nächste Test ausgeführt, um gezielt die Einsatzfähigkeit des Verfahrens unter reellen Bedingungen zu überprüfen.

\subsection{Überprüfung der Robustheit} \label{test_3}
Wie in Abschnitt ~\ref{Motivation} erwähnt, soll das Verfahren unter reellen Bedingungen verwendbar sein und unterschiedlichen Geometrien erkannt werden. Dieses verlangt eine hohe Robustheit des Verfahrens. Anhand der Erkenntnisse aus den vorherigen Tests wurde die Hypothese aufgestellt, dass das Verfahren unter rellen Bedingungen zuverlässig funktionieren würde. Um diese Hypothese zu überprüfen wurden zwei Tests konzipiert. Da reelle Aufnahmen durch einen Lasersensor häufig sehr unregelmäßig sind, wurde für den ersten Test der Einfluss einer steigenden Unregelmäßigkeit des Punkteabstandes auf die Genauigkeit des Verfahrens überprüft. Bei dem zweiten Test wurde das IEFD-Verfahren auf verschiedene Aufnahmen von reellen Bauteilen angewendet und anhand der Metriken aus Abschnitt ~\ref{evaluations_metrics} bewertet.

\subsubsection{Erste Teiluntersuchung} \label{test_3_part_1}
Für die erste Teiluntersuchung wurde die Ground-Truth verwendet. Hierbei wurde einen Punkteabstand von 0,00025\,\si{\m} festgelegt. Die reelle Punkte aus dem Lasersensor, wiesen keinen regelmäßigen Punkteabstand auf. Um diese Unregelmäßigkeit nachzuahmen, wurde eine künstliche Verzerrung des Punktmusters dem Ground-Truth implementiert. Hierfür wurde ein zufälliger Versatz \textit{d} auf dem ursprünglichen Positionen der Punkte aufaddiert. Zur Errechnung des Versatzes wurde eine zufällige reelle Zahl zwischen -1 und 1 gewählt und mit einem Verzerrungsfaktor \textit{\^{r}} multipliziert. Das Faktor ließ sich errechnen, indem der konstante Punkteabstand \textit{r} mit einer Zahl \textit{k} zwischen 0 und 1,5 multipliziert wurde. Für jeden Punkt wurde ein unterschiedlicher Versatz \textit{d} in allen drei Richtungen des Koordinatensystems errechnet, um den Abstand zwischen Punkten möglichst unregelmäßig zu gestalten. Abbildung ~\ref{fig: point_pattern_comparision} zeigt die unterschiedlichen Mustern der Punkte in dem Ground-Truth mit und ohne einem Rauschen. 

\begin{figure}[b]
	\centering
	\begin{subfigure}{0.49\textwidth}
		\includegraphics[width=\textwidth]{Abbildungen/point_pattern_without_dist.png}
		\centering
		\caption{Punktmuster von Ground-Truth ohne Verzerrung}
		\label{fig: point_pattern_without_dist}
	\end{subfigure}
	\hfill
	\begin{subfigure}{0.49\textwidth}
		\includegraphics[width=\textwidth]{Abbildungen/point_pattern_with_dist.png}
		\centering
		\caption{Punktmuster von Ground-Truth mit Verzerrung und \textit{\^{r}=1,5}}
		\label{fig: point_patter_with_dist}
	\end{subfigure}
	\caption{Vergleich der Punktmuster des Ground-Truths mit und ohne Verzerrung}
	\label{fig: point_pattern_comparision}
\end{figure}

Für diesen Test wurden die Parameter außer \distthresha und \distthreshb des IEFD-Verfahrens konstant gehalten und aus Abschnitt ~\ref{test_1} übernommen. Zwecks der kürzeren gesamten Verarbeitungszeit wurde ein höherer Punktabstand von 0,00025\,\si{\m}, da es eine optimale Bilanz zwischen Punktedichte und Punktezahl angeboten hat sowie den durchschnittlichen Punkteabstand der Aufnahmen entsprochen hat, die mittels des Lasersensors aufgenommen wurden. Dementsprechend wurde \distthresha auf den Wert 0,00025 und \distthreshb auf den Wert 0,00075 festgelegt. Um das Verzerrungsfaktor zu manipulieren, wurde die Zahl \textit{k} diskret von der Untergrenze \textit{0} bis zur Obergrenze \textit{0,15} mit einer Schrittgröße \textit{0,1} inkrementiert. Die Größe der Iteration~\textit{n} wurde ähnlich wie zuvor dynamisch mit einer Mindestzahl von 80 errechnet. Die Anzahl der wiederholten Scan-Linien \textit{k} betrugen wiederum mindestens 15. Zur Entfernung falscher Kanten wurde eine Region festgelegt, die maximal zwei Scan-Linien am Anfang oder am Ende einer Iteration umfasste. Nach Festlegung dieser Parameter wurde der Test ausgeführt und die Ergebnisse wurden in Tabelle ~\ref{table: point_distortion_results} aufgelistet.

\begin{table}[t]
	\centering
	\begin{tabular}[width=\textwidth]{l *{5}{c}}
		\hline
		\textbf{Ausführung} & \textbf{Verzerrungsfaktor \^{r}} & \textbf{p\textsubscript{dc}} & \textbf{p\textsubscript{mj}} & \textbf{p\textsubscript{dct}} & \textbf{p\textsubscript{mjt}} \\
		\hline
		1 & 0,1 & 1 & 0 & 1 & 0 \\
		2 & 0,2 & 1 & 0 & 1 & 0 \\
		3 & 0,3 & 1 & 0 & 1 & 0 \\
		4 & 0,4 & 1 & 0 & 1 & 0 \\
		5 & 0,5 & 1 & 0 & 1 & 0 \\
		6 & 0,6 & 1 & 0 & 1 & 0 \\
		7 & 0,7 & 1 & 0 & 1 & 0 \\
		8 & 0,8 & 1 & 0 & 1 & 0 \\
		9 & 0,9 & 1 & 0,2 & 0,92 & 0,25 \\
		10 & 1,0 & 1 & 0,8 & 0,67 & 0,17 \\
		11 & 1,1 & 1 & 1 & 0,6 & 0,35 \\
		12 & 1,2 & 1 & 1,3 & 0,53 & 0,43 \\
		13 & 1,3 & 1 & 1,6 & 0,39 & 0,34 \\
		14 & 1,4 & 1 & 1,5 & 0,4 & 0,36 \\
		15 & 1,5 & 0,8 & 1,4 & 0,42 & 0,33 \\
		\hline
	\end{tabular}
	\caption{Die Genauigkeit des Verfahrens gegen steigender Verzerrung der Punkte}
	\label{table: point_distortion_results}
\end{table}

Die Ausführungen \textit{1-8} lieferten perfekte Ergebnisse, wo das IEFD-Verfahren trotz steigender Verzerrung der Punkte alle zehn Kanten des Ground-Truths erkennen sowie vollständig segmentieren konnte. Als die Verzerrung der Punkte erhöht wurde, wurden neben allen Kanten auch weitere Einzelpunkte als Randpunkte erkannt. Diese Einzelpunkte \(Störpunkte\) lagen zufällig gestreut und es wurde weder eine linienartige Kante noch ein anderes geometrisches Muster visuell erkannt. Ab der neunten Ausführung konnten Spuren der falschen Kanten aus Abschnitt ~\ref{false_edges} erkannt werden. Neben den zehn Kanten des Ground-Truths waren zwei falsche Kanten leicht erkennbar, weswegen die neunte Ausführung eine \textit{p\textsubscript{dc}} und \textit{p\textsubscript{mj}} Bewertung von \textit{1,0} beziehungsweise \textit{0,2} erhalten hat. Aufgrund der Dürftigkeit der falschen Kanten konnte eine dieser Kanten nicht vollständig segmentiert wurden. Stattdessen wurden an der Stelle zwei Segmente erzeugt. Darüber hinaus wurde die Innenkante des Ground-Truths auch nicht vollständig segmentiert. An den Stellen, wo die falschen Kanten die Innenkante geschnitten haben, wurde ein neues Segment erzeugt. Somit erhielt diese Variante eine Bewertung von \textit{0,92} und \textit{0,25} für die Metriken \textit{p\textsubscript{dct}} beziehungsweise \textit{p\textsubscript{mjt}}. Ab einem Verzerrungsfaktor \textit{\^{r}} von \textit{1,0} der Verzerrung konnten mehr falschen Kanten zwischen jeder Iteration erkannt werden. Diese waren allerdings ähnlich dürftig und undeutlich definiert.  Dieser Trend der steigenden Verzerrung wurde bis zu der 13. Ausführung bemerkt, wo der Höchstpunkt erreicht wurde. Relativ zur früheren Ausführungen gab es die meisten Störpunkte und sowie die meisten falschen Kanten, die nicht entfernt wurden. Ab dieser Stelle wurde ein Plateau erreicht. Als die Verzerrung der Punkte erhöht wurde, wurden ungefähr die gleichen Anzahl an falschen Kanten erkannt. Gleichzeitig stiegen die Anzahl der Störpunkte, die durch die Kantenerkennung erkannt wurden. Trotzdem wurden alle Randelemente sowie die Innen- und Außenkante des Ground-Truths durch das Verfahren richtig erkannt. Aufgrund der hohen Anzahl an Störpunkte waren allerdings die Innen- und Außenkanten leicht dürftig. Insgesamt erzielten die Ausführungen 13 bis 15 einen \textit{p\textsubscript{dc}} Wert von 1.0, allerdings waren die Werte für \textit{p\textsubscript{mj}} deutlich höher. Die höhere Anzahl der falschen Kanten hat dazu geführt, dass die Randelemente in der y-Richtung teilweise nicht vollständig segmentiert wurden, sondern wurden bis zu zwei zusätzlichen Segmente erzeugt. Darüber hinaus hat die hohe Anzahl der Störpunkte dazu geführt, dass weder die Innenkante noch die Außenkante vollständig segmentiert wurde. Stattdessen wurde in jeder Iteration ein neues Segment gestartet. Dadurch erhielten beide Metriken \textit{p\textsubscript{dct}} und \textit{p\textsubscript{mjt}} nahezu die gleiche Werte von ca. 0,4 beziehungsweise 0,34. Abbildung ~\ref{fig: point_distortion_comparision} visualisiert den Trend der Metriken. 

\begin{figure}[h]
	\includegraphics[scale=0.85]{Abbildungen/point_distortion_influence_line.png}
	\centering
	\caption{Das Verhältnis zwischen der Genauigkeit des Verfahrens und dem Verzerrungsfaktor der Punktverzerrung.}
	\label{fig: point_distortion_comparision}
\end{figure}

\subsubsection{Zweite Teiluntersuchung} \label{test_3_part_2}
Nach der Überprüfung der Genauigkeit bei einer Änderung der Punktverzerrung wurde die Robustheit des IEFD-Verfahrens gegen unterschiedlichen Geometrien überprüft. Dabei wurden reelle 3D-Aufnahmen von vier verschiedenen Bauteilen verwendet. Ähnlich wie zuvor wurden alle Parameter außer \distthresha und \distthreshb aus Abschnitt ~\ref{test_1} übernommen. Vor Festlegung von \distthresha und \distthreshb war noch einen Schritt nötig. Die Punktedichte der Aufnahmen war an manchen Stellen der Punktwolke deutlich höher als im Durchschnitt. Diese waren die Stellen, wo der Lasersensor im Vergleich zu anderen Stellen länger aufhielt, wodurch mehrere Scan-Linien an den Stellen generiert wurden. Abbildung ~\ref{fig: point_density_before} visualisiert diese Anomalie. 

\begin{figure}[t]
\includegraphics[width=1\textwidth]{Abbildungen/point_density_distribution_blech_before.png}
\centering
\caption{Die Anzahl der Nachbarpunkte jedes Punktes vor dem Downsampling im Umfang von einem Millimeter}
\label{fig: point_density_before}
\end{figure}

Diese lokalisierten Fällen der hohen Punktedichte hatten eine schlechte Auswirkung auf der Kantensegmentierung und führte dazu, dass diese Kanten nicht vollständig segmentiert wurden. Durch die Anwendung der \textit{UniformSampling} Methode konnte dieses Problem gelöst werden und eine einheitlichere Punktedichte sichergestellt werden. Es wurde für jede Aufnahme eine Blattgröße festgelegt, die den durchschnittlichen Punkteabstand nach dem Downsampling regelte. Somit hatte jeder Punkt ungefähr die gleichen Anzahl an Nachbar innerhalb eines bestimmten Radius von 0,001\,\si{\m}. Dieses wird in Abbildung~\ref{fig: point_density_after} dargestellt. Dieser Wert wurde für den Parameter \distthresha festgelegt. Der Parameter \distthreshb wurde für alle Bauteile unterschiedlich gewählt, durfte allerdings weder kleiner als \distthresha noch größer als $1,5 \cdot \distthresha$ sein. Die Parameter für jedes Bauteil können aus der Tabelle~\ref{table: test_3-2_results} entnommen werden. Die Argumentation hinter dem Auswahl dieser Wert wird in der Diskussion in Kapital~\ref{diskussion} diskutiert.

\begin{figure}[t]
	\includegraphics[width=\textwidth]{Abbildungen/point_density_distribution_blech_after.png}
	\centering
	\caption{Die Anzahl der Nachbarpunkte jedes Punktes nach dem Downsampling}
	\label{fig: point_density_after}
\end{figure}

Nachdem die Parameter des Verfahrens festgelegt wurden, wurde jeder Scan drei Mal durch das Verfahren verarbeitet und die Ergebnisse davon wurden gemittelt, um den Einfluss von Ausreißer zu verringern. Diese Scans enthielten Bauteile mit kreisförmigen Löchern, Innenkanten, Nuten, Schlitze und weiteren geometrischen Merkmalen. Die Erkennung dieser Merkmale galt als erfolgreich, wenn die Kanten zu diesen Merkmalen erkannt wurden. Für jede Punktwolke wurden auch die die Anzahl der Punkte sowie die Anzahl der erkennbaren Kanten \textit{N\textsubscript{gc}} manuell bestimmt´. Drei der vier Punktwolken der Bauteile wiesen einen höheren durchschnittlichen Punkteabstand von ca. 0,00028\,\si{\m} in einem bestimmten Bereich am Ende des Scans auf. Die restlichen Punkte standen im Gegensatz ca. 0,00008\,\si{\m}m voneinander entfernt. Dieses war vor dem Downsampling der Fall. Aus diesem Grund wurde für diese drei Punktwolken eine \textit{Leaf-Size} von 0,0003m gewählt. Die Tabelle ~\ref{table: test_3-2_results} fasst diese Erkenntnisse und Ergebnisse zusammen. 

\begin{table}[t]
	\centering
	\begin{tabularx}{\textwidth}{l c >{\centering}X *{7}{c}}
		\hline
		\textbf{Id.} & \textbf{Punktezahl} & \textbf{Punkteabstand} & \textbf{d\textsubscript{t1}} & \textbf{d\textsubscript{t2}} & \textbf{N\textsubscript{gc}} & \textbf{p\textsubscript{dc}} & \textbf{p\textsubscript{mj}} & \textbf{p\textsubscript{dct}} & \textbf{p\textsubscript{mjt}}\\
		\hline
		1 & 2324097 & 0,0001 & 0,0001 & 0,00015 & 22 & 0,82 & 0 & 0,83 & 0,39 \\
		2 & 813997 & 0,0003 & 0,0003 & 0,0003 & 49 & 0,94 & 0 & 0,89 & 0,19 \\
		3 & 795356 & 0,0003 & 0,0003 & 0,00045 & 92 & 0,95 & 0 & 0,92 & 0,09 \\
		4 & 707337 & 0,0003 & 0,0003 & 0,0003 & 80 & 0,96 & 0,05 & 0,91 & 0,18 \\
		\hline
	\end{tabularx}
	\caption{Ergebnisse des Verfahrens mit vier reellen Punktwolken von Bauteilen}
	\label{table: test_3-2_results}
\end{table}

Im Vergleich zu den anderen drei wurde bei dem ersten Bauteil die wenigsten Kanten erkannt. Auch die Kantensegmentierung war für dieses Bauteil am wenigsten erfolgreich. Aus den 22 gezählten Kanten wurden nur 18 vollständig richtig erkennt. Darüber hinaus wurden nur 15 dieser 18 Kanten auch richtig segmentiert. Es wurden alle Ränder des Bauteils richtig erkannt sowie die kreisförmige Aussparung. Die Erkennung der Nuten lieferte im Gegensatz dazu mangelnde Ergebnisse. Die Qualität der Punktewolke verschlechterte sich in den Nuten und die Abbildung der Kanten war nicht sehr deutlich definiert. Die Auflösung hatte trotzdem gereicht, visuell die Kanten der Nuten zu erkennen. Aus den zwölf Kanten der Nuten konnten lediglich acht richtig erkannt werden. Die Segmentierung der Außenränder erfolgte auch sehr gut. Bis auf eine Kante, wurden alle vollständig segmentiert. Die Kantensegmentierung konnte zwischen zwei parallelen Kanten unterscheiden, die ca. einen Millimeter voneinander entfernt waren. Die Segmentierung der erkannten Kanten der Nuten erfolgte auch nur teilweise richtig. Während manche Kanten mit einer minderen Anzahl an lokalen Störungen vollständig segmentiert wurden, wurden Kanten mit einer höheren Anzahl an Störpunkte unvollständig segmentiert. Auf diesen Kanten wurden bis zu zwei weiteren Segmente erzeugt. Die kreisförmige Aussparung wurde auch unvollständig segmentiert. An der Stelle wurden drei zusätzliche Segmente erzeugt. 

Am Anfang der Punktwolken der restlichen drei Bauteile wurde eine Besonderheit erkannt. Das Anfangsstück des Bauteils wurde bereits einmal gescannt, bevor das gesamte Bauteil von Anfang an mit einer höheren Auflösung gescannt wurde. Dies resultierte in einem Überlappungsbereich, wo es Punktwolken mit zwei unterschiedlichen Auflösungen existierten. Dieses wird in Abbildung ~\ref{fig: scan_overlap} visualisiert.

\begin{figure}[h]
	\includegraphics[width=0.8\textwidth]{Abbildungen/scan_overlap.png}
	\centering
	\caption{Unterschiedliche Punktedichten der Bauteile zwei bis vier}
	\label{fig: scan_overlap}
\end{figure}

In der Punktwolke des zweiten Bauteils wurden 49 Kanten visuell erkannt. Die Kantenerkennung konnte 46 dieser Kanten vollständig erkennen. Besondere Schwierigkeiten hatte die Kantenerkennung bei der vollständigen Erkennung der drei kreisförmigen Löcher des Schweißtisches. Diese Kreise wurden nur teilweise korrekt als Halbkreise oder Teilkreise erkannt. Es wurden alle Ränder sowie die Innenkante des Bauteils richtig erkannt. Die Kantensegmentierung für das zweite Bauteil schnitt im Vergleich zu dem ersten Bauteil besser ab. Es wurden bis auf einem alle erkannten Ränder des Bauteils vollständig segmentiert. Die Innenkante des Bauteils wurde auch vollständig richtig segmentiert. Schwierigkeiten hatte die Kantensegmentierung wieder bei den Halb- und Teilkreisen. Auf diesen wurden insgesamt zwei bis vier Segmente erzeugt. Insgesamt konnte die Kantenerkennung 41 der 46 erkannten Kanten vollständig segmentieren, während es 9 zusätzliche falsche Segmente durch das Verfahren erzeugt wurden. Bei dem dritten Bauteil wurden ähnlich gute Ergebnisse bei der Kantenerkennung sowie Kantensegmentierung erzielt. Aus den 92 gezählten visuellen Kanten wurden 87 durch die Kantenerkennung erkannt. Schwierigkeiten hatte das Verfahren wiederum bei der vollständigen Erkennung der Löcher des Schweißtisches. Diese wurden ähnlich wie zuvor nur als Halb- oder Teilkreise erkannt. Die Ränder sowie die Innenkante des Bauteils wurden auch vollständig erkannt. Die Segmentierung dieses Bauteils hat auch einen ähnlichen Erfolgsquote wie zuvor. Aus den 87 erkannten Kanten wurden 80 vollständig segmentiert. Unter diesen Kanten zählten die Ränder des Bauteils. Die Stufen der stufenartigen Innenkante wurden auch überwiegend richtig voneinander unterschieden und richtig segmentiert. Diese Stufen konnten allerdings in dem Überlappungsbereich nicht mehr vollständig segmentiert werden. Die rechteckigen Aussparungen wurden auch durch das Verfahren überwiegend richtig und vollständig segmentiert. Bei dem letzten Bauteil mit ähnlichen geometrischen Merkmalen wie dem dritten wurde auch eine ähnliche Genauigkeit des Verfahrens beobachtet. Aus den 80 gezählten Kanten wurden 77 richtig erkannt. Dazu gehörten wiederum die Außenränder sowie die Innenkante des Bauteils. Gleichermaßen wurden die Löcher des Schweißtisches nicht vollständig erkannt. Die Segmentierung lieferte auch ein ähnlich gutes Ergebnis. Aus den 77 Kanten wurden 70 vollständig segmentiert. Hierunter zählten die Außenränder und eine überwiegenden Anzahl an Stufen der Innenkante. Die meisten rechteckigen Aussparungen wurden auch richtig segmentiert. Es wurde eine Längskante einer Aussparung detektiert, die nicht vollständig segmentiert wurde. Stattdessen wurden insgesamt zwei Segmente an der Stelle erzeugt. Die Kantenerkennung sowie Kantensegmentierung konnten die zwei Bereiche unterschiedlicher Punktedichten am Anfang der Punktwolken voneinander unterscheiden. Die Grenzen dieser Bereiche wurden richtig erkannt sowie vollständig segmentiert.

Die in diesem Kapitel entworfenen Untersuchungen der Genauigkeit des Verfahrens konnten die Leistung des Verfahrens unter diversen Bedingungen austesten. Aus diesen Untersuchungen konnten diverse Erkenntnisse gesammelt werden, die in dem folgenden Kapitel detaillierter behandelt werden.
	%Struktur:
% Zusammenfassung Methodik und Ergebnisse
% Erkenntnisse
%PCL-Version und UBuntu
%Bounding Box mit Lasersensor

\chapter{Diskussion}
\section{Zusammenfassung der Methodik und Ergebnisse}
Das in dieser Arbeit vorgelegte Verfahren wurde mit dem Hintergrund entwickelt, eine Methode zur Erkennung geometrischer Merkmale in wachsenden Punktwolken zu präsentieren. Dieses Verfahren sollte es ermöglichen, während der Erzeugung der Punktwolke relevante geometrische Informationen nebenbei bereitzustellen. Grundsätzlich sollte die sequenzielle Identifizierung der Geometrien eines Objektes während es Abtastung ermöglicht werden. Hierfür wurde zuerst das bereits vorhandene \textit{AGPN}-Verfahren nach \autocite{ni_edge_2016} aus der Literatur gewählt. Da das Programm des Verfahrens nicht durch die Autoren quelloffen zur Verfügung gestellt wurde, musste es durch den Verfasser implementiert werden. Das AGPN-Verfahren bestand aus zwei Teile - die Kantenerkennung sowie die Kantensegmentierung. Bei der Kantenerkennung wurden zuerst die \textit{K\textsubscript{1}} nächsten Nachbarpunkte eines Punktes \textit{o} gesucht um eine Nachbarschaft \textit{N\textsubscript{o}} zu erstellen. Aus diese Nachbarschaft wurden mittels eines RANSAC-Verfahrens die Punkte ausgesucht, die auf der gleichen Ebene \textit{E\textsubscript{N\textsubscript{o}}} lagen. Falls \textit{o} zu dieser Ebene gehörte, wurden alle Inliers der Ebene \textit{E\textsubscript{N\textsubscript{o}}} weiterhin auf ihren Winkelabstand überprüft. Mittels des Verfahrens aus Abschnitt \ref{edge_detection_reprod} wurden die Winkelabstände \textit{G\textsubscript{$\theta$}} von zwei konsekutiven Nachbarpunkten berechnet. Danach wurde es überprüft, ob der maximale Wert von \textit{G\textsubscript{$\theta$}} einen Schwellwert \textit{$\alpha$} überstieg. Falls dieses zutraf, wurde \textit{o} als einen Randpunkt markiert, der zu einer Kante gehörte. Diese Schritte wurden in einer Funktion namens \textit{FindEdgePoints} verpackt. 

Nach der Implementierung der Kantenerkennung wurde die Kantensegmentierung implementiert. Hierfür wurden zwei Methoden \textit{ComputeVectors} und \textit{ApplyRegionGrowing} entworfen. Die erste Methode kümmerte sich um die Bestimmung von den exakten benachbarten Randpunkte eines Randpunktes \textit{p} sowie die Bestimmung des Richtungsvektors von \textit{p}. Ähnlich wie bei \textit{FindEdgePoints} wurden mittels eines KD-Trees \textit{K\textsubscript{2}} benachbarten Randpunkte von \textit{p} bestimmt. Danach wurde mittels des RANSAC-Verfahrens eine Linie \textit{L\textsubscript{N\textsubscript{p}}} solange auf die Nachbarschaft \textit{N\textsubscript{p}} von \textit{p} angepasst, bis \textit{p} zu den Inliers von \textit{L\textsubscript{N\textsubscript{p}}} gehörte. Diese Inliers wurden als die exakten Nachbarpunkte von \textit{p} aufgespeichert. Die Linie \textit{L\textsubscript{N\textsubscript{p}}} wurde als Richtungsvektor von \textit{p} verwendet. Dieses wurde für alle erkannten Randpunkte wiederholt. Danach wurden die Randpunkte segmentiert. Hierfür wurde für einen unsegmentierten Randpunkt - der initialer Seedpunkt \textit{s\textsubscript{i}} ein neues Segment \textit{C} mit einem eindeutigen Kennzeichen erzeugt. Danach wurden die exakten Nachbarpunkte von \textit{s\textsubscript{i}} in der Methode \textit{GrowSegment} darauf geprüft, ob ihren Richtungsvektor mit dem von \textit{s\textsubscript{i}} übereinstimmte. Falls der Winkelabstand zwischen den beiden Richtungsvektoren kleiner als einen Schwellwert \textit{$\phi$} betrug, wurde der Nachbarpunkt zum \textit{C} hinzugefügt und als segmentiert markiert. Der übereinstimmende Nachbarpunkt wurde danach zu einer Sammlung nächster Seedpunkte \textit{s\textsubscript{c}} für \textit{C} hinzugefügt. Nachdem alle exakten Nachbarpunkte von \textit{s\textsubscript{i}} überprüft wurden, wurde das gleiche für die nächsten Seedpunkte \textit{s\textsubscript{c}} wiederholt, bis keine Punkte zum \textit{C} hinzugefügt werden konnten. Danach wurde ein neues Segment mit einem neuen initialen Seedpunkt erzeugt.

Nach der Reproduktion des AGPNs wurde das Verfahren für den Zweck dieser Arbeit erweitert. Zur Behebung der Anomalie der falschen Kanten wurden zwei Methoden präsentiert. Die erste Methode verwendete ein Bounding-Box, um die Randbereiche einer Iteration zu erkennen und gezielt die falschen Randpunkte aus diesen Bereichen zu entfernen. Die zweite Methode verwendete die Reihenfolge der empfangenen Punkte, um falschen Ränder zu erkennen und zu entfernen. Es wurden eine bestimmte Anzahl an Scan-Linien wahlweise am Anfang und/oder am Ende einer Iteration hierfür verwendet. Falls Punkte aus dieser Scan-Linien durch \textit{FindEdgePoints} als Randpunkte erkannt wurden, wurden diese entfernt. Die zweite Methode wies eine deutlich geringere Zeitkomplexität als die erste Methode auf, da es bei der ersten Methode um eine Matrizenrechnung handelte, während es bei der zweiten Methode um einfache Such- und Zugriffsoperationen handelte. Es wurden auch zusätzliche Filterverfahren wie \textit{UniformSampling} und \textit{StatisticalOutlierRemoval} implementiert, um die Anzahl der Punkte zu verringern, den Punkteabstand zu vereinheitlichen, sowie Ausreißer zu entfernen. Neben dieser Filterverfahren wurden auch Methoden implementiert, die das Verhältnis zwischen der Punkteindizes der Punkte vor und nach dem Filtern gemerkt haben. Dieses war insbesondere für die Korrektur der zu entfernenden Punkteindizes sowie der \textit{k} wiederholten Punkte wichtig. Die Methode \textit{FindEdgePoints} hat keine großen Änderungen erfahren. Als neue Funktionalität wurde die zweite Methode zur Entfernung falscher Kanten hier implementiert. Im Gegensatz dazu erfolgten die größten Änderungen in den zwei Verfahren von \textit{SegmentEdges}. Bei \textit{ComputeVectors} wurden in jeder Iteration die \textit{k} wiederholten der vorigen Iteration gefunden und die exakten Nachbarpunkte neu bestimmt, um die neuen Randpunkte aus der Randbereiche einzuschließen. Auch der Richtungsvektor dieser Punkte wurde neu berechnet. Bei \textit{ApplyRegionGrowing} wurden diese wiederholten Randpunkte gefunden und ihre Segment-Markierung zurückgesetzt. Daneben wurden auch die Markierungen der exakten Nachbarpunkte dieser wiederholten Randpunkte zurückgesetzt. Danach wurde die Methode \textit{GrowSegment} hier wiederverwendet, um das ursprüngliche Segment des wiederholten Randpunktes mit neuen Punkten zu erweitern. Nachdem alle vorhandenen Segmente um neue Punkte erweitert wurden, wurde als nächstes überprüft, ob noch unmarkierte Punkte vorhanden sind. In diesem Fall wurden anhand dieser Punkte neue Segmente erstellt, bis alle Randpunkte der neuen Iteration zu einem Segment hinzugefügt wurden. Somit wurden alle nötigen Methoden entwickelt, um ein erstes Prototyp für die Kantenerkennung und Kantensegmentierung wachsender Punktwolken zu entwerfen.

Um die Genauigkeit und Einsetzbarkeit des in dieser Arbeit entwickelten Verfahrens zu überprüfen, wurden auf Basis der Forschungsfrage sowie der Teilforschungsfragen drei Test konzipiert. Für diesen Tests wurden entweder reeller Aufnahmen von Bauteilen oder eine künstlich erzeugte Punktwolke - ein Ground-Truth - verwendet. Zur Auswertung der Genauigkeit der Verfahren wurden die Metriken nach \autocite[13]{ni_edge_2016} verwendet. Bei dem ersten Test wurde die Genauigkeit des IEFD-Verfahrens mit der des AGPN-Verdahrens überprüft. Hierfür wurde es anhand der Metriken ausgewertet, zu welchem Anteil alle Kanten der Testdatei richtig erkannt und vollständig segmentiert wurden. Es wurden für beiden Verfahren die gleichen Parameter verwendet. Das IEFD-Verfahren schnitt im Durchschnitt besser als das AGPN-Verfahren ab. Während das AGPN-Verfahren nur die Seitenrändern der Testdatei erkennen konnte, wurde durch das IEFD-Verfshren neben den Seitenrändern auch die Innen- und Außenkante erkannt. Die Segmentierung der erkannten Kanten erfolgte in beiden Verfahren sehr gut. Während das IEFD-Verfahren bis auf einen Testdurchlauf alle Kanten vollständig segmentiert hatte, konnte das AGPN-Verfahren alle erkannten Kanten in allen Testdurchläufen vollständig segmentieren. Bei dem zweiten Test wurde die Einfluss der Punktedichte auf die Genauigkeit des Verfahrens überprüft. Hierbe i wurden unterschiedliche Iterationen der Testdatei bei konstanten Dimensionen und variablen Punkteabständen erstellt. Der Punkteabstand wurde zwischen 0,00005m und 0,005m variiert. Bis zu einem Punkteabstand von 0,0025m blieb die Genauigkeit des IEFD-Verfahrens sehr hoch und lieferte sowohl für die Kantenerkennung als auch für die Kantensegmentierung gute Ergebnisse. Bei einem Punkteabstand von 0,005m stieg die Genauigkeit des Verfahrens deutlich ab. Ausnahmsweise wurden bei der niedrigsten Punkteabstand beziehungsweise die höchste Punktedichte die Innen- und Außenkante der Testdatei nicht erkannt. Beim dritten Test wurde die Robustheit des Verfahrens überprüft. Hierzu wurden zwei Untersuchungen durchgeführt. Bei der ersten Untersuchung wurde eine zufällige Verzerrung einer bestimmten Amplitude zu der Punktwolke eingeführt. Damit wurde eine Störung des regelmäßigen Punktemusters erzielt. Die Amplitude der Verzerrung wurde zwischen 0,1 und 1,5 variiert. Bis zu einer Amplitude von 0,8 wurde keine Beeinträchtigung der Leistung des Verfahrens beobachtet. Es wurden perfekte Ergebnisse für die Genauigkeit des Verfahrens erzielt. Auch bis zu einer Amplitude von 1,0 wurde für die Kantenerkennung sehr gute Ergebnisse beobachtet. Im Gegensatz dazu sind die Ergebnisse der Kantensegmentierung relativ schlechter gewesen. Bei der höheren Amplituden wurden auch teilweise die falschen Kanten zwischen einzelner Iterationen nicht vollständig entfernt und haben die Kantensegmentierung gestört. Demzufolge sind die Ergebnisse der Kantensegmentierung bei einer Erhöhung der Verzerrung schlechter geworden. Dieser Trend wurde bis zu einer Amplitude von 1,3 beobachtet, bis beide Metriken für die Kantensegmentierung gegen einen Wert von 0,4 beziehungsweise 0,34 eingependelt sind. Die Kantenerkennung lieferte im Gegensatz dazu deutlich bessere Ergebnisse. Es wurden alle Seitenränder sowie die Innen- und Außenkante richtig erkannt. Ab einer Amplitude von 0,9 wurden die Spuren falscher Kanten zwischen Iterationen auch erkannt. Diese Anzahl stieg bis zu einer Amplitude von 1,3 auf, bevor sie wieder leicht herunterstieg. Ab einer Amplitude von 1,3 konnten falsche Kanten weniger deutlich erkannt werden. Bei der zweiten Untersuchung des dritten Tests wurde die Genauigkeit des Verfahrens bei Anwendung auf vier Punktwolken reeller Bauteile überprüft. Es wurden hierfür Bauteile mit unterschiedlichen geometrischen Merkmalen wie Löcher, kreisförmige sowie rechteckige Aussparungen, Nuten und Innenkanten verwendet. Für alle vier Bauteile wurden gute bis sehr gute Ergebnisse für die Kantenerkennung sowie die Kantensegmentierung bemerkt. Die Seitenränder, Innenkante und Aussparungen der Bauteile wurden immer richtig erkannt. Die Erkennung aller Kanten der Nuten sowie der Löcher erfolgte gegenteilig nicht immer akkurat. Die Kantensegmentierung der Seitenränder, Innenkanten und der Aussparungen erfolgte in den überwiegenden Fällen richtig. Manchmal wurden diese Kanten nicht vollständig segmentiert, sondern wurden zusätzliche Segmente erzeugt. Die Kanten der Nuten mit einer höheren Anzahl an Störpunkte konnten durch das Verfahren nicht vollständig segmentiert werden. Auch die kreisförmigen Löcher stellten eine Hürde für die Kantensegmentierung dar. 

Bei der Konzipierung des Verfahrens sowie bei der Auswertung der Testergebnisse sind viele Einzelheiten zur Erkenntnis gekommen. Bei diesen Erkenntnissen handelte es sich unter anderem um Anmerkungen zu der Leistungsfähigkeit des Verfahrens sowie mögliche Erklärungen der Ergebnisse

\section{Erkenntnisse aus der Methodik und Ergebnisse}

Die wichtigste Anmerkung zu der Leistung und Rechenzeit des Verfahrens erfolgte bei der Parallelisierung der \textit{FindEdgePoints}-Methode. Auf einem Ryzen 5 3600 Prozessor \autocite{noauthor_amd_2022} mit sechs Kernen und 12 Threads und einer Basistaktrate von 3,6 GHz wurde eine siebenfache Leistungsverbesserung beobachtet. Eine Punktwolke mit ca. 450.000 Punkte konnte vor der Parallelisierung innerhalb 162 Sekunden verarbeitet werden. Nach der Parallelisierung erfolgte die Kantenerkennung innerhalb 22,5 Sekunden. Es ließ sich postulieren, dass Kanten durch die Verwendung eines Prozessors mit mehr Kernen noch schneller erkannt werden könnten. Die Verwendung eines Grafikprozessors, die deutlich mehr Kernen besitzen, hätte die Leistung der Kantenerkennung für sehr großen Punktwolken enorm steigern können. Hierzu hätte sich ein Grafikprozessor mit tausenden Kernen die Verarbeitung und Kantenerkennung von Punktwolken erheblich steigern können. Allerdings, durfte diese aufgrund der Software-Voraussetzungen in Abschnitt \ref{soft_voraus} nicht gemacht werden.

Die zweite Erkenntnis bei der Reproduktion des AGPN Verfahrens ist für die Implementierung auf einem Linux-basierten System relevant. Die Kantenerkennung erfolgte auf die neuere Version des Betriebssystems - Ubuntu Jammy Jellyfish (Version 22.04) - reibungslos und lieferte sehr gute Ergebnisse. Die Wiederholung des Programms auf eine ältere Generation des Betriebssystems - Ubuntu Focal Fossa (Version 20.04) - lieferte im Gegensatz schlechtere Ergebnisse. Ein Fehler seitens der Hardware wurde ausgeschlossen, indem der gleiche Rechner mit konstanten Spezifikationen für beide Betriebssysteme verwendet wurde. Auch der Einfluss fremder Software auf dem Programm wurde ausgeschlossen, indem das Programm an Betriebssysteme nur mit den notwendigen Softwareabhängigkeiten ausführt wurde. Eine genauere Untersuchung lieferte den Hinweis, dass die Standardversion der PCL-Bibliothek für beide Betriebssysteme unterschiedlich war. Die PCL-Bibliotheksversion 1.10 wurde Standardweise mit Ubuntu Focal Fossa geliefert, wobei die Version 1.12 Standardweise mit Ubuntu Jammy Jellyfish geliefert wurde. Das Downsampling-Verfahren aus der Bibliotheksversion 1.10 konnte sehr dichte Punktwolken nicht korrekt verarbeiten. Dieses Fehler wurde allerdings in der neueren Version der Bibliothek behoben. Aus diesem Grund lässt es sich empfehlen, entweder die Ubuntu Version 22.04 zu verwenden oder die Standardversion der PCL-Bibliothek zu entfernen und stattdessen die Version 1.12 zu installieren. Abbildung \ref{fig: pcl_version_comparision} zeigt die Randpunke und im weiteren Sinne die Kanten, die nach dem fehlerhaften Downsampling erkannt wurden.


\begin{figure}[t]
	\centering
	\begin{subfigure}[h]{0.49\textwidth}
		\includegraphics[width=\textwidth]{Abbildungen/blech_edges.png}
		\centering
		\caption{Randpunkte mit PCL 1.12}
		\label{fig: blech_edges}
	\end{subfigure}
	\hfill
	\begin{subfigure}[h]{0.49\textwidth}
		\includegraphics[width=\textwidth]{Abbildungen/blech_bad_edges.png}
		\centering
		\caption{Randpunkte mit PCL 1.10}
		\label{fig: bad_edges}
	\end{subfigure}
	\caption{Randpunkte, die durch beider Versionen von PCL erkannt werden}
	\label{fig: pcl_version_comparision}
\end{figure}

Während der Entwicklung der Methoden zur Entfernung falscher Kanten aus Abschnitt \ref{false_edges} wurden Anmerkungen zu der Leistung und Genauigkeit der ersten Methode gemacht. Obwohl dieses Verfahren theoretisch funktionieren sollte, lieferte es unzureichende Ergebnisse. Die Ungenauigkeit des Bounding-Boxes wurde durch die sehr hohe Dichte der Punktwolke beziehungsweise den sehr kurzen Abstand zwischen Punkten amplifiziert. Bei einem mittleren Punktabstand von weniger als 0,1 mm führte die inhärente Ungenauigkeit des Bounding-Boxes dazu, dass übermäßig viele Randpunkte entfernt wurden. Darüber hinaus kosteten die Transformation der Punktwolke und weiteren Berechnungen zusätzliche Rechenleistung und eigneten sich zu zeitintensiven Operationen nicht, weil sie eine Zeitkomplexität deutlich höher als $O(n)$ aufwiesen. Deswegen wurde die zweite Methode zur Entfernung falscher Segmente bevorzugt. Bei der Konzipierung beider Methoden zur Entfernung falscher Kanten wurde nebenbei auch ein zusätzliches Konzept entworfen. Bei diesem Verfahren sollte die Position sowie das Koordinatensystem des Sensors während der Abtastung ausgenutzt werden. Es sollten ähnlich wie bei der ersten Methode zwei Regionen definiert werden, die alle falschen Kanten umfassen würden. Hierzu sollten die Positionen des Sensors am Anfang und am Ende jeder Iteration aufgespeichert werden, um die Positionen der falschen Randpunkte an den jeweiligen Seiten zu bestimmen. Das Koordinatensystem des Sensors wäre zur Orientierung der Regionen verwendet werden. Die x-Achse würde als Stützvektor der Tiefe, die y-Achse als Stützvektor der Breite und die z-Achse als Stützvektor der Höhe dienen. Die y-Achse des Sensors würde auch seinem Richtungsvektor entsprechen. Zur Dimensionierung der Regionen sollten die Sensorspezifikationen verwendet werden. Der Lasersensor aus Abschnitt \ref{test_data} dieser Arbeit kann 290 mm in der x-Richtung und 460 mm in der z-Richtung scannen. Auf diesen Werten wird ein zusätzlicher Puffer addiert, sodass die Regionen möglichst alle falschen Randpunkte umfassen würden. Der Ausmaß für die Breite dieser Regionen dürfte durch den Benutzer angegeben werden. Während der Konzipierung dieses Verfahrens wurde jedoch enthüllt, dass die Sensorposition nicht immer Vertikal über das Bauteil lag, sondern meistens abgesetzt und in einer anderen Orientierung zu ihm lag. Als Lösungsansatz könnten die Randpunkte nach dem Koordinatensystem des Sensors transformiert werden, allerdings hätte diese wiederholte Berechnung für jeden Randpunkt einer Iteration zu einer Zeitkomplexität deutlich über $O(n)$ sowie zu einer Verlangsamung des gesamten Verfahrens geführt. Deswegen wurde diese Methode zur Entfernung falschen Kanten nicht vervollständigt und diese Zeit zur Entwicklung der zweiten Methode investiert. 

Das wichtigste Erkenntnis des ersten Tests ist die Bestätigung der ersten Hypothese. Das IEFD-Verfahren hat zu einer hohen Genauigkeit alle Kanten der Testdatei erkennt und segmentiert. Aufgrund der hohen Einheitlichkeit des Punkteabstandes, der gleichmäßigen Höhe aller Punkten einer Ebene sowie der klaren Trennung der Ebenen wurden alle Seitenränder und die Innen- sowie Außenkante vollständig erkannt und segmentiert. Allerdings ist das gleiche nicht für das AGPN-Verfahren geschehen. Trotz Verwendung der gleichen Testdatei hat das AGPN-Verfahren die Faltungen des Bauteils überhaupt nicht erkannt. Da alle Parameter beider Verfahren den gleichen Wert hatten und konstant gehalten wurden, lässt sich überlegen, ob die Anzahl der verarbeiteten Punkte einen Unterschied machen könnten. Während die Testdatei bei dem IEFD-Verfahren in kleineren Teilen zerlegt wurde, wurde bei dem AGPN-Verfahren die gesamte Punktwolke verarbeitet. Das Verhältnis zwischen der Anzahl der verarbeiteten Punkte und der Anzahl \textit{K\textsubscript{1}} an Nachbarpunkte ist somit für beide Verfahren unterschiedlich gewesen. Dieses Verhältnis ist für jede Iteration des IEFD-Verfahrens deutlich höher gewesen. Die Ergebnisse des AGPN-Verfahrens könnten durch die Wahl eines höheren Wertes für \textit{K\textsubscript{1}} verbessert werden. Die Ergebnisse des zweiten Tests haben die zweite Hypothese nicht bestätigt. Die Genauigkeit hat sich nicht proportional zu der Punktedichte verhält, sondern wurde kein Trend eindeutig erkannt. Der Verlauf der Genauigkeit wird einfach logarithmisch in Abbildung \ref{fig: point_density_trend} dargestellt und ertönt die Abwesenheit eines erkennbaren Trends. Je mehr der Punkteabstand erhöht wurde, desto weniger Punkte wurden in der Testdatei generiert. Die Ursache dieses Verhaltens sind die konstanten Dimensionen der Testdatei gewesen. Wie es auch in Abbildung \ref{fig: testdata_0,005m} zu sehen ist, ist die Erkennung geometrischer Merkmale der Testdatei schwerer geworden. Es lässt sich überlegen, ob die Genauigkeit des Verfahrens mit der Anzahl der Punkte statt der Punktedichte korreliert. Das obige Postulat aus dem ersten Test - das Verhältnis zwischen der Anzahl der verarbeiteten Punkte und \textit{K\textsubscript{1}} - würde die niedrigere Bewertung des Verfahrens bei einem sehr kürzen Punkteabstand von 0,00005m auch begründen. Bei diesem Punkteabstand wurden 48.000.000 Punkte in der Testdatei generiert, wodurch die Anzahl der zu verarbeitenden Punkte in jeder Iteration auch gestiegen ist. Dadurch hat wahrscheinlich der Wert für \textit{K\textit{1}} nicht gereicht, um die Innen- und Außenkante der Testdatei zu korrekt erkennen. Auch die Kantensegmentierung scheint von der Anzahl an Punkten abhängig zu sein. Aufgrund der niedrigen Punktezahl bei den höheren Punkteabständen wurden weniger Punkte in jeder Iteration wiederholt, welches die Genauigkeit der Kantensegmentierung beeinträchtigt hat. Für den dritten Test lässt sich die allgemeine Aussage treffen, dass die dritte Hypothese bestätigt wurde. In beiden teilen der dritten Untersuchung konnte das IEFD-Verfahren zuverlässige und gute Ergebnisse liefern. Bis zu einer Amplitude von 1,0 der Verzerrung lieferte die Kantensegmentierung sehr gute Ergebnisse. Die Erzielung einer perfekten Bewertung wurde dadurch verhindert, dass die falschen Kanten zwischen Iterationen nicht deutlich und lückenlos erkannt wurden. Dies führte dazu, dass die Segmentierung dieser falschen Kanten unvollständig erfolgt hat. Auch die Anzahl der Störpunkte ist mit der Amplitude der Verzerrung gestiegen. Dieses hat dazu geführt, dass die Erweiterung der vorhandener Segmente dadurch gestört worden ist. Die Werte für die Metriken  \textit{p\textsubscript{dct}} und \textit{p\textsubscript{mjt}} scheinen bei der höheren Verzerrungen gegen einem Wert zwischen 0,34 und 0,40 zu tendieren. Dies könnte daran liegen, dass die falschen Kanten zwischen Iterationen ab einer Amplitude von 1,4 weniger deutlich zu erkennen waren. Gleichzeitig sind die Anzahl der Störpunkte auch gestiegen. Aufgrund des großen Verzerrungsfaktors sind kleine Lücken entstanden, wodurch die angrenzenden Punkte als Randpunkte markiert wurden und für die Kantensegmentierung als Störpunkte wirkten. Es könnte argumentiert werden, dass der durchschnittliche Punkteabstand ab einer Amplitude von 1,0 sich stark geändert hat. Eine Ermittlung dieses Wertes sowie die entsprechende Anpassung von \distthresha und \distthreshb hätte zu besseren Ergebnissen geführt. Der zweite Teil der dritten Untersuchung hat gezeigt, dass das IEFD-Verfahren gute Ergebnisse bei der Verarbeitung von Punktewolken reeller Bauteile liefert. Dank der Wahl eines geeigneten Wertes für \distthreshb konnten auch nahegelegene Kanten richtig und getrennt voneinander segmentiert werden. Das Verfahren ermöglicht die Unterscheidung zwischen geometrischen Merkmale eines feinen Grades. Allerdings hatte das Verfahren Schwierigkeiten bei der Erkennung und Segmentierung von kreisförmigen Löchern des Ablagetisches. Der Grund hierfür lag daran, dass die Kanten der Kreise von einem Rand auf eine Faltung übergingen. Dieser Übergang ist in Abbildung \ref{fig: Loch} abgebildet. Es lässt sich behaupten, dass die Kantenerkennung bei solchen Übergängen fehlschlägt, allerdings könnte diesem Verhalten durch die Anpassung der Prozessparameter entgegengewirkt werden. Durch den zweiten Teil der dritten Untersuchung wurde auch das Erkenntnis gewonnen, dass die Anzahl der wiederholten Scan-Linien \textit{k} auch einen Einfluss auf die Genauigkeit der Kantensegmentierung hat. Bei einem unzureichenden Wert für \textit{k} wurden immer neue Segmente in neuen Iterationen erzeugt, obwohl vorhandene Kanten erweitert werden konnten. Eine höhere Anzahl von Scan-Linien würden die Wahrscheinlichkeit erhöhen, dass Punkte mit dem gleichen Richtungsvektor als vorhandenen Segmente gefunden werden. Es lässt sich behaupten, dass die Genauigkeit des Verfahrens nicht proportional zu dem Wert von \textit{k} steigt. Stattdessen würde sie asymptotisch zu einem bestimmten Wert von \textit{k} tendieren. Über den Umfang der dritten Untersuchung hinaus wurden die gleichen vier Bauteile durch mit dem AGPN-Verfahren verarbeitet. Ähnlich zu dem ersten Test wurde die Innenkante der Bauteile nicht durch das Verfahren erkannt. Die Kantensegmentierung erfolgte allerdings mit ähnlich guten oder leicht besseren Ergebnissen. 

\begin{figure}[t]
	\includegraphics[width=0.85\textwidth]{Abbildungen/point_density_log_scale.png}
	\centering
	\caption{Der Verlauf der vier Metriken bei einem steigenden Punkteabstand}
	\label{fig: point_density_trend}
\end{figure}

\begin{figure}[t]
	\includegraphics[scale=0.62]{Abbildungen/Loch.png}
	\centering
	\caption{Der Übergang von einem Rand auf eine Faltung des kreisförmigen Loches}
	\label{fig: Loch}
\end{figure}

Einen Vergleich der Genauigkeit des IEFD-Verfahrens mit anderen Verfahren aus \ref{Stand_der_technik} liefert die folgenden Erkenntnisse. Das Verfahren nach \textcite{ahmed_edge_2018} kann zu einer ähnlich hohen Genauigkeit Kanten erkennen. Das Verfahren wurde nach seiner Genauigkeit auf zwei Punktwolken geprüft. Hierfür wird der Wert für \textit{recall} verwendet, da es dem Metrik \textit{p\textsubscript{dc}} entspricht. In beiden Fällen sind die Kanten zu einer durchschnittlichen Genauigkeit von 0,8 beziehungsweise 80\% erkannt worden. \textcite{mineo_novel_2019} und \textcite{choi_rgb-d_2013} führen keine Überprüfung der Genauigkeit ihrer Verfahren auf Basis verschiedene reeller Objekte und bieten somit keine empirische Möglichkeit für einen Vergleich. Das Verfahren nach \textcite{bazazian_edc-net_2021} hat auch einen durchschnittlichen \textit{recall} Wert von 0,84 beziehungsweise 84\%. Auch das Verfahren nach \textcite{himeur_pcednet_2021} hat eine ähnliche Genauigkeit für die Kantenerkennung aufgewiesen. Das Verfahren nach \textcite{hu_jsenet_2020}  kann strukturelle Kanten deutlich schneller als alle anderen Verfahren aus der Literatur erkennen, allerdings ist die Erkennung von Kurven seine Schwachstelle. Das AGPN-Verfahren kann laut \textcite{ni_edge_2016} zu einer ähnlichen Genauigkeit wie das IEFD-Verfahren Kanten erkennen. Die Implementierung des AGPN-Verfahrens in dieser Arbeit weist abweichende Ergebnisse auf. Wie bereits in Abschnitt \ref{test_1} erwähnt, hat das Verfahren die Innen- und Außenkante der Testdatei nicht erkannt. Auch die Innenkanten der reellen Bauteile wurden sehr dürftig oder überhaupt nicht erkannt. Die Kantensegmentierung nach \textcite{lu_fast_2019} erzielt eine Genauigkeit von ca 0,7 oder 70\% und schneidet somit deutlich unter dem IEFD-Verfahren ab. Das AGPN-Verfahren weist im Gegensatz dazu eine vergleichbare hohe Genauigkeit bei der Kantensegmentierung wie das IEFD-Verfahren. Im Allgemeinen lässt sich die Aussage treffen, dass die Kantenerkennung und Kantensegmentierung des IEFD-Verfahrens überdurchschnittlich im Vergleich zu anderen Verfahren der Literatur abschneidet. Das Verfahren zeigt auch bessere Ergebnisse als das AGPN-Verfahren vor, worauf es basiert ist. Allerdings ist das Verfahren nicht ohne seine Schwachstellen. Die Verarbeitung von dürftigen Punktwolken mit wenig Punkten stellt eine Hindernis für das Verfahren dar. Dieses soll allerdings den Einsatz des Verfahrens unter reellen Bedingungen nicht verhindern, da die Aufnahme eines Objektes selten in solchen dürftigen Punktwolken resultiert. Auch eine sehr hohe Verzerrung der Punktwolke führt zu schlechteren Ergebnissen des IEFD-Verfahrens, welches durch eine Anpassung der Prozessparameter kompensiert werden könnte. Das wichtigste Erkenntnis aus dieser Arbeit liegt an der Tatsache, dass das Umfang dieser Arbeit nicht ausreichend war, das IEFD-Verfahren vernünftig und gründlich zu überprüfen.

\section{Limitationen dieser Arbeit}
Im Umfang dieser Arbeit wurden die Angaben aus dem Referenzwerk nach \textcite{ni_edge_2016} verwendet, um das AGPN-Verfahren zu reproduzieren sowie die drei Tests zur Überprüfung der Genauigkeit zu gestalten. Da die Implementierung des Verfahrens durch den Autoren der Referenzwerk nicht öffentlich zugängig gemacht wurde, musste die Methodik durch den Verfasser dieser Arbeit interpretiert werden. Deswegen kann es nicht gewährleistet werden, dass diese Implementierung des AGPN-Verfahrens identisch zu der des Referenzwerks funktioniert. Dies hätte beispielsweise dazu führen können, dass die Kantenerkennung des AGPN-Verfahrens in dieser Arbeit und in dem Referenzwerk abweichende Ergebnisse geliefert haben. Wie bereits ertönt, konnte das AGPN-Verfahren die Innen- und Außenkante der Testdatei nicht erkennen, obwohl es laut dem Referenzwerk zu erwarten wäre. Die künstlich erzeugte Testdatei setzt aus simplen Geometrien zusammen und weist keine komplexe Merkmale auf. Dank der niedrigen Komplexität ist es einfach gewesen, die Anzahl der Kanten sowie der Randpunkte visuell zu bestimmen. Die Abwesenheit von komplexen Geometrien hat allerdings dazu geführt, dass die Grenzen des IEFD-Verfahrens in den drei Untersuchungen nicht gründlich ausgetestet werden konnten. Die Auswahl eines breiteren oder umfangreicheres Spektrum an Punktwolken reeller Bauteile und Objekte hätten auch die Plausibilität des dritten Tests verbessern können und mehr Ergebnisse für den Vergleich mit der Literatur bereitstellen können. Die Untersuchung der Einflüsse der sechs Prozessparameter des IEFD-Verfahrens sowie der drei Parameter der Testumgebung auf die Genauigkeit hat innerhalb des Umfanges dieser Arbeit nicht gepasst. Eine solche Untersuchung könnte wichtige Erkenntnisse liefern. Diese Arbeit konzentriert sich auf die Überprüfung der Eignung des IEFD-Verfahrens für eine industrielle Anwendung hinsichtlich der Genauigkeit. Eine Überprüfung und einen Vergleich der Performanz des Verfahrens könnte allerdings auch wertvolle Erkenntnisse liefern. Die Evaluierungsmetriken sind für den Zweck dieser Arbeit sinnvoll gewählt worden. Für die angezielten Anwendungsgebiete ist die richtige Erkennung und Segmentierung der geometrischen Merkmale von Wert. Andere Evaluierungsmetriken wie die Präzision und Recall Metriken sowie der F-Maß hätten den Verfahren detaillierter ausgewertet, allerdings würden viele dieser für den Einsatzzweck irrelevant gewesen. Trotz der Limitationen sind die Erkenntnisse dieser Arbeit nicht zu vernachlässigen, sondern liefern wichtige Hinweise über die Einsatzfähigkeit und Genauigkeit des IEFD-Verfahrens.


%behaupten, dass die Anzahl der Nachbarpunkte \textit{K\textsubscript{1}} und \textit{K\textsubscript{2}} einen vernachlässigbaren Einfluss auf die Genauigkeit des Verfahrens haben.
	\chapter{Fazit}

Das Ziel dieser Arbeit war die Vorstellung und Evaluierung des IEFD-Verfahrens zur Erkennung geometrischer Merkmale in unvollständigen und wachsenden Punktwolken. Die Evaluierung des Verfahrens sollte ergeben, ob es für eine Anwendung unter reellen Bedingungen geeignet ist. Um eine möglichst ausgewogene Aussage über die Leistung und Genauigkeit des Verfahrens zu treffen, wurde das Verfahren hinsichtlich seiner Genauigkeit und Robustheit gründlich getestet. Dabei wurden verschiedene Testdateien verwendet und unterschiedliche reelle Störungen simuliert. 

Als Grundlage des IEFD-Verfahrens wurde das neuartige AGPN-Verfahren aus der Literatur implementiert, die unter Verwendung von kd-Bäumen, dem RANASC-Algorithmus sowie die Region-Growing-Segmentierung zu einer hohen Genauigkeit Kanten erkennen und segmentieren kann. Das AGPN-Verfahren wurde im Rahmen dieser Arbeit mit zusätzlichen Funktionalitäten bestattet, um Kanten korrekt und vollständig in iterativ wachsenden Punktwolken zu erkennen und zu segmentieren. Auch zusätzliche Korrekturmaßnahmen mussten implementiert werden, um eine Anomalie der Kantenerkennung zu kompensieren. Auch zur Verbesserung der Leistung und Rechenzeit wurden im Laufe dieser Arbeit wichtige Erkenntnisse gewonnen. Ausschlaggebend ist die siebenfache Leistungsverbesserung der Kantenerkennung, die durch eine Parallelisierung des Verfahrens erzielt werden kann.

Die Bestätigung der ersten Hypothese hat grundsätzlich unter Beweis gestellt, dass das Verfahren richtig funktioniert. Während jeder Randpunkt einer Punktwolke nicht korrekterweise markiert oder erkannt wird, kann das Verfahren ganzheitlich alle Kanten ausgeglichen und lückenlos erkennen sowie darstellen. Wichtiger ist die Erkenntnis, dass das IEFD-Verfahren unter Verwendung der gleichen Parameter wie das AGPN-Verfahren mehr Kanten richtig erkennen kann. Mit nahezu einer Genauigkeit von 100\% für die Kantenerkennung sowie die Kantensegmentierung kann die erste Teilforschungsfrage beantwortet werden. 

Es wurde auch nachgewiesen, dass die Genauigkeit des IEFD-Verfahrens nach keinem erkennbaren Trend mit dem Punkteabstand korreliert. Das Verfahren liefert trotz einem steigenden Punkteabstand eine sehr hohe Genauigkeit von 100\% für die Kantenerkennung sowie eine Genauigkeit von 90\% für die Kantensegmentierung. Einen wirkungsvolleren Einfluss hat möglicherweise die Anzahl der Punkte in der Punktwolke auf die Genauigkeit des IEFD-Verfahrens. Somit lässt sich die zweite Teilforschungsfrage beantworten: Der Punktabstand hat auf die Genauigkeit des Verfahrens keine beeinträchtigende Wirkung. 

Das IEFD-Verfahren ist gegen Verzerrungen innerhalb der Punktwolke sowie gegen andere reellen Bedingungen und Störfaktoren sehr Robust. Bei einer zufälligen Verzerrung der Punkte bis zu einer Amplitude von 1.0 liefert das IEFD-Verfahren für die Kantenerkennung sowie Segmentierung sehr gute Ergebnisse von 100\% beziehungsweise 92\%. Diese Ergebnisse können ohne Anpassung der Prozessparameter trotz einer Erhöhung des durchschnittlichen Punktabstands um ca. 1,5 erzielt werden. Die Kantenerkennung funktioniert für unterschiedlichen Geometrien auch sehr stabil und kann gerade sowie runde Kanten zu einer hohen Genauigkeit erkennen. Eine Schwachstelle des IEFD-Verfahrens präsentiert sich im Form von Übergangsbereiche, wo eine Kontur nahtlos von einer Seitenkante zur einer Innen- oder Außenkante übergeht. Diese sind für die Kantenerkennung besonders schwierig zu erkennen. Auch soll betont werden, dass die Anzahl der wiederholten Punkte in jeder Iteration einen unbestimmten Einfluss auf die Güte der Kantensegmentierung hat. Im Grunde lässt sich die Aussage zur Beantwortung der dritten Teilfrage treffen, dass das IEFD-Verfahren sehr Robust mit einer durchschnittlichen Genauigkeit von 92\% sowie 89\% Kanten unter reellen Bedingungen erkennen und segmentieren kann. 

Die Ergebnisse dieser Arbeit stellen unter Beweis, dass das IEFD-Verfahren zu eine gute Robustheit und eine hohe Genauigkeit aufweist, wodurch es für den Einsatz unter reellen Bedingungen eignet. Dieses Verfahren eignet sich insbesondere zur Erkennung der scharfen Konturen und Kanten von Bauteile in der Produktion und Verarbeitung. Das IEFD-Verfahren wird in dieser Arbeit in einer Prototypenphase vorgestellt und eignet sich in seinem aktuellen Stand nicht zur direkten Umsetzung in der Industrie oder für praktischen Probleme. 

Es besteht noch viel Potenzial, dieses Verfahren weiterhin zu überprüfen sowie weiterzuentwickeln. Mögliche Arbeiten könnten die Einflüsse der unterschiedlichen Prozessparameter auf die Genauigkeit des Verfahrens unter Betracht ziehen, sowie die Leistung des Verfahrens hinsichtlich der Performanz und Verarbeitungszeit auswerten. Das weitere Auslasten des Verfahrens mit diversen Aufnahmen reeller Objekte könnte die Grenzen des Verfahrens erproben und wichtige Erkenntnisse über die Leistung und Genauigkeit des Verfahrens liefern. Weitere Arbeiten sollen sich damit beschäftigen, die Performanz des Verfahrens zu optimieren, um seine Eignung für zeitkritischen Aufgaben zu verbessern. Eine weitere Möglichkeit zur Verbesserung der Genauigkeit besteht in der Nachbearbeitung der Segmente, um zwei kollineare Segmente zu vergleichen und zusammenzufügen, falls Sie auf der selben Kante liegen. Letztlich kann auf das IEFD-Verfahren aufgebaut werden, um beispielsweise Methoden zur Laufbahnplanung für Roboter auf Basis der erkannten Kanten zu implementieren. Dieser Arbeit stellt ein funktionsfähiges Konzept vor, welches in den Bereichen des maschinellen Sehens, der Robotik und Automatisierung große Implikationen für die Wissenschaft und Industrie haben kann.
	\printbibliography
%	\appendix
\chapter{Algorithmen} \label{Algorithmen}
\begin{algorithm}
	\caption{Ablauf des IEFD-Verfahrens}
	\label{alg: IEFD_Ablauf}
	\begin{algorithmic}[1]
		\State $\textbf{Input: } \textit{Point cloud with scan lines} = \{P\}, \text{Parameter: } K_1, d_{t1}, \alpha, K_2, d_{t2}, \phi$
		\State $point gap \gets CalculateAvgGap(P)$
		\State $n \gets 0,004/\text{point gap}$
		\State $k \gets 0,0008/\text{point gap}$
		\State $f \gets 0,00004/\text{point gap}$
		\State \textit{raw points \{R\}} $\gets$ \{\}
		\State \textit{reused points \{RE\}} $\gets$ \{\}
		\State $\textit{false edge start \{FS\}} \gets \{\}$
		\State $\textit{false edge end \{FE\}} \gets\{\}$
		\State \textit{segments \{S\}} $\gets$ \{\}
		\State \textit{previous reused edge points \{ER\textsubscript{p}\}} $\gets \{\}$
		\For{Scan line S\textsubscript{i} \textbf{in} \{P\}}
		\State $\textit{R} \gets \textit{R} + P[S_i]$
		\If{$\textit{\textbf{size(}R\textbf{)}} \geq n - k$}
		\State$\text{RE} \gets \textit{RE} + P[S_i]$
		\EndIf
		\If{$\textbf{size}(\textit{R}) \leq f$}
		\State $\textit{FS} \gets \textit{FS} + P[S_i]$
		\ElsIf{$\textbf{size}(\textit{R}) \geq n - f$}
		\State $\textit{FE} \gets \textit{FE} + P[S_i]$
		\EndIf	
		\If{$\textit{\textbf{size(}R\textbf{)}}\geq n$}
		\State \textit{Edge Points \{E\}} $\gets$ \{\}
		\State \textit{Adjusted reused edge points \{ER\}} $\gets$ \{\}
		\State \textit{E, ER} $\gets$ \textit{\hyperref[alg:find_edge_points]{FindEdgePoints}}(\textit{R, RE, FS, FE, K\textsubscript{1}, d\textsubscript{t1},}$\alpha$)
		
		\State \textit{neighbours map \{N\}} $\gets$ \{\}
		\State \textit{vectors map \{V\}} $\gets$ \{\}
		\State \textit{N, V} $\gets$ \textit{\hyperref[alg:compute_vectors]{ComputeVectors}}(\textit{E, ER\textsubscript{p},} $K_2$, $d_{t2}$)
		
		\State \textit{\{S\}} $\gets$ \textit{\hyperref[alg:apply_region_growing]{ApplyRegionGrowing}}(\textit{E, ER\textsubscript{p}, S, N, V, } $K_2, \phi$)
		
		\State $R \gets \{\}$
		\State $R \gets \{RE\}$
		\State $RE \gets \{\}$
		\State $ER\textsubscript{p} \gets \{ER\}$
		\State $FS \gets \{\}$
		\State $FE \gets \{\}$
		\EndIf
		\EndFor
		
		
	\end{algorithmic}
\end{algorithm}

\begin{algorithm}
	\caption{Ablauf der \textit{\hyperref[alg:find_edge_points]{FindEdgePoints}} Funktion}
	\label{alg:find_edge_points}
	\begin{algorithmic}[1]
		\Function{\textit{\hyperref[alg:find_edge_points]{FindEdgePoints}}}{\textit{R, RE, FS, FE, K\textsubscript{1}, d\textsubscript{t1},} $\alpha$}
		\State \textit{removed indices \{RI\}} $\gets$ \{\}
		\State \textit{(R, RI)} $\gets$ \textit{UniformSampling(R, d\textsubscript{t1})}
		\State \textit{removed indices map \{RM\}} $\gets$ \{\}
		\State \textit{point shifts \{PS\}} $\gets$ \{\}
		\State \textit{(RM, PS)} $\gets$ \textit{MarkPoints(RI)}
		\State \textit{FS\textsubscript{copy}} $\gets$ \textit{CorrectPoints(FS, RM, PS)}
		\State \textit{FE\textsubscript{copy}} $\gets$ \textit{CorrectPoints(FE, RM, PS)}
		\State \textit{RE\textsubscript{copy}} $\gets$ \textit{CorrectPoints(RE, RM, PS)}
		\State $\{E\} \gets \{\}$
		\For{$point\  o = 0 \textbf{ to size}(\{P\})$}
		\State $\textit{Nearest neighbours} \{N_o\} \gets \{\}$
		\State $N_o \gets$ \textit{NearestNeighbourSearch(P, o, K\textsubscript{1})}
		\State \textit{normal vector $\vec{n\textsubscript{o}}$} $\gets$ \{0, 0, 0\}
		\State \textit{Inliers \{I\textsubscript{N\textsubscript{o}}\}} $\gets$ \{\}
		\State \textit{I\textsubscript{N\textsubscript{o}}} $\gets$ \textit{ApplyRansacPlane(N\textsubscript{o}, d\textsubscript{t1})}
		\State $\vec{n\textsubscript{o}}$ $\gets$ \textit{OptimizeNormal(I\textsubscript{N\textsubscript{o}})}
		\If{$o \notin I\textsubscript{N\textsubscript{o}} \textbf{or size}(I\textsubscript{N\textsubscript{o}}) < 3$}
		\State \textit{\textbf{continue}}
		\EndIf
		\State $(\vec{u}, \vec{v}) \gets$ \textit{GetFrame($\vec{n\textsubscript{o}}$)}
		\State G\textsubscript{$\theta$} $\gets$ \textit{ComputeAngularGap(o, I\textsubscript{N\textsubscript{o}}, $\vec{u}$, $\vec{v}$)}
		\If{G\textsubscript{$\theta$} $\geq \frac{\pi}{2}$}
		\State $E \gets o$
		\EndIf
		\EndFor
		\State $E \gets \textit{RemoveFalseEdges(E, FS\textsubscript{copy}, FE\textsubscript{copy})}$
		\State \textit{ER} $\gets$ \textit{ReusedEdgePoints(E, RE\textsubscript{copy})}
		
		\State \Return $(E, ER)$
		\EndFunction
	\end{algorithmic}
\end{algorithm}

\begin{algorithm}
	\caption{Ablauf der \textit{\hyperref[alg:compute_vectors]{ComputeVectors}} Funktion}
	\label{alg:compute_vectors}
	\begin{algorithmic}[1]
		\Function{\textit{\hyperref[alg:compute_vectors]{ComputeVectors}}}{\textit{E, ER\textsubscript{P}, }$K\_2$, $d_{t2}$}
		\State \textit{first reused point r\textsubscript{1}} $\gets$ \{ER\textsubscript{P}\}[0]
		\State \textit{neighbours map \{N\}} $\gets$ \{\}
		\State \textit{vectors map \{V\}} $\gets$ \{\}
		\For{\textit{point p = r\textsubscript{1}} \textbf{to size}{(\{E\})}}
		\State \textit{nearest neighbour \{N\textsubscript{p}\}} $\gets \{\}$
		\State \textit{N\textsubscript{p}} $\gets$ \textit{NearestNeighbourSearch(E, p, K\textsubscript{2})}
		\If{$\textbf{size}(\textit{text}) < 3$}
		\State $\{N\}[p] \gets \{N\textsubscript{p}\}$
		\State $\{V\}[p] \gets \textit{CalculateVector(N\textsubscript{p})}$
		\State \textit{\textbf{continue}}
		\EndIf
		\State \textit{Inliers \{I\textsubscript{N\textsubscript{p}}\}} $\gets \{\}$
		\State \textit{I\textsubscript{N\textsubscript{p}}} $\gets$ \textit{ApplyRansacLine(N\textsubscript{p}, d\textsubscript{t2})}
		\While{\textit{p} \textbf{not in} \textit{I\textsubscript{N\textsubscript{p}}}}
		\State \textit{remove I\textsubscript{N\textsubscript{p}} from N\textsubscript{p}}
		\State \textit{I\textsubscript{N\textsubscript{p}}} $\gets$ \textit{ApplyRansacLine(N\textsubscript{p}, d\textsubscript{t2})}
		\EndWhile
		\State \textit{Direction vector $\vec{p_{p}}$} $\gets$ \textit{OptimizeVector(I\textsubscript{N\textsubscript{p}})}
		\State $\{N\}[p] \gets I_{N_p}$
		\State $\{V\}[p] \gets \vec{a_p}$
		\EndFor
		\State \Return (N, V)
		\EndFunction
	\end{algorithmic}
\end{algorithm}

\begin{algorithm}
	\caption{Ablauf der \textit{\hyperref[alg:apply_region_growing]{ApplyRegionGrowing}} Funktion}
	\label{alg:apply_region_growing}
	\begin{algorithmic}[1]
		\Function{\textit{\hyperref[alg:apply_region_growing]{ApplyRegionGrowing}}}{\textit{E, RE\textsubscript{p}, S, N, V, K\textsubscript{2}}, $\phi$}
		\State \textit{seed counter $x$} $\gets 0$
		\State \textit{point labels \{PL\}} $\gets$ \textbf{size}\textit{(\{E\} + \{S\})}
		\State \textit{All unlabeled points from E \textbf{in} PL get value }$-1$
		\State \textit{point residuals \{PR\}} $\gets$ \textit{E}
		\State \textbf{sort} \textit{PR by no. of neighbours}
		\State \textit{number of segmented points n\textsubscript{pts}} $\gets$ \textit{no. of labels} $\neq -1$
		\State \textit{No. of segments n\textsubscript{segs}} $\gets \textbf{max}\textit{(PL)} + 1$
		\State \textit{n\textsubscript{pts}} $\gets$ \textit{\hyperref[alg: extend_segments]{ExtendSegment}(S, ER\textsubscript{p}, PL, N, V, K\textsubscript{2}, $\phi$)}
		\State \textit{initial seed point s\textsubscript{i}} $\gets$ \textit{first point i \textbf{in} PR with \{PL\}[i]}$= -1$
		\While{$\textit{n\textsubscript{pts}} < \textbf{size}\textit{(\{E\})}$}
		\State \textit{segment id C} $\gets$ \textit{n\textsubscript{segs}}
		\State \textit{new points new\textsubscript{pts}} $\gets$ \textit{\hyperref[alg:grow_segment]{GrowSegment}(s\textsubscript{i}, C, PL, N, V,  K\textsubscript{2}, $\phi$)}
		\State \textit{\{S\}[C]} $\gets$ \textit{new\textsubscript{pts}}
		\State $n_{pts} \gets new_{pts}$
		\State \textit{n\textsubscript{segs}} $\gets + 1$
		\For{$\textit{seed i}= x + 1 \textbf{ to size}(PR)$}
		\If{$\{PL\}[i] = -1$}
		\State $s_i = \{PR\}[i]$
		\State $x = i$
		\State \textit{\textbf{break}}
		\EndIf
		\EndFor
		\EndWhile
		\State \Return \textit{S}
		\EndFunction
	\end{algorithmic}
\end{algorithm}

\begin{algorithm}
	\caption{Ablauf der \textit{\hyperref[alg: extend_segments]{ExtendSegment}} Funktion}
	\label{alg: extend_segments}
	\begin{algorithmic}[1]
		\Function{\textit{\hyperref[alg: extend_segments]{ExtendSegment}}}{\textit{S, ER\textsubscript{p}, PL, N, V, K\textsubscript{2}}, $\phi$}
		\State \textit{number of segmented points n\textsubscript{prev}} $\gets$ \textbf{size}\textit{(\{PL\}), for labels $\neq -1$ }
		\State \textit{number of new points added n\textsubscript{add}} $\gets 0$
		\State \textit{point labels copy \{PL\textsubscript{copy}\}} $\gets$ \{PL\}
		\State \textit{reset indices map \{RI\}} $\gets \{\}$
		\For{\textit{index i \textbf{in} \{ER\textsubscript{p}\}}}
		\For{\textit{neighbour n in \{N\}[i]}}
		\State \textit{label} $\gets \{PL\}[n]$
		\If{\textit{label }\textbf{in }\textit{\{RI\}}}
		\State \textit{\textbf{continue}}
		\EndIf
		\State \textit{\{PL\}[n]} $\gets -1$
		\State \textit{\{S\}[label]} $\gets\textit{ \{S\}[label]} - 1$
		\State $RM \gets n$
		\EndFor
		\EndFor
		\For{\textit{point/seed index s\textsubscript{i} }\textbf{in } \textit{\{ER\textsubscript{p}\}}}
		\If{$\textit{\{PL\}[s\textsubscript{i}]} > -1$}
		\State \textit{\textbf{continue}}
		\EndIf
		\If{$\textit{\{PL\textsubscript{copy}\}[s\textsubscript{i}]} < 0$}
		\State \textit{\textbf{continue}}
		\EndIf
		\State $C \gets \{PL\textsubscript{copy}\}[s\textsubscript{i}]$
		\State \textit{additional points in segment new\textsubscript{add}} $\gets$ \textit{\hyperref[alg:grow_segment]{GrowSegment}(s\textsubscript{i},C,PL,N,V,K\textsubscript{2},$\phi$)}
		\State \textit{\{S\}[C]} $\gets new_{add}$
		\State \textit{n\textsubscript{add}} $\gets$ \textit{new\textsubscript{add}}
		\EndFor
		\For{\textit{index i \textbf{in} \{ER\textsubscript{p}\}}}
		\For{\textit{neighbour n in \{N\}[i]}}
		\If{$\textit{\{PL\}[p\textsubscript{i}]} > -1$}
		\State \textit{\textbf{continue}}
		\EndIf
		\If{$\textit{\{PL\textsubscript{copy}\}[p\textsubscript{i}]} < 0$}
		\State \textit{\textbf{continue}}
		\EndIf
		\State $C \gets \{PL\textsubscript{copy}\}[n]$
		\State \textit{\{PL\}[n]} $\gets$ \textit{C}
		\State \textit{\{S\}[C]} $\gets + 1$
		\State $n_{add} \gets + 1$
		\EndFor
		\EndFor
		\State \Return $n_{add} - n_{prev}$
		\EndFunction
		
	\end{algorithmic}
\end{algorithm}

\begin{algorithm}
	\caption{Ablauf der \textit{\hyperref[alg:grow_segment]{GrowSegment}} Funktion}
	\label{alg:grow_segment}
	\begin{algorithmic}[1]
		\Function{\textit{\hyperref[alg:grow_segment]{GrowSegment}}}{\textit{s\textsubscript{i}, C, PL, N, V, K\textsubscript{2}, $\phi$}}
		\State \textit{Queue of seeds \{S\}} $\gets$ \textit{s\textsubscript{i}}
		\State \textit{\{PL\}[s\textsubscript{i}]} $\gets$ \textit{C}
		\State \textit{number of points in segment n\textsubscript{C}} $\gets 1$
		\While{$\textbf{size}\textit{(S)}\geq 0$}
		\State \textit{current seed s\textsubscript{C}} $\gets$ \textbf{dequeue}\textit{(S)}
		\For{$\textit{neighbour n}= 0 \textbf{to size}\textit{(\{N\}[s\textsubscript{C}])}$}
		\State \textit{index i} $\gets$ \textit{\{N\}[s\textsubscript{C}][n]}
		\State \textit{label} $\gets \{PL\}[i]$ 
		\If{$label \neq -1$}
		\State \textit{\textbf{continue}}
		\EndIf
		\State \textit{seed vector vec\textsubscript{s\textsubscript{C}}} $\gets \{V\}[s_C]$
		\State \textit{neighbour vector vec\textsubscript{i}} $\gets \{V\}[i]$
		\State \textbf{bool }\textit{added} $\gets$ \textit{CheckPoint(vec\textsubscript{s\textsubscript{C}}, vec\textsubscript{i}, $\phi$)}
		\If{\textbf{not} \textit{added}}
		\State \textit{\textbf{continue}}
		\EndIf
		\State \textit{\{PL\}[i]} $\gets C$
		\State $n_C \gets +1$
		\State $S \gets i$
		\EndFor
		\EndWhile
		\State \Return $n_C$
		\EndFunction
	\end{algorithmic}
\end{algorithm}

\chapter{Abbildungen} \label{Bilder}
\begin{figure}[h]
	\centering
	\begin{subfigure}{0.49\textwidth}
		\includegraphics[width=\linewidth]{Abbildungen/Bauteil_1.png}
		\centering
		\caption{Bauteil 1}
		\label{fig:bauteil_1}
	\end{subfigure}
	\hfill
	\begin{subfigure}{0.49\textwidth}
		\includegraphics[width=\linewidth]{Abbildungen/Bauteil_2.png}
		\centering
		\caption{Bauteil 2}
		\label{fig:bauteil_2}
	\end{subfigure}
	\vfill
		\begin{subfigure}{0.49\textwidth}
		\includegraphics[width=\linewidth]{Abbildungen/Bauteil_3.png}
		\centering
		\caption{Bauteil 3}
		\label{fig:bauteil_3}
	\end{subfigure}
	\hfill
	\begin{subfigure}{0.49\textwidth}
		\includegraphics[width=\linewidth]{Abbildungen/Bauteil_4.png}
		\centering
		\caption{Bauteil 4}
		\label{fig:bauteil_4}
	\end{subfigure}
\caption{Alle Bauteile für den Test 3 aus Abschnitt~\ref{test_3_part_2}}
\label{fig:bauteile_test_3_2}
\end{figure}

\begin{figure}[h]
	\centering
	\begin{subfigure}{0.49\textwidth}
		\includegraphics[width=\linewidth]{Abbildungen/Segmente_1_extra.png}
		\centering
		\caption{Segmente von Bauteil 1}
		\label{fig:Segmente_1}
	\end{subfigure}
	\hfill
	\begin{subfigure}{0.49\textwidth}
		\includegraphics[width=\linewidth, height=0.15\textheight]{Abbildungen/Segmente_2_extra.png}
		\centering
		\caption{Segmente von Bauteil 2}
		\label{fig:Segmente_2}
	\end{subfigure}
	\vfill
	\begin{subfigure}{0.49\textwidth}
		\includegraphics[width=\linewidth, height=0.15\textheight]{Abbildungen/Segmente_3_extra.png}
		\centering
		\caption{Segmente von Bauteil 3}
		\label{fig:Segmente_3}
	\end{subfigure}
	\hfill
	\begin{subfigure}{0.49\textwidth}
		\includegraphics[width=\linewidth, height=0.15\textheight]{Abbildungen/Segmente_4_extra.png}
		\centering
		\caption{Segmente von Bauteil 4}
		\label{fig:Segmente_4}
	\end{subfigure}
	\caption{Alle Segmente für den Test 3 aus Abschnitt~\ref{test_3_part_2}}
	\label{fig:Segmente_test_3_2}
\end{figure}

\end{document}