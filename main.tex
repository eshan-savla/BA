%\documentclass[]{report}
%\usepackage[ngerman]{babel}
%\usepackage{hyphenat}
%\hyphenation{Mathe-matik wieder-gewinnen}



% Title Page
%\title{Ingenieurmäßige Arbeit}
%\author{Eshan Savla}


%\begin{document}
%\maketitle

%\begin{abstract}
%\end{abstract}

%\tableofcontents
%\vspace{2cm}

%\chapter{Einleitung}
%Industrieroboter werden primär in der Industrie für die Aufgabe der Handhabung, Montage oder Verarbeitung von Werkstücken und Teile eingesetzt. Häufig finden Industrieroboter als Schweißroboter, Lackierroboter oder Montageroboter Anwendung. Diese Art der Roboter sind meistens nicht autonom und müssen bei Bedarf für ihre Aufgabenausführung an neuen Produkte oder Werkstücke angepasst und umprogrammiert werden.


%\end{document}          


\documentclass[
	fontsize=12pt, 		% Schriftgröße
	BCOR=8mm,			% Bindekorrektur (wird nicht benötigt, da Seitenrand definiert ist)
	DIV=calc,			% Satzspiegel automatisch berechnen
	ngerman, 			% für Umlaute, Silbentrennung etc.
	a4paper, 			% Papierformat
	oneside, 			% zweiseitiges Dokument (oneside - einseitig, z. B. für Bachelor-,Studien-, oder Masterarbeit)
	titlepage, 			% es wird eine Titelseite verwendet
	parskip=half, 		% Abstand zwischen Absätzen (halbe Zeile)
	headings=normal, 	% Größe der Überschriften verkleinern
	listof=totoc, 		% Verzeichnisse im Inhaltsverzeichnis aufführen
	bibliography=totoc, % Literaturverzeichnis im Inhaltsverzeichnis aufführen
	index=totoc, 		% Index im Inhaltsverzeichnis aufführen
	final, 				% Status des Dokuments (final/draft)
	numbers=noenddot,	% keine abschließenden Punkte bei Kapitelzahlen
	footheight=30pt,	% Höhe des Raumes für Fußnoten
	headheight=30pt,]{scrbook}
% PDF-einbinden ----------------------------------------------------------------
\usepackage{pdfpages} % FH-Dortmund Logo
\usepackage[ngerman]{babel} % für deutsche Sprache
\usepackage{csquotes} %Anführungszeichen
\usepackage[backend=biber,style=authoryear,sorting=nyt,autocite=inline, maxnames=2, minnames=2]{biblatex}%Literaturverzeichnis
\usepackage{amsmath} % matritzen
\usepackage{tabularx} % mehrzeilige Tabelle
\usepackage[a4paper]{geometry} %Seitengeometrie
\usepackage{changepage} %ändert Seiten Einstellung
\usepackage{lipsum}
\usepackage{lscape} %Querformat
\usepackage{acronym} % Abkürzungsverzeichnis
\usepackage{amssymb} % Symobl reelle Zahlen
\usepackage{placeins} %behalte Bilder in der zugehörigen Section
\usepackage{makecell} % Zeilenumbruch in Tabelle
\renewcommand{\cellalign}{tl}
\usepackage[T1]{fontenc}% wichtig für Trennung von Wörtern mit Umlauten
\usepackage{microtype}% verbesserter Randausgleich
\usepackage{booktabs} % schönere Tabellen
\usepackage{caption} % Bild aus Abbildungsverzeichnis entfernen
\usepackage{subcaption}
\usepackage{lmodern}
\usepackage{graphicx}
\usepackage{blindtext}
\usepackage{siunitx}
\usepackage{float}
\usepackage{listings}
\usepackage{xcolor}
\usepackage{algorithm}
\usepackage[noend]{algpseudocode}
%\usepackage{abstract}
%\usepackage[authoryear]{natbib}
% Meta-Informationen -----------------------------------------------------------
%   Definition von globalen Parametern, die im gesamten Dokument verwendet
%   werden können (z.B auf der Titelseite etc.).
%          
% ------------------------------------------------------------------------------
\newcommand{\iatitle}{Die intelligente Schweißrobotik}
\newcommand{\engltitle}{Intelligent welding robots}
\newcommand{\beschreibung}{Eine Untersuchung der Anforderungen und Funktionsweise eines autonomen Schweißroboters}
\newcommand{\autor}{Eshan Savla}
\newcommand{\Matrikelnr}{7203288}
\newcommand{\Erstpruefer}{Prof. Dr.-Ing. Dennis Ziegler}
\newcommand{\Zweitpruefer}{Max Daiber-Huppert, M.Sc.}
%\bibliography{Literatur/Quellen.bib}
\addbibresource{Literatur/Quellen.bib}
\DefineBibliographyStrings{german}{andothers = {et al.}}
\setlength\bibitemsep{1.5\itemsep}  % Spacing between bib entries

\hypersetup{
	pdftitle={Bachelorarbeit-Savla},
	pdfauthor={Eshan Savla},
	colorlinks=false,
	}

\algrenewcommand\textproc{}% Used to be \textsc

%%colour and style settings code blocks
%\definecolor{codegreen}{rgb}{0,0.6,0}
%\definecolor{codeteal}{rgb}{0,0.5,0.5}
%\definecolor{codeyellow}{rgb}{0,1,1}
%\definecolor{codedandelion}{rgb}{1,0.84,0.4}
%\definecolor{codepurple}{rgb}{0.79,0.17,0.57}
%\definecolor{codeblue}{blue}{0,0.43,0.72}
%
%\lstset{emph={%
%		class, def%
%	 	}
%}
%
%\lstdefinestyle{pythonstyle}{
%	commentstyle=\color{codegreen},
%	keywordstyle=\color{codeblue},
%	stringstyle=\color{color}
%	
%}
\geometry{paper=a4paper,left=30mm,right=20mm,top=30mm,bottom=30mm}
\onehalfspacing
\RedeclareSectionCommand[afterindent=false,beforeskip=-\topskip]{chapter}

%Worttrennung
\babelprovide[hyphenrules=ngerman-x-latest]{ngerman}
\hyphenation{Nach-bar-punkte fälsch-lich-er-weise falsch-mark-ierten Stand-ard-ab-weichung vor-ge-filterte Proto-typen-phase Exakti-fizierung Kanten-erkennung Kanten-segmentierung Pro-gramm-ier-schnitt-stelle Winkel-abstand Winkel-abstände Winkel-abstands Rand-punkten Rand-punkt Rand-punkte Hard-ware-beschleunigung Punkte-abstand Punkte-abstände Punkt-abständen}

\begin{document}
	\begin{titlepage}
% Titelseite ohne Seitenzahl
\vspace{2cm}

\begin{center}
\large\textbf{Fachhochschule Dortmund \\}
\vspace{2cm}
\large{Fachbereich: Maschinenbau \\}
\large{Studiengang: Maschinenbau (B.Eng.)}
\vspace{2cm}

\includegraphics[width=10cm]{"Abbildungen/FH_Dortmund-logo".png}  %Logo der FH einfügen
\vspace{1cm}

\large{\textbf{\iatitle}}\\
\vspace{0.5cm}
\large{\beschreibung}

\vspace{1.5cm}

\begin{tabular}{ll}
Erstprüfer: & \Erstpruefer \\
Zweitprüfer: & \Zweitpruefer \\
\end{tabular}

\vspace{1.5cm} 

\large{Vorgelegt von\\
\textbf{\autor}\\
Matrikelnummer: \Matrikelnr\\
am \today}
\end{center}
\end{titlepage}
	\pagenumbering{Roman}
	%\chapter*{Eidesstattliche Erklärung}

Hiermit erkläre ich, dass ich die vorliegende Arbeit eigenständig und ohne fremde Hilfe angefertigt habe. Textpassagen, die wörtlich oder dem Sinn nach auf Publikationen oder Vorträgen anderer Autoren beruhen, sind als solche kenntlich gemacht. Die Arbeit wurde bisher keiner anderen Prüfungsbehörde vorgelegt und auch noch nicht veröffentlicht.
\vspace{0.5cm}

Stuttgart, \today

\vspace{0.5cm}

\_\_\_\_\_\_\_\_\_\_\_\_\_\_\_\_

Eshan Savla


	\section*{\centering Zusammenfassung}
Die Erkennung von geometrischen Merkmalen eines Objektes, während es durch einen optischen Sensor abgetastet wird, hat für die sensor-basierte Programmierung von Robotern eine hohe Relevanz. In dieser Arbeit wird ein numerisches Verfahren für diesen Zweck vorgestellt. Das Ziel dieser Arbeit ist die Bewertung der Effektivität des Verfahrens. Dazu ergänzend wird diese Forschungsfrage gestellt: Wie Effektiv ist ein numerisches Verfahren bei der Kantenerkennung und Segmentierung von wachsenden Punktwolken? Zuerst wurde das AGPN-Verfahren aus der Literatur reproduziert sowie für den Einsatzzweck angepasst und erweitert. In dieser Arbeit werden die Schritte zur Erweiterung des Verfahrens detailliert beschrieben sowie auf Methoden zur Entfernung falsch-detektierten Kanten eingegangen. Zur Beantwortung der Forschungsfrage wurden drei Untersuchungen konzipiert. Konkret wurde die Genauigkeit des Verfahrens unter verschiedenen Bedingungen überprüft und auf Basis der erkannten und segmentierten Kanten bewertet. Hierfür wurden reelle sowie synthetische Objekte verwendet. Diese Arbeit lieferte das Ergebnis, dass das Verfahren zu einer hohen Genauigkeit Kanten von geometrischen Merkmalen erkennen und segmentieren kann. Daneben wurden auch wichtige Erkenntnisse über die Schwachstellen und Grenzen des Verfahrens gewonnen. Schließlich werden aufbauende Forschungsmöglichkeiten im Bereich der Robotik und Automatisierung besprochen.

\section*{\centering Abstract}
The recognition of geometric features of an object while being scanned by an optical sensor is of high relevance for sensor-based robotics programming. This paper presents a numerical method for this purpose and aims to evaluate its effectiveness. The research question posed is: How effective is a numerical method for edge detection and segmentation in growing point clouds? Initially, the AGPN Method from current literature was reproduced before being adapted and extended for the specific application. The steps taken to extend the method are detailed in this paper, including methods for removing falsely detected edges. Three studies were designed to answer the research question, where the accuracy of the method was tested under different conditions and evaluated based on detected and segmented edges. Real and synthetic objects were used for this purpose. The result showed that the method can detect and segment edges of geometric features with high accuracy, while also highlighting weaknesses and limitations. Finally, future research opportunities on the basis of this method in the field of robotics and automation are discussed.

	\newpage
\setuptoc{toc}{totoc} %Inhaltsverzeichnis im Inhaltsverzeichnis
\tableofcontents{}
\newpage
%\addchap{Abkürzungsverzeichnis}
%Führe lieber keine Abkürzungen ein, wenn du sie nicht mindestens dreimal verwendest. Das gilt allerdings nicht für Fachausdrücke, die standardmäßig abgekürzt werden.

%Deine nicht allgemein gebräuchlichen Abkürzungen solltest du im Abkürzungsverzeichnis einführen.
%\begin{acronym}[Grundlagen]
%\acro{tcp}[TCP]{Tool-Center-Point}
%\acro{cobots}[Cobot]{kollaborierender Roboter}
%\acro{ros}[ROS]{Robot Operating System}
%\acro{ur}[UR]{Universal Robots}
%\end{acronym}
\newpage
%\listoffigures
%\listoftables
	
	\pagenumbering{arabic}
%	\chapter{Einleitung}
Moderne Technologien wie das maschinelle Sehen finden in der heutigen Ära weit und breit in diversen industriellen und technischen Sektoren eine Anwendung. Optische Sensoren werden in diversen Prozessen eingesetzt, um Informationen aufzunehmen und zu verarbeiten, sodass darauf basierend (autonom) Entscheidungen über die Prozessreglung getroffen werden können. Einer der häufigsten Anwendungsfälle für optischen Sensoren und das maschinelle Sehen ist die Qualitätskontrolle. Hierzu werden Bilder aufgenommen und durch spezielle Verfahren verarbeitet, um beispielsweise Defekte zu erkennen. \autocite[3-11]{beyerer_machine_2015}

Maschinelles Sehen wird nicht nur auf die Auswertung und Verarbeitung zweidimensionaler Bilder beschränkt, sondern lässt es sich auch auf dreidimensionale Abbildungen anwenden \autocite{biegelbauer_model-based_2010}. Um solche Abbildungen zu erstellen werden Objekte häufig mit bestimmten Sensoren wie Lasersensoren abgetastet, um ihre Oberflächen in einem dreidimensionalen virtuellen Raum abzubilden \autocite[20-22]{savla_intelligente_2022}. Diese Aufnahmen werden in Punktwolken gespeichert, die alle Oberflächen des Objektes mit einer Vielzahl an dreidimensionalen Punkten modellieren. \Textcite{lougheed_3-d_1988} stellen Abbildungssysteme und Verarbeitungsverfahren für dreidimensionalen Abbildungen vor, die in der Robotik angewendet werden können. Es handelt sich dabei um die Entwicklung eines Verfahrens zur Erkennung von Flächen, wo Gegenstände durch den Roboter gegriffen werden sollten. 

Neben Flächen setzen sich Objekte aus mehr dreidimensionalen Merkmalen zusammen und die Erkennung solcher Merkmale eines Objektes hat viele relevante Anwendungen in der Industrie. Eine Teildisziplin des dreidimensionalen maschinellen Sehens beschäftigt sich mit der Erkennung von Kanten. Zum heutigen Stand sind in der Literatur diverse Beiträge zu finden, die unterschiedlichen Verfahren zur Erkennung dieser Kanten in Punktwolken vorstellen.

\section{Stand der Technik} \label{Stand_der_Technik}
Grundsätzlich bestehen zwei Gattungen von Verfahren zur Kantenerkennung, die unterschiedliche Stärken und Schwächen aufweisen. Die meisten Vorschläge für die Kantenerkennung machen sich entweder numerischen Methoden oder neuronalen Netzen zunutze.

\Textcite{hu_jsenet_2020} schlagen ein neuartiges Verfahren zur Kantenerkennung sowie Oberflächensegmentierung vor. Es wird als Eingabe eine Punktwolke vorausgesetzt und gleichzeitig durch zwei unterschiedlichen Operationen verarbeitet. Bei der ersten Operation werden mittels eines faltenden neuronalen Netzwerk (englisch: Convolutional Neural Network oder CNN) die Oberflächen der Punktwolke segmentiert. Bei der Segmentierung handelt es sich um die Gruppierung aller gleichartigen Punkte. Im zweiten Datenstrom werden die Kanten erkannt, die eine Punktwolke repräsentieren können. Schließlich werden beide Datenströme zusammengeführt, um die Informationen beider Ströme zu kombinieren. \Textcite{bazazian_edc-net_2021} schlagen auch ein Verfahren vor, das mittels eines neuronalen Netzes alle Randpunkte der Punktwolke klassifiziert. \Textcite{himeur_pcednet_2021} schlagen ähnlich ein Verfahren vor, welches auch ein CNN anwendet, um Kanten zu erkennen. 

\Textcite{choi_rgb-d_2013} schlagen in ihrem Werk ein neuartiges Verfahren zur Kantenerkennung in dreidimensionalen Farbbilder oder RGB-D Bilder vor. Diese Bilder haben neben den drei gewöhnlichen Farbkanäle auch einen vierten Kanal, welcher die Entfernung jedes Pixels von dem Sensorursprung angibt. Zur Erkennung der Kanten werden große Sprünge oder Unterbrechungen in der Tiefe des Bildes gesucht und die entsprechenden Pixel als Kanten erkannt. Darüber hinaus werden zweidimensionale Kantenerkennungsverfahren auf die Bildkomponente angewendet, um Kanten zwischen Flächen mit einer starken Krümmung zu erkennen. 

\Textcite{mineo_novel_2019} schlagen ein neuartiges Verfahren zur Kantenerkennung in unorganisierten Punktwolken, deren Punkte im Vergleich zu organisierten Punktwolken wie RGB-D Bilder nicht in einer vordefinierten Anordnung zu einander gespeichert werden. Das vorgeschlagene Verfahren bestimmt Randpunkte mittels der Analyse einer Nachbarschaft von Punkten innerhalb einer Sphäre. Das Verfahren ist in der Lage, konvexe und konkave Kanten zu erkennen, sowie den Umfang des Objekts nahtlos nachzuzeichnen.

\Textcite{lu_fast_2019} präsentieren ein Verfahren zur Kantenerkennung, welches etablierte Methoden wie Segmentierung, kleinste Quadrate (englisch: Least Mean Squares) und Projektion verwendet. In diesem Verfahren werden zuerst die Oberflächen der Punktwolke segmentiert. Danach werden alle Punkte der jeweiligen Oberflächen auf eine Ebene projiziert, um mittels der kleinsten Quadrate Methode die Kanten zu erkennen. Diese Kanten werden wieder zurück in die dritte Dimension projiziert.

\Textcite{ahmed_edge_2018} verwenden ein statistisches Verfahren zur Ermittlung eines mittleren Punktes oder eines Centroids einer Nachbarschaft von Punkten. Liegt der Abstand eines Punktes der Nachbarschaft über einen statistisch bestimmten Schwellwert zu dem Centroid, wird der Punkt als einen Randpunkt klassifiziert. So ermittelt das Verfahren nach den Autoren statistisch die Kanten. Es wird ein zusätzliches Verfahren zur Erkennung von Eckpunkten aus Randpunkten auf Basis der Krümmung der Randpunkte vorgeschlagen.

\Textcite{ni_edge_2016} legen ein Verfahren zur Erkennung und Segmentierung von Kanten dar. Die Kantenerkennung erfolgt durch die Bestimmung lokaler Nachbarschaften von Punkten und die Analyse der Winkelabstände zwischen den zugehörigen Punkte. Die Kantensegmentierung erweitert das Verfahren durch das sinnvolle Clustern von Randpunkte, die eine hohe Kollinearität aufweisen und lückenlos an einander anschließen. 

\section{Motivation und Zielsetzung} \label{Motivation}
Bemerkenswert ist die Tatsache, dass sehr viele Beiträge in der Literatur verschiedene neuartige Methoden zur Kantenerkennung und Segmentierung für offline Fällen präsentieren, während der Bereich der Kantenerkennung in online Fällen nach aktuellem Kenntnisstand weitgehend unerforscht blieb. Die Gewinnung von Informationen über Kanten in Punktwolken, während sie nebenbei wachsen, hat für viele Anwendungsgebiete eine große Bedeutung. Die Arbeit von \textcite{savla_intelligente_2022} untersucht eine Implementierung eines \textit{kognitiven} Schweißroboters, der die Schweißnaht des Bauteils während des Schweißvorgangs mittels eines Lasersensors bestimmt und verfolgt. Eine aktuelle Limitation dieses Roboters ist die fehlende Wahrnehmung der Bauteilgeometrie während des Schweißvorgangs. Mittels einer online Kantenerkennung wäre der Schweißroboter nicht nur in der Lage, Erkenntnisse über die Bauteilgeometrie zu sammeln, sondern auch die Laufbahnplanung und Prozessparameter dementsprechend anzupassen.

Für den Einsatz in einem automatischen Schweißvorgang des obigen Falls muss die Kantenerkennung einige Voraussetzungen erfüllen. Das wichtigste Kriterium für die Kantenerkennung ist die hohe Genauigkeit und Robustheit des Verfahrens. Das Verfahren soll unter reellen Bedingungen zuverlässige Ergebnisse liefern, um möglichst genau die Schweißnaht zu bestimmen sowie Hindernisse zu erkennen. Zur korrekten Erkennung von Kehlnähte soll das Verfahren auch Innen- sowie Außenkanten erkennen können. Die Anwendung des Verfahrens darf nicht nur auf organisierten Punktwolken beschränkt sein, sondern auch unorganisierten Punktwolken. Die online Funktionalitäten des Roboters setzen voraus, dass das Verfahren performant und in einer kurzen Zeitspanne die Kanten erkennen soll. Letztlich wird die Hardwarebeschleunigung mit einem Grafikprozessor ausgeschlossen, um Konflikte mit dem Echtzeitkernel der Robotersoftware zu verhindern. 

Im Rahmen dieser Arbeit wird angestrebt, ein Verfahren zur Erkennung der Kanten geometrischer Merkmale in wachsenden Punktwolken unter Einhaltung der obigen Anforderungen vorzulegen. Aus der Vielzahl der Verfahren aus Abschnitt~\ref{Stand_der_Technik} wird das AGPN-Verfahren (englisch: Analysis of Geometric Properties of Neighborhoods) nach \textcite{ni_edge_2016} als Grundlage für die online Kantenerkennung gewählt und mit weiteren Funktionalitäten für die Online-Erkennung erweitert. Für ihre Wahl über anderen Verfahren der Literatur hat die Methode ihre hohen Genauigkeit zu verdanken, insbesondere bei der Trennung zwei naheliegender Kanten. Darüber hinaus bietet das Verfahren der Autoren zur Kantensegmentierung eine elegante Möglichkeit an, zwischen den Kanten eines Objektes zu unterscheiden. Da das Verfahren auf numerischen Methoden zurückgreift und keine neuronale Netze verwendet, muss kein rechenaufwändiges Training mit einem großen Datensatz auf einem Grafikprozessor durchgeführt werden.  

Es wird im Laufe dieser Arbeit der Frage nachgegangen, wie ein nummerisches Verfahren aus der Literatur für die Kantenerkennung und Segmentierung für eine Anwendung auf wachsenden Punktwolken in der Online-Erkennung erweitert werden kann. Dabei ist das Ziel dieser Arbeit die Überprüfung der Effektivität eines solchen Verfahrens. Um die vagen Rahmen dieser Forschungsfrage zu konkretisieren wird sie in drei weiteren Forschungsfragen unterteilt. Im Umfang dieser Frage wird hinterfragt, zu welcher Genauigkeit das Verfahren Kanten erkennen kann, welcher Einfluss die Punktdichte auf die Genauigkeit hat und wie Robust das Verfahren gegen Unregelmäßigkeiten ist. Kurzgefasst sind dies die Forschungsfragen:

\begin{itemize}
	\item Wie Effektiv ist ein numerisches Verfahren bei der Kantenerkennung und Segmentierung von wachsenden Punktwolken?
	\begin{itemize}
		\item Zu welcher Genauigkeit kann ein solches Verfahren Kanten erkennen und segmentieren?
		\item Welcher Einfluss hat die Punktdichte auf die Genauigkeit?
		\item Wie Robust ist das Verfahren gegen Unregelmäßigkeiten?
	\end{itemize}
\end{itemize}

Anhand quantitativen Metriken wird das Verfahren nach den Richtlinien der drei Forschungsfragen ausgewertet. Dabei werden sowohl künstlich erzeugte Ground-Truth Dateien verwendet, als auch reelle Aufnahmen von echten Bauteilen. Durch die Verwendung eines Ground-Truths können die Ergebnisse der jeweiligen Untersuchungen mit höchster Zuversicht ausgewertet werden, während die reellen Aufnahmen eine Beurteilung der \textit{tatsächlichen} Effektivität bei der Anwendung unter reellen Bedingungen ermöglichen werden. Die quantitativen Ergebnisse der Untersuchungen ermöglichen auch eine vernünftige Gegenüberstellung des Verfahrens, um seine Genauigkeit mit anderen Verfahren aus der Literatur zu vergleichen. 

\section{Aufbau und Struktur dieser Arbeit}
Im folgenden Kapitel werden die nötigen theoretischen Grundlagen zur Methodik besprochen. Hierbei werden wichtige Kenntnisse über allgemeine Datenstrukturen und Algorithmen vermittelt. Darüber hinaus werden besondere Datenstrukturen und Algorithmen detaillierter behandelt, die in der Methodik eine Anwendung finden. 

In der Methodik wird darauf eingegangen, wie auf einem aktuell vorhandenen Verfahren aufgebaut wird, um sie mit neuen Funktionalitäten zur Kantenerkennung in wachsenden Punktwolken zu bereichern. Dabei wird zuerst behandelt, wie das Verfahren aus der Literatur nachgestellt wird. Danach werden die konkreten Schritte zur Erweiterung des Verfahrens genannt. Dabei werden konkrete Maßnahmen vorgestellt, die die Leistung oder Genauigkeit des Verfahrens verbessert haben. 

Im nächsten Kapitel wird die Genauigkeit und Einsatzfähigkeit des Verfahrens überprüft. Hierzu werden anhand der Forschungsfrage sowie der drei Teilforschungsfragen Tests konzipiert, die gezielt die Genauigkeit und Robustheit des Verfahrens ausmessen und darlegen. Neben der Beschreibung der drei Tests werden auch Testdateien und Auswertungsmetriken vorgestellt, die für die Tests verwendet werden. Unter den Testdaten wird eine künstliche Ground-Truth Datei sowie vier Aufnahmen reeller Bauteile vorgestellt. 

Als Vorletztes werden die Methodik und Ergebnisse dieser Arbeit diskutiert. Zuerst werden diese zusammengefasst und die wichtigsten Erkenntnisse aus den jeweiligen Kapiteln nochmals genannt. Darüber hinaus werden besondere Erkenntnisse der Methodik genannt, welcher zu einer Leistungsverbesserung geführt haben. Es werden die Ergebnisse der drei Tests interpretiert. Letztlich werden Limitationen dieser Arbeit diskutiert, die die Entwicklung des Verfahrens sowie seine Auswertung beeinflusst haben. 

Zuletzt wird der Beitrag dieser Arbeit zur Forschung zusammengefasst und über die Zukunftspotentiale hinsichtlich weiterer Untersuchungen sowie Erweiterungsmöglichkeiten diskutiert.
%	\chapter{Grundlagen}
Die Robotik deckt ein sehr breites Spektrum ab. Unter dem Begriff sind verschiedene Roboterarten mit unterschiedlichen Merkmalen und Eigenschaften inbegriffen. Aufgrund dieser Diversität ist es wichtig, dass Roboter auf Basis gewisse Merkmale untereinander differenziert werden. Eine Unterscheidung kann nach der Konstruktionsweise, Funktionsweise oder Verwendungszweck erfolgen. Unter Annahme einer Differenzierung nach der Konstruktionsweise können Roboter in vier Arten eingeteilt werden - stationäre Roboter, mobile Roboter, kognitive Roboter und humanoide Roboter. \autocite[24]{maier2022grundlagen}. Da humanoide Roboter keine breite Anwendung in der Industrie finden und durch Firmen wie Tesla, Inc. \autocite{morris2022} entwickelt werden, werden sie zuerst vernachlässigt. Hauptsächlich sind Industrieroboter in der Form von mehrachsigen stationären Robotern in der Produktion zu finden \autocite[25]{maier2022grundlagen}.

\section{Stationärer Industrieroboter}
Die Hauptaufgabe des Industrieroboters befasst das Bewegen und Führen eines Endeffektors zu einer Zielposition in einem vordefinierten Raum. Dieser Endeffektor kann ein Mittel zum Zweck des Greifens, Messens oder der Bearbeitung eines Werkstückes sein. Das Tool Center Point (TCP) oder der Werkzeugmittelpunkt bezeichnet einen charakteristischen Punkt des Endeffektors, welcher als einen Bezugspunkt für die Bewegungen des Roboters benutzt wird. Um den Endeffektor bewegen zu können, wird er als das letzte Armteil einer Aneinanderreihung von Armteilen angebracht, die miteinander durch Gelenke verbunden sind. Diese Gelenke zusammen mit den Größenverhältnisse der Arme spezifizieren die Bewegungsmöglichkeiten des Roboters. \autocite[64]{maier2022grundlagen}

Die Aneinanderreihung der Armteile des Industrieroboters definiert auch gleichzeitig seine kinematische Kette. Die kinematische Kette eines Roboters gibt alle Bewegungsmöglichkeiten und Positionen, die der Roboter annehmen kann, vor. Auf Basis der kinematischen Kette und der mechanische Aufbau kann eine Einteilung nach Bauart und sich daraus darbietendem Arbeitsraum erfolgen. Die Bauart eines Roboters ergibt sich aus der Art der Gelenke, die in der mechanischen Konstruktion benutzt wurden. Schubgelenke ermöglichen dem Roboter eine translatorische Bewegung mittels translatorischen Achsen. Drehgelenke ermöglichen dem Roboter im Gegensatz eine rotatorische Bewegung, indem sie eine rotatorische Achse bilden. \autocite[114-115]{maier2022grundlagen}

Knickarm-Roboter oder Gelenkarmroboter werden aus drei serielle Drehgelenke zusammengesetzt und besitzen somit eine RRR-Kinematik. Die serielle Kinematik bezeichnet eine reihenweise Anordnung einzelner Bewegungsachsen - eine Serie von Bewegungsachsen \autocite[84]{maier2022grundlagen}. Diese Bauart des Roboters findet am häufigsten eine Anwendung in der Industrie. Das Fundament eines vertikalen Knickarm-Roboters ist eine horizontal drehbare Basis, worauf weitere vier bis fünf Gelenke aufgebaut werden. Diese Gelenke sind nicht nur drehbar, sondern auch vertikal beweglich. Eine parallele Bewegung dieser Achsen ermöglicht dem Roboter das Ausführen von komplexen Bewegungsabläufen. Diese Flexibilität führt zu einem fast kugelsegmentförmigen Arbeitsraum, indem der Roboter sich bewegen kann. Aufgrund dieser hohen Beweglichkeit lässt sich der Knickarm-Roboter sehr gut für Produktionsaufgaben wie das Beschichten, Fräsen oder Schweißen einsetzen. \autocite[116-117]{maier2022grundlagen}

\begin{figure}[h]
	\includegraphics[width = \textwidth]{Abbildungen/Knickarm-Roboter.png}
	\centering
	\caption{Kinematische Kette eines Knickarm-Roboters \autocite[116]{maier2022grundlagen}}
\end{figure}

\subsection{Cobots} \label{ssec:cobots}
Einige Aufgaben der automatisierten Produktion sind manchmal sehr komplex. Um diese Komplexität zu überwinden, wird ein kollaboratives Arbeitsverhalten zwischen dem Roboter und Mensch gefördert. Kollaborierende Roboter, oder Cobots, sind Roboter die speziell für die Zusammenarbeit mit Menschen ausgelegt sind. Cobots sind Industrieroboter, die hinsichtlich der Sicherheitsgewährleistung nach dem Norm DIN EN ISO 10218 angepasst und geändert werden. Es bestehen hohe Sicherheitsanforderungen für Cobots, wie die Anwendung einer Leichtbauweise, zur Vermeidung jegliche Menschenverletzungen, da diese Roboter im Gegensatz zu herkömmlichen Industrieroboter in direktem Kontakt mit Menschen kommen. \autocite[181-183]{maier2022grundlagen}

In der Mensch-Roboter-Kollaboration überschneiden sich die Arbeitsräume des Menschen und des Roboters. Deswegen werden sie auch von einander durch eine physische Separation wie einen Schutzzaun oder eine Trennwand nicht abgetrennt. Aufgrund der fehlenden physischen Barriere steigt das Risiko einer Kollision zwischen Mensch und Roboter enorm. Deswegen müssen einige Sicherheitsanforderungen für die Zusammenarbeit erfüllt werden. Bei einer fehlenden physischen Schutztrennung soll innerhalb des Arbeitsraums des Roboters einen Gefahrenbereich identifiziert werden. Diese Zone soll mittels Schutzeinrichtungen wie Lichtschranken oder Laserscanner überwacht werden, sodass beim Betreten der Zone, ein Stoppsignal an dem Roboter sofort versendet werden kann. Cobots werden öfter zur Zielposition oder zwecks der Teach-In-Programmierung durch den Mensch per Hand geführt. Hierbei sollen Kraftsensoren im Roboter die Schubrichtung erkennen und die Bewegung in der Richtung mittels Motoren erleichtern. Gleichzeitig soll dabei die Geschwindigkeit durch den Roboter geregelt werden. Wichtig ist der Einbau einer Kollisionserkennung. Hierbei soll der Roboter eine Kollision mit einem Gegenstand über Kraft- und Impulssensoren erkennen und sofort aufhalten. Somit können ernsthafte lebensgefährliche Verletzungen vermieden werden. \autocite[12-16]{Schleicher2020}

Die Menschen-Roboter-Kollaboration ist neben der herkömmlichen Anwendung in der Produktion auch häufig in der Intralogistik zu sehen. Hinsichtlich der pünktlichen und kostengünstigen Warenlieferung werden Roboter zunehmend in diesem Bereich eingesetzt. Zur Kanalisierung des Warendurchflusses innerhalb des Warenhauses werden mobile Roboter für den Transport verwendet. Allerdings teilen sich diese Roboter eine Fläche mit anderen Menschen und müssen auf diese achten. Somit entsteht ein Netzwerk mobiler Roboter, die dem kollaborierenden Menschen keinen Gefahr stellen. Cobots werden auch anderweitig als Trag- und Hebehilfen für die menschlichen Mitarbeiter verwendet. \autocite[52]{Glück2022}

Eine Kollaboration mit Robotern bringt mit sich einige Verbesserungen der Prozessflexibilität, Prozesseffizienz und Ergonomie der Arbeit. Durch die Verwendung von Cobots können Fähigkeiten der Mitarbeiter und Roboter miteinander harmonisiert werden, sodass es sowohl von den Stärken der Menschen als auch der Robotern profitiert werden kann. Cobots dienen nicht nur als Instrumente der Prozessverbesserung sondern auch zur Erleichterung anstrengender Aufgaben. Somit können für die Mitarbeiter mehr Arbeitssicherheit gewährleistet sowie berufsbedingte körperliche Verletzungen und Krankheiten vermeidet werden. Die Menschen-Roboter-Kollaboration kann auch als Mittel gegen demographischen Probleme wirken. Ländern wo der Durchschnittsalter höher wird stehen vor einem künftigen Fachkräftemangel. Dieser kann durch die Verwendung von Cobots kompensiert werden, die die fehlende Arbeitskraft ersetzen können. \autocite[50]{Glück2022}

Die angesprochenen Problematiken sind in der Schweißindustrie besonders stark zu finden, weswegen Schweiß-Cobots zuletzt einen regelrechten Boom erlebt haben, und insbesondere als erleichterten Einstieg in die Robotik zu sehen sind. Nach dem Verstehen der grundsätzlichen Funktionsweise eines Industrieroboters lohnt sich ein Einblick in einem seiner Anwendungsfälle. Industrieroboter werden neben anderen Bereiche auch in der Schweißtechnik eingesetzt. In diesem Gebiet werden verschiedene übliche Schweißverfahren durch mehrere Modifizierungen angepasst, sodass sie durch Roboter ausgeführt werden können. Das Verstehen in dieser Prozesse und ihrer Parameter sowie deren Effekte ist für einen effektiven Einsatz der Schweißrobotik notwendig.

\section{Die Schweißrobotik}

\subsection{Warum die Schweißrobotik?}
Die wirtschaftlichen Vorteile der Industrieroboter in Einsatz für die Produktion wurden schon in ~\ref{sec:Industrieroboter} diskutiert. Heutige Marktanforderungen für das Schweißen setzen es voraus, dass auch nicht standardisierte, kundenspezifische Werkstücke schnell, kostengünstig und in kleineren Losgrößen hergestellt und verarbeitet werden können. Roboter sind dank ihrer Programmierbarkeit und diversen Effektoren für unterschiedlichen Aufgaben sehr flexibel und eignen sich für den Einsatz in der Produktion unter heutigen Anforderungen. Sie können für das Schweißen von unterschiedlichen Werkstücken mit abweichenden Geometrien schnell angepasst werden. Roboter besitzen auch speicherprogrammierbare Steuerungen (SPS) und sind somit in der Lage, mit Sensoren sowie der Schweißquelle zu kommunizieren und diese zu steuern. Das automatisierte Schweißen benötigt eine Bahnsteuerung und sensorische Information über den Prozess. Eine Steuereinheit und Sensorintegration lassen dem Roboter diese Fähigkeiten zuteilwerden und ermöglichen die Koordination zwischen der Schweißquelle, Schweißpistole und dem Schweißverfahren. \autocite[17-22]{Pires_WeldingRobots_2006}

Es gibt unterschiedlichen Schweißverfahren, die in der Industrie für die Metallverarbeitung verwendet werden. Aufgrund der unterschiedlichen Schweißmethodiken gibt es verschiedenartige Parameter, die den Prozess beeinflussen und somit durch den Schweißroboter kontrolliert werden müssen.

\subsection{Wolfram-Inertgas-Schweißen}
\subsubsection{Verfahren}
Dieses Schweißverfahren (\emph{WIG}) ist auf dem Lichtbogenschweißverfahren basiert und verwendet dazu einen inerten Gas zum Schutz des Schweißbades. Bei dieser Methode entsteht ein Lichtbogen zwischen ein´er Wolfram-Elektrode und den Werkstücken, wodurch Wärme generiert wird. Diese Wärme führt zu dem Schmelzen der Werkstückoberflächen und einer Zusatzelektrode sowie zu der Formation eines Schweißbades, das aus flüssigem Metall besteht. Dieses Schweißbad wird durch das Inertgas umhüllt und wird somit gegen Fremdstoffe und Verunreinigungen in der Luft beschützt. Durch das Abkühlen des Schweißbades entsteht eine solide und starke Verbindung der Werkstücke. Eine wesentliche Merkmale dieses Verfahrens ist die Wolfram-Elektrode. Diese ist eine Dauerelektrode und wird während des Prozesses nicht verbraucht. Die, durch dieses Verfahren hergestellten Teile haben eine höhere Schweißqualität, weniger Bauteilverzug und kaum Metallspritzflecken. \autocite[27-28]{Pires_WeldingRobots_2006}

\subsubsection{Geräte}
Das WIG-Schweißverfahren besteht hauptsächlich aus vier Komponenten. Die Schweißquelle oder Stromquelle ist für die Bereitstellung der nötigen Spannung und Strom für das Schweißen zuständig. Die Schweißquelle besteht aus Stromrichter, Wechselrichter und einen Transformator. Der Wechselstrom aus dem Hauptnetz wird zuerst in Gleichstrom umgewandelt. Danach folgend wird er wieder rückgängig in Wechselstrom mit einer höheren Frequenz als den Netzstrom gewandelt. Dank einer Frequenzerhöhung kann der Transformator die gleiche Leistung bei einem kleineren Form und niedrigeren Gewicht leisten. Somit bleibt die Schweißquelle leicht transportable. Nach einer Invertierung wird der Hochspannungsstrom zu einer niedrigeren, zum Schweißen geeigneten, Spannung gebracht. Dieser Strom wird wahlweise danach zum Gleichstrom verwandelt. Mittels eines Regelkreises wird die Spannung des Ausgangsstroms überwacht. Meisten Stromquellen bieten die Möglichkeit an, zwischen Gleich- oder Wechselstrom für den Ausgangsstrom zu wählen. \autocite[28]{Pires_WeldingRobots_2006}

Die Schweißpistole dient nicht nur als Halterung der Dauerelektrode, sondern auch zur Übertragung des Stroms zu der Dauerelektrode sowie zur richtigen Führung des Inertgases zum Schweißbad. Je nach Leistung und Einsatzdauer der Schweißpistole wird besteht der Bedarf einer Gas- oder Wasserkühlung. \autocite[29]{Pires_WeldingRobots_2006}

Dauerelektroden bestehen aus pures Wolfram oder eine Wolframlegierung. Während pures Wolfram ein paar Vorteile in der Anwendung hat, wird eine Wolframlegierung mit zwei prozentiges Thorium-Oxid (Thorium-Oxid Wolfram) weit in der Industrie verwendet. Diese Legierung hat eine hohe Beständigkeit gegen Kontaminierungen, einen kleiner Wartungsbedarf und erzeugt stabileren Lichtbogen. Aufgrund der Radioaktivität von Thorium-Oxid werden auch andere Alternativlegierungen verwendet. Beispielsweise findet eine Wolframlegierung mit Zirkonium auch eine häufige Anwendung, da sie sehr ähnlichen Eigenschaften zu Thorium-Oxid Wolfram hat. \autocite[29-30]{Pires_WeldingRobots_2006}

Der Inertgas-Regulator wird zur Reglung des Gasdruckes verwendet. Schwankungen in dem Druck der Gasquelle können somit ausgeglichen werden. Die Druckreduzierung und Regulierung kann in einem oder zwei Phasen erfolgen, wobei die zweite Variante eine stabilere Ausströmung anbietet. \autocite[31]{Pires_WeldingRobots_2006}

\subsubsection{Prozessparameter}
Nachfolgend wird diskutiert, wie Prozessparameter des Schweißverfahrens die Qualität der Schweißnaht beeinflussen. Schweißgeräte verleihen die Kontrolle über diese Parameter, wodurch die Qualität des Prozesses geregelt werden kann. 

Der Schweißstrom wirkt auf die Form der Schweißnaht ein. Diese ist als den Querschnitt der Naht vorzustellen, die orthogonal zu der Schweißrichtung ist. Durch die Verwendung von Wechselstrom oder Gleichstrom unterschiedlicher Polaritäten entsteht. Einen Gleichstrom an der negativen Elektrode liefert die besten Ergebnisse mit einer besseren Einbrandtiefe und Schweißgeschwindigkeit. Schweißen mit Gleichstrom an der positiven Elektrode liefert im Gegensatz eine sehr geringe Einbrandtiefe. Schweißen mit Wechselstrom stellt das mittlere Ergebnis dar. Wechselstrom-Schweißen liefert beim Schweißen von Metallen wie Aluminium und Mangan gewisse Vorteile, indem es die oxidierte Oberflächenschicht dieser Stoffe beim Schweißen entfernt. Gepulster Gleichstrom wird häufig zur Reduzierung des Bauteilverzuges verwendet. Die Periode und Stärke des Stromes wird anhand der Materialeigenschaften und Werkstückprofilgeometrie festgelegt. Die Frequenz hat einen Einfluss auf diversen Eigenschaften, wie die Einbrandtiefe, maximale Schweißgeschwindigkeit und reduzierte Porosität der Schweißnaht. Die Einstellung der Schweißstrom-Eigenschaften erfolgt über die Stromquelle. \autocite[31-32]{Pires_WeldingRobots_2006}

Die Schweißgeschwindigkeit hat einen direkten Einfluss auf die Erwärmung der Werkstücke. Dies geschieht ohne eine Änderung der Lichtbogen-Eigenschaften. Mit einer Erhöhung der Schweißgeschwindigkeit wird der Querschnitt der Schweißnaht reduziert sowie, zu einem niedrigeren Grad, die Einbrandtiefe und Breite der Naht. Anhand des Materials und Geometrie des Werkstückes sowie die Eigenschaften des Schweißstroms werden standardmäßig Schweißgeschwindigkeiten von 100 bis zu 500 mm/min gewählt. Da die Schweißpistole als Endeffektor eines Roboters eingesetzt wird, wird die Schweißgeschwindigkeit durch die Roboterbewegung bestimmt. \autocite[33]{Pires_WeldingRobots_2006}

Die Lichtbogenlänge wird aus der Distanz zwischen der Werkstückoberfläche und Dauerelektrode ermittelt. Die Lichtbogenlänge und Spannung aus der Stromquelle stehen in einem direkten Zusammenhang mit einander. Mit einer Vergrößerung der Lichtbogenlänge muss zur Gewährleistung der Lichtbogenstabilität die Phasendifferenz zwischen der Elektrode und Werkstück erhöht werden. Die Lichtbogenlänge darf vergrößert werden, wenn die Erwärmung der Werkstücke reduziert werden soll, da mehr Wärme durch Strahlung verloren wird. Demzufolge wird die Einbrandtiefe und der Querschnitt der Schweißnaht reduziert. Die Lichtbogenlänge hängt von einer physikalischen Größe ab - die Distanz. Deswegen kann sie durch den Schweißroboter beeinflusst werden, indem die Position der Schweißpistole manipuliert wird. \autocite[33]{Pires_WeldingRobots_2006}

Der Auswahl des Schutzgases spielt eine wesentliche Rolle bei der Prävention von Verschmutzung der Schweißnaht durch Umwelteinflüsse. Eine Verunreinigung kann die Qualität der Schweißnaht und die Stärke der Schweißverbindung deutlich beeinträchtigen. Zusätzlich hat der Schutzgas auf die Lichtbogenzündung sowie -Stabilität eine Ausprägung. Am häufigsten wird der Inertgas Argon aufgrund seiner niedrigen Ionisation und schwerem Gewicht als Schutzgas verwendet. Diese Eigenschaften fördern die Lichtbogenentzündung und Stabilität der Luftblase um den Schweißbad herum. Ein weiterer Vorteil dieses Gases ist sein niedriger Preis im Vergleich zu anderen Inertgasen wie Helium. Wenn höherer Temperaturen zum Schweißen dickerer Werkstücke oder leitfähigerer Materialien erfordert werden, wird Helium als Schutzgas verwendet. Aufgrund seiner höheren Ionisation muss die Spannung des Schweißstroms für eine erfolgreiche Lichtbogenentzündung und stabileren Lichtbogen erhöht werden. Folglich entstehen, die zum Schweißen nötigen höheren Temperaturen. \autocite[33-34]{Pires_WeldingRobots_2006}

Nach \textcite[499-503]{Abbasi2018} steht die Flussrate des Schutzgases in einer inversen Korrelation mit der Mikrostruktur und Beschaffenheit der Schweißnaht, zwar die Bildung von Kleinrissen und Pori. Während eine niedrige Flussrate zur Formung von Rissen und Poren führt, können sie durch höhere Flussrate vermeidet und die Zugfestigkeit sowie Duktilität des Materials verbessert werden. Laut \textcite{Tesfaw2022} sind bei anderen Schutzgas-Schweißmethoden auch Korrelationen zwischen der Gasflussrate und Schweißnaht-Qualität zu sehen, beispielsweise die Materialhärte. \textcite{Mvola2017} diskutieren über Methoden zur intelligenten und adaptiven Steuerung des Gasflusses. Mittels verschiedenen Sensoren wird Information über den Gasfluss gesammelt und diese in einer Reglung eingekoppelt. Somit ist nicht nur eine Kontrolle über die Gasflussrate möglich, sondern auch eine Anpassung der Gasmischung.

Der Winkel und die Stellung der Elektrode hat eine Einwirkung auf die Einbrandtiefe und Lichtbogendruck. Je kleiner der Winkel, desto besser ist die Einbrandtiefe und Lichtbogendruck. Allerdings führt es auch zur Verschlechterung der Schweißpistolenspitze. \autocite[34]{Pires_WeldingRobots_2006}

\subsection{Metall-Schutzgas-Schweißen}
\subsubsection{Verfahren}
Dieses Schweißverfahren (\emph{MSG}) ist in der Funktionsweise sehr ähnlich zum WIG-Schweißen. Auch hier wird das Werkstück und die Elektrode mittels eines Lichtbogens erhitzt. Der Unterschied zwischen den beiden Verfahren liegt an der Art der Elektrode und des Schutzgases. Die Elektrode dieses Verfahrens unterscheidet sich von der Dauerelektrode des WIG-Verfahrens dadurch, dass sie abschmelzt und während des Schweißens verbraucht wird. Somit ist kein externes Schweißzusatz für diese Methode benötigt. Aus diesem Zusatz entsteht die Schweißnaht. Es können außer Inertgase auch Aktivgase als Schutzgas verwendet werden. \autocite[36]{Pires_WeldingRobots_2006}

Das MSG-Verfahren findet weit und breit in der Industrie eine Anwendung. Es ist für das Schweißen dünne sowie dicke Materialien in allen möglichen Stellungen der Schweißpistole gleicht effektiv. Aufgrund der höheren Schweißgeschwindigkeit und die niedrigere Anforderung an Erfahrung und Fähigkeit des Bedieners ist diese Methode auch sehr wirtschaftlich. Da die Zufuhr des Schweißzusatzes automatisch erfolgt, kann sie maschinell gesteuert werden. All diese Vorteile machen das MSG-Verfahren für den Einsatz in der Schweißrobotik sehr geeignet. \autocite[37]{Pires_WeldingRobots_2006}

\subsubsection{Geräte}
Da die MSG-Verfahren eine sehr ähnliche Funktionsprinzip zu dem WIG-Verfahren haben, gibt es bei den beiden Verfahrensarten ähnliche Geräte, die dem gleichen Zweck erfüllen.

Die Stromquellen der MSG-Verfahren haben eine ähnliche elektrische Architektur wie die des WIG-Verfahrens. Sie liefern eine konstante Spannung, die zusammen mit einem konstanten Drahtvorschub für die Stabilität des Lichtbogens sorgt und einen Ausgleich bei abweichender Entfernung zwischen der SNeben Geräte teilen sich die beiden Schutzgasschweißverfahren auch sehr ähnliche Prozessparameter. chweißpistole und Werkstückoberfläche ermöglicht. Diese Quellen bieten auch die Möglichkeit an, den Strom in Intervallen zu erzeugen und leiten. Typischerweise werden Strompulse in einer Frequenz von 100 Hz bis 200 Hz generiert. \autocite[38]{Pires_WeldingRobots_2006}

Das Drahtvorschubgerät ist für diese Art des Schutzgasschweißens ein Alleinstellungsmerkmal, da die Zuführung des Schweißzusatzes in dem WIG-Verfahren durch den Anwender kontrolliert wird. Dieses Gerät ist meistens innerhalb des Schweißgerätes integriert. Es ist für die Abwickelung der abschmelzenden Elektrode aus einer Spule und deren Begradigung sowie Führung zu der Schweißpistole zuständig. \autocite[39-40]{Pires_WeldingRobots_2006}

Die Schweißpistole der MSG-Verfahren ist im Aufbau und in der Funktion sehr ähnlich zu der des WIG-Verfahrens. Ihre Hauptfunktion ist die Leitung von Strom zu der abschmelzenden Elektrode und die Schutzgasführung. Abhängig von dem Einsatz wird die Schweißpistole Wasser- oder Gasgekühlt. Diese Pistole ist einzigartig, indem deren Taster Steuersignale nicht nur an der Stromquelle und dem Gasventil sendet, sondern auch an dem Drahtvorschubgerät. \autocite[40]{Pires_WeldingRobots_2006}

\subsubsection{Prozessparameter}
Neben Geräte teilen sich die beiden Schutzgasschweißverfahren auch sehr ähnliche Prozessparameter. Ähnlich zu dem WIG-Verfahren wird abhängig von der Material und dem Einsatz entweder die abschmelzende Elektrode mit Gleichstrom sowie Wechselstrom oder das Werkstück mit Gleichstrom beliefert. Eine höhere Spannung führt zu einer flacheren und breiteren Schweißnaht, wobei eine zu hohe Spannung die Lichtbogenstabilität negativ beeinflusst und die Schweißnaht-Qualität verschlechtert. Eine niedrige Spannung verursachen eine höhere Schweißraupe, die zur besseren Bewehrung der Schweißnaht führt. Bei der Wahl einer richtigen Spannung und eines Stroms wird das Übertragungsmodus ins Betracht genommen. Die Übertragungsmodi bestimmen wie die Wärme sowie das Schweißzusatz an der Werkstückoberfläche angebracht werden und somit zum Werkstück übertragen werden. Der Wahl eines Übertragungsmodus wird auf Basis der Werkstückmaterial, gezielten Schweißqualität und anderen Parameter bestimmt. \autocite[41-42]{Pires_WeldingRobots_2006}

Im Gegensatz zu dem WIG-Verfahren hat die Schweißgeschwindigkeit der MSG-Verfahren keinen linearen Zusammenhang mit der Wärmeeinbringung zu dem Werkstück und Ablagerungsrate des Schweißzusatzes. Eine initiale Erhöhung der Schweißgeschwindigkeit zeigt einen höheren Einbrand in dem Werkstück, da die Werkstückoberfläche eine längere Exposition zu dem Lichtbogen hat. Eine Steigerung dieses Parameters ab einer Grenze führt zu einer Verschlechterung der Einbrandtiefe und Schweißnaht-Qualität, da das Schweißbad nicht mit genügendem Schweißzusatz beliefert wird. \autocite[42]{Pires_WeldingRobots_2006}

Elektrodenverlängerung steht für die Länge der Elektrode, die von der Schweißpistole herausragt. Sie wird durch die Distanz zwischen der Pistolenende und Werkstückoberfläche bestimmt. Dieses Parameter hat eine direkte Korrelation mit der Schmelzrate der abschmelzenden Elektrode und wird anhand des gewählten Übertragungsmodus angepasst. Das Elektrodendurchmesser wird auf Basis des Schweißstromes, der Materialeigenschaften und Werkstückgeometrie gewählt. \autocite[42-43]{Pires_WeldingRobots_2006}

Der Drahtvorschub regelt die Geschwindigkeit, mit der die abschmelzende Elektrode zu der Schweißpistole geführt wird. Somit ist auch die Masse des Schweißzusatzes kontrolliert, die zu dem Schweißbad und der Schweißnaht hinzugefügt wird. Integral für eine gute Schweißqualität ist eine konstante und ruckfreie Zuführung der Elektrode. Nach \textcite{Senthilkumar2017} stehen die Abschmelzrate des Schweißzusatzes und der benötigte Schweißstrom mit dem Drahtvorschub im Zusammenhang. Auch in der gleichen Studie wurde festgestellt, dass ein höherer Drahtvorschub in einer höheren Schweißraupe resultiert.

Neben der in dem WIG-Verfahren verwendetem Inertgas werden üblicherweise auch aktive Gase in den MSG-Verfahren verwendet. Während Inertgase zum Schweißen reaktive Materialien wie Kupfer und Nickel verwendet wird, werden Aktivgase für diverse Stähle benutzt. Aktive Gase werden häufig auch mit Inertgasen gemischt. Das Sauerstoff in Argon-Sauerstoff-Mischungen für Edelstähle und das Kohlendioxid in Argon-Kohlendioxid-Mischungen für Kohlenstoffstähle trägt zur Stabilisierung des Lichtbogens bei und beeinflusst die Schweißnaht-Geometrie. Auch ternäre Gasmischungen werden zur Verbesserung der Materialbeständigkeit gegenüber Verschmutzung häufig für bestimmte Materialien verwendet. \autocite[42-43]{Pires_WeldingRobots_2006}

Es soll bemerkt werden, dass diese Schweißparameter nicht vollständig unabhängig voneinander angepasst und justiert werden können. Im Gegensatz dazu stehen einzelne Schweißparameter in konkreten Zusammenhängen mit einander. Die Änderung gewisser Parameter führt zu einer automatischen Anpassung eines oder mehrerer anderer Parameter. \autocite[40]{Pires_WeldingRobots_2006}

\subsection{Laserstrahlschweißen}
\subsubsection{Verfahren}
Bezugnehmend auf Abschnitt ~\ref{ssec:lasersensoren} besteht ein Laser aus Lichtstrahlen, die durch Stimulierung von Atomen entstehen, verstärkt und ausgestrahlt werden. Die Energie aus dem Laserstrahl wird entweder durch Reflexion abgeführt oder durch das Werkstück absorbiert. Durch diese Energie wird das Bauteil erwärmt, welches zum Abschmelzen oder sogar zur Verdampfung des Materials führt. Auf Basis der Energiedichte kann das Laserstrahlschweißen in zwei Verfahren aufgeteilt werden. Laserschweißen mit einer niedrigeren Energiedichte führt zu einem Energieverlust von bis zu neunzig Prozent. Die absorbierte Energie schmelzt die Werkstückoberfläche aber reicht für die Verdampfung nicht aus. Das Schweißbad dieser Variante kann durch sein breites aber flaches Profil charakterisiert werden. Die zweite Prozessvariante betrifft Laserschweißen mit Energiedichten über $10^{10}\ Wm^{-2}$. Aufgrund dieser hohen Energie schmelzt sowie verdampft sich das Oberflächenmaterial teilweise. Dadurch verteilt sich die Hitze tiefer im Material und schafft eine schmale und tiefe Schweißnaht, als das flüssige Material hinter dem Laser abkühlt. Die zweite Prozessvariante wird häufig zum Schweißen dickere Werkstücke mit hohen Schweißgeschwindigkeiten eingesetzt und kann mit oder ohne Schweißzusatz erfolgen. \autocite[45-47]{Pires_WeldingRobots_2006}

Das Laserstrahlschweißen lässt sich sehr gut einsetzen, wo eine sehr hohe Präzision mit geringem Bauteilverzug erforderlich ist. Allerdings ist die Qualität der Ergebnisse dieses Verfahrens von der Materialeigenschaften wie Reflektivität und Lichtabsorption. Daneben ist die Investition für dieses Verfahren ein Vielfaches der Kapitalanlage für die Lichtbogenschweißverfahren. \autocite[47]{Pires_WeldingRobots_2006}

\subsubsection{Geräte}
Grundsätzlich werden die gewöhnlichen Geräte aller Schweißverfahren auch in diesem Verfahren verwendet. Eine Schweißpistole dient dem Zweck des Führens und Richtens des Laserstrahles sowie des Schutzgases. Ein Schutzgas wird zur Vermeidung von Porenbildung in der Schweißnaht und Verschlechterung der Schweiß-Qualität benutzt. Auch der optionale Einsatz eines Schweißzusatzes kann über eine Drahtvorschubeinrichtung erfolgen. Die Stromquelle bei diesem Verfahren dient allerdings nicht zur Generierung eines Lichtbogens, sonder zur Erzeugung des Laserstrahls. In diesem Schweißverfahren werden zwei Arten von Lasern benutzt - Festkörperlaser und Gaslaser \autocite[47]{Pires_WeldingRobots_2006}.

In Festkörperlasern befinden sich spezielle Kristalle, die erregt werden und einen starken Laserstrahl ausstrahlen. Die Erregung erfolgt mittels Lichtblitze aus Laserdioden. Grundsätzlich kann der Laserstrahl bei geringer Leistung im Dauermodus emittiert werden, allerdings werden sie aufgrund beschränkter Kühlleistung mit sehr hohen Leistungen und konstanter Frequenz gepulst. Der Laserstrahl wird mittels einer Glasfaserleitung zu der Schweißpistole geleitet. Durch die Nutzung solch einer Leitung kommt der Lasersystem eine höhere Flexibilität zugute. \autocite[47-48]{Pires_WeldingRobots_2006}

In Gaslasern wird der Laserstrahl durch die Erregung eines zirkulierenden Gases erzeugt. Dieser Strahl wird mittels Spiegeln und teil-reflektierenden Spiegeln fokussiert und gerichtet. Die Leistung eines Lasers dieser Art wird auch durch die Kühlleistung des Systems begrenzt. Gaslaser werden selten kontinuierlich ausgestrahlt, sondern häufig gepulst. Ähnlich wir bei Festkörperlasern wird der Laserstrahl mittels einer Glasfaserleitung übertragen. \autocite[48-49]{Pires_WeldingRobots_2006}

\subsubsection{Parameter}
Die verfahren-übergreifende Geräte wie die Schweißpistole und Drahtvorschubeinrichtungen bieten Anpassungsmöglichkeiten an, die auch verfahren-übergreifend sind. Effekte dieser Parameter wie der Drahtvorschub und die Elektrodenverlängerung auf die Qualität sind gleich wie bei anderen Schweißverfahren. Einige Parameter zeichnen sich allerdings aus, da sie einzigartig zu den verwendeten Lasern sind. 

Die Leistung des Laserstrahls hat bei einem konstanten Durchmesser einen direkten Einfluss auf die Einbrandtiefe. Mit einer Erhöhung der Laserleistung steigt auch die Energiedichte an, sodass mehr Material verschmolzen und verdampft wird. Auf die Energiedichte hat auch die Brennweite der fokussierenden Linse einen Einfluss. Mit einer Erhöhung dieses Werts wird das Durchmesser des Laserstrahls größer, welches zu einer Reduzierung der Energiedichte führt. Durch die Einstellung der Tiefe des Brennpunktes kann die Geometrie und Tiefe der Schweißnaht kontrolliert werden. Die Setzung des Brennpunktes über die Werkstückoberfläche führt zu einer kleineren Einbrandtiefe und dünneren Schweißnaht. Allerdings steigt die Genauigkeitsanforderung der Werkstückplatzierung bei der tieferen Setzung des Brennpunktes in dem Werkstück. \autocite[50]{Pires_WeldingRobots_2006}

Die Effizienz dieses Schweißverfahrens wird auch durch Materialeigenschaften beeinflusst. Eine hohe Absorptionsfähigkeit des Materials resultiert in einer hohen Energieaufnahme durch das Material, die zu einer besseren Schmelzleistung führt. Konträr ist eine hohe Reflektivität zum Schmelzen des Werkstückes ungeeignet. Die geringe Absorption von Metallen erklärt die Erforderlichkeit der hohen Laserleistung. \autocite[52-53]{Pires_WeldingRobots_2006}

Die Schweißgeschwindigkeit spielt analog zu anderen Schweißverfahren bei der Einbrandtiefe eine Rolle. Eine höhere Schweißgeschwindigkeit reduziert die Einbrandtiefe. Merkwürdig ist die Entstehung eines flacheren Einbrands bei sehr geringen Schweißgeschwindigkeiten. Dies passiert aufgrund Plasmawolken, die den Laserstrahl stören und somit die Wärmeleistung verringern. \autocite[51]{Pires_WeldingRobots_2006}

Der Schutzgas dient bei diesem Verfahren nicht nur zum Schutz des Schweißbades vor atmosphärische Verunreinigungen, sondern auch zur Entfernung von Plasmawolken. Dies geschieht, indem ein Inertgas wie Helium lateral über das Schweißbad gepustet wird. Je nach Reaktivität des Werkstückmetalls wird ein passendes Schutzgas gewählt. \autocite[52]{Pires_WeldingRobots_2006}

\subsection{Durchbrüche in der Schweißrobotik}

\textcite{Wilmsmeyer_2018} haben ein Design für einen Schweißroboter vorgeschlagen, der fähig ist, Bauteilen mit großen Toleranzen präzise ohne Nachverarbeitung zu fertigen. Es werden Geometriedaten des Bauteils über CAD-Daten eingelesen. Für die hoch-genaue Bahnplanung wird allerdings die Bauteilgeometrie mittels 3D-Sensoren optisch getastet. Auch zur Erkennung der Schweißnaht werden diverse Sensoren wie Gasdüsensensoren, Lichtbogensensoren und Lasersensoren benutzt. Obwohl die Autoren zugeben, dass das optische Messverfahren zur Erfassung der Bauteilgeometrie nicht die Genauigkeit des taktilen Verfahrens nachahmen kann, ist dieses bei vielen Fällen deutlich schneller. Diese Arbeit schaffte es, sinnvolle Geometriedaten optisch mittels Sensoren zu erfassen.

Eine andere Arbeit durch \textcite{Yang_2019} schlägt eine weitere Methode für die intelligente Naht-Erkennung vor. Es referenziert auch diverse andere Arbeiten, die mittels unterschiedlichen Sensoren wie Ultraschallsensoren, Infrarotsensoren und RGB-D Tiefenkameras diverse Erkennungen während des Schweißprozesses machen könnten. Die Autoren dieser Arbeit entwickelten eine Methode zur Extraktion der Schweißnaht-Geometrie aus einer Punktewolke des Bauteils. Ihre Methode für die Erkennung könnte im Vergleich zu der traditionellen Teach-In-Methode eine deutliche Effizienzverbesserung und Zeitersparnis bringen. 

Eine ähnliche Arbeit wurde durch \textcite{Geng2021} geleistet, wo sie eine Methode zur Bahnplanung des Schweißroboters ohne den Bedarf an Teach-In-Programmieren vorschlagen. Dies geschieht unter der Verwendung eines industriellen 3D-Kameras, welches zum Einscannen der Werkstückoberfläche verwendet wird. Diese Kamera ermittelt Entfernungen auf Basis der Stereo-Vision-Technologie. Aus der 3D-Daten des Werkstückes wird eine Punktwolke erstellt. Nach einer Vorverarbeitung, wird mathematisch die Naht aus zwei deckungsgleiche Ebenen ermittelt. Somit kann eine präzise Bahnplanung der Schweißnaht entlang ohne eine Teach-In-Programmierung erfolgen. Mit einer geringen Zeitanforderung ist diese Methode auch für die Industrie interessant. 

Mittlerweile ist die Sensorik ein integraler Bestandteil der Schweißrobotik geworden, sondern auch anderer Roboteranwendungen. Die Sensorik ermöglicht sowohl die kontinuierliche Überwachung der Position und Stellung der Roboterarme, als auch die Wahrnehmung der Umwelt um den Roboter. Um die Sensordaten aus der Umgebung sinnvoll für die Roboterprozesse zu verwenden wird eine zentrale Steuerung notwendig. Dies erfolgt mittels einer zentralen Steuereinheit, die die Aufgabe der Überwachung aller Bewegungen übernimmt und diese anhand der Sensordaten steuert. Letztendlich ist auch eine Programmiereinrichtung, meistens ein multifunktionales Programmierpanel nötig, das zur Weisungsbefugnis dient. \autocite[114]{maier2022grundlagen}

\section{Sensoren eines Industrieroboters}
Analog zu den Sinnesorgane der Menschen ermöglichen Sensoren den Maschinen oder Geräten es, auf die Umwelt zu reagieren. Konkret bereichern Sensoren Maschinen, Geräte oder Anlagen mit der Fähigkeit, bestimmte Zustände oder Zustandsänderungen wahrzunehmen und auf diese zu reagieren. \autocite[1-2]{HesseStefan2018SFDP}. Robotersteuerungen können mit der aus Sensoren gewonnen Informationen den Ist-Zustand des Roboters mit dem Soll-Zustand vergleichen und darauf basierend Anweisungen generieren. Sensoren für Roboter können auf Basis ihrer erfassten Datenarten in zwei Klassen unterteilt werden - interne und externe Sensoren. \autocite{maier2022grundlagen}

Interne Sensoren fassen Information über den Zustand der intrinsischen physikalischen Eigenschaften des Roboters auf, beispielsweise dessen Innentemperatur, Geschwindigkeit oder Position. Sie liefern Aktualisierungen über die Position und Achsenstellung des Roboters sowie den Status des Endeffektors. Beschleunigungssensoren und Gyroskopen werden häufig zur Erfassung der Beschleunigung, Kräfteverhältnis oder Orientierung eingesetzt. \autocite[220]{maier2022grundlagen}

Externe Sensoren erfassen die qualitativen und quantitativen physikalischen Eigenschaften der Umgebung des Roboters. Somit ermöglichen sie die Wahrnehmung des Arbeitsraumes sowie des zu verarbeitenden Objektes. Die Beschaffenheit von Oberflächen, Ausrichtung von Objekten oder die Entfernung zu einem Gegenstand zählen unter gängigen physikalischen Größen, die durch externen Sensoren wie Kameras, optische Sensoren oder Tastensensoren erfasst werden. Diese Sensoren nehmen die Werten der physikalischen Größen auf und wandeln diese in eindeutig verwertbare, herkömmlicherweise in elektrische Signale um\autocite[221]{maier2022grundlagen}

\subsection{Lasersensoren zur Distanzermittlung} \label{ssec:lasersensoren}
Das Laser - ein Kürzel für \emph{light amplification by stimulated emission of radiation} - ist ein kohärenter und monochromatischer Lichtstrahl, der im Vergleich zu Lichtstrahlen aus anderen Quellen intensiver und feiner ist. Ein Laserstrahl kann über großen Entfernungen mit wenigem Verlust reflektiert werden, welches dem Lasersensor zu einem geeigneten Distanzermittler macht. Mittels einer Laserdiode kann ein Laserstrahl induziert werden. \autocite[126]{HesseStefan2018SFDP}

\subsubsection{Die Laserdiode}
Die Laserdiode setzt sich aus drei integralen Bestandteile zusammen - die p- und n-Elektroden, einen Kristall, und einen Spiegel. Die p-Elektrode wird durch Strom erregt, welches dazu führt, dass erregte Elektronen ein Photon freilassen, um wieder zu stabilisieren. Die freigesetzten Photonen werden mehrmalig in dem Kristall reflektiert, welches dazu führt, dass dieser erregt wird. Der Kristall setzt wiederum Photonen der gleichen Frequenz und Phase als die aus der p-Elektrode freigesetzten Photonen frei. Aus der Resonanz dieser gleichphasigen Photonen entsteht ein intensiver Lichtstrahl. \autocite[4-5]{PeterWEpperlein20013} \autocite[16-21]{Hooshang2004} \autocite[221-232]{DomingoGeorge2007}

Die Distanzermittlung funktioniert grundsätzlich auf Basis der Reflektion. Der Laserstrahl, meistens ein rotes oder unsichtbares infrarotes Licht, wird auf dem Objekt projiziert. Dieser Strahl wird durch das Objekt zurück zu dem Lasersensor reflektiert, welcher aus dieser Information ein elektrisches Signal berechnet. Dieses Signal dient zur Ermittelung der Objektdistanz. Das Messverfahren zu Grunde liegend kann der Lasersensor in drei Unterklassen eingeteilt werden.\autocite[167]{Hering2018}

\subsubsection{Pulszeitverfahren}
Bei dem Pulszeitverfahren wird ein Lichtpuls in Intervallen ausgestrahlt und es wird die Zeit bis zum Rückkehr des Lichts gemessen. Aufgrund zwei bekannte und konstant bleibende Größen lässt sich diese Zeit \emph{$\Delta$t} als einen Maß für den Objektabstand \emph{d} verwenden. Die Lichtgeschwindigkeit \emph{c} ist bekannt und konstant sowie das Brechungsindex \emph{n} des Mediums, in dem der Laserstrahl durchläuft. Somit lässt sich der Objektabstand einfach errechnen (Gl. 2.1). \autocite[171]{Hering2018}

\begin{equation}
	d = \frac{c \cdot \Delta t}{2 \cdot n}
\end{equation}

Um Fehlmessungen zu verringern wird intern in dem Lasersensor mittels einer Stoppuhr und Signale werden Lichtpulse von weiter entfernten Objekten elektronisch ausgeblendet. Es wird bei dem Aussenden des Lichtes einen Startsignal ausgelöst und die Uhr gestartet. Sobald der Empfänger einen Lichtimpuls mit einer Mindestlichtintensität erhält, wird ein Stoppsignal gesendet und die Stoppuhr aufgehalten und ausgelesen. \autocite[171]{Hering2018}.

\subsubsection{Frequenzlaufzeitverfahren}

Ein ähnliches Verfahren zu dem Pulszeitverfahren ist das Phasen- oder Frequenzlaufzeitverfahren. Der Objektabstand \emph{d} wird durch Vermessen der Phasendifferenz \emph{$\Delta$ $\varphi$} zwischen dem ausgesendeten und eingegangenen Licht errechnet (Gl. 2.3). Integral für die Berechnung ist die Modulationswellenlänge \emph{$\lambda_m$} des Lichtpulses, welches mit der Periodendauer \emph{$T_m$} moduliert wird (Gl 2.2). \autocite[171]{Hering2018}

\begin{equation}
	\lambda_m = \frac{c \cdot T_m}{n}
\end{equation}

\begin{equation}
			d = \frac{\Delta \varphi}{2 \cdot \pi} \cdot \frac{\lambda_m}{2} + i \cdot \frac{\lambda_m}{2}, i=0,1,2,...
\end{equation}

Um sowohl möglichst großen Abstände eindeutig zu messen und als auch eine hohe Messgenauigkeit zu erzielen müssen mehrfache Modulationswellenlängen verwendet werden. Dies wird möglich, indem der Puls eine konstante und zyklische Frequenzänderung zwischen einem Minimum- und Maximumwert durchlebt. \autocite[172-173]{Hering2018}

\subsubsection{Interferometrisches Verfahren}

Bei dem interferometrischen Messverfahren wird das Laserlicht mittels eines teildurchlässigen Spiegels in zwei Strahlen aufgeteilt. Diese Strahlen kehren voneinander ab und legen unterschiedliche Wege zurück. Ein Teilbündel wird innerhalb des Gerätes mittels eines anderen Spiegels zurückreflektiert, während das andere Teilbündel mit der Objektoberfläche begegnet. Nach dem Rückkehr beider Teilbündel werden sie zusammengeführt und die Lichtintensität nach der Überlagerung ermessen. Bei einer Änderung des Objektabstandes ändert sich die Lichtintensität proportional. 

\subsubsection{Triangulationsverfahren}

Das Triangulationsverfahren zur Abstandsermittlung ist ein geometrisches Verhalten. Hierbei spielen zeitabhängige Größen wie die Lichtgeschwindigkeit keine Rolle. Ein dichtes Lichtbündel wird von dem Lasersensor gesendet und trifft die Oberfläche des zu detektierenden Objektes. Nach der Reflektion, die von der Oberflächenkontur und Eigenschaften abhängt, wird der Strahl durch den Sensor empfangen und in die Detektionsebene abgebildet. Auf diese Ebene entsteht eine Lichtverteilung mit einem Schwerpunkt \emph{x}, dessen Position mit einer Änderung der Objektdistanz \emph{d} auch verschiebt. Da der Basisabstand \emph{B} zwischen dem Sender und Empfänger des Lasersensors sowie der Abstand \emph{F} zwischen der Empfänger-Optik und Detektionsebene bekannt ist, kann mit dem Vermessen des Schwerpunktes \emph{x} auch der Objektabstand \emph{d} errechnet werden (Gl. 2.4). \autocite[170]{Hering2018}

\begin{equation}
	x = \frac{B \cdot F}{d}
\end{equation}

Optoelektronischen Abstandssensoren auf Basis des Triangulations- sowie Laufzeitverfahrens finden aufgrund ihrer einfacherer Technik und gute Reichweite Anwendung in der Abstandmessung, Konturbestimmung, Dickenbestimmung sowie die Füllstandskontrolle. Interferometrische Abstandssensoren können aufgrund ihrer intrinsischen Eigenschaften sehr hohe Auflösungen realisieren, welches sie zur Abstandsmessung in dem Mikrometerbereich geeignet macht. Diese Art der Lasersensoren finden bei der Feineinstellung von Maschinen und Anlagen sowie in der Oberflächenqualitätskontrolle eine Anwendung. \autocite[174-177]{Hering2018}

Mit dem Einsatz Lasersensoren in Industrierobotern können diese Geräte die Fähigkeit verleiht werden, mit einer hohen Genauigkeit Werkstücke zu erkennen und innerhalb ihren eigenen Koordinatensysteme zu lokalisieren und abzubilden. Allerdings ist für das Zusammenspiel zwischen dem Roboter und dem Lasersensor die richtige Auslegung von Kommunikationswegen zwischen den jeweiligen einzelnen Komponenten sehr wichtig.

\section{Das Robot Operating System (ROS)} \label{sec:ROS}
Das Robot Operating System (ROS) ist ein Sammelwerk von Werkzeugen, Software-Bibliotheken und Normen, die zusammen einen Rahmenwerk für die Erstellung von Robotersoftware bilden. Der Hintergrund hinter ROS ist es, komplexe aber robuste Lösungen für Roboteraufgaben plattformübergreifend zu ermöglichen. Auch simpel erscheinende Probleme umfassen die Lösung sehr komplexer Teilaufgaben wie Planungs-, Koordinierungs- und Erkennungsaufgaben, die auf einander Einflüsse haben, mit einander zusammenarbeiten sollen und jeweils komplexe Teilprobleme mit sich bringen. Zwecks der Simplifizierung und Förderung von Kollaboration ermöglicht ROS die separate Entwicklung von Lösungen der Teilaufgaben und übernimmt die Aufgabe der Kollaboration der jeweiligen Lösungen. \autocite[3]{QuigleyROS2015}

\subsection{Philosophie hinter der Entwicklung}

ROS wurde auf Basis einiger Grundprinzipien geschaffen, die zur Effizienz, Anpassbarkeit, Flexibilität und Wiederverwendbarkeit eines Systems beitragen. 

\subsubsection{Peer-to-Peer}
	ROS-Systeme bilden sich aus verschiedenen kleinen Teilprogramme zusammen, die auf die Lösung einzige Probleme spezialisiert sind. Diese Teilprogramme kommunizieren mittels eines direkten Nachrichtenaustausches miteinander, ohne einer zentralen Routing. Dank dieses \emph{peer-to-peer} Prinzips können ROS-Systeme auch mit enormen Datenmengen umgehen. \autocite[3]{QuigleyROS2015} 
\subsubsection{Tools-basiert}
	ROS besitzt auch keine kanonische, übergeordnete Entwicklungs- und Laufzeitumgebung. Inspiriert durch das \emph{Tools-basierte} Prinzip, werden die jeweiligen Aufgaben einer solchen Umgebung durch kleinen, eigenständigen Programme oder Werkzeuge erledigt. Somit ist nicht nur eine Weiterentwicklung oder Verbesserung des Programms einfach, sondern auch das Ersetzen oder die Überarbeitung. \autocite[3]{QuigleyROS2015} 
\subsubsection{Multilingual}
	Die Entwickler von ROS haben die Stärken und Schwächen unterschiedlicher Programmiersprachen erkannt, wodurch ihren situationsabhängigen Einsatz gewisse Vorteile bringen können. Deswegen wurde ein \emph{multilingualer} Ansatz übernommen, sodass ROS-Module in verschiedenen unterstützten Hochsprachen geschrieben werden können. Auch neue Programmiersprachen können jederzeit hinzugefügt werden, solange Software-Pakete für die Interpretation und Implementierung der ROS-Kommunikationsnormen auch bereitgestellt werden. \autocite[3]{QuigleyROS2015} 
\subsubsection{Schlank}
	Zur Realisierung einer besseren Wiederverwendbarkeit der ROS-Module werden Entwickler empfohlen einen \emph{schlanken} Ansatz zu verwenden, indem eigenständige Module ohne den Bedarf einer ROS-Umgebung entwickelt werden. Zur Integration mit ROS sollen die bestehenden Module mit Methoden zur Kommunikation mit der ROS-Umgebung und anderen ROS-Modulen erweitert werden. \autocite[3]{QuigleyROS2015} 
\subsubsection{Quelloffen}
	Letztlich wird ROS unter einem \emph{quelloffenen} BSD Lizenz veröffentlicht, sodass kommerzielle Systeme mit proprietären Modulen sowie quelloffene akademische und Hobbyprojekte ohne Hindernisse entwickelt werden können. \autocite[4]{QuigleyROS2015}
	
\subsection{ROS-Bausteine}
Zur besseren Verständnis seiner Funktionsweise lohnt sich ein tieferer Blick in den Aufbau von ROS.

\subsubsection{ROS-Nodes}
Ein ROS-System setzt sich aus verschiedenen unabhängigen Programmen oder Modulen zusammen. Diese Module erfüllen roboter-spezifische Aufgaben der Navigation, Computervision, Greifen usw. Graphisch können diese Module als Knoten dargestellt werden. Deswegen werden sie auch als \emph{ROS-Nodes} genannt. ROS-Nodes können mit einander kommunizieren und Information in Form von Nachrichten austauschen. \autocite[9-10]{QuigleyROS2015} 

\subsubsection{ROS-Master}
Das ROS-Master, auch \emph{roscore} genannt, ist ein ROS-Node, deren Aufgabe die eines DNS-Servers ähnelt. Der ROS-Master koordiniert die Kommunikation zwischen einzelnen Nodes, indem es einen Sammelwerk von Information über aktiven ROS-Nodes erstellt. Dies enthält unter anderem den Namen des aktiven ROS-Nodes sowie dessen Adresse. Die Adresse des ROS-Masters wird durch den Benutzer vorgegeben und somit allen Nodes verfügbar. Sobald ein ROS-Node gestartet wird, nutzt sie diese Adresse, um sich bei dem ROS-Master zu registrieren. Dieser Node gibt dem ROS-Master bekannt, mit welchen anderen ROS-Nodes sie kommunizieren will und die Art der Nachrichten, die sie sendet. Der ROS-Master meldet sich mit der Adresse des relevanten ROS-Nodes zurück. Somit können ROS-Nodes mit anderen Nodes Kontakt aufnehmen. Sobald der Kontakt zwischen den Nodes entsteht, tretet der ROS-Master zurück. Änderungen in aktiven Nodes werden auch dem ROS-Master bekannt gegeben. In einem ROS-System darf es nur eine Instanz des ROS-Masters laufen.  \autocite[11-12]{QuigleyROS2015} \autocite[6-7]{NewmanWyattS2018ASAt} \autocite[14]{LentinMasteringROS2021}

\subsubsection{ROS-Messages}
ROS-Messages sind Nachrichten, die während der Kommunikation zwischen Nodes ausgetauscht werden. Ein ROS-Message wird durch den Sender in Form einer serialisierten Datenpaket gesendet, welches mit einem entsprechenden Schlüssel interpretiert und rekonstruiert werden kann. ROS-Messages können standardmäßig Information als Zahlen in Form von (vorzeichenlosen) Integern und Gleitkommazahlen, als Text in Form von Strings und als boolesche Werte enthalten. Allerdings ist es auch möglich, benutzerdefinierte ROS-Messages zu erstellen. Das ermöglicht beispielsweise die Übermittlung von translatorischen und rotatorischen Geschwindigkeiten eines Roboters in Form drei dimensionale Matrizen oder Arrays. \autocite[6]{NewmanWyattS2018ASAt}

\subsubsection{ROS-Topics}
ROS-Topics sind speziell ausgelegte Wege, die analog zu Datenbussen funktionieren. Innerhalb eines ROS-Topics befinden sich ROS-Messages zu einem spezifischen Thema, beispielsweise die Temperaturwerte aus einem Temperatursensor. ROS-Topics unterscheiden sich von Datenbussen, indem sie eine entkoppelte Kommunikation auf Basis des TCP/IP zwischen ROS-Nodes ermöglichen. ROS-Messages werden zu einem Topic durch einen ROS-Node veröffentlicht und können durch anderen ROS-Nodes empfangen werden, indem sie dem ROS-Topic abonnieren. Somit bleiben diese ROS-Nodes anonym. Aufgrund der entkoppelten Kommunikation können mehrere ROS-Nodes gleichzeitig zu dem ROS-Topic veröffentlichen sowie dem abonnieren. \autocite[14]{LentinMasteringROS2021} \autocite[23]{LentinMasteringROS2018} \autocite[6]{NewmanWyattS2018ASAt}

\subsubsection{Publisher und Subscriber}
Publishers und Subscribers sind ROS-Nodes, die eine unidirektionale Kommunikation über ROS-Topics ausführen. Publishers sind meistens ROS-Nodes, die unter allem Daten erfassen und sammeln. Subscribers sind ROS-Nodes, die diese Daten für die Ausführung verschiedener Aufgaben benötigen. Die Daten des Publishers werden als ROS-Messages strukturiert und in einem kontinuierlichen Datenstrom zu einem ROS-Topic veröffentlicht und stehen dem Subscriber zur Verfügung. Um diese Daten zu erhalten, muss der Subscriber dem Topic abonnieren.

\subsubsection{ROS-Services}
Die Kommunikation über ROS-Topics erfolgt unidirektional, die für den Einsatz bestimmter Zwecke ungeeignet ist. Manche Roboteraufgaben erfordern eine Kommunikation basiert auf dem Anfrage/Antwort Modell. Dies wird durch den Einsatz von ROS-Services ermöglicht. ROS-Services setzen sich aus zwei ROS-Nodes zusammen - einen Server und einen Client. Der Client kann dem Server die Ausführung eines Dienstes, zwar eine Aufgabe, anfordern. Nachdem der Server diesen Dienst ausgeführt hat, werden dessen Ergebnisse als eine Antwort an dem Client zurückgesendet. Die Anfrage und Antwort werden beide als ROS-Messages gesendet, die Information eines bestimmten Datentypes oder Formates enthalten. \autocite[24]{LentinMasteringROS2018}. Ein Beispiel für die Anwendung eines ROS-Services in der Robotik sind Bestückungsaufgaben. Der Greifer und ein optischer Sensor eines Roboters werden in einem ROS-System als ROS-Nodes abgebildet. Der Sensor-Node, in Rolle eines Clients, erkennt ein Werkstück. Der Client fordert das Aufheben des Werkstückes an, indem er eine ROS-Message mit den Werkstückkoordinaten dem Greifer-Node sendet, der die Rolle des Servers spielt. Nachdem der Greifer das Werkstück aufgehoben hat, schickt er dem Client eine Antwort in Form einer ROS-Message zurück. Da ROS-Services eine synchrone Kommunikationsform anwenden, führt der Client bis zur Erhaltung einer Antwort keine weitere Schritte aus. Eine asynchrone Form der Kommunikation auf Basis der ROS-Services kann mittels ROS-Actions realisiert werden.

Dank der quelloffenen Philosophie und die Möglichkeit der Kollaboration in ROS sind viele quelloffene Module - sogenannte ROS-Packages - entstanden, die sehr geschickte und umfangreiche Lösungen zu verschiedenen Roboter spezifische Probleme wie die Simulation, Koordinatentransformation und Visualisierung präsentieren. Einige dieser ROS-Packages sind in vielen Entwicklungen Roboterlösungen ein integraler Bestandteil geworden.

\subsection{Simulation mit Gazebo}
Die Simulation in der Robotik dient hauptsächlich zwei Zwecke - zur Verständnis des Systemverhaltens und zur Analyse der Systemleistung \autocite[110]{DatteriSchiaffonati2019}. Beispielsweise kann zur Verständnis der Roboterbewegungsmöglichkeiten, das Robotermodell in einer Simulationsumgebung modelliert werden. Wichtig ist die Festlegung der Knoten oder Gelenke sowie der Festkörper des Modells. In der Simulation können die unterschiedlichen Bewegungsmöglichkeiten geprüft sowie Kollisionsgefahren untersucht werden. \autocite[11]{LentinMasteringROS2018}. Zur Überprüfung der Sicherheitskriterien von kollaborativen Industrierobotern, sogenannten Cobots, wird Simulation als eine kostengünstigere Alternative zu physischen Testverläufe benutzt. Da Cobots Aufgaben in Zusammenarbeit mit Menschen erledigen und ihrer Arbeitsraum nicht abgesichert ist, müssen sie einige Sicherheitskriterien erfüllen, damit sie nicht den Menschen verletzen. Eine Simulation ermöglicht die Überprüfung der Sicherheitsgewährleistung, ohne dass teure Cobots physikalisch beschädigt werden oder die Menschen in Gefahr kommen. Ein geometrisches Modell des Cobots kann in der Simulationsumgebung geladen werden und sein Verhalten bei dem Auftreten von simulierten Störungen und risikobehaftete Situationen beobachtet und bewertet werden. \autocite{ore_vemula_hanson_wiktorsson_fagerström_2019}

Simulatoren können auf Basis der Anzahl an abgebildeten Dimensionen unterteilt werden - 2D- und 3D-Simulatoren. 2D-Simulatoren sind sehr rechen-effizient und eignen sich insbesondere für die Simulation planarer Navigation. Hierbei wird die Fläche eines Raumes zweidimensional abgebildet mit Verzicht auf die Höhendimension. Allerdings ist für vielen Aufgaben, beispielsweise die Manipulationsaufgaben eines Industrieroboters, eine dreidimensionale  Simulation unabdingbar. \emph{Gazebo} ist eine quelloffene Software für Physiksimulation, die häufig mit ROS benutzt wird. Innerhalb der Simulationsumgebung werden physischen Zusammenhänge der Festkörperphysik imitiert. Eine besondere Eigenschaft dieser Software ist, dass sie eine Vielzahl an verschiedenen Sensoren realistisch simulieren kann. Wichtig für eine Simulation mit Gazebo ist auch eine Modellierung der Umgebung des Roboters - das sogenannte Weltmodell. Dieses Modell enthält statische Gegenstände wie Gebäude, Gelände und Hindernisse sowie dynamische Objekte wie greifbare und bewegbare Werkstücke, weitere Roboter und andere bewegliche Gegenstände. Das \emph{gazebo \textunderscore ros} Modul ermöglicht eine bidirektionale Kommunikation zwischen der Gazebo-Software und ROS. Somit können die künstlichen Sensordaten sowie Physikdaten aus der Simulationsumgebung in ROS einfließen sowie Aktuator-Befehle von ROS zu Gazebo zurückgeleitet werden. Im Falle der korrekten Konfiguration könnte ein ROS-System in der Simulationsumgebung identisch zu dem realen Roboter laufen, und das ohne großen Anpassungen. \autocite[95-96]{QuigleyROS2015} \autocite[114]{NewmanWyattS2018ASAt}

\subsection{Koordinatentransformation mit Transform-Library} \label{ssec:tf2}
Wie in Kapitel 2.1 Festgestellt, besteht ein Roboter aus einer Aneinanderreihung von Armteile, die miteinander mit Gelenken verbunden sind. Aufgrund der Freiheitsgrade der einzelnen Gelenke vervielfachen sich Bewegungsmöglichkeiten des Roboters. Um die Gesamtbewegung des Roboters für die Robotersteuerung auszurechnen, wird jedes Arm sein eigenes Koordinatensystem verleiht. Mit Hilfe mehrerer Koordinatentransformationen wird aus der Stellung der einzelnen Arme, beginnend mit dem ersten, die Stellung des Endeffektors berechnet. Dieses nennt sich die Vorwärtstransformation. Bei der Rückwertskinematik handelt es sich um die Berechnung der Gelenkstellungen der jeweiligen Armteile aus der Stellung des Endeffektors. Diese Berechnung ist komplexer, schwieriger und rechenintensiver als die Vorwärtstransformation. Entscheidend für die Trajektorienplanung und Bewegung in der Robotik ist die Berechnung der Rückwertskinematik in Echtzeit. \autocite[65-66]{maier2022grundlagen}

Da Koordinatentransformationen in der Robotik sehr gebräuchlich sind, bietet ROS ein mächtiges Modul namens Transform-Library (tf2). Dieses Modul ist die zweite Generation und erweitert die erste Version dieses Moduls, \emph{tf}. \autocite{foote_tf2_wiki_2019}. Die grundsätzliche Funktion bleibt allerdings unverändert. Das tf2 Modul ermöglicht es, auf die Transformation zwischen zwei Koordinatensysteme des Roboters jederzeit zuzugreifen. Diese berechneten Werte werden im Form einer ROS-Message zu einem ROS-Topic regelmäßig veröffentlicht. Diese Message wird ein \emph{TFMessage} genannt. Die Transformation wird in zwei Komponenten aufgeteilt - die Translationswerte als einen dreidimensionalen Vektor und die Orientierungswerte als eine Quaternion. Außerdem enthält das TFMessage einen Zeitstempel sowie Angaben zu dem Referenz- und transformierten Koordinatensystem. Der Zeitstempel ist für statische sowie dynamische Transformationen wichtig. Eine statische Transformation zwischen zwei Koordinatensysteme bleibt immer gleich, bis deren Ausrichtung zu einander explizit geändert wird. Der Zeitstempel einer statischen Transformation gibt an, ab wann eine Transformation als gültig gilt. Sie bleibt valid, bis sie aktualisiert wird. Eine dynamische Transformation bezeichnet eine ständig wechselnde Ausrichtung zweier Koordinatensysteme zu einander. Der Zeitstempel dieser Art der Transformation gibt an, an welchem Zeitpunkt sie gültig ist. Die Simulationssoftware Gazebo bietet hierzu ein paar hilfreiche Funktionalitäten. Mit der richtigen Aufstellung eines ROS-Systems ahmt Gazebo die Bewegung des Roboters zeitgleich nach. Diese Gelegenheit wahrnehmend werden die aktuellen Orientierungen und Winkeln der Roboter-Gelenke und deren Koordinatensysteme zu einem ROS-Topic \emph{joint\textunderscore states} veröffentlicht. Mittels der Information aus diesem ROS-Topic lassen sich diverse Koordinatentransformationen berechnen. Da diese Berechnungen sehr häufig gebraucht werden, wird durch ROS ein Modul namens \emph{robot \textunderscore state\textunderscore publisher} angeboten, welches die Winkeln der Gelenke annimmt und aus diesen die drei-dimensionale Pose der Roboterarme berechnet. Daraus werden dann die Koordinaten"-transformationen automatisch errechnet und publiziert. \autocite[156-160]{NewmanWyattS2018ASAt}

Das tf2 Modul baut auf die bestehende Publisher/Subscriber Kommunikation in ROS auf. Diese Kommunikationsarchitektur wurde für die Design- und Funktionsanforderungen des tf2 Moduls erweitert. Es besteht aus \emph{Broadcasters} und \emph{Listeners}. Ein Broadcaster dient dem Zweck des Übertragens. Er übertragt mit einer regelmäßigen Frequenz Aktualisierungen über die Transformationen. Eine Änderung der Transformation sei dabei nicht vorausgesetzt, sondern gilt das Vergehen der Zeit auch als eine Aktualisierung. Ein Listener sammelt die, durch den Broadcaster publizierten Werte in einer, nach der Zeit sortierten Liste und speichert sie. Mittels einer Abfrage an dem Listener können Auskünfte über spezifischen Transformationen zu einem Zeitpunkt gewonnen werden. Da Transformationen diskret publiziert werden, besteht die Möglichkeit, dass der Listener keine Transformationsdaten zu einem Zeitpunkt besitzt. In dem Fall wird mittels einer sphärischen linearen Interpolation (SLERP) die Transformation interpoliert. Dies trägt zusätzlich zu der Robustheit dieses Moduls bei. \autocite{Foote_tfPaper_2013} \autocite[161-167]{NewmanWyattS2018ASAt}

Das tf2 Modul vereinfacht das Berechnen, Speichern und Organisieren mehrere Koordinatentransformationen, allerdings ist es für einige Aufgaben der Robotik ungeeignet. Tf2 lässt sich für die Roboterbahnplanung schlecht einsetzen, da es zukünftige hypothetische kinematische Transformationen nicht vorausberechnen kann. Ein weiteres Nachteil dieses Moduls ist es, dass die Transformationen zeitverzögert gespeichert und bereitgestellt werden. Dies macht es für die Anwendung in zeitkritischen Reglungen wie Kraftreglungen ungeeignet. In diesen Fällen wird es empfohlen, die kinematischen Transformationen programm-intern auszurechnen. \autocite[174-175]{NewmanWyattS2018ASAt}

\subsection{Visualisierungen mit Rviz}
\emph{Rviz}, eine Abkürzung von \emph{ROS Visualization}, ist eine Umgebung für die dreidimensionale Visualisierung von reellen oder virtuellen Sensoren, Robotern und ihren Bewegungen sowie Aktionen. Dieses Modul ermöglicht das Überwachen aller Prozesse aus der Sichtweise des Roboters, indem Dateneingänge aus Sensoren visualisiert werden. Beispielsweise werden Bilddaten aus einer Kamera als Bilder oder Videos dargestellt sowie Tiefenwerte aus einem Lasersensor als eine dreidimensionale Punktwolke angezeigt werden. Auch zurückgelegte Bahnen und Bewegungen des Roboters können mittels visueller Anhaltspunkte verfolgt werden. Die Rviz Erweiterung der später vorgestellten MoveIt-Bibliothek bietet auch einige Funktionalitäten für die Fernsteuerung des Roboters an. Das Modul lässt sich auch sehr gut in einem ROS-System und mit anderen ROS-Modulen integrieren. Sensordaten sowie die Pose des Roboters erhält Rviz, indem es den entsprechenden ROS-Topics abonniert. \autocite[177-180]{NewmanWyattS2018ASAt} \autocite[126-127]{QuigleyROS2015} \autocite[243]{LentinMasteringROS2018} \autocite[36-40]{FairchildROS2017}
\begin{figure}[h]
	\includegraphics[width = \textwidth]{Abbildungen/Rviz_vis.png}
	\centering
	\caption{Visualisierung eines Roboters in Rviz}
	\label{fig:Rviz_vis}
\end{figure}

In Abbildung~\ref{fig:Rviz_vis} zu sehen ist ein in Rviz visualisierter Schweißroboter. In dieser Umgebung wird die aktuelle Pose des Roboters während einer Schweißaufgabe illustriert. Hierbei werden nicht nur die Transformationen und Gelenkwinkeln des Roboters eingelesen, sondern auch die Sensordaten aus einem Triangulation-Lasersensor. Diese Daten können sowohl aus dem physischen Roboter und Sensor entstehen, als auch in Gazebo emuliert werden. In Abbildung~\ref{fig:Gazebo_vis} wird der Roboter zusammen mit dem Werkstück (in Orange dargestellt) modelliert. Der Laser aus dem Lasersensor ist ein Linienlaser und wird als die blaue Projizierung abgebildet. In Rviz werden die Tiefenwerte aus dem Sensor ausgewertet und farblich kodiert. Die aktuelle Projizierung des Lasers auf dem Werkstück wird als einen roten Streifen nachgebildet. Durch die Bewegung des Roboters in der dritten Dimension des Lasersensors werden dreidimensionale Punktewolken erstellt. Die Kontur des Werkstückes wird aus dieser Punktewolken interpretiert und als das gelbe Profil nachgezeichnet. Als letztes ist die, durch den Endeffektor beziehungsweise die Schweißpistole zurückgelegte Bahn als die grüne Linie reproduziert worden. Mit Hilfe dieser Animationen kann die Roboterbewegung aus einer sicheren Entfernung überwacht werden sowie die Sensordaten auf Fehlern und Störungen überprüft werden.

\begin{figure}[t]
		\includegraphics[width = \textwidth]{Abbildungen/Gazebo_vis.png}
		\centering
		\caption{Visualisierung des gleichen Roboters in Gazebo}
		\label{fig:Gazebo_vis}
\end{figure}

\subsection{Bahnplanung mit MoveIt}
In Abschnitt~\ref{ssec:tf2} wurde die Signifikanz der Vorwärts- und Rückwertstransformation angedeutet. Diese sind für die Berechnung der direkten und inversen Kinematik des Roboters sehr wichtig \autocite[41-118]{MareczekRobKin2020}. Das tf2 Modul wurde für die Berechnung Koordinatentransformationen spezialisiert, allerdings eignet sich das Modul nicht für die Planung und Berechnung der Kinematik. Für die Erfüllung dieser Aufgaben wurde \emph{MoveIt} konzipiert. Das ist ein ROS-Modul für die Berechnung inverser Kinematik, Bewegungs- und Bahnplanung, Kollisionsprüfung und weiteres mittels unterschiedlicher Ansätze wie stochastische Algorithmen, deterministische Bahnplaner und echtzeitfähige, lokale Planungsmodule. \autocite[160]{GandhinathanROSProjects2019}

Das wichtigste Element des MoveIt-Moduls ist der \emph{move \textunderscore group} ROS-Node. Dieser Node erfüllt die Aufgabe des Integrators, indem es alle anderen Komponenten und Funktionalitäten des MoveIt Moduls zusammen bündelt. Dem Benutzer wird die Interaktion mit diesem Modul mittels ROS-Services und ROS-Actions ermöglicht. ROS-Actions sind eine Erweiterung der ROS-Services und ermöglichen den asynchronen Aufruf von ROS-Services sowie, unter anderem, den Abbruch eines Aufrufs \autocite[61]{QuigleyROS2015}. Den Aufruf dieser Services und Actions könnte mittels eines Python-Skriptes, C++ Programms oder der graphischen Benutzeroberfläche von Rviz gesteuert werden. Details über die Gestaltung und Form des Roboter erhält der \textit{move \textunderscore group} Node über ein 3D-Modell des Roboters. \autocite[161]{GandhinathanROSProjects2019}

Allerdings reicht nur ein Modell des Roboters für eine online Bahnplanung nicht aus. Andere Details besorgt sich MoveIt von dem Roboter mittles der Publisher/Subscriber Kommunikation und ROS-Topics sowie ROS-Actions. Information über die Gelenkstellungen wird per Abonnement zu dem \textit{/joint\textunderscore states} ROS-Topic erhalten. MoveIt überwacht die einzelnen Transformationen des Roboters mit Abfragen zu dem tf2 Modul um aktuelle Information über die Roboterpose zu erhalten. Diese Werte werden intern für die Berechnung der Kinematik verwendet. Das MoveIt Modul kann auch mit der Steuereinheit des Roboters über ROS-Actions interagieren. Der Steuerung des Roboters wird über diese Schnittstelle Befehle für die Bewegung der Gelenke gegeben. Um einen Überblick über die Stellung des Roboters und die Welt zu behalten erstellt MoveIt eine virtuelle Szene, überwacht diese und benutzt sie für die Planung. \autocite{moveit_concepts_2021}

\subsection{ROS für Industrieroboter} \label{ssec:ros_industrial}
ROS-Module erweitern und bereichern die Fähigkeiten von ROS und realisieren eine deutliche Erleichterung des Aufstellens eines ROS-basierten Robotersystems. Wie es aus dem Roboter-System der Abbildung~\ref{fig:Rviz_vis} und Abbildung~\ref{fig:Gazebo_vis} zu entnehmen ist, könnten ROS-basierte Roboter unter Anwendung dieser Module auch industrielle Fertigungsaufgaben erledigen. Um diesen Konzept zu realisieren und ROS für Industrieroboter anwendbar zu machen, wurde \emph{ROS-Industrial} vorgestellt. Dieses Konzept bringt mit sich einige Ziele. Es soll die Entstehung einer Gemeinschaft bestehend aus Akademiker, Forscher und Experten für Industrieroboter unterstützen. Die fortgeschrittenen Fähigkeiten von ROS sollen möglichst gut mit existierenden Industrietechnologien harmonisiert werden und somit ein robustes und zuverlässiges Software-Paket für Industrieforschung sowie Industrieanwendungen anbieten. Grundsätzlich soll es ROS dazu unterstützen, einen Industriestandard für die Robotik zu werden. \autocite[476]{LentinMasteringROS2018}. Die meisten gängigen Roboterherstellern bieten mittlerweile ROS-Schnittstellen an. Diese befinden sich allerdings in unterschiedlich weiten Entwicklungs- und Wartungszuständen.

Die Theorie hinter der Funktionsweise eines Roboters, der automatisierten Schweißverfahren, optischen Sensoren und ROS dient dem Aufbau von Grundlagen, die zur Verständnis der Funktionsweise des kollaborierenden Schweißroboters des Fraunhofer-IPA notwendig sind. 

	%Struktur
% -Festlegung von Software-Kriterien
% -Auswahl des Verfahrens
% -Reproduktion des Verfahrens aus der Literatur
% 	-Kantenerkennung
% 	-Segmentierung
%	-Probleme & besondere Erkenntnisse
% -Erweiterung des Verfahrens
%	-Grundlegende Änderungen des Verfahrens
%	-False edge removal
%	-Point marking and cloud filtering
%	-misc.


\chapter{Entwicklungsprozess der Software}
Die Rücksichtnahme des Einsatzzwecks bei der Design und Entwurf des Verfahrens sowie die Entwicklung der Software war erforderlich, um die gewünschte Funktionalitäten gewährleisten zu können. Das Verfahren soll Kanten und Geometrien nicht nur in vollständig generierten Punktwolken erkennen, sondern auch in unvollständige Punktwolken, die iterativ wachsen. Hierbei wird ein Laserliniensensor eine Kante eines Werkstücks oder Objektes entlang geführt und somit sequentiell abgetastet. Deswegen wird die räumliche Struktur des Objektes nicht in einer einzigen Aufnahme abgebildet, sondern durch mehrere kleine Einzelaufnahmen. Der intelligente Schweißroboter, der durch das Fraunhofer Institut für Produktions- und Automatisierungstechnik entwickelt wird, verwendet ein solches Verfahren zum Scannen eines Werkstückes und zur Erkennung Schweißnähte \autocite[39]{savla_intelligente_2022}. Mittels eines Lasersensors wird die Oberfläche des Werkstückes dreidimensional abgebildet. Aktuell wird eine Schweißkegelnaht durch die Erkennung der Schnittlinie zwei Ebenen markiert, die mittels RANSAC-Algorithmen auf die Punktwolke des Werkstückes gefittet werden. Dieses Verfahren zur Erkennung der Schweißnaht bietet allerdings kaum detaillierte Informationen über die Geometrie des Werkstückes an.\autocite[39-52]{savla_intelligente_2022}. Das, in dieser Arbeit entwickelte Verfahren soll das bestehende Verfahren ersetzen und somit seine Limitationen überwinden.

\begin{figure}[h]
	\includegraphics[width = \textwidth]{Abbildungen/collage.jpg}
	\centering
	\caption{Der Laserliniensensor} 
\end{figure}

\section{Vorbereitungen}
\subsection{Software-Voraussetzungen}\label{soft_voraus}
Bei der Auswahl eines geeigneten Verfahrens zur Detektierung Kanten in einer Punktwolke wurden einige Voraussetzungen festgelegt. Die Methode sollte in der Lage sein, nicht nur Außenkanten zu erkennen, sondern auch Innenkanten beziehungsweise Faltungen. Neben dem originellen Einsatzzweck sollte das Verfahren möglichst breit anwendbar sein und eine hohe Modularität aufweisen. Die Funktionen der Kantendetektierung und Punktesegmentierung sollten unabhängig von einander aufrufbar gestaltet werden, um dem Benutzer eine möglichst hohe Flexibilität anzubieten. Die Kantenerkennung sollte performant erfolgen und Punktwolken innerhalb eines praktischen Zeitraums verarbeiten. Letztlich soll das Programm in dem bestehenden Programmpaket des Schweißroboters integrierbar sein. Die Hardwarebeschleunigung des Verfahrens mittels eines Grafikprozessors wurde ausgeschlossen, da ihrer Verwendung mit dem Echtzeitkernels des Programmpakets zur Konflikte führt. 

\subsection{Auswahl eines Verfahrens}
Eine Literatursuche nach Verfahren zur adaptiven Erkennung von Kanten in wachsenden 3D Punktwolken für den Einsatzzweck ergab nichts. Die meisten Verfahren eigneten sich für die Kantenerkennung nur in vollständigen Punktwolken. Aus diesem Grund wurde die Entscheidung getroffen, ein vorhandenes Verfahren aus der Literatur zu wählen und es für den Einsatzzweck anzupassen. Drei unterschiedlichen Verfahren nach \textcite{bazazian_edc-net_2021}, \textcite{himeur_pcednet_2021} und \textcite{rachmadi_road_2017} zeigten viel versprechende Ergebnisse. Allerdings wurden neuronale Netze in dieser Verfahren verwendet, welches zu zwei Problemen geführt hätte. Aufgrund der Funktionsweise neuronaler Netze wäre es schwierig gewesen, diese für den Einsatzzweck ohne eine umständliche Anpassung des neuronalen Netzes anzupassen. Das zweite Hindernis entsteht durch die Einschränkung bei der Verwendung von Grafikprozessoren. Diese Prozessoren hätten die Rechenzeit neuronaler Netze sehr stark verringert und die schnelle Performanz des Verfahrens gewährleistet \autocite[625]{luo_artificial_2005}. Das numerische Verfahren nach \textcite{choi_rgb-d_2013} war auch für den Einsatzzweck ungeeignet, da es als Eingangsparameter eine RGB-D Datei erfordert. Somit wäre das Verfahren nur für eine Anwendung auf organisierten, gefärbten Punktwolken eingeschränkt. Es wurden zwei weitere Verfahren gefunden, die sich zur Erkennung Kanten in organisierten sowie unorganisierten Punktwolken eignen würden. \textcite{mineo_novel_2019} stellten eine numerische Methoden vor, welche zu einer hohen Genauigkeit Kanten erkennen konnte. Allerdings wurden keine Angaben über die Erkennung Innenkanten in dieser Arbeit gemacht. \textcite{ni_edge_2016} schlagen im Gegensatz eine Methode namens AGPN vor, die nicht nur Außen- sowie Innenkanten und Faltungen erkennt, sondern die erkannten Randpunkte zusammen clustert, um Kannten voneinander zu trennen. Diese Studie präsentierte ein Verfahren mit einer hohen Genauigkeit sowie eine Möglichkeit, die Randpunkte sinnvoll zusammen zu gruppieren. Aus diesem Grund wurde dieses Verfahren als Grundlage für das adaptive Verfahren dieser Arbeit gewählt.

\section{Reproduktion des AGPNs}
Bevor das Verfahren für den Einsatzzweck angepasst wurde, wurde es zuerst zwecks einer Überprüfung unverändert implementiert. Es sollte sichergestellt werden, dass das Verfahren für die Erkennung Innenkanten und potenzielle Schweißnähte geeignet ist. Da die Autoren das Quellcode ihres Verfahrens nicht öffentlich zugängig gemacht haben, musste das Programm händisch reproduziert werden. Die Reproduktion des Programms erfolgte in zwei Schritten - die Reproduktion des Verfahrens zur Kantenerkennung und dessen zur Kantensegmentierung. Obwohl andere Skriptsprachen wie Python und MATLAB hinsichtlich des Prototypings Vorteile anbieten, wurde das Programm in C++ wegen seiner besseren Leistungsfähigkeit implementiert \autocite{svensson_performance_2021}. Viele Funktionalitäten der PCL-Bibliothek \autocite{rusu_3d_2011} wurden auch zum Entwurf des Verfahrens verwendet.

\subsection{Verfahren zur Erkennung Randpunkte}
Während Randelemente in zweidimensionale Bilder als eine klare Definition haben, fehlt eine solche Definition für Randelemente und Kanten in 3D-Punktwolken. In diesem Verfahren wurden die geometrischen Eigenschaften einer Kollektion von Punkten zur Erkennung Randpunkte berücksichtigt. Randpunkte weisen eine besondere geometrische Eigenschaft auf - der Winkelabstand zwischen benachbarten Randpunkte ist im Vergleich zu anderen benachbarten Punkten deutlich größer. Faltungen stellen den Grenzbereich zwischen zwei angrenzenden Ebenen dar, deren Normale in unterschiedlichen Richtungen zeigen. Diese geometrischen Eigenschaften wurden zur Erkennung Randpunkte verwendet. \autocite[1-2]{ni_edge_2016}

Im folgenden wird das Verfahren zur Erkennung Randpunkte detaillierter erläutert. Für einen Punkt \textit{o} wurde eine Sammlung von \textit{K} benachbarten Punkten mittels eines kd-trees erstellt. Diese Sammlung wird als eine Nachbarschaft \textit{N\textsubscript{o}} referiert. Danach wurde mittels eines RANSAC-Algorithmus eine Ebene \textit{E\textsubscript{N}} auf diese Nachbarschaft gefittet, um Ausreißer herauszufiltern und zwei angrenzenden Flächen innerhalb der Nachbarschaft voneinander zu trennen. Danach wurde Überprüft, ober der Punkt \textit{o} auf der RANSAC-Ebene lag. Falls dieser Punkt ein Ausreißer der Ebene \textit{E\textsubscript{N}} war, wurde er nicht als einen Randpunkt markiert. Ansonsten wurden weiterhin die geometrischen Eigenschaften der Nachbarschaft überprüft. Abbildung \ref{RANSAC-Ebene} visualisiert die Trennung zwischen unterschiedlichen Flächen einer Punktwolke mittels des RANSAC-Verfahrens. 

\begin{figure}[t]
	\includegraphics[width=\textwidth]{Abbildungen/RANSAC-Ebene.png}
	\centering
	\caption{Eine lokale RANSAC-Ebene (rot dargestellt) neben anderen Oberflächen (blau dargestellt). In \textbf{a} sind drei Ebenen zu sehen, wobei in \textbf{b} nur zwei zu sehen sind. \autocite{ni_edge_2016}}
	\label{RANSAC-Ebene}
\end{figure} 

Im Falle, dass der Punkt \textit{o} ein Inlier war und zu der Ebene \textit{E\textsubscript{N\textsubscript{o}}} gehörte, fang die tatsächliche Überprüfung der geometrischen Eigenschaften der Nachbarschaft \textit{N\textsubscript{o}} an. Um die Ebenengleichung von \textit{E\textsubscript{N}} näherungsweise zu schätzen, wurde zuerst die Normale \textit{$\vec{n}$} der Ebene geschätzt. In einer effizienten Weise wurde die Ebenengleichung durch das RANSAC-Verfahren näherungsweise geschätzt. Diese Gleichung wurde weiterhin auf die RANSAC-Inliers optimiert und daraus die Normale \textins{$\vec{n}$} ermittelt. Danach erfolgte die Errechnung des Winkelabstands zwischen den jeweiligen Punkten von \textit{N\textsubscript{o}}. Hierfür wurden für die Ebene die jeweiligen Eigenvektoren \textit{E\textsubscript{N}} $\vec{u}$ und $\vec{v}$ aus der Normale $\vec{n}$ errechnet. Das fertige Verfahren von PCL zur Ausrechnung der Eigenvektoren lieferte ungenaue Ergebnisse. Stattdessen wurden zur Ermittelung \textit{$\vec{u}$} zwei zufällig gewählte Punkten aus der Inliers verwendet. Es wurde dabei sichergestellt, dass keiner der Punkten den Punkt \textit{o} entsprachen. Zur Errechnung des Winkelabstands wurden zuerst die Winkel aller Punkte der lokalen Ebene \textit{E\textsubscript{N}} errechnet. Mit dem Punkt \textit{o} als Ursprung wurde für jeden Punkt \textit{p\textsubscript{i}}, aus \textit{N\textsubscript{r}} Punkten, der Winkel \textit{$\theta_i$} zu einer Nulllinie errechnet. Danach wurde die Differenz zwischen zwei konsekutiver Punktwinkel $\theta_i$ und $\theta_{i+1}$ errechnet, welcher den Winkelabstand \textit{G\textsubscript{$\theta$}} zwischen zwei Punkten \textit{p\textsubscript{i}} und \textit{p\textsubscript{i+1}} betrug. Abbildung \ref{edge_boundary} zeigt, wie der Winkelabstand zwischen Punkten am Rand der Punktwolke aussieht.

\begin{figure}[h]
	\includegraphics[width=0.5\textwidth]{Abbildungen/angular_gap_boundary}
	\centering
	\caption{Der Winkelabstand \textit{G\textsubscript{$\theta$}} zwischen Punkten am Rand der Punktwolke. \textbf{\(a\)} zeigt ein interner Punkt \textit{o}und ein Nachbarpunkt \textit{p\textsubscript{i}}. Im Vergleich dazu zeigt \textbf{\(b\)} \textit{o} am Rand und den großen Winkelabstand \textit{G\textsubscript{$\theta$}} zwischen Punkte \textit{p\textsubscript{i}} und \textit{p\textsubscript{i + 1}}. \autocite{ni_edge_2016}}
	\label{edge_boundary}
\end{figure}

Auch die Erkennung von Punkten in Innen- und Außenkanten war durch diese Berechnungen möglich. Wie bereits erwähnt, wurden zwei angrenzenden Flächen mittels das RANSAC-Verfahren voneinander getrennt. Falls der Punkt \textit{o} auf der Schnittlinie beider Flächen sowie auf der lokalen RANSAC-Ebene \textit{E\textsubscript{N}} liegt, dann gehört es zum lokalen Rand der Ebene. Falls der Punkt \textit{o} auf der Schnittlinie beider Flächen liegt, aber nicht zu der RANSAC-Ebene gehört, wird es automatisch nicht als einen Randpunkt gemerkt. \textit{E\textsubscript{N}}. Die Abbildung \ref{edge_fold} zeigt, wie der Winkelabstand zwischen Punkten auf einer Schnittlinie zwischen zwei Flächen der Punktwolke aussieht. Die Errechnungen des Winkelabstands erfolgte nach den Gleichungen \ref{first_equation} - \ref{last_equation}.

\begin{figure}[h]
	\includegraphics[width=\textwidth]{Abbildungen/angular_gap_fold}
	\centering
	\caption{Der Winkelabstand zwischen Punkten auf einer Schnittlinie zwei angrenzender Flächen. \textbf{\(a\)} zeigt ein interner Punkt \textit{o} der RANSAC-Ebene. \textbf{\(b\)} zeigt \textit{o} am lokalen Rand der RANSAC-Ebene und den Winkelabstand \textit{G\textsubscript{$\theta$}} zwischen Punkte \textit{p\textsubscript{i}} und \textit{p\textsubscript{i + 1}}. \textbf{\(c\)} zeigt \textit{o} als ein Ausreißer der RANSAC-Ebene. \autocite{ni_edge_2016}}
	\label{edge_fold}
\end{figure}

\begin{equation}
\label{first_equation}
d_i^u = \vec{{op}_i} \cdot \vec{u}
\end{equation}
\begin{equation}
d_i^v = \vec{{op}_i} \cdot \vec{v}
\end{equation}
\begin{equation}
\theta_i = \arctan{\frac{d_i^u}{d_i^v}}
\end{equation}
\begin{equation}
G_\theta = \max(\theta_{i + 1} - \theta_i), i \in \{1, \ldots, N_r\},
\label{last_equation}
\end{equation}



Zur Korrekten Ausrechnung des maximalen Winkelabstands einer Nachbarschaft \textit{G\textsubscript{$\theta$}} war eine aufsteigende Sortierung der Winkel \textit{$\theta_i$} notwendig. Diese Sortierung entsprach eine Sortierung der Punkte \textit{p\textsubscript{i}} nach ihrer aufsteigenden polaren Entfernung von der Nulllinie. Die Abbildung \ref{} dient zur Visualisierung der Methode zur Ausrechnung von $\theta_i$.

%	\chapter{Technischer Stand des Robotersystems}
In diesem Kapitel wird der, in Abschnitt~\ref{sec:motivation} referenzierter kollaborierender Schweißrobotersystem des Fraunhofer Institutes für Produktions- und Automatisierungstechnik en détail untersucht und seine Funktionsweise erklärt. Hierbei liegt der Fokus nicht nur auf dem Roboter, sondern auch auf die anderen mitwirkenden Hardware- sowie Softwarekomponenten des Systems.

\section{Die Roboterzelle}

\begin{figure}[b!]
	\includegraphics[width = \textwidth]{Abbildungen/roboter_zelle.png}
	\centering
	\caption{Die Roboterzelle}
\end{figure}
\emph{Zelle} wird hier als ein Oberbegriff für die restlichen Komponenten des Schweißrobotersystems verwendet. Der Grund für die Verwendung dieses Begriffs liegt daran, dass der Roboter sich in einem geschlossenem Raum mit den anderen Komponenten befindet und zusammen mit den als eine Einheit agiert. Der Cobot ist auf einem Metalltisch festmontiert, der auch als den Arbeitsplatz des Roboters sowie die elektrische Masse für das Schweißverfahren dient. Zu schweißende metallische Werkstücke werden auf dem Tisch fixiert und haben aufgrund des direkten Kontaktes mit der elektrische Masse ein Potential von null. Außer dem Schweißtisch befinden sich in der Zelle auch anderen Komponenten, die zum Schweißen erforderlich sind, sensorischen Zwecke erfüllen oder zur Gewährleistung der Sicherheit dienen. Die Schweißquelle, Drahtvorschubeinrichtung und Schutzgaszylinder befinden sich in diese Zelle und sind mittels Kabel und Schläuche mit der Schweißpistole verbunden. Die Kabel für Stromführung und Schläuche für die Überführung des Schutzgases und Schweißzusatzes werden zusammen gebündelt und zu der Schweißpistole so geführt, dass sie möglichst entfernt von den Roboterarmen sind. Dieses Bündel wird mittels gefederten Seilrollen, die an der Decke montiert und horizontal fahrbar sind, weit über den Roboter gehalten. Somit kann der Roboter während eines Bewegungsvorganges in beliebigen Richtungen ohne Hindernisse fahren. Am Endeffektor des Roboters ist auch ein Lasersensor fixiert, welches für die optische Sinnesempfindung verwendet wird. Der Sensor wird über den M8 8-Pin Anschluss an dem Roboterarm mit der Steuerung und über einen Ethernet-Kabel mit einem leistungsstarken Rechner, ein Industrie-PC (IPC), verbindet. Dieses Kabel wird auch mit den restlichen Kabeln zusammen gebündelt und von den Roboterarmen ferngehalten. Schließlich die Sicherheitsvorrichtungen, die bezugnehmend auf Abschnitt~\ref{ssec:cobots} dem Mitarbeiter schützen sollen. Die Einzelkomponenten der Roboterzelle werden jeweils detaillierter betrachtet

\subsection{Der Roboter}
Der Roboter dieses Systems ist die zentrale Hardwarekomponente dieses Systems. Dieser ist für die Bewegung und Handhabung der Schweißpistole und des Lasersensors zuständig. Verwendet wird hierfür ein sechs achsiges kollaborierender Roboter mit der Bezeichnung 10e von dem dänischen Unternehmen \emph{Universal Robots}. Mit einer Reichweite von 1300 mm und eine Traglast von 12,5 kg eignet sich der Cobot für die Bearbeitung größerer Werkstücke. Um die notwendigen Sicherheitsanforderung zu gewährleisten, kann dieser Cobot mit verschiedenen Sicherheitsfunktionen konfiguriert werden. Auch zur Kraftmessung und Kollisionserkennung ist er mit Kraft-Moment-Sensoren ausgestattet. Diese können Kräfte bis zu 100 N oder Momente bis zu 10 Nm mit einer Genauigkeit von 5,5 N beziehungsweise 0,5 Nm und einer diskreten Auflösung von 5,0N beziehungsweise 0,2 Nm in allen dreidimensionalen Richtungen ermessen. Jeder der sechs Achsen hat einen Arbeitsradius von $\pm \ang{360}$, allerdings sind die jeweiligen maximalen Geschwindigkeiten abweichend. Der Fußgelenk und der Schultergelenk können maximal $\pm \ang{120} s^{-1}$ in beiden Laufrichtungen drehen. Im Gegensatz dazu haben die drei Handgelenke eine maximale Rotationsgeschwindigkeit von $\pm \ang{180} s^{-1}$. Der Roboter hat eine Wiederholgenauigkeit von $\pm 0,05 mm$. Laut dem Hersteller ist es möglich, jeden möglichen Endeffektor mit dem richtigen Aufsatz am Roboterende zu montieren .Mit einem standardisierten M8 8-Pin Anschluss dürfen diese Endeffektoren oder andere Sensoren mit dem Roboter interagieren. Über den 8-Pin Stecker wird werden diese Geräte auch mit einem Strom beliefert. Eine Interaktion mit dem Werkzeug kann über die vorhandenen digitalen oder analogen Ein- und Ausgänge erfolgen. 

Die Steuereinheit des Roboters befindet sich in einem Schaltkasten, der mit dem Roboter über einen Kabel verbunden wird. Dieser Kabel liefert nicht nur den Strom zum Roboter, sondern ermöglicht es der Steuereinheit, Steuersignale zu senden und digitale sowie analoge Signale zu empfangen. Der Schaltkasten mit der Steuereinheit verfügt auch über jeweils sechzehn digitale Eingänge und Ausgänge sowie zwei analoge. Diese dienen als Schnittstelle zur Kommunikation mit anderen Geräte wie Lichtschranke, Notaustaster oder andere speicherprogrammierbare Steuerungen. Es verfügt auch über eine Ethernet-Schnittstelle zur Kommunikation mit anderen Rechnern über TCP/IP sowie zwecks der Fernsteuerung. Der Schaltkasten wird über einen Ethernet-Kabel mit dem IPC vernetzt.

Als letztes ist auch kleiner Tablett, das Teach-In-Pendant mit dem Schaltkasten verbunden. Auf diesem Pendant läuft die vorinstallierte Software zur Programmierung des Roboters, und bietet über einen Touch-Bildschirm die Möglichkeit der Interaktion an. Daneben befindet sich auch eine Notaustaste auf dem Tablett zum sofortigen Aufhalten des Roboters. Ein anderer Knopf, der sich auch am Roboter befindet, lässt beim Eindrücken eine einfache Positionierung und Bewegung des Roboters per Handführen. 

Die technischen Daten des Roboters wurden aus dem Produktdatenblatt sowie die Bedienungsanleitung erhoben.

%\subsection{Die Zelle}
%\emph{Zelle} wird hier als ein Oberbegriff für die restlichen Komponenten des Schweißrobotersystems verwendet. Der Grund für die Verwendung dieses Begriffs liegt daran, dass der Roboter sich in einem geschlossenem Raum mit den anderen Komponenten befindet und zusammen mit den als eine Einheit agiert. Der Cobot ist auf einem Metalltisch festmontiert, der auch als den Arbeitsplatz des Roboters sowie die elektrische Masse für das Schweißverfahren dient. Zu schweißende metallische Werkstücke werden auf dem Tisch fixiert und haben aufgrund des direkten Kontaktes mit der elektrische Masse ein Potential von null. Außer dem Schweißtisch befinden sich in der Zelle auch anderen Komponenten, die zum Schweißen erforderlich sind, sensorischen Zwecke erfüllen oder zur Gewährleistung der Sicherheit da stehen. Die Schweißquelle, Drahtvorschubeinrichtung und Schutzgaszylinder befinden sich in diese Zelle und sind mittels Kabel und Schläuche mit der Schweißpistole verbunden. Die Kabel für Stromführung und Schläuche für die Überführung des Schutzgases und Schweißzusatzes werden zusammen gebündelt und zu der Schweißpistole so geführt, dass sie möglichst entfernt von den Roboterarmen sind. Dieses Bündel wird mittels gefederten Seilrollen, die an der Decke montiert und horizontal fahrbar sind, weit über den Roboter gehalten. Somit kann der Roboter während eines Bewegungsvorganges in beliebigen Richtungen ohne Hindernisse fahren. Am Endeffektor des Roboters ist auch ein Lasersensor fixiert, welches für die optische Sinnesempfindung verwendet wird. Der Sensor wird über den M8 8-Pin Anschluss an dem Roboterarm mit der Steuerung und über einen Ethernet-Kabel mit einem Rechner verbindet. Dieses Kabel wird auch mit den restlichen Kabeln zusammen gebündelt und von den Roboterarmen ferngehalten. Eine detaillierte Behandlung des Sensors ist in Abschnitt~\ref{sec:lasersensor} zu finden. Schließlich die Sicherheitsvorrichtungen, die bezugnehmend auf Abschnitt~\ref{ssec:cobots} dem Mitarbeiter schützen sollen. 

\subsection{Sicherheitsvorrichtungen}
Zuerst sind die Sicherheitsfunktionen des Roboters zu betrachten. Es lässt sich die Bewegungsmöglichkeiten, maximale Gelenk- sowie Robotergeschwindigkeiten des Ellbogens, Endeffektors und der Mitte, und  maximales Drehmoment des Roboters auf benutzerdefinierte Werte einschränken. Weiterhin lässt sich der Roboter in Notfällen in verschiedenen Weisen aufhalten, beispielsweise mit einem Notschalter oder einer Abschaltung durch das System. Der Roboter kann auch in einem \emph{reduzierten} Modus betrieben werden. Es sollen Grenzwerte für die Geschwindigkeit, Kräfte und Momente des Roboters vordefiniert werden, die beim Aktivieren des Modus innerhalb von 500 ms übernommen werden. Dieser Modus kann entweder automatisch durch die Vordefinition einer Auslöseebene innerhalb des Roboter-Koordinatensystems oder durch einem anderen Gerät aktiviert werden. Wenn der Roboter die Grenze der Auslöseebene überschreitet, wird dieser Modus automatisch eingeschaltet. 

Zum Schutz vor dem gesundheitsschädlichen Schweißrauch, wird ein Filter- und Absauggerät der Firma Kemper GmbH. Dieses Gerät entfernt die beim Schweißen entstehenden Rauchpartikeln mit einer Absaugleistung von $18,33\ m^3s^{-1}$. Es ist wissenschaftlich nachgewiesen worden, dass das Einatmen der aerosolierten Schweißpartikeln das Risiko von Lungenkrebs in einem beruflichen Schweißer um zwanzig bis vierzig Prozent erhöht \autocite{Health_and_safety_2011}. Die kontaminierte Luft werden in einem zweistufigen Prozess gefiltert, gereinigt und danach wieder im Raum eingeführt. Die Kontamination bleibt dabei in einer selbstreinigende Filterpatrone gefangen. Über Druckluftimpulse und zentrifugal Kräfte des Rotationsabscheiders werden die Partikeln in einer Entsorgungskartusche gesammelt. Dies ermöglicht eine möglichst geringe Exposition zu den Krebserreger während der Entsorgung. 

Zur Gewährleistung allgemeiner Sicherheit beim Betreiben des Robotersystems werden alle Mitarbeiter gefordert, entsprechende Sicherheitsausrüstung anzuziehen. Der Roboter und Schweißtisch befinden sich in einem abschließbaren Raum mit dunkel-folierten transparenten Kunststofflamellen-Wände. Somit ist nicht nur der Arbeitsraum des Roboters physisch abtrennbar, sondern wird das Auge vor dem blendenden Blitzlicht während eines Schweißvorganges geschützt.

\begin{figure}[h]
	\includegraphics[width = \textwidth]{Abbildungen/Schweißvorgang .jpg}
	\centering
	\caption{Verdunkelte Ansicht eines Schweißvorganges durch eine folierte Wand}
\end{figure}

\subsection{Schweißquelle}

Dieses Robotersystem ist für das Schweißen mit dem MSG-Verfahren ausgelegt. Hierfür wird die Schweißquelle S5-RoboMIG XT der Lorch Schweißtechnik GmbH. Mit dieser Stromquelle ist es möglich, einen Schweißbereich von 25-400 A mit einer stufenlosen Spannungseinstellung zu realisieren. Somit können Stahl-Schweißzusätze mit einem Durchmesser zwischen 0,8 und 1,6 mm und Aluminimum-Schweißzusätze zwischen 1,0 und 1,6 mm dick  problemlos verwendet werden. Eine Gas- sowie Wasserkühlung sind eine der möglichen Kühloptionen für die Stromquelle. Es sind auch einige smarte Funktionalitäten zur Reglung des Schweißprozesses in der Schweißquelle eingebaut. Beispielsweise ermöglicht eine Lichtbogenreglung die intelligente Anpassung der Schweißparameter zur optimalen Einstellung der Lichtbogenlänge. Somit ist nach Bedarf der Aufgabe eine Einflussnahme auf die Schweißqualität und Einbrandtiefe möglich. Die Puls-Schweißtechnik der Schweißquelle sorgt für eine nahezu spritz-freie Qualität der Schweißnaht und spart somit den Aufwand einer Nachbearbeitung. Es ist eine allgemein hohe Schweißgeschwindigkeit sowie tieferer Einbrand dank der hohen Energiedichte, des durch die Stromquelle erzeugten Lichtbogens, realisierbar. Die Schweißquelle bietet auch eine Schnittstelle zur netzwerkbasierten Kommunikation für alle gängigen Busprotokolle an. Sie wird für die Interaktion mit der Robotersteuerung verwendet. Über diese Schnittstelle kann ist die Steuereinheit des Roboters in der Lage, Steuersignale an der Schweißpistole zu senden.

Zu der Schweißquelle gehört auch die Drahtvorschubeinrichtung. Sie wird direkt mit der Stromquelle verbunden und wird durch sie mit Strom betrieben. Als Schweißzusatz wird das Edelstahl Thermanit GE-316L Si der Böhler Welding Germany GmbH verwendet. Die Herstellervorgaben zum Schweißen mit dieses Material ist die Verwendung von Gleichstrom mit der Elektrode als Kathode und ein Kohlendioxid-Argon-Schutzgas nach Norm EN ISO 14175 mit Kohlendioxidgehalt von $2,5 \%$. Die Drahtvorschubeinrichtung wird durch die Regelungstechnik der Schweißquelle kontrolliert. Sie regelt nicht nur den Drahtvorschub sondern kann auch die Elektrodenverlängerung anpassen. Somit ist die Reglung der Schweißquelle in der Lage, viele der bewegungs- und positionsunabhängigen Schweißparameter selbst zu regeln. 

Technische Daten der Schweißquelle und komplementären Geräte wurden von dem Produktdatenblatt, Bedienungsanleitung, und Produktseite erhoben.

\subsection{Lasersensor}

Die Rolle des technischen Auges wird durch ein Laserliniensensor der Micro-Epsilon Messtechnik GmbH \& Co. KG gespielt. Dieses Sensor dient zur Profilerkennung mit einer hohen Genauigkeit und wird zur Erkennung der Schweißnaht durch den Roboter verwendet. Die Produktbezeichnung des Laserliniensensors ist scanControl 3000 und hat einen Messbereich von 300 mm in der z-Richtung sowie 290 mm in der x-Richtung. Es können 2.048 Punkte in der x-Achse pro Profil aufgenommen mit einer maximalen Wiederholrate von 10.000 Hz werden. Dieser Sensor misst die Objektentfernung nach dem Prinzip des Triangulationsverfahrens aus Abschnitt~\ref{ssec:lasersensoren}. Es wird eine Linie über die entsprechende Optik auf die Werkstückoberfläche projiziert, deren Spiegelung auf eine Sensormatrix zweidimensional abgebildet wird. Nach einer Auswertung dieser Abbildung wird Positionsinformation des Werkstückprofils in Form einer Punktewolke erhalten. 

Die ausgewerteten Messergebnisse sind über eine Ethernet-Schnittstelle verfügbar. Die Schnittstelle kann auch für die Sensorsteuerung verwendet werden. Somit ist es möglich, jegliche Sensorparameter extern über ein anderes Gerät anzupassen. Der M8 8-Pin Anschluss wird zur Umschaltung zwischen den unterschiedlichen Betriebsmodi sowie zur Auslösung eines Messvorganges verwendet. Der Sensor ist mittels einer speziell ausgelegten und konstruierten Halterung an dem Endeffektor des Roboters montiert. Er wird über den M8 8-Pin Anschluss mit dem Digitalausgang am Roboterende und über den Ethernet-Anschluss mit dem IPC verknüpft. Der, im Sensor verwendeten Laser ist einer der Laserklasse 2R. Eine direkte Ausstrahlung in das Auge kann zur Irritation oder Verletzung des Auges führen. Aus diesem Grund wird nicht nur die Stromversorgung sondern auch das Einschalten des Laserliniensensors über externe Schalter getätigt.

\begin{figure}[h]
	\includegraphics[width=\textwidth]{Abbildungen/collage.jpg}
	\centering
	\caption{Der Laserliniensensor auf einem Werkstück projiziert}
\end{figure}

\section{Software}
Die Erstellung des Softwarepakets namens \emph{processit} zur Realisierung der erzielten Funktion des Schweißroboters ist die Kernarbeit der Fraunhofer IPA. Hierfür wird das ROS-Industrial-Paket aus Abschnitt~\ref{ssec:ros_industrial} als Fundament verwendet. Dies ermöglicht eine einwandfreie Integration des Cobots und enthüllt ein Rahmenwerk für die Kommunikation und Steuerung des Roboters über Programmierschnittstellen. Die komplexe Aufgabe der automatischen Naht-Erkennung und des Schweißens kann in überschaubarer Teilaufgaben zerlegt werden. Bezugnehmend auf Abschnitt~\ref{sec:ROS} können somit Teilaufgaben separat erarbeitet werden und die Aufgabe der Zusammenführung und Koordinierung an ROS überlassen werden.

\subsection{Abstraktion des Softwarepakets}
Bevor die genauen Funktionen der einzelnen Teilmodule angeschaut werden, lohnt sich eine Vereinfachung der komplexen Funktionsweise des ganzen Softwarepakets. Die höhere Softwarearchitektur leitet sich aus der drei vernetzten Geräte und somit drei Kernmodule. Die Software verkoppelt den Roboter, Laserlinienscanner und über das Teach-In-Pendant die Benutzeroberfläche mit einander und verwendet dafür die Pakete \emph{processit\textunderscore sensors}, \emph{processit\textunderscore detection} und \emph{processit\textunderscore adapt}. Das Modul processit\textunderscore sensors ist für Steuerung des Sensors sowie die Vorverarbeitung der Sensordaten zuständig. In die Vorverarbeitung wird das Störgeräusch ausgefiltert und die Koordinaten der Punktewolke zur Weltkoordinaten transformiert. Dieses Koordinatensystem auf das globale System und ist der Bezugspunkt für andere Koordinatensysteme. Diese Information wird an dem nächsten Modul - processit\textunderscore detection - übertragen, welches die Aufgabe der Naht-Erkennung übernimmt. Deren Funktionsweise wird bei der genaueren Behandlung des dafür zuständigen Teilmoduls erklärt. Nach der Erkennung der Naht wird diese Erkenntnis durch \emph{processit\textunderscore adapt} verwendet, um zusammen mit externe Module wie MoveIt eine Bahn für den Roboter auszurechnen. Die geplante Trajektorie wird dem Roboter über einen ROS-Treiber bereitgestellt, der durch den Hersteller zur Verfügung gestellt wird. Dieses Kernmodul kommuniziert auch mit dem Teach-In-Pendant, sodass das Starten der Schweißaufgabe sowie die Parametersetzung des Roboters und Schweißprozesses über die Benutzeroberfläche erfolgen kann. Einen Blick auf den Datenfluss stellt klar, wie der autonome Schweißprozess von der Erkennung bis zur Bahnplanung abgewickelt wird.

\begin{figure}[htp]
	\includegraphics[width = \textwidth]{Abbildungen/data_flow_diagram.png}
	\centering
	\caption{Datenfluss innerhalb processit}
	\label{fig:Datenfluss}
\end{figure}

Die Legende zur Interpretierung von Abbildung~\ref{fig:high-level-architecture} ist im Anhang~\ref{a:legende} zu finden. Ein Verständnis der höheren Funktionsweise von processit dient dem Überblick bei einer Untersuchung der einzelnen Teilmodule, die zur Kernfunktionalitäten des Programmpakets beitragen. 

\begin{figure}[tp]
	\includegraphics[width = \textwidth]{Abbildungen/architecture_rough.png}
	\centering
	\caption{Höhere Softwarearchitektur des processit-Softwarepakets}
	\label{fig:high-level-architecture}
\end{figure}

\subsection{ROS-Module}
Vor einer Betrachtung der einzelnen Teilmodule ist es wichtig, ein paar Konzepte zu diskutieren, die ein wesentliches Aspekt des Entwicklungsprozesses gewesen sind. Die Objektorientierung in der Softwareentwickelung dient zur Verbesserung der Erweiterbarkeit, Testbarkeit und des Wartungsaufwands einer Software. Objektorientierte Ansätze machen Objekte zur Lösung komplexere Probleme zu Nutze. Ein Objekt besteht aus verschiedene Eigenschaften und Funktionen, die Methoden genannt werden, die Alleinstellungsmerkmale des Objekts sind. Diese Eigenschaften und Methoden werden somit mit dem Objekt verbunden und dürfen nur durch ihm verwendet werden. Objekte ermöglichen die Abstrahierung und Modellierung komplexer, vielseitiger Daten und vereinfachen ihrer Anwendung in einer Software. Objekte können in objektorientierten Programmiersprachen mit Klassen erstellt werden. \autocite[27-28]{Lahres2021} \autocite[415-416]{Kaiser2022}

Konzepte wie die lose Kopplung und Abhängigkeitsinjektion sowie -Inversion wurden in der Entwicklung von processit sehr häufig benutzt. Die lose Kopplung ist ein Konzept der serviceorientierten Architektur bei der Softwareentwickelung, die eine Reduzierung des Abhängigkeitsgrad zwei Module oder Objekte beziehungsweise Klassen fördert. Somit können Klassen und Objekte mehr angepasst werden, ohne große Änderungen in den abhängigen Klassen vorzunehmen. Mit einer losen Kopplung ist es möglich, einzelne Objekte nach ihrer Hauptfunktionen zu abstrahieren. Somit ist eine Anwendung unterschiedlicher Objekte mit den gleichen Hauptfunktionen in einer Klasse oder Modul ohne deren beziehungsweise dessen Anpassung möglich. \autocite{Hockkoon2010}

Die Abhängigkeitsinversion ist ein Entwurfsprinzip, wo die Abhängigkeitsrelationen zweier oder mehrerer Module oder Klassen wiederkehrt werden. Somit hängen sie nicht von konkreten Objekte ab, sondern von Abstraktionen  \autocite[67]{Noback2018}. Abhängigkeitsinjektion ist ein Konzept dieses Entwurfsprinzips, welches die Verbesserung der Wiederverwendbarkeit, Wartbarkeit, und Testbarkeit von Software erzielt. Diese Vorgehensweise ermöglicht die Entwicklung von lose-gekoppelter Software, indem Klassen Objekte mit bestimmten Funktionalitäten auffordern können und diese ihnen geliefert werden. Somit reduzieren sich die Verantwortungen der Klasse, indem sie nicht für die Erstellung des erforderten Objektes zuständig ist. \autocite[204-1112]{Gregoire2021}

Zwecks der Wiederverwendbarkeit sind die einzelne Teilmodule von processit als ROS-Module gestaltet. Jedes Modul erstellt nach dem Start je nach Aufgabenvielfalt eigene ROS-Nodes. Innerhalb jedes Moduls werden die Codedateien, die von ROS abhängig sind, von anderen ROS-unabhängigen Codedateien zwecks des Organisierens getrennt.

\subsubsection{processit\textunderscore sensors}
Dieses Modul ist für den Informationsaustausch mit dem Laserliniensensor zuständig. Das Starten und Stoppen sowie die Parametersetzung des Lasers erfolgt über dieses Modul. Ein gesondertes Modul \emph{scanconctrol\textunderscore handler} übernimmt die Funktion als Schnittstelle zwischen dem \emph{processit\textunderscore sensors} Node und ROS Treiber des Sensors. Über ein Control-Register können die Stärke des Lasers und verschiedene Sensorparameter eingestellt werden.

Ein externer ROS-Modul, der einen Treiber für den Laserliniensensor anbietet, wird zur Gewinnung der Geometriedaten in Form einer Punktewolke verwendet. Dieses Modul stellt einen ROS-Publisher bereit, der regelmäßig neue Punktwolken aussendet, während der Laserscanner über das Werkstückprofil verläuft. Dieser Punktewolken sind allerdings Rohdaten und müssen zuerst vorverarbeitet werden und zum Weltkoordinatensystem transformiert werden. Ein Präprozessor ist für diese Aufgabe zuständig. Ein eingebauter ROS-Subscriber abonnierten Punktewolken von dem Treiber, prüft sie auf Datengültigkeit und filtert sie auf Basis einer statistischen Methode. Hierbei werden Punkte entfernt, die innerhalb eines bestimmten Radius nicht eine Mindestanzahl an benachbarten Punkten haben. Nach der Entfernung von Ausreißern wird die Punktewolke in das Weltkoordinatensystem transformiert und mittels eines eingebauten ROS-Publishers für die Weiternutzung zu einem ROS-Topic (\emph{scan\textunderscore world}) veröffentlicht.  

Die Lasersensordaten werden auch in einer Simulation des Roboters in Gazebo simuliert werden, um eine Testumgebung ohne die Einbindung der Hardwarekomponenten zu ermöglichen.

\subsubsection{processit\textunderscore detection} \label{sssec:processit_detection}
Dieses Softwarepaket stellt den vierten Baustein des Datenflusses in Abbildung~\ref{fig:Datenfluss} dar. Hier werden die gefilterten Geometriedaten der Punktewolke verwendet, um eine Naht zu erkennen. Der ROS-Node \emph{scan\textunderscore processor} dient als eine Schnittstelle zur Verbindung mit anderen ROS-Nodes außerhalb dieses ROS-Moduls. Neben seine Rolle als Schnittstelle und Kommunikationsagent steuert er auch den Prozess zur Detektierung der Naht. Die Möglichkeit des Startens, Pausierens, Fortsetzens und Stoppens der Detektion wird auch durch dieser Node in Form von ROS-Services angeboten. Auch die Detektionsmodi, die die Geschwindigkeit und Genauigkeit des Detektionsverfahrens regeln, können über den \emph{scan\textunderscore processor} Node geregelt werden. Die eigentliche Detektion der Naht wird allerdings durch den Prozessor an das jeweilige Erkennungsmodul für den entsprechenden Anwendungsfall delegiert. 

Da es unterschiedliche Arten einer Schweißnaht geben, werden unterschiedliche Verfahren zur Erkennung der Schweißnaht verwendet. Hierbei wird das Entwurfsprinzip der losen Kopplung angewendet. Die Klasse des Detektors wird abstrahiert, sodass mehrere Erkennungsverfahren unter dieser abstrakten Klasse zugeordnet werden können. Somit verfügt der Prozessor beim Abruf der abstrakten Klasse automatisch über die implementierten Erkennungsmethoden und muss nicht situationsabhängig die einzeln aufrufen. Abhängig von der abstrakten Basisklasse wurden Klassen für die Erkennungsverfahren einer Kehlnaht, Stumpfnaht und einer offline erzeugten Werkzeugbahn. Die Erkennung einer vordefinierten Werkzeugbahn ist für das Abfahren einer Freiformfläche notwendig. Die Stumpfnaht-Erkennung und Werkzeugbahn-Erkennung befinden sich gerade in einer Prototyp-Phase. Eine Abstrahierung des Codes zum Verständnis der Struktur ist im Anhang~\ref{a:abstraktion_nahterkennung} zu finden. Der Auswahl einer Erkennungsmethode erfolgt nicht bei der Initialisierung der Klasse, da es bei einer Änderung des Schweißnahttypes ein Neustart des Programms nötig wäre. Stattdessen erfolgt der Auswahl dynamisch über eine Funktion, die die Konfigurationsdaten einliest und eine Wahl trifft. Diese Daten werden als Teil der Request-Nachricht beim aufrufen des Services zum Starten des Detektionsverfahrens erhalten.
´
Schließlich wird zur Aufnahme, Protokollierung und Speicherung der Punktewolke-Daten eine Logger-Klasse implementiert. Diese abonniert ein ROS-Topic namens scan\textunderscore world und erhält somit gleichzeitig wie der \emph{scan\textunderscore processor} die vorverarbeitete Geometrieinformation. Eingebaut ist auch ein ROS-Subscriber, der dem ROS-Topic \emph{stopRecording} abonniert. Diese signalisiert dem Logger über das Ende des Detektionsverfahrens und löst den Speichervorgang der aufgenommenen Punktewolke aus. 

\subsubsection{processit\textunderscore adapt}
Dieses ROS-Modul enthält die wichtigsten Komponenten des processit-Projekts. Es ist die Hauptschnittstelle zur Kommunikation mit der Benutzeroberfläche des Teach-In-Pendants. Ein selbsterstelltes Plugin für die Software des Pendants ermöglicht die Aktivierung der smarten Funktionen von processit. Dieses Plugin kann ROS-Services aufrufen, die sich in diesem Modul befinden und die Roboterbewegung sowie den Schweißprozess steuern. Darüber hinaus können auch Vorgänge zur Kalibrierung und Kommissionierung des Roboters hieraus gestartet werden.

Das \emph{robot\textunderscore controller\textunderscore bridge} funktioniert analog zu einer Brücke zwischen dem ganzen processit-Programmpaket und der Steuereinheit des Roboters. Hierüber kann der Schweißvorgang gesteuert werden, indem gezielt ROS-Services aufgerufen werden. Diese Services sind für vier Hauptaufgaben ausgelegt worden. Ein Service zur Initialisierung des Vorganges wird zuerst aufgerufen. Hier werden unter anderem die, durch den Benutzer gesetzten Parameter eingelesen. Der nächste Service startet eine Vorfahrt des Roboters, die zur Erkennung der Schweißnaht und ihrer Position dient. Danach fängt der eigentliche Schweißprozess an, wo der Roboter die Naht verfolgt. Nach Ende der Naht wird der letzte Service aufgerufen, der den Schweißvorgang beendet. Daneben wird die Kalibrierung des Lasersensors auch in diesem Programm als einen ROS-Service angeboten. Die eigentliche Aufgabe der Aufgabenausführung, Bahnplanung und Kalibrierung wird an anderen Unterprozesse delegiert. 

Ähnlich wie das Detektionsverfahren aus~\ref{sssec:processit_detection}, gibt es unterschiedliche Verfahrensarten zum Schweißen eines Bauteils. Es wird nicht über die Schweißverfahren geredet, sondern die unterschiedlichen Varianten, in der ein Roboter an einer Schweißaufgabe herangehen könnte. Das processit-Paket bietet drei Varianten der Herangehensweisen zum Schweißen eines Bauteils an. Diese werden \emph{task\textunderscore appliers} genannt. Mit dem gleichen Zweck der Simplifizierung wird auch hier das Entwurfsprinzip der losen Kopplung mittels Abhängigkeitsinjektion angewendet. Die Services aus robot\textunderscore controller\textunderscore bridge rufen eine abstrakte Klasse auf, die konkret für die drei Herangehensweisen implementiert wird. Die erste Variante (online\textunderscore following) ist die Online-Verfolgung. Hierbei erfolgt die Erkennung der Naht während einer Bewegung des Roboters. Die Trajektorie des Roboters wird soeben während des Schweißvorganges ermittelt. Es sind hier zwei Bewegungsplaner nötig. Der erste, ein industrieller Bewegungsplaner von Pilz, dient zur Errechnung einer vorläufige Bahn nach der Erkennung der Schweißnaht-Position und Orientierung. Der zweite, ein selbsterstellter Planer, dient der weiteren Planung einer Trajektorie mit Tiefeninformation aus dem Laserliniensensor in echter Zeit. Bei der zweiten Variante (Scan \& Plan) wird zuerst ein Messvorgang durchgeführt. Aus den Erkenntnissen dieses Vorganges wird mit einem Bewegungsplaner eine Trajektorie generiert. Die dritte Variante (toolpath\textunderscore following) ist für das Schweißen einer Freiformoberfläche und ist wie bei dessen Detektionsverfahren in einer Prototyp-Phase. Es gibt eine vierte Variante des Prozesses, die der Kalibrierung des Laserliniensensors dient und für den eigentlichen Schweißprozess wenig relevant ist. Es werden über die Basisklasse auch die ROS-Services für die Detektionsaufgabe aus~\ref{sssec:processit_detection} gestartet und stehen somit erst mit dem Start des Schweißvorganges zur Verfügung. Der Auswahl einer Herangehensweise erfolgt auch hier dynamisch auf Basis der Konfigurationsdaten, die in der Request-Nachricht des aufgerufenen Services stehen. Ein Beispiel zur Veranschaulichung der Code-Struktur ist im Anhang unter %TODO: add ref to appendix

Innerhalb der einzelnen Implementationen wird die Aufgabe der Bahngenerierung für den Roboter weiter delegiert. Bei der Erstellung einer Bahn wird die Pose der Schweißnaht rechnerisch ermittelt. Es wird die Tiefeninformation der Schweißnaht benutzt, die der Detektionsprozess aus~\ref{sssec:processit_detection} gewonnen wird, um die Position und Orientierung der Schweißnaht zu erkennen. Für jede Herangehensweise sind Generatoren zuständig, die mit Hinsicht auf die speziellen Anforderungen jeder Variante erstellt wurden. Diese werden \emph{path\textunderscore generators} genannt. Unter Verwendung des gleichen Prinzips der Abhängigkeitsinversion wird auch hier eine abstrakte Klasse erstellt, die in den einzelnen Implementationen konkretisiert wird. Wie bei den task\textunderscore appliers wird die Bahngenerierung auch für das online\textunderscore following, das Scan \& Plan, toolpath\textunderscore following und den Kalibriervorgang des Laserliniensensors implementiert. Eine Veranschaulichung der Code-Struktur ist im Anhang unter %TODO: add ref to appendix

Ein detaillierter Programmablauf der obigen Module ist im Anhang~\ref{a:processit_detailablauf} zu finden. Diese drei Hauptmodule des processit-Pakets werden durch weitere Stützmodule ergänzt, die verschiedene Aufgaben der Simulation, Bahnplanung, Visualisierung und jeglicher Berechnungen übernehmen. Die effektive Zusammenarbeit der drei Hauptmodule sowie Stützmodule funktioniert nur mit der effektiven Kommunikation zwischen diesen Modulen. Deswegen werden zunächst die unterschiedlichen ROS-Messages diskutiert, die den Kernpartikel der zwischen-modularen Kommunikation bilden.


\subsection{ROS-Messages von processit}
Die in processit verwendeten ROS-Messages können grundsätzlich nach ihrer Verwendung unterschieden werden. Es werden für die Publisher-Subscriber-Kommunikation innerhalb von ROS Nachrichten verwendet sowie als Austauschmittel zwischen Clients und Servers. Die Behandlung aller ROS-Messages beider Arten wird zu umfangreich und ist nicht das Ziel dieses Abschnitts. Es werden nur die kritischen Nachrichtenarten behandelt, die zum Verständnis der Kommunikation zwischen den Hauptprozessen sind. 

Es werden zuerst die ROS-Messages angeschaut, die Information aus Sensoren des Robotersystems vermitteln. Die ROS-Message \emph{PointCloud2} enthält Information über die Punktewolke, die aus der Messwerte des Laserliniensensors erstellt wird. Dies enthält nicht nur die Punktwolke selbst, sondern auch Metadaten dazu. Darunter zählt die Aufnahmezeit der Punktewolke, deren Höhe sowie Breite, und die Datengültigkeit der Punktewolke unter anderem. Die Lage der Robotergelenke aus den internen Sensoren steht auch den ROS-Modulen in Form einer ROS-Message zur Verfügung. \emph{JointState} speichert Daten über die Position, Geschwindigkeit, und Belastungsgrad eines Gelenks, das über einen einzigartigen Namen identifiziert wird. Die Rohdaten aus einem Laserscan werden in einer ROS-Message namens \emph{LaserScan} gespeichert, die nicht nur die Messwerte in Form von Abständen enthält, sondern auch Metadaten wie der Winkel am Anfang und Ende des Scans, die Dauer und die Grenzwerte der Messungen. Aus der Information dieser ROS-Message kann eine Punktewolke auf Basis der PointCloud2 ROS-Message erstellt werden.

Als nächstes werden ROS-Messages betrachtet, die geometrische Information enthalten. Ein paar grundlegende Messages müssen zuerst definiert werden, da sie häufig in anderen ROS-Messages eingebettet werden. Die ROS-Message \emph{Vector3} wird zur Sendung von dreidimensionaler Vektorinformation verwendet. Vector3 wird für den Ausdruck einer räumlichen Geschwindigkeit als linearer- und Winkelgeschwindigkeit in der \emph{TwistStamped} Message angewendet. Die Koordinatentransformationen zwischen den einzelnen Komponenten des Robotersystems sowie den Gelenken des Roboters werden auch über ROS-Messages ausgetauscht. Eine Koordinatentransformation wird in einer ROS-Message namens \emph{transform} gespeichert, die Details über die translatorische und rotatorische Transformation als Vektoren und Quaternionen enthält. Diese Message wird kaum eigenständig verwendet, sondern in einer \emph{TransformStamped} Message eingebettet. Diese Message enthält zusätzliche Header-Daten, die eine ID des Eltern-Koordinatensystems hat, und die Identifikation des Koordinatensystems, das sich auf dem Eltern-Koordinatensystem bezieht. Auch geometrische Information über eine Pose wird als eine ROS-Message strukturiert. Diese enthält Details über eine Position als einen Punkt in einem dreidimensionalen Raum und die Orientierung des Punktes als eine Quaternion. 

ROS-Messages können auch zur Bekanntgabe von Details über den Roboter verwendet werden. In der Message \emph{RobotState} werden Details über die Position von Gelenke und Armteile sowie Objekte mit Kollisionsgefahr kompiliert. In der \emph{RobotTrajectory} ROS-Message sind Auskünfte über die aktuelle Bahn der Robotergelenke enthalten. Weitere ROS-Messages bestehen auch zur Übermittlung von Information über Trajektorien. Diese Information stellt sich aus dem Bahnverlauf, der Geschwindigkeit, Beschleunigung und dem Ruck zusammen. 

Viele Aktionen des processit-Pakets werden durch ROS-Services ausgelöst. ROS-Services erhalten Anfragen und senden Antworten in Form von Nachrichten. Hierfür werden auch ROS-Messages verwendet. Die Parametersetzung des Laserliniensensors erfolgt über einen ROS-Service. Hierzu wird eine Anfrage gesendet, die Angaben zu der Laserfrequenz, Belichtungszeit und Operationsmodus enthält. Bei dem Aufruf dieses Services wird keine Antwort gesendet. Auch zum Starten des Detektionsverfahrens aus Abschnitt~\ref{sssec:processit_detection} wird eine Abfrage gesendet. Diese enthält Auskünfte über die zu suchende Geometrie (eine Kehlnaht oder Stumpfnaht), eine Identifikationszeichenkette, die kartesische Trajektorie, die Pose des Startpunkts und eine Reihe weiterer Posen, die für das Schweißen einer Freiformoberfläche wichtig sind. Als eine Antwort nach der Ausführung wird eine Statusnachricht über den Prozesserfolg erhalten. Der Service zur Initialisierung des Schweißvorganges erhält eine detailreiche Anfrage. In dieser Nachricht ist Information über die zu schweißende Geometrie, Bewegungssteuerende Eigenschaften wie die Geschwindigkeit und andere prozessrelevante Details enthalten. Als Antwort wird hier auch eine Statusnachricht über den Prozesserfolg erhalten. Zum Fortsetzen des Schweißvorganges nach einer Pause wird auch eine Anfrage zu dem zuständigen ROS-Service gesendet. Da alle prozessrelevante Details in der Anfrage für die Prozessinitialisierung gesendet wurden, steht in dieser Anfrage nichts. Auch dieser Service sendet eine Statusnachricht über den Prozesserfolg zurück. Der \emph{AddPoseMarker} ROS-Service ist speziell für die Werkzeugbahnverfolgung bei dem Schweißen einer Freiformoberfläche ausgelegt. Er wird zur Setzung von virtueller Markierungen auf der Werkzeugbahn verwendet. Dieser erhält per Anfrage verschiedene geometrische sowie andere prozessrelevante Details und sendet eine einzigartige Identifikation für die gesetzte Markierung. 

Viele dieser ROS-Messages werden durch externe Module sowie Stützmodule des processit-Pakets verwendet, um Funktionen innerhalb des Programmpakets aufzurufen oder mit dem Programm zu kommunizieren.

%\subsection{Support-Module} \label{ssec: support_modules}
%Die Stützmodule des Programmpakets bieten zusätzliche Funktionalitäten an. Es können die Sensormessungen und Roboterbewegungen visualisiert werden. Die Berechnung einer Trajektorie für den Roboter wird auch durch ein Stützmodul übernommen. Letztlich kann das Verhalten des Robotersystems mittels einer Simulationsumgebung auch detailliert getestet werden, welches besonders hilfreich bei dem Testen neuer Funktionalitäten ist. Hierfür werden die diversen Zusatzmodule von ROS aus Abschnitt~\ref{sec:ROS} verwendet.
%
%Mittels Rviz kann nicht nur die Roboterbewegung in seiner Umgebung überwacht werden, sondern auch die Messungen des Lasersensors. Diese werden innerhalb von processit\textunderscore sensors bearbeitet, sodass sie durch das Rviz Programm visualisiert werden können. Mittels STL-Dateien wird die Geometrie der Einzelkomponenten des Robotersystems dem Rviz-Programm übermittelt. Hierunter zählt nicht nur der Roboter und seiner Umgebung, sondern auch der Laserliniensensor und die Schweißpistole. Diese werden möglichst präzise modelliert. Die Visualisierung der Einzelkomponenten sowie der Sensor- und Koordinateninformation kann mittels Konfigurationsdateien in Rviz angepasst werden. Die Integration von Rviz in dem processit-Programmpaket ermöglicht die Prozessüberwachung aus der Roboter-Sichtweise und dient dem Verstehen des Roboterverhaltens während eines Programmablaufs. 
%
%Das Programm Gazebo wird als eine Simulationsumgebung für das processit-Programmpaket verwendet. Dieses Programm wird für die Erschaffung eines digitalen Zwilling für das Robotersystem verwendet. Es können virtuelle Bauteile in der Simulationsumgebung eingeladen werden, die virtuell durch den Roboter verarbeitet werden können. In dieser Umgebung werden auch die physikalischen Zusammenhänge der physischen Komponenten vordefiniert, sodass die Bewegung des Roboters möglichst realitätsnah abgebildet werden kann. Auch Sensoren wie der Laserliniensensor werden hier modelliert und sind dazu fähig, das virtuelle Bauteil optisch anzutasten und Sensorwerte in Form der ROS-Message, LaserScan, zu emulieren. Ähnlich wie Rviz erfolgt die Einstellung von Gazebo mittels 3D-Objektdateien wie STL-Dateien und Konfigurationsdateien in form von YAML- sowie XML-Dateien. 
%
%Andere Module wie das \emph{ur\textunderscore robot\textunderscore driver} und MoveIt dienen unterschiedliche Zwecke zur Realisierung der Roboterfunktion. das ur\textunderscore robot\textunderscore driver agiert als einen ROS-Treiber für die Robotersteuerung und ermöglicht die Kommunikation zwischen den beiden Teilen. MoveIt wird zur Planung verschiedener Trajektorien des Roboters verwendet. Diese Module ersparen viel Aufwand und Zeit für den Entwicklern des processit-Programmpakets und lassen sie auf das wesentliche konzentrieren. 

%	\chapter{Fazit und Ausblicke}
\section{Rückschlüsse}
Über die ganze Arbeit lässt sich einen Sachverhalt klarstellen - der Aufbau und die Funktion eines intelligenten Schweißroboters ist vielfältig und komplex. Es gibt mehrere Hardware- und Softwarekomponenten, die für eine richtige Funktionsweise des Roboters integral sind. Der Roboter ist der Hauptteil der Hardwarekomponenten. Er verleiht dem System eine Handhabungseinrichtung, die für die Manipulation der Werkzeuge, die Schweißpistole, wichtig ist. Um mit Menschen zu kollaborieren muss der Roboter allerdings ein paar Sicherheitsanforderung erfüllen. Neben einer ungefährlichen Geometrie wird die Geschwindigkeit sowie Kraft und das Drehmoment des Roboters beschränkt. Er wird mit verschiedenen Funktionen zur Kollisionsdetektion ausgestattet, um lebensgefährliche Verletzungen zu vermeiden.

Die Sicherheitsanforderungen sind nicht nur durch den Roboter zu erfüllen, sondern durch die ganze Zelle zu gewährleisten. Es gibt Notaustaster, die den Roboter zu einem sofortigen Stopp bringen würden und die Schweißpistole sofort ausschalten würden. Der Arbeitsbereich des Cobots wird während des Betriebs durch eine physische Zelle getrennt, die transparente folierte Wände hat. Dies ermöglicht eine Überwachung des Systems während des Betrieb, ohne die Gefahr einer Verblendung. Der giftige Schweißrauch, welcher bei Schweißen entsteht, enthält aerosolierte krebserzeugende Partikeln. Dieser Rauch wird letztlich durch ein Absauggerät entfernt und gefiltert bevor die Luft wieder in der Umgebung eingeführt wird.

Die Schweißquelle liefert nicht nur den erforderten Strom für die Schweißaufgabe, sondern bietet auch zusätzliche Funktionalitäten zur Stabilisierung und Verbesserung des Schweißprozesses an. Die Stromquelle des Robotersystems erfüllt genau diese Anforderungen und steuert darüber hinaus nicht nur den Drahtvorschub, sondern auch den Fluss des Schutzgases für das Metall-Schutzgas-Schweißen. Letztendlich ist die Schweißquelle auch Kommunikationsfähig. Die Schweißquelle des Robotersystems ist in der Lage, über mehrere gängige Industrieprotokolle mit anderen Geräten zu kommunizieren, die die Schweißquelle auch steuern können. 

Innerhalb des Netzwerks des Robotersystems kommunizieren die Steuereinheit des Roboters sowie der Schweißquelle, der Laserliniensensor und ein Industrierechner. Neben der Kommunikationsschnittstelle, die der Lasersensor anbietet, wurde er aufgrund seiner hohen Ausgabeauflösung, anpassbare Pulsfrequenz und Genauigkeit gewählt. Der IPC, worauf das processit-Programm läuft, ist für die rechenintensive Aufgabe der Detektion sehr leistungsfähig.

Das processit-Programm bietet drei Hauptfunktionen an. Das erste Teil ist dient zur Interaktion mit dem Laserliniensensor. Es übernimmt die Aufgabe der Parametersetzung, vorverarbeitet die Messergebnisse aus dem Sensor und stellt sie zur Verfügung bereit. Das zweite Teil des Programmpakets enthält Methoden zur Erkennung unterschiedlicher Schweißnähte oder einer Offline Werkzeugbahn für die Bearbeitung einer Freiformoberfläche. Das Detektionsverfahren kann durch externe Prozesse gesteuert werden. Der dritte Teil von processit widmet sich der Prozessteuerung und Bahngenerierung. Es steuert die Teilvorgänge des Detektionsverfahrens indem es die Methoden zur Detektion aufruft und kontrolliert. Letztendlich werden die Erkenntnisse aus dem Detektionsverfahren für die Erstellung einer Bahn für den Roboterarm verwendet, die danach an dem Programm für die Trajektorienplanung übermittelt wird. Die Basis von processit ist ROS, welches die Kommunikation zwischen den unterschiedlichen Teilprogramme und Hardwarekomponenten ermöglicht. Hierfür werden verschiedene standard sowie benutzerdefinierte ROS-Messages verwendet, die den Kernpartikel der ROS-basierten Kommunikation bilden. 

Diese Software- sowie Hardwarekomponente erfüllen die Anforderungen eines intelligenten kollaborierenden Robotersystems für die automatische Erkennung der Schweißnaht.

\section{Implikationen für die Industrie und Wissenschaft}

Die Robotik findet bislang überwiegend bei großen Unternehmen in der Produktion eine Anwendung. Roboter bieten bei der Abwicklung Produktionsaufgaben eine hohe Wiederholgenauigkeit an und ermüden sich nicht. Sie steigern durchaus die Effizienz und Produktivität in vielen Prozessen. Konventionelle Ansätze waren allerdings nur für die Fertigung in großen Losgrößen rentabel. Mit einem intelligentem kollaborierendem Schweißroboter-System kann die Einsetzbarkeit der Robotik für kleine und mittelständische Unternehmen verbessert werden. Das intelligente System ist in der Lage, das Schweißverfahren mit der Genauigkeit eines Roboters und die Intuition eines Menschen zu beherrschen. Mit dem Einbinden eines Laserliniensensor wird das System mit einem technischen Auge ausgestattet, welches die Programmierung des Roboters um ein Vielfaches erleichtert. Bauteile können schnell in kleinen Losgrößen ohne hohen Zeit-, Kosten- und Menschenaufwand gefertigt werden. Das demographische Problem des Fachkräftemangels kann auch durch dieses Robotersystem gelöst werden. Es ist fähig, mit anderen Mitarbeitern zusammenzuarbeiten, ohne sie in Gefahr zu bringen.

Da das processit-Paket mit starkem Einfluss des Entwurfsprinzips der losen Kopplung entwickelt wird, kann es sehr einfach durch zusätzliche Funktionalitäten erweitert werden. Mittels ROS können sehr schnell diese Funktionalitäten in das Hauptprogramm integriert oder ausgeschlossen werden. Die Hardwarekomponenten sind genauso modular wie die Softwareteile. Das System lässt sich für die Abwicklung anderer Aufgaben schnell umrüsten. Somit bietet sich das System als eine gute Basis zur Entwicklung neuer intelligenter Verfahren zur Abwicklung anderer Herstellungsprozesse an. Entwicklungen im Bereich der Computer-Vision, Trajektorienberechnung, Robotergenauigkeit und maschinelles Lernen sind wichtige Beitragsleistungen zu der Wissenschaft. Diese bieten Plattformen für weitere Forschungen in diesen Bereichen an.

\section{Limitationen und Zukunftspotenzial}
Der Zweck dieser Arbeit war es, die Zusammensetzung und Funktion des Robotersystems klarzulegen. Dies ist ein großes Projekt, welches durch mehreren Beteiligten betreut wird. Aus diesem Grund reicht die Umfang dieser Arbeit nicht aus, sehr detailliert die einzelnen Komponenten des Systems anzuschauen. Daher konzentriert sich diese Arbeit nur auf die wichtigsten Komponenten des Systems

Das Robotersystem hat auch technische Limitationen. Aufgrund der physischen Einschränkungen des Roboters ist die Größe des zu bearbeitenden Werkstückes beschränkt. Das Schweißen sehr großer Bauteile ist somit nicht mit diesem System möglich. Die Erkennung einer Stumpfnaht sowie einer offline Werkzeugbahn befinden sich in primären Phasen. Diese Methoden sind noch nicht ausgereift und müssen verbessert werden. Das Programm ist aktuell nicht in der Lage, die Schweißparameter während eines laufenden Prozesses auf Basis der Werkstückgeometrie anzupassen. Dies kann Einflüsse auf die Schweißqualität haben.

Es besteht das Potenzial, diese Limitationen grundsätzlich zu beseitigen. Die Methoden zur Erkennung der Stumpfnaht und Offline Werkzeugbahn sollen weiterentwickelt werden, um die Einsatzmöglichkeiten des Robotersystems zu verbessern. Weiterhin soll die Hardware und Software des Systems erweitert werden, um neue Werkzeuge und Herstellungsprozesse einzugliedern. 

Aufgrund des Ansatzes zur Detektion der Schweißnaht ist das Verfahren nur auf die Erkennung von Ebenen aus der Punktwolke beschränkt. Aus diesem Grund können geometrische Merkmale des Werkstückes wie Löcher, Schlitze, Stoßgeometrien, etc. nicht durch das Programm erkannt werden. Dies führt dazu, dass die Schweißparameter auf Basis der Werkstückgeometrie nicht angepasst werden. Es besteht hier das Potenzial, durch eine Überarbeitung des Erkennungsverfahrens, diese geometrische Merkmale ins Betracht zu ziehen. Somit können Schweißparameter auf Basis dieser Merkmale angepasst werden.

Die Automatisierung ist ein brandaktuelles Zukunftsthema. Flexible und intelligente Produktionssysteme, die sich einfach einrichten und anwenden lassen, werden zukünftig ausschlaggebend für die Konkurrenzfähigkeit der kleinen und mittelständischen Unternehmen sein. Die Entwicklung solcher Systeme ist im Gegensatz dazu die Herausforderung von Heute.

	\printbibliography
%	\appendix
\chapter{Algorithmen} \label{Algorithmen}
\begin{algorithm}
	\caption{Ablauf des IEFD-Verfahrens}
	\label{alg: IEFD_Ablauf}
	\begin{algorithmic}[1]
		\State $\textbf{Input: } \textit{Point cloud with scan lines} = \{P\}, \text{Parameter: } K_1, d_{t1}, \alpha, K_2, d_{t2}, \phi$
		\State $point gap \gets CalculateAvgGap(P)$
		\State $n \gets 0,004/\text{point gap}$
		\State $k \gets 0,0008/\text{point gap}$
		\State $f \gets 0,00004/\text{point gap}$
		\State \textit{raw points \{R\}} $\gets$ \{\}
		\State \textit{reused points \{RE\}} $\gets$ \{\}
		\State $\textit{false edge start \{FS\}} \gets \{\}$
		\State $\textit{false edge end \{FE\}} \gets\{\}$
		\State \textit{segments \{S\}} $\gets$ \{\}
		\State \textit{previous reused edge points \{ER\textsubscript{p}\}} $\gets \{\}$
		\For{Scan line S\textsubscript{i} \textbf{in} \{P\}}
		\State $\textit{R} \gets \textit{R} + P[S_i]$
		\If{$\textit{\textbf{size(}R\textbf{)}} \geq n - k$}
		\State$\text{RE} \gets \textit{RE} + P[S_i]$
		\EndIf
		\If{$\textbf{size}(\textit{R}) \leq f$}
		\State $\textit{FS} \gets \textit{FS} + P[S_i]$
		\ElsIf{$\textbf{size}(\textit{R}) \geq n - f$}
		\State $\textit{FE} \gets \textit{FE} + P[S_i]$
		\EndIf	
		\If{$\textit{\textbf{size(}R\textbf{)}}\geq n$}
		\State \textit{Edge Points \{E\}} $\gets$ \{\}
		\State \textit{Adjusted reused edge points \{ER\}} $\gets$ \{\}
		\State \textit{E, ER} $\gets$ \textit{\hyperref[alg:find_edge_points]{FindEdgePoints}}(\textit{R, RE, FS, FE, K\textsubscript{1}, d\textsubscript{t1},}$\alpha$)
		
		\State \textit{neighbours map \{N\}} $\gets$ \{\}
		\State \textit{vectors map \{V\}} $\gets$ \{\}
		\State \textit{N, V} $\gets$ \textit{\hyperref[alg:compute_vectors]{ComputeVectors}}(\textit{E, ER\textsubscript{p},} $K_2$, $d_{t2}$)
		
		\State \textit{\{S\}} $\gets$ \textit{\hyperref[alg:apply_region_growing]{ApplyRegionGrowing}}(\textit{E, ER\textsubscript{p}, S, N, V, } $K_2, \phi$)
		
		\State $R \gets \{\}$
		\State $R \gets \{RE\}$
		\State $RE \gets \{\}$
		\State $ER\textsubscript{p} \gets \{ER\}$
		\State $FS \gets \{\}$
		\State $FE \gets \{\}$
		\EndIf
		\EndFor
		
		
	\end{algorithmic}
\end{algorithm}

\begin{algorithm}
	\caption{Ablauf der \textit{\hyperref[alg:find_edge_points]{FindEdgePoints}} Funktion}
	\label{alg:find_edge_points}
	\begin{algorithmic}[1]
		\Function{\textit{\hyperref[alg:find_edge_points]{FindEdgePoints}}}{\textit{R, RE, FS, FE, K\textsubscript{1}, d\textsubscript{t1},} $\alpha$}
		\State \textit{removed indices \{RI\}} $\gets$ \{\}
		\State \textit{(R, RI)} $\gets$ \textit{UniformSampling(R, d\textsubscript{t1})}
		\State \textit{removed indices map \{RM\}} $\gets$ \{\}
		\State \textit{point shifts \{PS\}} $\gets$ \{\}
		\State \textit{(RM, PS)} $\gets$ \textit{MarkPoints(RI)}
		\State \textit{FS\textsubscript{copy}} $\gets$ \textit{CorrectPoints(FS, RM, PS)}
		\State \textit{FE\textsubscript{copy}} $\gets$ \textit{CorrectPoints(FE, RM, PS)}
		\State \textit{RE\textsubscript{copy}} $\gets$ \textit{CorrectPoints(RE, RM, PS)}
		\State $\{E\} \gets \{\}$
		\For{$point\  o = 0 \textbf{ to size}(\{P\})$}
		\State $\textit{Nearest neighbours} \{N_o\} \gets \{\}$
		\State $N_o \gets$ \textit{NearestNeighbourSearch(P, o, K\textsubscript{1})}
		\State \textit{normal vector $\vec{n\textsubscript{o}}$} $\gets$ \{0, 0, 0\}
		\State \textit{Inliers \{I\textsubscript{N\textsubscript{o}}\}} $\gets$ \{\}
		\State \textit{I\textsubscript{N\textsubscript{o}}} $\gets$ \textit{ApplyRansacPlane(N\textsubscript{o}, d\textsubscript{t1})}
		\State $\vec{n\textsubscript{o}}$ $\gets$ \textit{OptimizeNormal(I\textsubscript{N\textsubscript{o}})}
		\If{$o \notin I\textsubscript{N\textsubscript{o}} \textbf{or size}(I\textsubscript{N\textsubscript{o}}) < 3$}
		\State \textit{\textbf{continue}}
		\EndIf
		\State $(\vec{u}, \vec{v}) \gets$ \textit{GetFrame($\vec{n\textsubscript{o}}$)}
		\State G\textsubscript{$\theta$} $\gets$ \textit{ComputeAngularGap(o, I\textsubscript{N\textsubscript{o}}, $\vec{u}$, $\vec{v}$)}
		\If{G\textsubscript{$\theta$} $\geq \frac{\pi}{2}$}
		\State $E \gets o$
		\EndIf
		\EndFor
		\State $E \gets \textit{RemoveFalseEdges(E, FS\textsubscript{copy}, FE\textsubscript{copy})}$
		\State \textit{ER} $\gets$ \textit{ReusedEdgePoints(E, RE\textsubscript{copy})}
		
		\State \Return $(E, ER)$
		\EndFunction
	\end{algorithmic}
\end{algorithm}

\begin{algorithm}
	\caption{Ablauf der \textit{\hyperref[alg:compute_vectors]{ComputeVectors}} Funktion}
	\label{alg:compute_vectors}
	\begin{algorithmic}[1]
		\Function{\textit{\hyperref[alg:compute_vectors]{ComputeVectors}}}{\textit{E, ER\textsubscript{P}, }$K\_2$, $d_{t2}$}
		\State \textit{first reused point r\textsubscript{1}} $\gets$ \{ER\textsubscript{P}\}[0]
		\State \textit{neighbours map \{N\}} $\gets$ \{\}
		\State \textit{vectors map \{V\}} $\gets$ \{\}
		\For{\textit{point p = r\textsubscript{1}} \textbf{to size}{(\{E\})}}
		\State \textit{nearest neighbour \{N\textsubscript{p}\}} $\gets \{\}$
		\State \textit{N\textsubscript{p}} $\gets$ \textit{NearestNeighbourSearch(E, p, K\textsubscript{2})}
		\If{$\textbf{size}(\textit{text}) < 3$}
		\State $\{N\}[p] \gets \{N\textsubscript{p}\}$
		\State $\{V\}[p] \gets \textit{CalculateVector(N\textsubscript{p})}$
		\State \textit{\textbf{continue}}
		\EndIf
		\State \textit{Inliers \{I\textsubscript{N\textsubscript{p}}\}} $\gets \{\}$
		\State \textit{I\textsubscript{N\textsubscript{p}}} $\gets$ \textit{ApplyRansacLine(N\textsubscript{p}, d\textsubscript{t2})}
		\While{\textit{p} \textbf{not in} \textit{I\textsubscript{N\textsubscript{p}}}}
		\State \textit{remove I\textsubscript{N\textsubscript{p}} from N\textsubscript{p}}
		\State \textit{I\textsubscript{N\textsubscript{p}}} $\gets$ \textit{ApplyRansacLine(N\textsubscript{p}, d\textsubscript{t2})}
		\EndWhile
		\State \textit{Direction vector $\vec{p_{p}}$} $\gets$ \textit{OptimizeVector(I\textsubscript{N\textsubscript{p}})}
		\State $\{N\}[p] \gets I_{N_p}$
		\State $\{V\}[p] \gets \vec{a_p}$
		\EndFor
		\State \Return (N, V)
		\EndFunction
	\end{algorithmic}
\end{algorithm}

\begin{algorithm}
	\caption{Ablauf der \textit{\hyperref[alg:apply_region_growing]{ApplyRegionGrowing}} Funktion}
	\label{alg:apply_region_growing}
	\begin{algorithmic}[1]
		\Function{\textit{\hyperref[alg:apply_region_growing]{ApplyRegionGrowing}}}{\textit{E, RE\textsubscript{p}, S, N, V, K\textsubscript{2}}, $\phi$}
		\State \textit{seed counter $x$} $\gets 0$
		\State \textit{point labels \{PL\}} $\gets$ \textbf{size}\textit{(\{E\} + \{S\})}
		\State \textit{All unlabeled points from E \textbf{in} PL get value }$-1$
		\State \textit{point residuals \{PR\}} $\gets$ \textit{E}
		\State \textbf{sort} \textit{PR by no. of neighbours}
		\State \textit{number of segmented points n\textsubscript{pts}} $\gets$ \textit{no. of labels} $\neq -1$
		\State \textit{No. of segments n\textsubscript{segs}} $\gets \textbf{max}\textit{(PL)} + 1$
		\State \textit{n\textsubscript{pts}} $\gets$ \textit{\hyperref[alg: extend_segments]{ExtendSegment}(S, ER\textsubscript{p}, PL, N, V, K\textsubscript{2}, $\phi$)}
		\State \textit{initial seed point s\textsubscript{i}} $\gets$ \textit{first point i \textbf{in} PR with \{PL\}[i]}$= -1$
		\While{$\textit{n\textsubscript{pts}} < \textbf{size}\textit{(\{E\})}$}
		\State \textit{segment id C} $\gets$ \textit{n\textsubscript{segs}}
		\State \textit{new points new\textsubscript{pts}} $\gets$ \textit{\hyperref[alg:grow_segment]{GrowSegment}(s\textsubscript{i}, C, PL, N, V,  K\textsubscript{2}, $\phi$)}
		\State \textit{\{S\}[C]} $\gets$ \textit{new\textsubscript{pts}}
		\State $n_{pts} \gets new_{pts}$
		\State \textit{n\textsubscript{segs}} $\gets + 1$
		\For{$\textit{seed i}= x + 1 \textbf{ to size}(PR)$}
		\If{$\{PL\}[i] = -1$}
		\State $s_i = \{PR\}[i]$
		\State $x = i$
		\State \textit{\textbf{break}}
		\EndIf
		\EndFor
		\EndWhile
		\State \Return \textit{S}
		\EndFunction
	\end{algorithmic}
\end{algorithm}

\begin{algorithm}
	\caption{Ablauf der \textit{\hyperref[alg: extend_segments]{ExtendSegment}} Funktion}
	\label{alg: extend_segments}
	\begin{algorithmic}[1]
		\Function{\textit{\hyperref[alg: extend_segments]{ExtendSegment}}}{\textit{S, ER\textsubscript{p}, PL, N, V, K\textsubscript{2}}, $\phi$}
		\State \textit{number of segmented points n\textsubscript{prev}} $\gets$ \textbf{size}\textit{(\{PL\}), for labels $\neq -1$ }
		\State \textit{number of new points added n\textsubscript{add}} $\gets 0$
		\State \textit{point labels copy \{PL\textsubscript{copy}\}} $\gets$ \{PL\}
		\State \textit{reset indices map \{RI\}} $\gets \{\}$
		\For{\textit{index i \textbf{in} \{ER\textsubscript{p}\}}}
		\For{\textit{neighbour n in \{N\}[i]}}
		\State \textit{label} $\gets \{PL\}[n]$
		\If{\textit{label }\textbf{in }\textit{\{RI\}}}
		\State \textit{\textbf{continue}}
		\EndIf
		\State \textit{\{PL\}[n]} $\gets -1$
		\State \textit{\{S\}[label]} $\gets\textit{ \{S\}[label]} - 1$
		\State $RM \gets n$
		\EndFor
		\EndFor
		\For{\textit{point/seed index s\textsubscript{i} }\textbf{in } \textit{\{ER\textsubscript{p}\}}}
		\If{$\textit{\{PL\}[s\textsubscript{i}]} > -1$}
		\State \textit{\textbf{continue}}
		\EndIf
		\If{$\textit{\{PL\textsubscript{copy}\}[s\textsubscript{i}]} < 0$}
		\State \textit{\textbf{continue}}
		\EndIf
		\State $C \gets \{PL\textsubscript{copy}\}[s\textsubscript{i}]$
		\State \textit{additional points in segment new\textsubscript{add}} $\gets$ \textit{\hyperref[alg:grow_segment]{GrowSegment}(s\textsubscript{i},C,PL,N,V,K\textsubscript{2},$\phi$)}
		\State \textit{\{S\}[C]} $\gets new_{add}$
		\State \textit{n\textsubscript{add}} $\gets$ \textit{new\textsubscript{add}}
		\EndFor
		\For{\textit{index i \textbf{in} \{ER\textsubscript{p}\}}}
		\For{\textit{neighbour n in \{N\}[i]}}
		\If{$\textit{\{PL\}[p\textsubscript{i}]} > -1$}
		\State \textit{\textbf{continue}}
		\EndIf
		\If{$\textit{\{PL\textsubscript{copy}\}[p\textsubscript{i}]} < 0$}
		\State \textit{\textbf{continue}}
		\EndIf
		\State $C \gets \{PL\textsubscript{copy}\}[n]$
		\State \textit{\{PL\}[n]} $\gets$ \textit{C}
		\State \textit{\{S\}[C]} $\gets + 1$
		\State $n_{add} \gets + 1$
		\EndFor
		\EndFor
		\State \Return $n_{add} - n_{prev}$
		\EndFunction
		
	\end{algorithmic}
\end{algorithm}

\begin{algorithm}
	\caption{Ablauf der \textit{\hyperref[alg:grow_segment]{GrowSegment}} Funktion}
	\label{alg:grow_segment}
	\begin{algorithmic}[1]
		\Function{\textit{\hyperref[alg:grow_segment]{GrowSegment}}}{\textit{s\textsubscript{i}, C, PL, N, V, K\textsubscript{2}, $\phi$}}
		\State \textit{Queue of seeds \{S\}} $\gets$ \textit{s\textsubscript{i}}
		\State \textit{\{PL\}[s\textsubscript{i}]} $\gets$ \textit{C}
		\State \textit{number of points in segment n\textsubscript{C}} $\gets 1$
		\While{$\textbf{size}\textit{(S)}\geq 0$}
		\State \textit{current seed s\textsubscript{C}} $\gets$ \textbf{dequeue}\textit{(S)}
		\For{$\textit{neighbour n}= 0 \textbf{to size}\textit{(\{N\}[s\textsubscript{C}])}$}
		\State \textit{index i} $\gets$ \textit{\{N\}[s\textsubscript{C}][n]}
		\State \textit{label} $\gets \{PL\}[i]$ 
		\If{$label \neq -1$}
		\State \textit{\textbf{continue}}
		\EndIf
		\State \textit{seed vector vec\textsubscript{s\textsubscript{C}}} $\gets \{V\}[s_C]$
		\State \textit{neighbour vector vec\textsubscript{i}} $\gets \{V\}[i]$
		\State \textbf{bool }\textit{added} $\gets$ \textit{CheckPoint(vec\textsubscript{s\textsubscript{C}}, vec\textsubscript{i}, $\phi$)}
		\If{\textbf{not} \textit{added}}
		\State \textit{\textbf{continue}}
		\EndIf
		\State \textit{\{PL\}[i]} $\gets C$
		\State $n_C \gets +1$
		\State $S \gets i$
		\EndFor
		\EndWhile
		\State \Return $n_C$
		\EndFunction
	\end{algorithmic}
\end{algorithm}

\chapter{Abbildungen} \label{Bilder}
\begin{figure}[h]
	\centering
	\begin{subfigure}{0.49\textwidth}
		\includegraphics[width=\linewidth]{Abbildungen/Bauteil_1.png}
		\centering
		\caption{Bauteil 1}
		\label{fig:bauteil_1}
	\end{subfigure}
	\hfill
	\begin{subfigure}{0.49\textwidth}
		\includegraphics[width=\linewidth]{Abbildungen/Bauteil_2.png}
		\centering
		\caption{Bauteil 2}
		\label{fig:bauteil_2}
	\end{subfigure}
	\vfill
		\begin{subfigure}{0.49\textwidth}
		\includegraphics[width=\linewidth]{Abbildungen/Bauteil_3.png}
		\centering
		\caption{Bauteil 3}
		\label{fig:bauteil_3}
	\end{subfigure}
	\hfill
	\begin{subfigure}{0.49\textwidth}
		\includegraphics[width=\linewidth]{Abbildungen/Bauteil_4.png}
		\centering
		\caption{Bauteil 4}
		\label{fig:bauteil_4}
	\end{subfigure}
\caption{Alle Bauteile für den Test 3 aus Abschnitt~\ref{test_3_part_2}}
\label{fig:bauteile_test_3_2}
\end{figure}

\begin{figure}[h]
	\centering
	\begin{subfigure}{0.49\textwidth}
		\includegraphics[width=\linewidth]{Abbildungen/Segmente_1_extra.png}
		\centering
		\caption{Segmente von Bauteil 1}
		\label{fig:Segmente_1}
	\end{subfigure}
	\hfill
	\begin{subfigure}{0.49\textwidth}
		\includegraphics[width=\linewidth, height=0.15\textheight]{Abbildungen/Segmente_2_extra.png}
		\centering
		\caption{Segmente von Bauteil 2}
		\label{fig:Segmente_2}
	\end{subfigure}
	\vfill
	\begin{subfigure}{0.49\textwidth}
		\includegraphics[width=\linewidth, height=0.15\textheight]{Abbildungen/Segmente_3_extra.png}
		\centering
		\caption{Segmente von Bauteil 3}
		\label{fig:Segmente_3}
	\end{subfigure}
	\hfill
	\begin{subfigure}{0.49\textwidth}
		\includegraphics[width=\linewidth, height=0.15\textheight]{Abbildungen/Segmente_4_extra.png}
		\centering
		\caption{Segmente von Bauteil 4}
		\label{fig:Segmente_4}
	\end{subfigure}
	\caption{Alle Segmente für den Test 3 aus Abschnitt~\ref{test_3_part_2}}
	\label{fig:Segmente_test_3_2}
\end{figure}

\end{document}